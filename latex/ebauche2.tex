\documentclass[a4paper]{article} % papier A4
\usepackage[utf8]{inputenc}      % accents dans le source
\usepackage[T1]{fontenc}         % accents dans le pdf
\usepackage{textcomp}            % symboles complémentaires (euro)
\usepackage[frenchb]{babel}      % titres en français
\usepackage{amsmath}
\usepackage{amsthm}
\usepackage{amssymb}
\usepackage[colorlinks=false]{hyperref}
\usepackage{enumerate}
\usepackage{algorithmic}
\usepackage{pgf}
\usepackage{tikz}
\usepackage{tikz-cd}
\usetikzlibrary{matrix,arrows,decorations.pathmorphing}
\numberwithin{equation}{section}
\newcommand\nroot[1]{\textit{#1}\up{\textit{ième}}}
\newcommand\zmodn[1]{\mathbb{Z}/#1\mathbb{Z}}
\newcommand\zmodninv[1]{(\mathbb{Z}/#1\mathbb{Z})^{\times}}
\newcommand\GF[1]{\mathbb{F}_{#1}}
\newcommand\Irr[2]{\textup{Irr}_{#1}(#2)}
\renewcommand{\algorithmicrequire}{\textbf{Input:}}
\renewcommand{\algorithmicensure}{\textbf{Ouput:}}
\newcommand\Tr[1]{\textup{Tr}\left(#1\right)}
\begin{document}
\newtheorem{thm}{Thèorème}[section]
\newtheorem{lem}[thm]{Lemme}
\newtheorem{cor}{Corollaire}[thm]
\newtheorem{prop}[thm]{Proposition}
\theoremstyle{definition}
\newtheorem{defn}[thm]{Définition}
\newtheorem*{ex}{Exemple}
\theoremstyle{remark}
\newtheorem{rem}{Remarque}[thm]
\section*{Remerciements}
\section*{Introduction}
Soit $\GF{q}$ le corps fini à $q = p^r$ éléments, avec $p$ un nombre premier. On
définit alors $\overline{\GF{q}}$ sa clôture algébrique. Lorsqu'on a besoin de
faire des calculs dans cette clôture, il est peut être plus aisé de calculer
directement sur un corps finis en particulier. Il est donc important de pouvoir
se déplacer rapidement entre les différents sous-corps de cette clôture.\par
Il est connu et prouvé que de deux corps finis de même cardinal sont reliés par
un isomorphisme. Cependant la preuve de ce résultat n'est pas constructive,
trouver un tel isomorphisme demande alors un travail supplémentaire.\par
La situation est la suivante, nous avons deux extensions de $\GF{q}$ de même
degré et définies par deux polynôme irréductible distincts $f$ et $g$. On notera
:
\[k_1=\GF{q}[X]/(f)\textup{ et }k_2=\GF{q}[Y]/(g)\]
le but est donc de trouver un isormophisme reliant ces deux corps. Une méthode
simple et immédiate est d'envoyer $x = \bar{X}$ sur une racine de $f$ dans
$k_2$. Le problème est que cela revient à factoriser le polynôme $f$ ce qui est
beaucoup trop lent.\par
On va donc dans ce rapport étudier deux méthodes permettant, \textit{via}
l'utilisation de racines de l'unité et de courbes elliptique, de calculer
rapidement de tels isomorphismes.


\section{Préliminaire/contexte théorique}

\subsection{Corps finis}
Dans cette section on va rappeler les définitions et démontrer certains 
résultats liés aux corps finis. Principalement, on étudiera les extensions 
de corps et la théorie de Galois de base sur les corps finis. On ne considérera
que des corps commutatifs.

\subsubsection{Définitions}

Soit $\Omega = (K, +, .)$ un triplet où $K$ est un ensemble et $+$ et $.$ deux
lois de composition interne sur $K$. On dit que $\Omega$ est un corps si : 
\begin{enumerate}[(i)]
\item $(K, +)$ est un groupe abélien,
\item $(K\setminus\lbrace0\rbrace,.)$ est un groupe (abélien ou non, dans la 
littérature francophone),
\item la première loi est distributive par rapport à la seconde,
\item $K$ n'a pas de diviseurs de $0$.
\end{enumerate}
Par abus de notation, on désignera désormais le corps $\Omega$ par son ensemble
$K$. On notera $0$ et $1$ les éléments neutres de la première et seconde loi 
respectivement; on omettra régulièrement de noter la seconde loi dans les 
opérations sur $K$.\par
On appelle la caractéristique d'un corps $K$, le plus petit entier $p$ tel que pour 
tout $x\in K$ on ait :
\[p(x) = \underbrace{x + x + \dots + x}_{p fois} = 0\]
Elle peut être nulle ou égale à nombre premier. Dans notre cas, on s'intéressera
au corps finis, c'est-à-dire tels que $K$ soit un ensemble fini et la caractéristique 
est non-nulle.\\\par
Un homomorphisme de corps est nécessairement injectif. En effet, soit deux corps 
$K$ et $L$ de caractéristique positive ou nulle. Soit $f : K \to L$, on a alors : 
\[f(x) = 0 \Leftrightarrow x.f(1) = 0 \Leftrightarrow x = 0\]
puisque un homomorphisme doit conserver la structure, $f(1) = 1$. S'il existe un
tel homomorphisme entre deux corps, on a $K\subset L$ et on dit que $L$ est un
surcorps de $L$ ou $K$ est un sous-corps de $L$.\par
On appelle corps premier un corps qui n'a aucun sous-corps. Dans le cas des
corps finis, il s'agit du corps $\GF{p}$ à $p$ éléments, pour $p$ premier. Ce
qui nous donne le théorème suivant :

\begin{thm}
\label{corfincarprem}
Soit $K$ un corps fini, alors $K$ a $p^n$ éléments, où $p$ premier est la
caractéristique du corps premier et $n$ le degré de $K/\GF{p}$.
\end{thm}
\begin{proof}
De façon informelle, on appelle le degré de $L$ sur $K$ la dimension de $L$
en tant que $K$-espace vectoriel. On discutera plus en détails de cette
notion dans le point suivant \textit{(réf au résultat)}.\par
Montrons le lemme suivant :
\begin{lem}
Un corps fini a pour caractéristique un nombre premier.
\end{lem}
\begin{proof}
Comme $K$ est intègre en tant qu'anneau, sa caractéristique est plus grande ou
égale à $2$. Supposons que la caractéristique de $K$ soit $p = km$ pour $k,
m\in\mathbb{Z}$. Alors pour $e\in K$ on a e :
\[0 = ne = (km)e = (ke)(me)\]
puisque $K$ n'a pas de diviseurs de $0$, alors soit $ke = 0$ ou $me = 0$. Il
s'ensuit que pour tout $r\in K$, on a $kr = (ke)r = 0$ ou $mr = (me)r = 0$ ce
qui contredit la minimalité de la caractéristique.
\end{proof}
Donc $K$ est de caractéristique $p$ premier et son corps premier est (isomorphe
à) $\GF{p}$. On fini avec le lemme qui suit :
\begin{lem}
Soit $K$ un corps fini contenant un corps fini $k$ à q éléments. Alors $K$ a
$q^m$ élément où $m$ est le degré de $K$ en tant que $k$-espace vectoriel.
\end{lem}
Si $K$ est un $k$-espace vectoriel de dimension fini, il admet alors une base
d'éléments de $k$. Si on écrit chaque éléments de $K$ comme une combinaison
linéaire des éléments de cette base, chaque coeffecient de cette combinaison a
$q$ valeurs possibles, d'où $K$ à $q^m$ éléments; ce qui achève la démonstration
du lemme. Il suffit alors de remplacer $k$ par $\GF{p}$ et $q$ par $p$ et le
théorème est démontré.\\
\end{proof}
Le prochain théorème assure la commutativité des corps finis, et même d'autre
structures plus générales, on ne va pas le démontrer. Pour une preuve se référer
par exemple à \cite{

Si $K$ est un corps à $q$ éléments, alors pour tout $x\in K$ on a $x^q = x$. En
effet, pour $x = 0$ c'est immédiat. Pour $x\neq 0$, on sait que le groupe
multiplicative de $K$, noté $K^{\times} = K\setminus\lbrace0\rbrace$, est un
groupe fini à $q-1$ éléments par définition, donc $x^{q-1} = 1$ pour tout $x\in
K^{\times}$ et il suffit de multiplier encore une fois par $x$ pour obtenir le
résultat voulu.\par
On en déduit immédiatement que le polynôme $X^q - X$ se scinde dans $K$
puisque étant de degré $q$, il a au maximum $q$ racines qui sont déjà les
éléments de $K$. On peut dire que $K$ est un corps de décomposition
\textit{(mettre une réf à la définition)} de $X^q - X$ sur le corps premier de
$K$.\par
On notera que dans les corps finis et leurs extensions, l'élévation à une
puissance égal au cardinal d'un corps fini est un automorphisme particulier. On
l'appelle l'automorphisme de Frobenius et on le note $Fr_q$ pour $Fr_q(x) =
x^q$. Ce n'est pas immédiat qu'il s'agit là d'un homomorphisme, mais un simple
calcul permet de s'en assurer.\\\par
Finissons ce point par le théorème suivant qui assure l'existence et l'unicité
à isomorphisme près des corps finis; c'est à partir de ce théorème que le
travail de tout ce rapport va se faire :

\begin{thm}
Pour tout nombre premier $p$ et tout entier positif $n$, il existe un corps
fini à $p^n$ éléments. Tout corps fini à $q = p^n$ éléments est isomorphe au
corps de décomposition de $X^q - X$ sur $\GF{p}$. On parlera alors du corps à
$q$ éléments et on le notera $\GF{q}$.
\end{thm}
\begin{proof}
\textit{(Existence)} Pour $q = p^n$, on considère le polynôme $X^q - X$ dans 
$\GF{p}[X]$ et on note $K$ son corps de décomposition sur $\GF{p}$. Le polynôme 
est séparable ou n'a aucune racine multiple puisque sa dérivée est égale à $qX^{q-1}
- 1 = -1$. Posons $S = \lbrace x\in K : x^q - x = 0\rbrace$, alors $S$ est un 
sous-corps de $K$; $0$ et $1$ sont dans $K$ et d'après les propriétés sur le 
Frobenius et les résultats obtenus plus haut on a :
\[(a - b)^q = a^q - b^q = a - b \textup{ et }(ab^{-1})^q = a^q - b^{-q} =
ab^{-1}\]
Ainsi, $S$ contient toutes les racines de $X^q - X$ mais comme $K$ a déjà $q$
élémens, alors $K = S$ est un corps à $q$ éléments.\par
\textit{(Unicité)} Soit $K$ le corps à $q = p^n$ éléments, d'après le théorème
\ref{corfincarprem}, est de caractéristique $p$ et contient $\GF{p}$. On en
déduit que $K$ est un corps de décomposition de $X^q - X$ sur $\GF{p}$,
puisqu'il est scindé sur $K$, et l'unicité de déduit de l'unicité des corps de
décomposition, théorème \ref{cordec}.
\end{proof}


\subsubsection{Extensions de corps}
On dit que $K$ est une extension de corps de $k$ s'il existe un morphisme de corps 
$\varphi : k \to K$. Ou de façon équivalente, si $k \subseteq K$ alors $K$ est une 
extension (de corps) de $k$. On note aussi $K/k$ une extension de corps.
Si on a $k\subseteq L \subseteq K$, alors $L/k$ est une sous-extension de
$K/k$.\par
Soit $K/k$ une extension de corps et $S$ une partie de $K$. Le sous-corps $L := 
k(S)$ de $K$ engendré par $S$ sur $k$ est le plus petite sous-corps de $K$ contenant 
$S$ et $l$. Si $S = \lbrace x_1,\dots,x_n \rbrace$ est fini, alors on note $L = 
K(x_1,\dots,x_n)$, on dit alors que l'extension est de type fini. L'extension $L/k$ 
est dite monogène ou simple si elle est engendré par un seul élément.\par
Si $K/k$ est une extension on peut voir $K$ comme un $k$-espace vectoriel ou une 
$k$-algèbre. On appelle $[K:k] := dim_k(K)$ le degré de l'extension. On dit qu'une 
extension $K/k$ est de degré fini si $[K:k] < \infty$.\par

\begin{thm}
Soit $k \subseteq L \subseteq L$ des extensions de corps de degré fini. Alors on a :
\[[K:k] = [K:L][L:k]\]
\end{thm}
\begin{proof}
Soit $[K:L] = m$ et $[L:k] = n$. On a donc que $K$ est un $L$-espace vectoriel de dimension $m$ et $L$ est un $k$-espace vectoriel de dimension $n$, le théorème revient à montrer que $K$ est un $k$-espace vectoriel de dimension $mn$. Or, d'après ce qui précède, on a $L \simeq k^n$ et $K \simeq L^m$, d'où
\[K \simeq \underbrace{L \oplus\dots\oplus L}_{m fois}\simeq\underbrace{k^n\oplus\dots\oplus k^n}_{m fois} \simeq k^{nm}\]
ce qui prouve le théorème.\\
\end{proof}

On dit qu'un élément $x\in K$ est algèbrique sur $k$ s'il existe un polynôme unitaire à coefficient dans $k$ qui annule $x$. L'ensemble des éléments algébrique d'un corps (sur un sous-corps) forme un corps \textit{(CF. Cours d'algèbre de Perrin ou autre)}. On dit qu'une extension $K/k$ est algébrique si tous les éléments de $K$ sont algébriques sur $k$.

\begin{prop}
Toute extension de degré fini est algébrique et de type fini.
\end{prop}
\begin{proof}
Si l'extension $K/k$ est de degré fini alors elle admet une base finie $(\alpha_1,\dots,\alpha_n)$ en tant que $k$-espace vectoriel, on a alors $K = k(\alpha_1,\dots,\alpha_n)$. Comme $[K:k] = n < \infty$, si $\alpha\in K$ alors $1, \alpha, \dots, \alpha^n$ satisfont une relation de dépendance linéaire, \textit{i.e.} il existe $a_0, \dots, a_n$ dans $k$ tels que :
\[a_0 + a_1.\alpha + \dots + a_n.\alpha^n = 0\]
puisque $K$ est un $k$-espace vectoriel. D'où tout $\alpha$ est algébrique sur $k$.\\
\end{proof}

\begin{prop}
Soit $\alpha\in L$ un élément algébrique sur $k$ et $f = X^d + a_1X^{d-1} + \dots + a_d$ son polynôme minimal sur $k$. Si on pose $d := \textup{deg }f$ alors on a :
\begin{enumerate}[(i)]
\item Les éléments $1, \alpha,\dots,\alpha^{d-1}$ forment une base de $k[\alpha]$ en tant qu'espace vectoriel.
\item $k[\alpha]$ est un corps, on a alors $k[\alpha] = k(\alpha)$.
\item On a $[k(\alpha):k] = d$.
\end{enumerate}
\end{prop}
\begin{proof}
(i) Il suffit de multiplier l'identité $\alpha^d = -a_1\alpha^{d-1} - \dots - a_d$ par $\alpha^i$ pour $i\geq0$ et on montre par récurrence que :
\[\alpha^{d+i} \in k\cdot1 + \dots k\cdot\alpha^{d-1}\]
d'où $k[\alpha] = k\cdot1 + \dots k\cdot\alpha^{d-1}$. De plus, les éléments $1,\alpha,\dots,\alpha^{d-1}$ sont linéairement indépendant puisque si :
\[u_0\cdots1 + u_1\cdots\alpha + \dots u_{d-1}\cdots\alpha^{d-1} = 0\]
avec les $u_i\neq0$ cela contredirait la minimalité du degré de $f$.\par
(ii) $k[\alpha] \subset L$ est le sous-anneau d'un corps, il est donc intègre. De plus, pour tout $\beta\in k[\alpha]$ l'application $m_{\beta}$ de la multiplication par $\beta$ dans $k[\alpha]$ est $k$-linéaire et injective, puisque l'anneau est intègre. Pour finir, elle est aussi surjective puisque nous sommes en dimension finie, donc il existe $x\in k[\alpha]$ tel que $m_{\beta}(x) = \beta x = 1$.\par
(iii) Il résulte directement des deux points précédents.\\
\end{proof}

\subsubsection{Corps de rupture et corps de décomposition}

On peut aussi construire des extensions de corps \textit{via} des polynômes irréductibles. Concrètement, soit $k$ un corps et $P$ un polynôme irréductible unitaire dans $k[X]$. Comme $k[X]$ est principal (il est même euclidien) alors $(P)$ est un idéal maximal, on peut alors faire le quotient $k[X]$ par $(P)$ est on obtient un corps $K \simeq k[X]/(P)$. Dans ce cas, la classe de $X$ qu'on notera $x := \overline{X}$, est une racine du polynôme $P$ et engendre $K$ sur $k$.\par
 Pour encore aller plus loin, considérons une extension $L/k$ telle que $P$ admette une racine $\alpha\in L$. Si on note $\Irr{k}{\alpha}$ le polynôme minimal de $\alpha$, alors $\Irr{k}{\alpha}$ divise $P$, donc est égal à $\lambda P$ pour $\lambda\in k^{\times}$, puisque $P$ est irréductible. Alors le morphisme $k$-algèbre $\phi : k[X] \to L$ défini par $\phi(X) = \alpha$ induit un morphisme de $k$-algèbre $\varphi : K \to L$ tel que $\varphi(x) = \alpha$. Ce morphisme est unique puisque $x$ engendre $K = k(x)$.\\\par
 On vient de montrer en particulier le théorème suivant :

\begin{thm}
Soit $k$ un corps et $P$ un polynôme irréductible dans $k[X]$. Alors $K := k[X]/(P)$ est un sur-corps de $k$ dans lequel $P$ a au moins une racine, la classe de $\overline{X} = x$. On l'appelle le corps de rupture de $P$ sur $k$.\par
\end{thm}

Une autre notion importante est celle de corps de décomposition d'un polynôme non constant sur un corps $k$. Il s'agit du sur-corps $K \supset k$ contenant toutes les racines de $P$, qui est engendré par les racines susnommées.

\begin{thm}
\label{cordec}
Tout $P\in k[X]$ non constant admet un corps de décomposition unique à isomorphisme près.
\end{thm}
\begin{proof}
%TODO: À toi de voir si tu fais toute la démonstration ou non. C'est un peu long, mais il y a des résultats sympas.
Pour prouver l'existence du corps de décomposition, il suffit de se placer dans la clôture algèbrique de $k$, alors si on note $K_0$ le corps engendré par $\alpha_1,\dots,\alpha_n$ les racines de $P$, il est clair que $K_0$ est le corps de décomposition de $P$.\par
\end{proof}

\subsubsection{Théorie de Galois}
Définitions et résutlats de base (sans forcément tout prouver), Frobenius, 
Groupe de Galois, élément primitif, extensions galoisiennes, etc.

\subsubsection{Extensions et polynômes cyclotomiques}
Racines de l'unités et quelques mots sur les polynômes cyclotomiques et leurs 
extensions. Peut-être rajouter un ou deux mots sur le cas des corps finis (ça 
pourrait se mettre dans la section du dessus).

\subsection{Théorie des nombres/Idéaux}
\textit{Résumé :} Tout ce qui concerne la preuve de Rains avec les idéaux 
premiers, les normes d'ideaux, la factorisation unique, Galois, les périodes de 
Gauss, les périodes elliptiques (Mihailescu et al.) ou plus précisément ce qui 
permet d'y arriver.
\subsubsection{Corps de nombres}
Définitions (normes, traces, éléments algébriques, etc.)
\subsubsection{Idéaux premiers}
 Idéaux premiers sur les anneaux d'entiers de corps de nombres,
 décomposition unique, résultat de la théorie de Galois.
\subsubsection{Périodes de Gauss, elliptiques}
Énoncé et justifier les résultats donnés par Rains (notamment le lemme
expliquant la forme de $\GF{q}[X]/I$ pour $I$ sous-groupe du groupe de Galois);
peut-être à mettre directement dans la partie cyclotomique de la section
algorithme de Rains cyclotomique.

\subsection{Courbes elliptiques}
\textit{Résumé :}Définitions (j-invariant, nombres de points, tordues, module de
Tate (?) etc.), résultats généraux, sur les corps finis
\subsubsection{Équation de Weierstrass}
Rapide énoncé des changements de variable, comment on arrive à la forme
"simple"; liste des différents paramètres/invariants qui nous intéressent.
\subsubsection{Loi de groupe}
Je suis pas sûr pour cette partie, c'est un peu long à tout justifier et ce
n'est pas ce qu'on veut faire. Mais peut-être expliquer rapidement.\par
Cela dit, je suis très tenté de renvoyer au Silverman pour toute la théorie de
base en ne mettant uniquement les résultats dont on aurait besoin (donc on saute
cette partie).
\subsubsection{Sur les corps finis}
Nombres de points, théorème de Hasse, Frobenius, trace
\subsubsection{Tordues}
Twists quadratiques, courbes $y^2 = x^3 + x$ et $y^2 = x^3 + 1$ et leurs
tordues, pour préparer le terrain pour la partie elliptique de Rains


\section{Isomorphismes de corps finis}
Présentation du problème, énonciation des difficultés etc.

\subsection{Algorithme de Pinch (peut-être à mettre dans l'introduction de 
la section ?)}
Énonciation et descriptions de la méthode de Pinch, justification des 
résultats, explication des inconvénients et autre.
\subsubsection{Principe}
Explication de la méthode et illustration avec l'exempe de l'article, peut-être
?
\subsubsection{Problèmes/limitations}
Taille de m, choisir un élément au hasard

\subsection{Algorithme de Rains : méthode cyclotomique}
\textit{Résumé :}Commencer par une explication concise de la méthode, ce 
qu'elle change par rapport à Pinch.\par
Présenter pas à pas l'article de Rains (en gros), en enlevant ce qui nous 
concernerait moins et en rajoutant ce qu'il n'a pas mis (tous les détails 
sur l'algorithme convert et sa complexité).
\subsubsection{Principe}
Passer par une petite extension pour réduire la taille de m, utilisation des
éléments normaux, utilisation des périodes de Gauss.
\subsubsection{Éléments normaux}
Définitions et résultats utiles pour l'algorithme, justification de
l'algorithme (peut-être à mettre dans la partie Galois de la section corps
finis), conversion de la base polynomiale à la base normale.
\subsubsection{Algorithme \& analyse de complexité}
Détailler l'algorithme de l'article/code de Luca + analyser la complexité de ces
algos et les "prouver" (montrer que ça marche bien, cela dit c'est peut-être
largement faisable avant).

\subsection{Algorithme de Rains : méthode elliptique}
\textit{Résumé :}Résumer/Détailler le papier de Luca, rajouter les résultats sur
les twists, les périodes, détaillés l'algorithme et sa complexité, le prouver 
(probablement fait avec le papier de Luca et Mihailescu pour les périodes ?)
, détailler les différentes étapes, les paramètres, la façon de trouver m 
etc.
\subsubsection{Principe}
Choix d'un point d'ordre m sur une bonne courbe elliptique engendre 
l'extension, utilisation des périodes elliptiques afin d'avoir un élément 
stable par l'action du groupe de Galois (Énoncer les lemmes etc.)
\subsubsection{Choix des paramètres}
Justifier le choix de m par rapport à n et q, tout ce qui concerne la valeur 
propre d'ordre n et donc le choix de traces qui s'en suit, expliquer comment on
pick les courbes (à ce moment là on pourra aussi justifier avec le résultat 
sur les tordues).
\subsubsection{Algorithme \& Analyse de complexité}
Décrire l'algorithme et donner sa complexité.

\section{Comparatif/Implémentation/Résultats numériques}
Comparer les deux méthodes ou même avec l'implémentation de base de SAGE 
(retrouver une simple racine). Détailler les limitations dû à SAGE, peut-être
prendre une partie de la conclusion en expliquant ce qui pourrait être changer.

\section{"Conclusion"}
Expliquer ce qu'il peut rester à faire (cas général, car(K) = 2 et n 
composé, couper et appliquer l'une des deux méthodes selon la nature du m ou 
s'il y a besoin de calculer une extension, etc.), où est-ce qu'on peut 
récupérer des performances (produit scalaire).

\begin{thebibliography}{LC}
\bibitem{Sam} \emph{Théorie algébrique des nombres}, \bsc{Pierre Samuel}, Hermann, 1971.
\bibitem{Nek} \emph{Théorie de Galois}, \bsc{Jan Nekov\'a\v{r}}, Université Pierre et Marie Curie, 2003, \bsc{url :} \url{http://www.math.jussieu.fr/~nekovar/co/ln/gal/g.pdf}.
\bibitem{Per} \emph{Cours d'algèbre}, \bsc{Daniel Perrin}, ellipses, 1996.
\bibitem{Rai} \emph{Efficient computation of isomorphism between finite fields}, \bsc{Eric M. Rains}, 2008.
\bibitem{LiNi} \emph{Finite fields}, \bsc{Rudolf Lidl} \& \bsc{Harald Niederreiter}, Encyclopedia of mathematics and its applications vol. 20, Cambridge, 1983.
\bibitem{Pin} \emph{Recognising elements of finite fields}, \bsc{Richard G.E. Pinch}, Cryptography and coding II, p. 193-197, Oxford University Press, 1992.
\bibitem{Pol} \emph{Algèbre et théorie de Galois}, \bsc{Patrick Polo}, Université Pierre et Marie Curie, 2007, \bsc{url :} \url{http://www.math.jussieu.fr/~polo/M1/ATG07chIV.pdf}.
\bibitem{Law} \emph{Introduction to cyclotomic fields}, \bsc{Lawrence C. Washington}, Graduate texts in mathematics, Springer-Verlag, 1982.
\bibitem{Sil} \emph{The arithmetic of elliptic curves}, \bsc{Joseph H. Silverman}, Graduate texts in mathematics, Springer, 2nd ed. 2009.
\bibitem{GarD} \emph{Handbook of finite fields}, \bsc{Gary L. Mullen} \& \bsc{Daniel Panario}, Discrete mathematics and its applications, Series Editor Kenneth H. Rosen, CRC Press.
\end{thebibliography}
\end{document}

\documentclass[a4paper]{article} % papier A4
\usepackage[utf8]{inputenc}      % accents dans le source
\usepackage[T1]{fontenc}         % accents dans le pdf
\usepackage{textcomp}            % symboles complémentaires (euro)
\usepackage[frenchb]{babel}      % titres en français
\usepackage{amsmath}
\usepackage{amsthm}
\usepackage{amssymb}
\usepackage[colorlinks=false]{hyperref}
\usepackage{enumerate}
\usepackage{algorithmic}
\usepackage{pgf}
\usepackage{tikz}
\usepackage{tikz-cd}
\usetikzlibrary{matrix,arrows,decorations.pathmorphing}
\numberwithin{equation}{section}
\newcommand\nroot[1]{\textit{#1}\up{\textit{ième}}}
\newcommand\zmodn[1]{\mathbb{Z}/#1\mathbb{Z}}
\newcommand\zmodninv[1]{(\mathbb{Z}/#1\mathbb{Z})^{\times}}
\newcommand\GF[1]{\mathbb{F}_{#1}}
\newcommand\Irr[2]{\textup{Irr}_{#1}(#2)}
\renewcommand{\algorithmicrequire}{\textbf{Input:}}
\renewcommand{\algorithmicensure}{\textbf{Ouput:}}
\newcommand\Tr[1]{\textup{Tr}\left(#1\right)}
\begin{document}
\newtheorem{thm}{Thèorème}[section]
\newtheorem{lem}[thm]{Lemme}
\newtheorem{cor}{Corollaire}[thm]
\newtheorem{prop}[thm]{Proposition}
\theoremstyle{definition}
\newtheorem{defn}[thm]{Définition}
\newtheorem*{ex}{Exemple}
\theoremstyle{remark}
\newtheorem{rem}{Remarque}[thm]
\section*{Remerciements}
\section*{Introduction}
Soit $\GF{q}$ le corps fini à $q = p^r$ éléments, avec $p$ un nombre premier. On
définit alors $\overline{\GF{q}}$ sa clôture algébrique. Lorsqu'on a besoin de
faire des calculs dans cette clôture, il est peut être plus aisé de calculer
directement sur un corps finis en particulier. Il est donc important de pouvoir
se déplacer rapidement entre les différents sous-corps de cette clôture.\par
Il est connu et prouvé que de deux corps finis de même cardinal sont reliés par
un isomorphisme. Cependant la preuve de ce résultat n'est pas constructive,
trouver un tel isomorphisme demande alors un travail supplémentaire.\par
La situation est la suivante, nous avons deux extensions de $\GF{q}$ de même
degré et définies par deux polynôme irréductible distincts $f$ et $g$. On notera
:
\[k_1=\GF{q}[X]/(f)\textup{ et }k_2=\GF{q}[Y]/(g)\]
le but est donc de trouver un isormophisme reliant ces deux corps. Une méthode
simple et immédiate est d'envoyer $x = \bar{X}$ sur une racine de $f$ dans
$k_2$. Le problème est que cela revient à factoriser le polynôme $f$ ce qui est
beaucoup trop lent.\par
On va donc dans ce rapport étudier deux méthodes permettant, \textit{via}
l'utilisation de racines de l'unité et de courbes elliptique, de calculer
rapidement de tels isomorphismes.


\section{Préliminaire/contexte théorique}

\subsection{Corps finis}
Dans cette section on va rappeler les définitions et démontrer certains 
résultats liés aux corps finis. Principalement, on étudiera les extensions 
de corps et la théorie de Galois de base sur les corps finis. On ne considérera
que des corps commutatifs.

\subsubsection{Définitions}

Soit $\Omega = (K, +, .)$ un triplet où $K$ est un ensemble et $+$ et $.$ deux
lois de composition interne sur $K$. On dit que $\Omega$ est un corps si : 
\begin{enumerate}[(i)]
\item $(K, +)$ est un groupe abélien,
\item $(K\setminus\lbrace0\rbrace,.)$ est un groupe (abélien ou non, dans la 
littérature francophone),
\item la première loi est distributive par rapport à la seconde,
\item $K$ n'a pas de diviseurs de $0$.
\end{enumerate}
Par abus de notation, on désignera désormais le corps $\Omega$ par son ensemble
$K$. On notera $0$ et $1$ les éléments neutres de la première et seconde loi 
respectivement; on omettra régulièrement de noter la seconde loi dans les 
opérations sur $K$.\par
On appelle la caractéristique d'un corps $K$, le plus petit entier $p$ tel que 
pour tout $x\in K$ on ait :
\[p(x) = \underbrace{x + x + \dots + x}_{p fois} = 0\]
Elle peut être nulle ou positive, en pratique elle sera nulle ou égale à un
nombre premier. Dans notre cas, on s'intéressera au corps finis, c'est-à-dire 
tels que $K$ soit un ensemble fini et la caractéristique est non-nulle.\par
\vspace{0.3cm}
Un homomorphisme de corps est nécessairement injectif. En effet, soit deux corps
$K$ et $L$ de caractéristique positive ou nulle. Soit $f : K \to L$, on a alors 
:
\[f(x) = 0 \Leftrightarrow x.f(1) = 0 \Leftrightarrow x = 0\]
puisque un homomorphisme doit conserver la structure et que $K$ n'a pas de
diviseurs de zéro, $x = 0$ et $f(1) = 1$. S'il existe un
tel homomorphisme entre deux corps, on a $K\subseteq L$ et on dit que $L$ est un
surcorps de $L$ ou $K$ est un sous-corps de $L$.\par
On appelle corps premier un corps qui n'a aucun sous-corps. Dans le cas des
corps finis, il s'agit du corps $\GF{p}$ à $p$ éléments, pour $p$ premier. Ce
qui nous donne le théorème suivant :

\begin{thm}
\label{corfincarprem}
Soit $K$ un corps fini, alors $K$ a $p^n$ éléments, où $p$ premier est la
caractéristique du corps premier et $n$ le degré de $K/\GF{p}$.
\end{thm}
\begin{proof}
De façon informelle, on appelle le degré de $L$ sur $K$ la dimension de $L$
en tant que $K$-espace vectoriel. On discutera plus en détails de cette
notion dans le point suivant \ref{defdegext}.\par
Montrons le lemme suivant :
\begin{lem}
Un corps fini a pour caractéristique un nombre premier.
\end{lem}
\begin{proof}
Comme $K$ est intègre en tant qu'anneau, sa caractéristique est plus grande ou
égale à $2$. Supposons que la caractéristique de $K$ soit $p = km$ pour $k,
m\in\mathbb{Z}$. Alors pour $e\in K$ on a e :
\[0 = ne = (km)e = (ke)(me)\]
puisque $K$ n'a pas de diviseurs de $0$, alors $ke = 0$ ou $me = 0$, ce
qui contredit la minimalité de la caractéristique.\\
\end{proof}
Donc $K$ est de caractéristique $p$ premier et son corps premier est (isomorphe
à) $\GF{p}$. On fini avec le lemme qui suit :
\begin{lem}
Soit $K$ un corps fini contenant un corps fini $L$ à q éléments. Alors $K$ a
$q^m$ élément où $m$ est le degré de $K$ en tant que $L$-espace vectoriel.
\end{lem}
Si $K$ est un $L$-espace vectoriel de dimension fini, il admet alors une base
d'éléments de $L$. Si on écrit chaque éléments de $K$ comme une combinaison
linéaire des éléments de cette base, chaque coeffecient de cette combinaison a
$q$ valeurs possibles, d'où $K$ à $q^m$ éléments; ce qui achève la démonstration
du lemme. Il suffit alors de remplacer $L$ par $\GF{p}$ et $q$ par $p$ et le
théorème est démontré.\\
\end{proof}

\begin{thm}[Wedderburn]
Tout corps fini est commutatif.
\end{thm}
On passera la preuve bien qu'elle soit intéressante, se référer par exemple à 
\cite[p.~70-73]{LiNi} ou \cite[p.~82]{Per}.\par
\vspace{0.3cm}
Si $K$ est un corps à $q$ éléments, alors pour tout $x\in K$, on a $x^q = x$. En
effet, pour $x = 0$ c'est immédiat. Pour $x\neq 0$, on sait que le groupe
multiplicatif de $K$, noté $K^{\times} = K\setminus\lbrace0\rbrace$, est un
groupe fini à $q-1$ éléments par définition; donc $x^{q-1} = 1$ pour tout $x\in
K^{\times}$ et il suffit de multiplier encore une fois par $x$ pour obtenir le
résultat voulu.\par
On en déduit immédiatement que le polynôme $X^q - X$ se scinde dans $K$
puisque étant de degré $q$, il a au maximum $q$ racines qui sont déjà les
éléments de $K$. On peut dire que $K$ est un corps de décomposition
\ref{defdec} de $X^q - X$ sur $\GF{p}$.\par
On notera que dans les corps finis et leurs extensions, l'élévation à une
puissance égal au cardinal d'un corps fini est un automorphisme particulier. On
l'appelle l'automorphisme de Frobenius et on le note $Fr_q$ pour $Fr_q(x) =
x^q$. Ce n'est pas immédiat qu'il s'agit là d'un homomorphisme, mais un simple
calcul permet de s'en assurer.\par
\vspace{0.3cm}
Finissons ce point par le théorème suivant qui assure l'existence et l'unicité
à isomorphisme près des corps finis; c'est à partir de ce théorème que le
travail de tout ce rapport va se faire :

\begin{thm}
\label{thisomGF}
Pour tout nombre premier $p$ et tout entier strictement positif $n$, il existe 
un corps fini à $p^n$ éléments. Tout corps fini à $q = p^n$ éléments est 
isomorphe au corps de décomposition de $X^q - X$ sur $\GF{p}$. On parlera alors 
du corps fini à $q$ éléments et on le notera $\GF{q}$.
\end{thm}
\begin{proof}
\textit{(Existence)} Pour $q = p^n$, on considère le polynôme $X^q - X$ dans 
$\GF{p}[X]$ et on note $K$ son corps de décomposition sur $\GF{p}$. Le polynôme 
est séparable ou n'a aucune racine multiple puisque sa dérivée est égale à 
$qX^{q-1} - 1 = -1$. Posons $S = \lbrace x\in K : x^q - x = 0\rbrace$, alors $S$
est un sous-corps de $K$; $0$ et $1$ sont dans $K$ et d'après les propriétés sur
le Frobenius et les résultats obtenus plus haut on a :
\[(a - b)^q = a^q - b^q = a - b \textup{ et }(ab^{-1})^q = a^qb^{-q} = ab^{-1}\]
Ainsi, $S$ contient toutes les racines de $X^q - X$ mais comme $K$ a déjà $q$
élémens, alors $K = S$ est un corps à $q$ éléments.\par
\textit{(Unicité)} Soit $K$ le corps à $q = p^n$ éléments, d'après le théorème
\ref{corfincarprem}, est de caractéristique $p$ et contient $\GF{p}$. On en
déduit que $K$ est un corps de décomposition de $X^q - X$ sur $\GF{p}$,
puisqu'il est scindé sur $K$, et l'unicité se déduit de l'unicité des corps de
décomposition, théorème \ref{cordec}.
\end{proof}

\subsubsection{Extension de corps}
\label{defdegext}
On dit que $K$ est une extension de corps de $k$ s'il existe un morphisme de 
corps $\varphi : k \to K$. Ou de façon équivalente, si $k \subseteq K$ alors $K$
est une extension (de corps) de $k$; on note aussi $K/k$ une extension de corps.
Si on a $k\subseteq L \subseteq K$, alors $L/k$ est une sous-extension de
$K/k$.\par
Soit $K/k$ une extension de corps et $S$ une partie de $K$. Le sous-corps $L := 
k(S)$ de $K$ engendré par $S$ sur $k$ est le plus petit sous-corps de $K$ 
contenant $S$ et $k$. Si $S = \lbrace x_1,\dots,x_n \rbrace$ est fini, alors on 
note $L = K(x_1,\dots,x_n)$, on dit alors que l'extension est de type fini. 
L'extension $L/k$ est dite monogène ou simple si elle est engendré par un seul 
élément.\par
Si $K/k$ est une extension on peut voir $K$ comme un $k$-espace vectoriel ou une
$k$-algèbre. On appelle $[K:k] := dim_k(K)$ le degré de l'extension. On dit 
qu'une extension $K/k$ est de degré fini si $[K:k] < \infty$.\par

\begin{thm}
Soit $k \subseteq L \subseteq L$ des extensions de corps de degré fini. Alors on
a :
\[[K:k] = [K:L][L:k]\]
\end{thm}
\begin{proof}
Soit $[K:L] = m$ et $[L:k] = n$. On a donc que $K$ est un $L$-espace vectoriel 
de dimension $m$ et $L$ est un $k$-espace vectoriel de dimension $n$, le 
théorème revient à montrer que $K$ est un $k$-espace vectoriel de dimension 
$mn$. Or, d'après ce qui précède, on a $L \simeq k^n$ et $K \simeq L^m$, d'où
\[K \simeq \underbrace{L \oplus\dots\oplus L}_{m fois}\simeq\underbrace
{k^n\oplus\dots\oplus k^n}_{m fois} \simeq k^{nm}\]
ce qui prouve le théorème.\\
\end{proof}

On dit qu'un élément $x\in K$ est algèbrique sur $k$ s'il existe un polynôme 
unitaire à coefficient dans $k$ qui annule $x$. L'ensemble des éléments 
algébrique d'un corps (sur un sous-corps) forme un corps \cite[p.~64, théorème
1.14]{Per}. On dit qu'une extension $K/k$ est algébrique si tous les éléments 
de $K$ sont algébriques sur $k$.

\begin{prop}
Toute extension de degré fini est algébrique et de type fini.
\end{prop}
\begin{proof}
Si l'extension $K/k$ est de degré fini alors elle admet une base finie 
$(\alpha_1,\dots,\alpha_n)$ en tant que $k$-espace vectoriel, on a alors $K 
= k(\alpha_1,\dots,\alpha_n)$. Comme $[K:k] = n < \infty$, si $\alpha\in K$ 
alors $1, \alpha, \dots, \alpha^n$ satisfont une relation de dépendance 
linéaire, \textit{i.e.} il existe $a_0, \dots, a_n$ dans $k$ tels que :
\[a_0 + a_1.\alpha + \dots + a_n.\alpha^n = 0\]
puisque $K$ est un $k$-espace vectoriel. D'où tout $\alpha$ est algébrique 
sur $k$.\\
\end{proof}

\begin{prop}
Soit $\alpha\in L$ un élément algébrique sur $k$ et $f = X^d + a_1X^{d-1} + 
\dots + a_d$ son polynôme minimal sur $k$. Si on pose $d := \textup{deg }f$ 
alors on a :
\begin{enumerate}[(i)]
\item Les éléments $1, \alpha,\dots,\alpha^{d-1}$ forment une base de 
$k[\alpha]$ en tant qu'espace vectoriel.
\item $k[\alpha]$ est un corps, on a alors $k[\alpha] = k(\alpha)$.
\item On a $[k(\alpha):k] = d$.
\end{enumerate}
\end{prop}

\begin{proof}
(i) Il suffit de multiplier l'identité $\alpha^d = -a_1\alpha^{d-1} - \dots 
- a_d$ par $\alpha^i$ pour $i\geq0$ et on montre par récurrence que :
\[\alpha^{d+i} \in k\cdot1 + \dots + k\cdot\alpha^{d-1}\]
d'où $k[\alpha] = k\cdot1 + \dots + k\cdot\alpha^{d-1}$. De plus, les éléments 
$1,\alpha,\dots,\alpha^{d-1}$ sont linéairement indépendants puisque si :
\[u_0\cdot1 + u_1\cdot\alpha + \dots + u_{d-1}\cdot\alpha^{d-1} = 0\]
avec les $u_i\neq0$ cela contredirait la minimalité du degré de $f$.\par
(ii) $k[\alpha] \subset L$ est le sous-anneau d'un corps, il est donc intègre. 
De plus, pour tout $\beta\in k[\alpha]$ l'application $m_{\beta}$ de la 
multiplication par $\beta$ dans $k[\alpha]$ est $k$-linéaire et injective, 
puisque l'anneau est intègre. Pour finir, elle est aussi surjective puisque 
nous sommes en dimension finie, donc il existe $x\in k[\alpha]$ tel que 
$m_{\beta}(x) = \beta x = 1$.\par
(iii) Il résulte directement des deux points précédents.\\
\end{proof}

On vient de décrire plus ou moins la construction d'extension de corps 
\textit{via} des polynômes irréductibles ou par "adjonction de racines". 
Concrètement, soit $k$ un corps et $P$ un polynôme irréductible unitaire 
dans $k[X]$. Comme $k[X]$ est principal (il est même euclidien) alors $(P)$ est 
un idéal maximal, on peut alors faire le quotient $k[X]$ par $(P)$ est on 
obtient un corps $K \simeq k[X]/(P)$. Dans ce cas, la classe de $X$ qu'on notera
$x := \overline{X}$, est une racine du polynôme $P$ et engendre $K$ sur $k$.\par
Pour encore aller plus loin, considérons une extension $L/k$ telle que $P$ 
admette une racine $\alpha\in L$. Si on note $\Irr{k}{\alpha}$ le polynôme 
minimal de $\alpha$, alors $\Irr{k}{\alpha}$ divise $P$, donc est égal à 
$\lambda P$ pour $\lambda\in k^{\times}$, puisque $P$ est irréductible. Alors 
le morphisme $k$-algèbre $\phi : k[X] \to L$ défini par $\phi(X) = \alpha$ 
induit un morphisme de $k$-algèbre $\varphi : K \to L$ tel que $\varphi(x) 
= \alpha$. Ce morphisme est unique puisque $x$ engendre $K = k(x)$.\par
\vspace{0.3cm}
On a montré en particulier le théorème suivant :

\begin{thm}
Soit $k$ un corps et $P$ un polynôme irréductible dans $k[X]$. Alors $K := 
k[X]/(P)$ est un sur-corps de $k$ dans lequel $P$ a au moins une racine, la 
classe de $\overline{X} = x$. On l'appelle le corps de rupture de $P$ sur 
$k$.\par
\end{thm}

Une autre notion importante est celle de corps de décomposition d'un polynôme 
non constant sur un corps $k$.

\begin{defn}
\label{defdec}
Le corps de décomposition d'un polynôme $P\in k[X]$ est le sur-corps $K \supset 
k$ contenant toutes les racines de $P$, qui est engendré par les racines 
susnommées.
\end{defn}

\begin{thm}
\label{cordec}
Tout $P\in k[X]$ non constant admet un corps de décomposition unique à 
isomorphisme près.
\end{thm}
\begin{proof}
\textit{(Existence)}
Pour prouver l'existence du corps de décomposition, il suffit de se placer 
dans la clôture algèbrique de $k$, alors si on note $K_0$ le corps engendré 
par $\alpha_1,\dots,\alpha_n$ les racines de $P$, il est clair que $K_0$ est 
le corps de décomposition de $P$.\par
\textit{(Unicité)}
On veut réalité montrer que si on a deux corps de décompositions, alors ils ont
isomorphes. Soit deux corps $k$ et $k^{\prime}$ isomorphes, on note $\phi$
l'isomorphise, qu'on étend alors de façon évidente à un isomorphismes de $k[X]$ 
dans $k^{\prime}[X]$; \textit{i.e.} on applique $\phi$ au coefficients des 
polynômes. On pose  $L$ et $L'$ les corps de décomposition de $P\in k[X]$ et $P'
= \phi(P) \in k^{\prime}[X]$ respectivement.\par
Raisonnons par récurrence sur $[L:k]$, si $L = k$ alors $L^{\prime} =
k^{\prime}$. Sinon prenons $\alpha\in L$ une racine de $P$ qui n'est pas dans
$k$ et notons $Q$ son polynôme minimal sur $k$. On a alors $\alpha'$ racine de 
$P'$ et $Q'$ son polynôme minimal sur $k'$. Posons $M = k(\alpha)$ et $M' =
k'(\alpha)$, ainsi $M$ (resp. $M'$) est un corps de décomposition de $Q$ (resp.
$Q'$) sur $k$ (resp. $k'$). Puisque les corps de rupture sont uniques, modulo
l'isomorphisme qui envoie une racine sur l'autre, il existe donc un isomorphisme
$\psi : M \to M'$ tel que $\psi(\alpha) = \alpha'$. On a la situation suivante :

\begin{align*}
&P = (X - \alpha)S(X)\in M[X]\\
&P' = (X - \alpha')S'(X)\in M'[X]
\end{align*}

Comme $\alpha' = \psi(\alpha)$ alors $S'=\psi{\alpha}$, ainsi $L$ est un corps
de décomposition pour $S$ et $L'$ pour $S'$; d'où par l'hypothèse de récurrence
sur les degrés, on a un isomorphisme de $L \to L'$ prolongeant $\psi$ et on a 
démontré ce qu'il fallait.\\
\end{proof}

Penchons-nous alors plus en détails sur les corps finis. On peut se demander
étant donné un corps finis $\GF{p^n}$ quels peuvent être les corps qui le
contiennent. Dans un premier temps, il faut que les deux corps aient la même
caractéristique. Supposons qu'il existe un isomorphisme $f : \GF{p^n} \to 
\GF{l^m}$ pour $p$ et $l$ premiers. En particulier, on a pour $x\in\GF{p^n}$ :
\[f(p.x) = 0 = p.f(x)\]
pour $f(x)\in\GF{l^m}$, donc nécessairement $l = p$.\par
Ainsi, si le morphisme existe alors $\GF{p^n}^{\times}$ est un sous-groupe
d'ordre $p^n - 1$ de $\GF{p^m}^{\times}$ d'ordre $p^m - 1$. Donc, on a $p^n - 1$
qui divise $p^m - 1$. Si on écrit $m = q.n + r$ avec $r < n$ alors on obtient :
\[p^m - 1 = p^r((p^n)^q - 1) + p^r - 1\]
Or $p^n - 1$ divise $(p^n)^q - 1$ donc comme $p^n - 1$ divise aussi $p^m - 1$ la
seule possibilité pour $r$ est $0$ et donc $n|m$. On vient de prouver la
nécessité du critère suivant :

\begin{prop}
Soit $\GF{q}$ un corps fini à $q = p^m$ éléments, alors tout ces sous-corps sont
de la forme $\GF{p^n}$ avec $n|m$. Réciproquement, soit $n|m$ alors il existe un
(et un seul) sous-corps de $\GF{q}$ de cardinal $p^n$.
\end{prop}
\begin{proof}
Pour montrer l'autre sens, il suffit de remarquer que $\GF{p^n}$ correspond aux
éléments satisfaisant l'équation $X^{p^n} - X$ puisque $p^n - 1$ divise $p^m -
1$. L'unicité est à isomorphisme près comme énoncé dans le point précédent.\\
\end{proof}
%TODO : Théorème de l'élément primitif et... ?

\begin{thm}
\label{thelemprim}
Toute extension séparable et finie est simple.
\end{thm}
\begin{proof}
On va se contenter de le prouver dans le cas d'un corps fini; le cas général
demande un peu plus de travail et pourra être démontrer après le point
suivant.\par
Si $K$ est un corps fini alors son groupe mulitplicatif $K^{\times}$ est
cyclique (voir par exemple la preuve de la proposition \ref{proprootcycl} ou
\cite[p.~50]{LiNi}). Dans ce cas, il suffit de prendre pour générateur d'une
extension $L/k$, le générateur du groupe mulitplicatif de $L$ et c'est gagné.\\
\end{proof}

Cela implique en particulier que toutes extensions de corps finis peut s'écrire
comme combinaison linéaire de puissances d'un unique élément, en pratique cet
élément sera la racine du polynôme par lequel on quotiente le corps premier afin
d'obtenir l'extension.

\subsubsection{Théorie de Galois}
%TODO :Définitions et résutlats de base (sans forcément tout prouver),
%Frobenius, action de groupes ? Ça me parait raisonnable...
%Groupe de Galois, élément primitif, extensions galoisiennes, etc.

Soit $L/K$ 
\subsubsection{Extensions et polynômes cyclotomiques}
%TODO : Racines de l'unités et quelques mots sur les polynômes 
%cyclotomiques et leurs extensions. Peut-être rajouter un ou deux mots sur le 
%cas des corps finis.
On définit une racine \nroot{n} de l'unité dans un corps $K$ 
comme un élément $u\in K$ tel que pour $n\in\mathbb{N}$ on ait $u^n = 1$. Ce 
sont exactement les racines du polynômes $X^n - 1$.
On note $U_n$ l'ensemble des racines \nroot{n} de l'unité dans une clôture 
algébrique de $K$ et $U_n(K)$ l'ensemble des racines \nroot{n} de l'unité 
appartenant à $K$.

\begin{prop}
\label{proprootcycl}
Soit $K$ un corps, s'il contient $n$ racines \nroot{n} de l'unité alors elles 
forment un groupe cyclique $U_n(K)$ d'ordre $n$ isomorphe à $\zmodn{n}$.
\end{prop}
\begin{proof}
La proposition découle d'un théorème plus général disant que tout sous-groupe 
fini $G$ du groupe multiplicatif $K^*$ est formé de racine de l'unité et est 
cyclique. Lui-même est tiré de deux corollaires du théorème suivant (voir 
\cite[p.~26-27]{Sam} pour une démonstration détaillée) :

\begin{thm}
Soit $A$ un anneau principal, $M$ un $A$-module libre de type fini et $M'$ un 
sous-module de $M$. Alors :
\begin{enumerate}[1)]
\item $M'$ est un module libre de rang $\leq n$.
\item Il existe une base $(e_1,\dots,e_n)$ de $M$, un entier $q\leq n$ et des 
éléments $a_1,\dots,a_q$ non nuls de $A$ tels que $(a_1e_1,\dots,a_qe_q)$ soit 
une base $M'$ avec $a_i|a_{i+1}$ pour tout $1\leq i\leq q-1$.
\end{enumerate}
\end{thm}

\begin{cor}
Soit $E$ un $A$-module de type fini. Alors $E$ est isomorphe à un produit 
$(A/\mathfrak{a}_1)\times\dots\times(A/\mathfrak{a}_n)$, où les $\mathfrak{a}_i$
sont des idéaux de $A$ tels que $\mathfrak{a}_1\supset\mathfrak{a}_2\supset
\dots\supset\mathfrak{a}_n$.
\end{cor}

\begin{cor}
Soit $G$ un groupe commutatif fini. Il existe $x\in G$ dont l'ordre est le 
$ppcm$ des ordres des éléments de $G$.
\end{cor}
\begin{proof}
On sait qu'un groupe commutatif est un $\mathbb{Z}$-module (si on le note 
additivement). D'après le corollaire ci-dessus, $G$ est donc isomorphique à 
un produit $\zmodn{a_1}\times\dots\times\zmodn{a_n}$ où $a_1|a_2|\dots|a_n$. 
Aucun des $a_i$ n'est nul, sinon $G$ serait infini; ce serait le produit de 
$\mathbb{Z}^r$ avec des groupes finis. On note $y$ la classe de $1$ dans 
$\zmodn{a_n}$ et on pose $x = (0,\dots,0,y)$. Alors $x$ est d'ordre $a_n$ et si 
on prend un élément $z\in G$, avec $z = (z_1,\dots,z_n)$, on a aussi $a_nz = 0$ 
car $a_i|a_n$ pour tout $1\leq i \leq n$. $x$ est donc l'élément recherché.\\
\end{proof}

De là, on en déduit qu'il existe un $z\in G$ d'ordre $n$ tel que $y^n = 1$ pour 
tout $y\in G$. Comme le nombre de racines de $X^n - 1$ sur un corps est au plus 
$n$ alors $G$ a au plus $n$ éléments. Or $z$ est d'ordre $n$ donc $G$ contient 
les éléments $z, z^2,\dots,z^n = 1$ qui sont distincts. Donc $G$ est formé par 
ces éléments et est cyclique d'ordre $n$. La proposition en découle 
naturellement.\\
\end{proof}

\begin{defn}
On appelle racine primitive \nroot{n} de l'unité les racines qui engendrent 
$U_n$.
\end{defn}

\begin{prop}
Les racines primitives \nroot{n} de l'unité forment un sous-groupe $U^{\times}_n
$ 
isomorphe à $(\zmodn{n})^*$. En particulier, il y en a $\varphi(n)$.
\end{prop}
\begin{proof}
D'après \ref{proprootcycl} les éléments qui engendrent $U_n$ sont exactement 
les éléments dont l'image engendre $\zmodn{n}$. D'où l'isomorphisme et l'ordre 
de $U^{\times}_n$.\\
\end{proof}

Lorsqu'on les considère sur $\mathbb{C}$, les racines \nroot{n} de l'unité sont 
de la forme $e^{2ik\pi/n}$ et les racines primitives définissent un polynôme 
qu'on appelle polynôme cyclotomique défini comme suit : 
\[\phi_n(X) = \prod_{\zeta\in U_n^{\times}}{(X - \zeta)}\]
On va montrer qu'il s'agit du polynôme minimal de ces racines sur $\mathbb{Q}$, 
commençons par la proposition suivante :

\begin{prop}
Le polynôme $\phi_n$ appartient à $\mathbb{Z}[X]$ et est de degré égal à 
$\varphi(n)$.
\end{prop}
\begin{proof}
Nous allons utiliser les deux résultats suivants; on ira voir 
\cite[p.~72, 80]{Per} pour des démonstrations détaillées.

\begin{lem}
Soit $n\in\mathbb{N}^*$ alors on a $n = \sum_{d|n}{\varphi(d)}$.
\end{lem}

\begin{prop}
On a $X^n - 1 = \prod_{d|n}{\phi_d(X)}$.
\end{prop}

On va alors raisonner par récurrence sur $n$. On a $\phi_1(X) = X-1 
\in\mathbb{Z}[X]$. Supposons alors que le résultat soit vrai pour $d<n$. Posons 
$F(X) = \prod_{d|n\atop{d\neq n}}{\phi_d(X)}$, alors $F\in\mathbb{Z}[X]$ et est 
unitaire. Si on fait la division euclidienne de $X^n - 1$ par $F$ dans 
$\mathbb{Z}[X]$, alors on obtient :
\[X^n - 1 = F(X)P(X) + R(X)\]
avec $P, R \in\mathbb{Z}[X]$ et $\textup{deg~}R < \textup{deg~}F$. Or, on a déjà
$X^n - 1 = \phi_n(X).F(X)$ dans $\mathbb{Q}[X]$, \textit{i.e.} $F(X)(\phi_n(X) -
P(X)) = R(X)$, mais le degré de $R$ étant plus petit que celui de $F$, on a 
nécessairement $\phi_n = P \in\mathbb{Z}[X]$. Le résultat sur le degré résulte 
de la proposition et du lemme ci-dessus.\\
\end{proof}
\begin{thm}
Le polynôme $\phi_n$ est irréductible sur $\mathbb{Q}$.
\end{thm}
\begin{proof}
Soit $K$ un corps de décomposition de $\phi_n$ sur $\mathbb{Q}$, $\zeta$ une 
racine primitive \nroot{n} de l'unité et $p$ un nombre premier ne divisant pas 
$n$. Les racines primitives de l'unité sont toutes de la forme $\zeta^m$ avec 
$(m, n) = 1$ puisqu'elles doivent être d'ordre exactement égal à $n$; cela ce 
soit sur leur expression exponentielle. On veut alors montrer que $\zeta$ et 
$\zeta^p$ ont exactement le même polynôme minimal.\par
Posons alors $f$ et $g$ les polynômes minimaux sur $\mathbb{Q}$ de $\zeta$ et 
$\zeta^p$ respectivement. Ils sont dans $\mathbb{Z}[X]$ puisqu'ils divisent tous
les deux $\phi_n(X)$. En effet, comme $\mathbb{Z}[X]$ est factoriel, on a :
\[\phi_n = f_1^{a_1}\dots f_r^{a_r}\]
avec $f_i$ unitaire, irreductible et dans $\mathbb{Z}[X]$; puisque $\phi_n$ est
unitaire et dans $\mathbb{Z}[X]$. Alors un $f_i$ annule $\zeta$ et un $f_j$ 
annule $\zeta^p$; or ils sont, d'après Gauss, irréductibles et unitaires dans 
$\mathbb{Q}[X]$, il s'agit donc de $f$ et $g$. En particulier, $f$ et $g$ 
divisent $\phi_n$ dans $\mathbb{Z}[X]$.\par
Supposons alors que $f \neq g$. Comme ils sont irréductibles et disctincts, leur
produit $f.g$ divise $\phi_n$. Comme $g(\zeta^p) = 0$, alors $\zeta$ est racine 
de $g(X^p)$, ainsi $f$ divise $g(X^p)$ dans $\mathbb{Q}[X]$ mais aussi dans 
$\mathbb{Z}[X]$ d'après le lemme de Gauss sur les contenus de polynômes. On va 
alors considérer les polynômes dans $\GF{p}[X]$, on note $\bar{f}$ et $\bar{g}$ 
leurs réductions modulo $p$. Par Frobenius, on obtient que $\bar{g}(X^p) = 
\bar{g}(X)^p$, ainsi si on prend $r$ un facteur irréductible de $\bar{f}$ et on 
écrit :
\[\bar{g}(X)^p = \bar{f}(X)\bar{h}(X)\]
alors $r$ divise aussi $\bar{g}$ par le lemme d'Euclide. Puique $f.g$ divise 
$\phi_n$ alors $\bar{f}.\bar{g}$ divise $\bar{\phi_n}$, on a $r^2$ qui divise 
$\bar{\phi_n}$. Ceci implique en particulier que $\bar{\phi_n}$ aurait une 
racine double, donc $\bar{X}^n - 1$ aussi. Or, $\bar{X}^n - 1$ est séparable sur
$\GF{p}$ puisque $(X^n - 1)' = nX^{n-1}$ et $p$ ne divise pas $n$, il ne peut 
donc pas avoir de racines multiples, d'où une contradiction.\par
Reste alors à montrer que tous les conjugués de $\zeta$ sont exactement les 
autres racines primitives \nroot{n} de l'unité. Si on prend une racine primitive
\nroot{n} $\zeta'$ quelconque, alors on a $\zeta' = \zeta^m$ où $m = p_1^{a_1}
\dots p_l^{a_l}$ avec $p_i \nmid n$. Il suffit alors de faire de procéder par 
récurrence en utilisant le raisonnement ci-dessus pour s'appercevoir que $\zeta$
et $\zeta'$ ont effectivement le même polynôme minimal. De sorte que $f$ admet 
toutes les racines primitives comme zéros, \textit{i.e.} $\textup{deg~}f \geq 
\textup{deg~}\phi_n$, et $f\mid\phi_n$ alors on a bien $f = \phi_n$ irréductible
sur $\mathbb{Q}$ comme on voulait.\\
\end{proof}

\begin{rem}
Comme $\phi_n$ est unitaire alors son contenu est égal à $1$ ce qui implique, 
avec ce qui précède, qu'il est alors aussi irréductible dans $\mathbb{Z}[X]$.
\end{rem}
\vspace{0.3cm}
Grâce à ce résultat on peut alors introduire une notion qui nous sera utile dans
la suite, celle de corps cyclotomique.

\begin{defn}
Soit $K$ un corps commutatif, on appelle corps cyclotomique $K^{(n)}$ sur $K$ le
corps de décomposition de $X^n - 1$ sur $K$.
\end{defn}

\begin{rem}
D'après les résultats ci-dessus, on voit que dans le cas $K = \mathbb{Q}$, le 
corps de décomposition de $X^n - 1$ est exactement $\mathbb{Q}(\zeta_n) \simeq 
\mathbb{Q}[X]/\phi_n(X)$, où $\zeta_n$ est une racine primitive. C'est en 
particulier une extension algébrique simple.
\end{rem}

%TODO: Parler des racines de l'unité dans les corps finis avant de foutre le
%résultat.
Le théorème suivant sera cruciale dans la suite du mémoire, il servira notament 
à justifier théoriquement la méthode de Rains.

\begin{thm}
\label{polycycldecomp}
Si $K = \GF{q}$ et $(n,q) = 1$, alors $\phi_n$ se factorise en $\varphi(n)/d$ 
polynômes unitaires irréductibles dans $\GF{q}[X]$ de même degré égal à $d$. 
$K^{(n)}$ est le corps de décomposition de n'importe lequel de ces polynôme et 
on a :
\[[K^{(n)}:K] = d\]
avec $d$ l'ordre multiplicatif de $q$ dans $\zmodn{n}$.
\end{thm}
\begin{proof}
Soit $\zeta$ une racine primitive \nroot{n} de l'unité dans $\GF{q}$, alors 
$\zeta$ appartient a un sur-corps $\GF{q^k}$ si et seulement si $\zeta^{q^k} = 
\zeta$; ce qui est équivalent à $q^k \equiv 1 \bmod n$ puisque $\zeta^n = 1$ par
définition. On pose alors $d$ égal au plus petit $K$ satisfaisant cette 
condition, dans ce cas $\zeta\in\GF{q^d}$ et ne peut pas être dans un sous-corps
de celui-ci. Ainsi, le polynôme minimal de $\zeta$ est de degré $d$ et comme on 
a choisi $\zeta$ arbitrairement, on obtient le résultat voulu.\\
\end{proof}

\subsection{Théorie des nombres/Idéaux}
%TODO: Tout ce qui concerne la preuve de Rains avec les idéaux 
%premiers, les normes d'ideaux, la factorisation unique, Galois, les périodes de
%Gauss, les périodes elliptiques (Mihailescu et al.) ou plus précisément ce qui 
%permet d'y arriver.

\subsubsection{Corps de nombres}
%TODO :Définitions (normes, traces, éléments algébriques, etc.)

\subsubsection{Idéaux premiers}
%TODO: Idéaux premiers sur les anneaux d'entiers de corps de nombres,
%décomposition unique, résultat de la théorie de Galois.

\subsubsection{Périodes de Gauss, elliptiques}
%TODO: Énoncé et justifier les résultats donnés par Rains (notamment le lemme
%expliquant la forme de $\GF{q}[X]/I$ pour $I$ sous-groupe du groupe de Galois);
%peut-être à mettre directement dans la partie cyclotomique de la section
%algorithme de Rains cyclotomique.

\subsection{Courbes elliptiques}
%TODO: Définitions (j-invariant, nombres de points, tordues, module de
%Tate (?) etc.), résultats généraux, sur les corps finis

\subsubsection{Équation de Weierstrass}
%TODO: Rapide énoncé des changements de variable, comment on arrive à la forme
%"simple"; liste des différents paramètres/invariants qui nous intéressent.

\subsubsection{Loi de groupe}
%TODO : Je suis pas sûr pour cette partie, c'est un peu long à tout justifier et
%ce n'est pas ce qu'on veut faire. Mais peut-être expliquer rapidement.\par
%Cela dit, je suis très tenté de renvoyer au Silverman pour toute la théorie de
%base en ne mettant uniquement les résultats dont on aurait besoin 
%(donc on saute cette partie).

\subsubsection{Sur les corps finis}
%TODO: Nombres de points, théorème de Hasse, Frobenius, trace

\subsubsection{Courbes supersingulières}
%TODO: Définitions Twists quadratiques, courbes $y^2 = x^3 + x$ et 
%$y^2 = x^3 + 1$ et leurs tordues, pour préparer le terrain pour la partie 
%elliptique de Rains

\section{Isomorphismes de corps finis}
%TODO: Présentation du problème, énonciation des difficultés etc.
On a vu, théorème \ref{thisomGF}, qu'un corps fini était unique à isomorphisme
près. De façons équivalente, deux corps finis de même cardinal sont donc 
isomorphes. Malheureusement, la preuve n'est pas constructive et ne permet donc 
pas d'expliciter ou de construire un tel isomorphisme.\par
On a également vu, théorème \ref{thelemprim}, que tout corps fini, excépté le 
corps premier, s'obtient par adjonction de la racine d'un polynôme irréductible
du degré de l'extension. Il y a donc autant de corps fini "différents" de même
cardinal que de polynôme irréductible de bon degré sur le corps premier (ou 
$\GF{q}$ si on ne part pas du corps premier). Le problème est donc comment 
passer de l'un à l'autre.\par

\subsection{Algorithme de Pinch (peut-être à mettre dans l'introduction de 
la section ?)}
%TODO: Énonciation et descriptions de la méthode de Pinch, justification des 
%résultats, explication des inconvénients et autre.
Soit $f$ et $g$ deux polynômes irréductibles de degré $n$ sur $\GF{q}$ et 
$k_1 := \GF{q}[X]/(f)$ et $k_2 := \GF{q}[Y]/(g)$, on veut trouver un 
isomorphisme qui relie ces deux corps de cardinal $q^n$. Une première approche 
naïve est de chercher une racine de $f$ dans $k_2$ et d'envoyer $x = \bar{X}$
sur cette même racine. Cependant, cela implique de factoriser $f$ ce qui prend
beaucoup trop de temps; on veut quelque chose de plus rapide. Une autre façon de
s'en sortir est dû à Pinch, elle utilise les racines de l'unité.

\subsubsection{Principe}
%TODO: Explication de la méthode et illustration avec l'exempe de l'article, 
%peut-être ?
Pour qu'une racine \nroot{m} de l'unité soit dans un corps fini, il faut et il 
suffit que $m$ divise l'ordre du groupe multiplicatif. Si c'est le cas, il y a
parmis ces racines, des racines primitives ou des éléments exactement d'ordre
$m$. Si jamais il existe un morphisme entre deux corps finis, alors les points
d'ordre $m$ sont envoyés sur des points d'ordre $m$.

\subsubsection{Problèmes/limitations}
%TODO: Taille de m, choisir un élément au hasard

\subsection{Algorithme de Rains : méthode cyclotomique}
%TODO  Commencer par une explication concise de la méthode, ce 
%qu'elle change par rapport à Pinch.\par
%Présenter pas à pas l'article de Rains (en gros), en enlevant ce qui nous 
%concernerait moins et en rajoutant ce qu'il n'a pas mis (tous les détails 
%sur l'algorithme convert et sa complexité).

\subsubsection{Principe}
%TODO: Passer par une petite extension pour réduire la taille de m, utilisation
%des éléments normaux, utilisation des périodes de Gauss.

\subsubsection{Éléments normaux}
%TODO: Définitions et résultats utiles pour l'algorithme, justification de
%l'algorithme (peut-être à mettre dans la partie Galois de la section corps
%finis), conversion de la base polynomiale à la base normale.

\subsubsection{Algorithme \& analyse de complexité}
%TODO: Détailler l'algorithme de l'article/code de Luca + analyser la complexité
%de ces algos et les "prouver" (montrer que ça marche bien, cela dit c'est 
%peut-être largement faisable avant).

\subsection{Algorithme de Rains : méthode elliptique}
%TODO: Résumer/Détailler le papier de Luca, rajouter les résultats sur
%les twists, les périodes, détaillés l'algorithme et sa complexité, le prouver 
%(probablement fait avec le papier de Luca et Mihailescu pour les périodes ?)
%, détailler les différentes étapes, les paramètres, la façon de trouver m 
%etc.
\subsubsection{Principe}
%TODO: Choix d'un point d'ordre m sur une bonne courbe elliptique engendre 
%l'extension, utilisation des périodes elliptiques afin d'avoir un élément 
%stable par l'action du groupe de Galois (Énoncer les lemmes etc.)

\subsubsection{Choix des paramètres}
%TODO: Justifier le choix de m par rapport à n et q, tout ce qui concerne la 
%valeur propre d'ordre n et donc le choix de traces qui s'en suit, expliquer 
%comment on pick les courbes (à ce moment là on pourra aussi justifier avec le 
%résultat sur les tordues).

\subsubsection{Algorithme \& Analyse de complexité}
%TODO: Décrire l'algorithme et donner sa complexité.

\section{Implémentation}
%TODO: Comparer les deux méthodes ou même avec l'implémentation de base de SAGE 
%(retrouver une simple racine). Détailler les limitations dû à SAGE, peut-être
%prendre une partie de la conclusion en expliquant ce qui pourrait être changer.
\subsection{Résultats numériques}

\subsection{Comparaison avec d'autres méthodes}

\section{Conclusion}
%TODO: Expliquer ce qu'il peut rester à faire (cas général, car(K) = 2 et n 
%composé, couper et appliquer l'une des deux méthodes selon la nature du m ou 
%s'il y a besoin de calculer une extension, etc.), où est-ce qu'on peut 
%récupérer des performances (produit scalaire).

\begin{thebibliography}{LC}
\bibitem{Sam} \emph{Théorie algébrique des nombres}, \bsc{Pierre Samuel}, 
Hermann, 1971.

\bibitem{Nek} \emph{Théorie de Galois}, \bsc{Jan Nekov\'a\v{r}}, Université 
Pierre et Marie Curie, 2003, \bsc{url :} 
\url{http://www.math.jussieu.fr/~nekovar/co/ln/gal/g.pdf}.

\bibitem{Per} \emph{Cours d'algèbre}, \bsc{Daniel Perrin}, ellipses, 1996.
\bibitem{Rai} \emph{Efficient computation of isomorphism between finite fields},
\bsc{Eric M. Rains}, 2008.

\bibitem{LiNi} \emph{Finite fields}, \bsc{Rudolf Lidl} \& 
\bsc{Harald Niederreiter}, Encyclopedia of mathematics and its applications vol.
20, Cambridge, 1983.

\bibitem{Pin} \emph{Recognising elements of finite fields}, \bsc{Richard G.E. 
Pinch}, Cryptography and coding II, p. 193-197, Oxford University Press, 1992.

\bibitem{Pol} \emph{Algèbre et théorie de Galois}, \bsc{Patrick Polo}, 
Université Pierre et Marie Curie, 2007, \bsc{url :} 
\url{http://www.math.jussieu.fr/~polo/M1/ATG07chIV.pdf}.

\bibitem{Law1} \emph{Introduction to cyclotomic fields}, \bsc{Lawrence C. 
Washington}, Graduate texts in mathematics, Springer-Verlag, 1982.

\bibitem{Law2} \emph{Elliptic curves, number theory and cryptography},
\bsc{Lawrence C. Washington}

\bibitem{Sil} \emph{The arithmetic of elliptic curves}, 
\bsc{Joseph H. Silverman}, Graduate texts in mathematics, Springer, 2nd ed. 
2009.

\bibitem{GarD} \emph{Handbook of finite fields}, \bsc{Gary L. Mullen} \& 
\bsc{Daniel Panario}, Discrete mathematics and its applications, Series Editor 
Kenneth H. Rosen, CRC Press.
\end{thebibliography}
\end{document}

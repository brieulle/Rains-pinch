\documentclass[a4paper]{article} % papier A4
\usepackage[utf8]{inputenc}      % accents dans le source
\usepackage[T1]{fontenc}         % accents dans le pdf
\usepackage{textcomp}            % symboles complémentaires (euro)
\usepackage[frenchb]{babel}      % titres en français
\usepackage{amsmath}
\usepackage{amsthm}
\usepackage{amssymb}
\usepackage[colorlinks=false]{hyperref} %Apparemment ça sert à rien...
\usepackage{enumerate}
\usepackage{tocloft}             % Pour la table des matières
\usepackage{algorithm}
\usepackage{algorithmic}
\usepackage{pgf}
\usepackage{tikz}
\usepackage{tikz-cd}
\usepackage{rotating}
\usetikzlibrary{matrix,arrows,decorations.pathmorphing}

% Numérotation des sections, sous-sections, équations, etc.
\numberwithin{section}{part}
\numberwithin{equation}{section}

% Maccros pour les commandes de math
\newcommand\nroot[1]{\textit{#1}\up{\textit{ième}}}
\newcommand\zmodn[1]{\mathbb{Z}/#1\mathbb{Z}}
\newcommand\zmodninv[1]{(\mathbb{Z}/#1\mathbb{Z})^{\times}}
\newcommand\GF[1]{\mathbb{F}_{#1}}
\newcommand\Irr[2]{\textup{Irr}_{#1}(#2)}
\renewcommand{\algorithmicrequire}{\textbf{Input:}}
\renewcommand{\algorithmicensure}{\textbf{Ouput:}}
\newcommand\Tr[1]{\textup{Tr}\left(#1\right)}
\newcommand\QQ{\mathbb{Q}}
\newcommand\ZZ{\mathbb{Z}}
\newcommand\NN{\mathbb{N}}
\newcommand\CC{\mathbb{C}}
\newcommand\RR{\mathbb{R}}
\newcommand\EO{\mathcal{O}}
\newcommand\PP[1]{\mathbb{P}^{#1}}
\newcommand\etmath{\textup{\quad et \quad}}

% Éviter l'overlap dans la table des matières pour les sections, sous-sections
% etc.
\setlength{\cftsecnumwidth}{3em}    
\setlength{\cftsubsecnumwidth}{3em} 


\begin{document}
\title{Calcul d'isomorphisme pour les corps finis}
\author{Ludovic Brieulle}
\newtheorem{thm}{Thèorème}[section]
\newtheorem{lem}[thm]{Lemme}
\newtheorem{cor}{Corollaire}[thm]
\newtheorem{prop}[thm]{Proposition}
\theoremstyle{definition}
\newtheorem{defn}[thm]{Définition}
\newtheorem*{ex}{Exemple}
\theoremstyle{remark}
\newtheorem*{rem}{Remarque}

\maketitle
\part*{Remerciements}

\addcontentsline{toc}{part}{Introduction}
\part*{Introduction}
Soit $\GF{q}$ le corps fini à $q = p^r$ éléments, avec $p$ un nombre premier. On
définit alors $\overline{\GF{q}}$ sa clôture algébrique. Lorsqu'on a besoin de
faire des calculs dans cette clôture, il peut être plus aisé de calculer
directement sur un corps fini en particulier. Il est donc important de pouvoir
se déplacer rapidement entre les différents sous-corps de cette clôture.\par
Il est connu et prouvé que de deux corps finis de même cardinal sont reliés par
un isomorphisme. Cependant la preuve de ce résultat n'est pas constructive,
trouver un tel isomorphisme demande alors un travail supplémentaire.\par
La situation est la suivante, nous avons deux extensions de $\GF{q}$ de même
degré et définies par deux polynômes irréductibles distincts $f$ et $g$. 
On notera :
\[k_1=\GF{q}[X]/(f)\etmath k_2=\GF{q}[Y]/(g)\]
le but est donc de trouver un isormophisme reliant ces deux corps. Une méthode
simple et immédiate est d'envoyer $x = \bar{X}$ sur une racine de $f$ dans
$k_2$. Le problème est que cela revient à factoriser le polynôme $f$ ce qui est
beaucoup trop lent.
On va donc dans ce rapport étudier deux méthodes permettant, \textit{via}
l'utilisation de racines de l'unité et de courbes elliptiques, de calculer
rapidement de tels isomorphismes. 

\newpage
\addcontentsline{toc}{part}{Table des matières}
\tableofcontents
\newpage

\part{Contexte théorique}
\label{un}

Cette partie consistera principalement à rappeler et énoncer les résultats dont
nous aurons besoin afin de justifier et mettre en place les deux algorithmes 
pour calculer les isomorphismes de corps finis. Cette partie n'a pas pour 
vocation de faire un exposé complet des notions abordées, pour un traité plus 
détaillé, le lecteur pourra consulter au choix \cite{LiNi1}, \cite{MuPa} ou 
\cite[chap.~III]{Per} pour les corps finis; \cite{Nek}, \cite[chap.~VIII]{Pol},
\cite[chap. VI]{Sam} ou \cite{Esc} pour la théorie de Galois; \cite{Sam}, 
\cite{Was1} ou \cite{Lan} pour la théorie des nombres; \cite{Sil} ou 
\cite{Was2} pour la théorie sur les courbes elliptiques; et pour finir 
\cite{GaGe} pour tout ce qui concerne le calcul formel en général. On 
s'efforcera néamoins de démontrer les résultats les plus importants ou 
fondamentaux. Le c\oe ur du sujet se trouve lui dans la partie \ref{deux}.

\section{Corps finis}
Dans cette section on va rappeler les définitions et démontrer certains 
résultats liés aux corps finis. Principalement, on étudiera les extensions 
de corps et la théorie de Galois sur les corps finis. On ne considérera
que des corps commutatifs.

\subsection{Définitions}

Soit $\Omega = (K, +, .)$ un triplet où $K$ est un ensemble, $+$ et $.$ deux
lois de composition interne sur $K$. On dit que $\Omega$ est un corps si : 
\begin{enumerate}[(i)]
\item $(K, +)$ est un groupe abélien,
\item $(K\setminus\lbrace0\rbrace,.)$ est un groupe (abélien ou non, dans la 
littérature francophone),
\item la première loi est distributive par rapport à la seconde,
\item $K$ n'a pas de diviseurs de $0$.
\end{enumerate}
Par abus de notation, on désignera désormais le corps $\Omega$ par son ensemble
$K$. On notera $0$ et $1$ les éléments neutres de la première et seconde loi 
respectivement; on omettra régulièrement de noter la seconde loi dans les 
opérations sur $K$.\par
On appelle la caractéristique d'un corps $K$, le plus petit entier $p$ tel que 
pour tout $x\in K$ on ait :
\[p(x) = \underbrace{x + x + \dots + x}_{p fois} = 0\]
Elle peut être nulle ou positive; en pratique elle sera nulle ou égale à un
nombre premier. Dans notre cas, on s'intéressera au corps finis, c'est-à-dire 
tels que $K$ soit un ensemble fini et de caractéristique non-nulle.\par
\vspace{0.3cm}
Un homomorphisme de corps est nécessairement injectif. En effet, soit deux corps
$K$ et $L$ de caractéristique positive ou nulle. Soit $f : K \to L$, on a 
alors :
\[f(x) = 0 \Leftrightarrow x.f(1) = 0 \Leftrightarrow x = 0\]
puisque un homomorphisme doit conserver la structure et que $K$ n'a pas de
diviseurs de zéro, $x = 0$ et $f(1) = 1$. S'il existe un
tel homomorphisme entre deux corps, on a $K\subseteq L$ et on dit que $L$ est un
surcorps de $L$ ou $K$ est un sous-corps de $L$.\par
On appelle corps premier un corps qui n'a aucun sous-corps. Dans le cas des
corps finis, il s'agit du corps $\GF{p}$ à $p$ éléments, pour $p$ premier. Ce
qui nous donne le théorème suivant :

\begin{thm}
\label{th:corfincarprem}
Soit $K$ un corps fini, alors $K$ a $p^n$ éléments, où $p$ premier est la
caractéristique du corps premier et $n$ le degré de $K/\GF{p}$.
\end{thm}
\begin{proof}
De façon informelle, on appelle le degré de $L$ sur $K$ la dimension de $L$
en tant que $K$-espace vectoriel. On discutera plus en détails de cette
notion dans le point suivant \ref{def:degext}.\par
Montrons le lemme suivant :
\begin{lem}
Un corps fini a pour caractéristique un nombre premier.
\end{lem}
\begin{proof}
Comme $K$ est intègre en tant qu'anneau, sa caractéristique est plus grande ou
égale à $2$. Supposons que la caractéristique de $K$ soit $p = km$ pour $k,
m\in\ZZ$. Alors pour $e\in K$ on a e :
\[0 = ne = (km)e = (ke)(me)\]
puisque $K$ n'a pas de diviseurs de $0$, alors $ke = 0$ ou $me = 0$, ce
qui contredit la minimalité de la caractéristique.\\
\end{proof}
Donc $K$ est de caractéristique $p$ premier et son corps premier est (isomorphe
à) $\GF{p}$.
\begin{lem}
Soit $K$ un corps fini contenant un corps fini $L$ à q éléments. Alors $K$ a
$q^m$ élément où $m$ est le degré de $K$ en tant que $L$-espace vectoriel.
\end{lem}
Si $K$ est un $L$-espace vectoriel de dimension fini, il admet alors une base
d'éléments de $L$. Si on écrit chaque éléments de $K$ comme une combinaison
linéaire des éléments de cette base, chaque coeffecient de cette combinaison a
$q$ valeurs possibles, d'où $K$ à $q^m$ éléments; ce qui achève la démonstration
du lemme. Il suffit alors de remplacer $L$ par $\GF{p}$ et $q$ par $p$ et le
théorème est démontré.\\
\end{proof}

\begin{thm}[Wedderburn]
Tout corps fini est commutatif.
\end{thm}
On passera la preuve bien qu'elle soit intéressante; se référer par exemple à 
\cite[p.~70-73]{LiNi1} ou \cite[p.~82]{Per}.\par
\vspace{0.3cm}
Si $K$ est un corps à $q$ éléments, alors pour tout $x\in K$, on a $x^q = x$. En
effet, pour $x = 0$ c'est immédiat. Pour $x\neq 0$, on sait que le groupe
multiplicatif de $K$, noté $K^{\times} = K\setminus\lbrace0\rbrace$, est un
groupe fini à $q-1$ éléments par définition; donc $x^{q-1} = 1$ pour tout $x\in
K^{\times}$ et il suffit de multiplier encore une fois par $x$ pour obtenir le
résultat voulu.\par
On en déduit immédiatement que le polynôme $X^q - X$ se scinde dans $K$
puisque étant de degré $q$, il a au maximum $q$ racines qui sont déjà les
éléments de $K$. On peut dire que $K$ est un corps de décomposition
\ref{def:dec} de $X^q - X$ sur $\GF{p}$.\par
On notera que dans les corps finis et leurs extensions, l'élévation à une
puissance égal à la caractéristique d'un corps fini ou une puissance de celle-ci
est un automorphisme particulier. On l'appelle l'automorphisme de Frobenius et 
on le note $\textup{Fr}_q$ pour $\textup{Fr}_q(x) = x^q$. Ce n'est pas immédiat
qu'il s'agit là d'un homomorphisme, mais un simple calcul permet de s'en 
assurer.\par
\vspace{0.3cm}
Finissons ce point par le théorème suivant qui assure l'existence et l'unicité
à isomorphisme près des corps finis; c'est à partir de ce théorème que le
travail de tout ce rapport va se faire :

\begin{thm}
\label{th:isomGF}
Pour tout nombre premier $p$ et tout entier strictement positif $n$, il existe 
un corps fini à $p^n$ éléments. Tout corps fini à $q = p^n$ éléments est 
isomorphe au corps de décomposition de $X^q - X$ sur $\GF{p}$. On parlera alors 
du corps fini à $q$ éléments et on le notera $\GF{q}$.
\end{thm}
\begin{proof}
\textit{(Existence)} Pour $q = p^n$, on considère le polynôme $X^q - X$ dans 
$\GF{p}[X]$ et on note $K$ son corps de décomposition, définition \ref{def:dec},
sur $\GF{p}$. Le polynôme est séparable ou n'a aucune racine multiple puisque sa
dérivée est égale à $qX^{q-1} - 1 = -1$, proposition \ref{prop:sepderiv}. Posons
$S=\lbrace x\in K : x^q - x = 0\rbrace$, alors $S$ est un sous-corps de $K$; $0$
et $1$ sont dans $K$ et d'après les propriétés sur le Frobenius et les résultats
obtenus plus haut on a :
\[(a - b)^q = a^q - b^q = a - b \etmath(ab^{-1})^q = a^qb^{-q} = ab^{-1}\]
Ainsi, $S$ contient toutes les racines de $X^q - X$ mais comme $K$ a déjà $q$
éléments, alors $K = S$ est un corps à $q$ éléments.\par
\textit{(Unicité)} Soit $K$ le corps à $q = p^n$ éléments, d'après le théorème
\ref{th:corfincarprem}, est de caractéristique $p$ et contient $\GF{p}$. On en
déduit que $K$ est un corps de décomposition de $X^q - X$ sur $\GF{p}$,
puisqu'il est scindé sur $K$, et l'unicité se déduit de l'unicité des corps de
décomposition, théorème \ref{cor:dec}.
\end{proof}

\subsection{Extension de corps}
\label{def:degext}
On dit que $K$ est une extension de corps de $k$ s'il existe un morphisme de 
corps $\varphi : k \to K$. Ou de façon équivalente, si $k \subseteq K$ alors $K$
est une extension (de corps) de $k$; on note aussi $K/k$ une extension de corps.
Si on a $k\subseteq L \subseteq K$, alors $L/k$ est une sous-extension de
$K/k$.\par
Soit $K/k$ une extension de corps et $S$ une partie de $K$. Le sous-corps $L := 
k(S)$ de $K$ engendré par $S$ sur $k$ est le plus petit sous-corps de $K$ 
contenant $S$ et $k$. Si $S = \lbrace x_1,\dots,x_n \rbrace$ est fini, alors on 
note $L = K(x_1,\dots,x_n)$, on dit alors que l'extension est de type fini. 
L'extension $L/k$ est dite monogène ou simple si elle est engendré par un seul 
élément.\par
Si $K/k$ est une extension on peut voir $K$ comme un $k$-espace vectoriel ou une
$k$-algèbre. On appelle $[K:k] := dim_k(K)$ le degré de l'extension. On dit 
qu'une extension $K/k$ est de degré fini si $[K:k] < \infty$.\par

\begin{thm}
Soit $k \subseteq L \subseteq L$ des extensions de corps de degré fini. Alors on
a :
\[[K:k] = [K:L][L:k]\]
\end{thm}
\begin{proof}
Soit $[K:L] = m$ et $[L:k] = n$. On a donc que $K$ est un $L$-espace vectoriel 
de dimension $m$ et $L$ est un $k$-espace vectoriel de dimension $n$, le 
théorème revient à montrer que $K$ est un $k$-espace vectoriel de dimension 
$mn$. Or, d'après ce qui précède, on a $L \simeq k^n$ et $K \simeq L^m$, d'où
\[K \simeq \underbrace{L \oplus\dots\oplus L}_{m fois}\simeq\underbrace
{k^n\oplus\dots\oplus k^n}_{m fois} \simeq k^{nm}\]
ce qui prouve le théorème.\\
\end{proof}

On dit qu'un élément $x\in K$ est algèbrique sur $k$ s'il existe un polynôme 
unitaire à coefficient dans $k$ qui annule $x$. L'ensemble des éléments 
algébrique d'un corps (sur un sous-corps) forme un corps \cite[p.~64, théorème
1.14]{Per}. On dit qu'une extension $K/k$ est algébrique si tous les éléments 
de $K$ sont algébriques sur $k$.

\begin{prop}
Toute extension de degré fini est algébrique et de type fini.
\end{prop}
\begin{proof}
Si l'extension $K/k$ est de degré fini alors elle admet une base finie 
$(\alpha_1,\dots,\alpha_n)$ en tant que $k$-espace vectoriel, on a alors $K 
= k(\alpha_1,\dots,\alpha_n)$. Comme $[K:k] = n < \infty$, si $\alpha\in K$ 
alors $1, \alpha, \dots, \alpha^n$ satisfont une relation de dépendance 
linéaire, \textit{i.e.} il existe $a_0, \dots, a_n$ dans $k$ tels que :
\[a_0 + a_1.\alpha + \dots + a_n.\alpha^n = 0\]
puisque $K$ est un $k$-espace vectoriel. D'où tout $\alpha$ est algébrique 
sur $k$.\\
\end{proof}

\begin{prop}
Soit $\alpha\in L$ un élément algébrique sur $k$ et $f = X^d + a_1X^{d-1} + 
\dots + a_d$ son polynôme minimal sur $k$. Si on pose $d := \textup{deg }f$ 
alors on a :
\begin{enumerate}[(i)]
\item Les éléments $1, \alpha,\dots,\alpha^{d-1}$ forment une base de 
$k[\alpha]$ en tant qu'espace vectoriel.
\item $k[\alpha]$ est un corps, on a alors $k[\alpha] = k(\alpha)$.
\item On a $[k(\alpha):k] = d$.
\end{enumerate}
\end{prop}

\begin{proof}
(i) Il suffit de multiplier l'identité $\alpha^d = -a_1\alpha^{d-1} - \dots 
- a_d$ par $\alpha^i$ pour $i\geq0$ et on montre par récurrence que :
\[\alpha^{d+i} \in k\cdot1 + \dots + k\cdot\alpha^{d-1}\]
d'où $k[\alpha] = k\cdot1 + \dots + k\cdot\alpha^{d-1}$. De plus, les éléments 
$1,\alpha,\dots,\alpha^{d-1}$ sont linéairement indépendants puisque si :
\[u_0\cdot1 + u_1\cdot\alpha + \dots + u_{d-1}\cdot\alpha^{d-1} = 0\]
avec les $u_i\neq0$ cela contredirait la minimalité du degré de $f$.\par
(ii) $k[\alpha] \subset L$ est le sous-anneau d'un corps, il est donc intègre. 
De plus, pour tout $\beta\in k[\alpha]$ l'application $m_{\beta}$ de la 
multiplication par $\beta$ dans $k[\alpha]$ est $k$-linéaire et injective, 
puisque l'anneau est intègre. Pour finir, elle est aussi surjective puisque 
nous sommes en dimension finie, donc il existe $x\in k[\alpha]$ tel que 
$m_{\beta}(x) = \beta x = 1$.\par
(iii) Il résulte directement des deux points précédents.\\
\end{proof}

On vient de décrire plus ou moins la construction d'extension de corps 
\textit{via} des polynômes irréductibles ou par "adjonction de racines". 
Concrètement, soit $k$ un corps et $P$ un polynôme irréductible unitaire 
dans $k[X]$. Comme $k[X]$ est principal (il est même euclidien) alors $(P)$ est 
un idéal maximal, on peut alors faire le quotient $k[X]$ par $(P)$ est on 
obtient un corps $K \simeq k[X]/(P)$. Dans ce cas, la classe de $X$ qu'on notera
$x := \overline{X}$, est une racine du polynôme $P$ et engendre $K$ sur $k$.\par
Pour encore aller plus loin, considérons une extension $L/k$ telle que $P$ 
admette une racine $\alpha\in L$. Si on note $\Irr{k}{\alpha}$ le polynôme 
minimal de $\alpha$, alors $\Irr{k}{\alpha}$ divise $P$, donc est égal à 
$\lambda P$ pour $\lambda\in k^{\times}$, puisque $P$ est irréductible. Alors 
le morphisme $k$-algèbre $\phi : k[X] \to L$ défini par $\phi(X) = \alpha$ 
induit un morphisme de $k$-algèbre $\varphi : K \to L$ tel que $\varphi(x) 
= \alpha$. Ce morphisme est unique puisque $x$ engendre $K = k(x)$.\par
\vspace{0.3cm}
On a montré en particulier le théorème suivant :

\begin{thm}
Soit $k$ un corps et $P$ un polynôme irréductible dans $k[X]$. Alors $K := 
k[X]/(P)$ est un sur-corps de $k$ dans lequel $P$ a au moins une racine, la 
classe de $\overline{X} = x$. On l'appelle le corps de rupture de $P$ sur 
$k$.\par
\end{thm}

Une autre notion importante est celle de corps de décomposition d'un polynôme 
non constant sur un corps $k$.

\begin{defn}
\label{def:dec}
Le corps de décomposition d'un polynôme $P\in k[X]$ est le sur-corps $K \supset 
k$ contenant toutes les racines de $P$, qui est engendré par les racines 
susnommées.
\end{defn}

\begin{thm}
\label{cor:dec}
Tout $P\in k[X]$ non constant admet un corps de décomposition unique à 
isomorphisme près.
\end{thm}
\begin{proof}
\textit{(Existence)}
Pour prouver l'existence du corps de décomposition, il suffit de se placer 
dans la clôture algèbrique de $k$, alors si on note $K_0$ le corps engendré 
par $\alpha_1,\dots,\alpha_n$ les racines de $P$, il est clair que $K_0$ est 
le corps de décomposition de $P$.\par
\textit{(Unicité)}
On veut réalité montrer que si on a deux corps de décompositions, alors ils ont
isomorphes. Soit deux corps $k$ et $k^{\prime}$ isomorphes, on note $\phi$
l'isomorphisme, qu'on étend alors de façon évidente à un isomorphisme de $k[X]$ 
dans $k^{\prime}[X]$; \textit{i.e.} on applique $\phi$ au coefficients des 
polynômes. On pose  $L$ et $L'$ les corps de décomposition de $P\in k[X]$ et $P'
= \phi(P) \in k^{\prime}[X]$ respectivement.\par
Raisonnons par récurrence sur $[L:k]$, si $L = k$ alors $L^{\prime} =
k^{\prime}$. Sinon prenons $\alpha\in L$ une racine de $P$ qui n'est pas dans
$k$ et notons $Q$ son polynôme minimal sur $k$. On a alors $\alpha'$ racine de 
$P'$ et $Q'$ son polynôme minimal sur $k'$. Posons $M = k(\alpha)$ et $M' =
k'(\alpha)$, ainsi $M$ (resp. $M'$) est un corps de décomposition de $Q$ (resp.
$Q'$) sur $k$ (resp. $k'$). Puisque les corps de rupture sont uniques, modulo
l'isomorphisme qui envoie une racine sur l'autre, il existe donc un isomorphisme
$\psi : M \to M'$ tel que $\psi(\alpha) = \alpha'$. On a la situation suivante :

\begin{align*}
&P = (X - \alpha)S(X)\in M[X]\\
&P' = (X - \alpha')S'(X)\in M'[X]
\end{align*}

Comme $\alpha' = \psi(\alpha)$ alors $S'=\psi{\alpha}$, ainsi $L$ est un corps
de décomposition pour $S$ et $L'$ pour $S'$; d'où par l'hypothèse de récurrence
sur les degrés, on a un isomorphisme de $L \to L'$ prolongeant $\psi$ et on a 
démontré ce qu'il fallait.\\
\end{proof}

Penchons-nous alors plus en détails sur les corps finis. On peut se demander
étant donné un corps finis $\GF{p^n}$ quels peuvent être les corps qui le
contiennent. Dans un premier temps, il faut que les deux corps aient la même
caractéristique. Supposons qu'il existe un isomorphisme $f : \GF{p^n} \to 
\GF{l^m}$ pour $p$ et $l$ premiers. En particulier, on a pour $x\in\GF{p^n}$ :
\[f(p.x) = 0 = p.f(x)\]
pour $f(x)\in\GF{l^m}$, donc nécessairement $l = p$.\par
Ainsi, si le morphisme existe alors $\GF{p^n}^{\times}$ est un sous-groupe
d'ordre $p^n - 1$ de $\GF{p^m}^{\times}$ d'ordre $p^m - 1$. Donc, on a $p^n - 1$
qui divise $p^m - 1$. Si on écrit $m = q.n + r$ avec $r < n$ alors on obtient :
\[p^m - 1 = p^r((p^n)^q - 1) + p^r - 1\]
Or $p^n - 1$ divise $(p^n)^q - 1$ donc comme $p^n - 1$ divise aussi $p^m - 1$ la
seule possibilité pour $r$ est $0$ et donc $n|m$. On vient de prouver la
nécessité du critère suivant :

\begin{prop}
Soit $\GF{q}$ un corps fini à $q = p^m$ éléments, alors tout ces sous-corps sont
de la forme $\GF{p^n}$ avec $n|m$. Réciproquement, soit $n|m$ alors il existe un
(et un seul) sous-corps de $\GF{q}$ de cardinal $p^n$.
\end{prop}
\begin{proof}
Pour montrer l'autre sens, il suffit de remarquer que $\GF{p^n}$ correspond aux
éléments satisfaisant l'équation $X^{p^n} - X$ puisque $p^n - 1$ divise $p^m -
1$. L'unicité est à isomorphisme près comme énoncé dans le point précédent.\\
\end{proof}
%TODO : Théorème de l'élément primitif et... ?

\begin{thm}
\label{th:elemprim}
Toute extension séparable (\ref{def:sep}) et finie est simple.
\end{thm}
\begin{proof}
On va se contenter de le prouver dans le cas d'un corps fini; le cas général
demande un peu plus de travail. On invitera le lecteur a consulté
\cite[p.~87]{Esc} ou \cite[Chap. VIII]{Pol}.\par
Si $K$ est un corps fini alors son groupe mulitplicatif $K^{\times}$ est
cyclique (voir par exemple la preuve de la proposition \ref{prop:rootcycl} ou
\cite[p.~50]{LiNi1}). Dans ce cas, il suffit de prendre pour générateur d'une
extension $L/k$, le générateur du groupe mulitplicatif de $L$ et c'est gagné.\\
\end{proof}

Cela implique en particulier que toutes extensions de corps finis peut s'écrire
comme combinaison linéaire de puissances d'un unique élément, en pratique cet
élément sera la racine du polynôme par lequel on quotiente le corps premier afin
d'obtenir l'extension.
\vspace{0.3cm}
%TODO : Trace, norme, multiplication par un élément ?

Deux notions importantes pour les extensions de corps sont la trace et la norme
d'un élément d'une extension par rapport à un sous-corps. On les définit comme
suit:

\begin{defn}
Soit $L/K$ une extension de dimension finie et $\alpha\in L$. On appelle matrice
de multiplication par $\alpha$ la matrice de l'application :
\begin{align*}
m_{\alpha} :&L \longrightarrow L\\
&x\longmapsto\alpha.x
\end{align*}
On appellera alors trace et norme de $\alpha$ la trace et le déterminant 
de la matrice de $m_{\alpha}$. Les formules suivantes, pour $\alpha, \alpha'\in
L$ :
\[ m_{\alpha + \alpha'} = m_{\alpha} + m_{\alpha'} \etmath
m_{\alpha\alpha'} = m_{\alpha}m_{\alpha'}\]
induisent les résultats classiques de la norme et la trace.
\end{defn}

On a la proposition suivante :

\begin{prop}
Soit un $K$ un corps de caractéristique nulle ou finie, $L/K$ une extension
algébrique de degré $n$ sur $K$. Soit 
$x_1,\dots, x_n$ les racines du polynôme minimal de $x$, dans le corps de
décomposition de celui-ci, chacune répétée $[L:K(x)]$ fois; alors on a :
\[Tr_{L/K}(x) = x_1 + \dots + x_n \etmath N_{L/K}(x) = x_1\dots x_n\]
De plus, le polynôme caractéristique de $x$ relativement à $L$ et $K$ est $(X -
x_1)\dots(X - x_n)$.
\end{prop}
%TODO: La preuve, éventuellement.
%TODO: Je voulais en déduire que la trace est surjective, mais je sais plus trop
%comment, si tu t'en souviens un jours...

\subsection{Théorie de Galois}
%TODO :Définitions et résutlats de base (sans forcément tout prouver),
%Frobenius, action de groupes ? Ça me parait raisonnable...
%Groupe de Galois, élément primitif, extensions galoisiennes, etc.
Soit $L/k$ et $K/k$ deux extensions finies de corps avec $L\subseteq K$, alors 
il existe des morphismes qui laisse $k$ inchangé. On les appelles des 
$k$-morphismes et on note :
\[\textup{Hom}_k(L, K)\]
L'ensemble de tels morphismes. Si $L = K$, on parle alors de $k$-endomorphisme
et si ceux parmis ces endomorphismes qui sont inversibles forment un groupe,
qu'on appelle le groupe des automorphismes de $L$ et on le note :

\vspace{0.3cm}

On dit qu'un polynôme irréductible $f$ est séparable sur un corps $K$ si toutes
ses racines dans un corps de décompositions sont distinctes; la notion est
indépendante du corps de décomposition choisi, puisqu'il existe un $K$-morphisme
reliant les racines et dans ce cas leurs images sont aussi distinctes. On dit
qu'un élément $\alpha$ est spérable sur $K$ si son polynôme minimal est 
séparable sur $K$.

\begin{defn}
\label{def:sep}
On dit qu'une extension algébrique $L/K$ est séparable si tout les $\alpha\in L$
sont séparables.
\end{defn}

\begin{rem}
On peut définir la séparabilité d'un polynôme réductible en se basant sur ses
facteurs irréductibles, un polynôme sera séparable si tous ses facteurs
irréductibles le sont.
\end{rem}

\begin{prop}
\label{prop:sepderiv}
Soit $P$ irréductible sur un corps $K$, on a l'équivalence suivante :
\[P\textup{ séparable } \Leftrightarrow P'\neq 0\]
\end{prop}

\begin{prop}
Soit $k \subset L \subset K$ des extensions de corps. Si $K/k$ est séparable
alors $L/k$ et $K/L$ le sont aussi.
\end{prop}
\begin{proof}
Le corps $L$ étant contenu dans $K$, tous ses éléments sont aussi séparables.
Démontrons le lemme suivante :
\begin{lem}
\label{lem:divsep}
Soit $P$ un polynôme séparable sur k, $L$ une extension de $k$ et $Q$ un
diviseur de $P$ dans $L[X]$. Alors $Q$ est spérable sur $L$.
\end{lem}
\begin{proof}
Soit $K$ un corps de décomposition de $P$ sur $L$, alors $Q$ est scindé dans $L$
et ses racines sont parmi celles de $P$ qui sont déjà deux à deux distinctes.\\
\end{proof}
Pour finir, soit $x\in K$ alors $Irr_L(x)$ est un diviseur de $Irr_k(x)$ dans
$L[X]$ et on conclut en appliquant le lemme.\\
\end{proof}

\begin{prop}
Soit $K/k$ une extension algébrique simple, $P = Irr_k(\alpha)$ et $d =
\textup{deg\;}P = [K:k]$. Alors pour toute extension $L/k$, le nombre de
$k$-morphismes de $K \to L$ est égal au nombre de racine distinctes de $P$ dans
$L$. On a alors :
\[\#\textup{Hom}_k(K,L)\leq[K:k]\]
avec égalité si et seulement si $P$ a $d$ racines distinctes dans $L$.
\end{prop}
\begin{proof}
Pour tout $k$-morphsime $\phi : K\to L$, $\phi(\alpha)$ est une racine de $P$
dans $L$. Réciproquement, toute racine $\beta$ de $P$ dans $L$ définit un
$k$-morphisme, il suffit d'envoyer $\alpha$ sur $\beta$ et ces morphismes sont
disctints pour chaque racine distincte.\\
\end{proof}

On admettra le théorème suivant, il fait usage de la proposition ci-dessus et
des théorèmes 15.54 et 16.16 de \cite[Chap. VIII]{Pol}. Il nous permettra de
prouver le corollaire suivant.

\begin{thm}
Soit $K/k$ une extension de degré finie. 

\begin{enumerate}[(1)]
\item Si $x\in K$ est séparable sur $k$, alors $k(x)/k$ est séparable.

\item Soit $k \subset E \subset K$. Si $K/E$ et $E/k$ sont séparables alors
$K/k$ l'est aussi.

\item Si $K = k[x_1,\dots,x_r]$ avec chaque $x_i$ séparable sur $k$, alors $K/k$
est séparable.
\end{enumerate}
\end{thm}

\begin{cor}
Soit $P\in k[X]$ un polynôme séparable et $K$ un corps de
décomposition de $P$ sur $k$. Alors $K/k$ est séparable.
\end{cor}
\begin{proof}
Soit $x_1,\dots,x_r$ les racines de $P$ dans $K$, alors $K = k[x_1,\dots,x_r]$
par définition. On pose :
\[P = Q_1\dots Q_n\]
Par hypothèse, chaque $Q_j$ est séparable, lemme \ref{lem:divsep}, or les $x_i$ 
sont racines d'un des $Q_j$ donc leurs polynômes minimaux divisent ces $Q_j$ et 
en appliquant à nouveau le lemme, on montre que tous les $x_i$ sont séparables. 
On fini en appliquant le point (3) du théorème précédent.\\
\end{proof}

Soit $f$ un polynôme irréductible de degré $n$ sur un corps $K$. Soit $\alpha_1 
= a$ une racine de $f$ et $\alpha_2,\dots,\alpha_n$ les autres racines de $f$ 
dans une corps de décomposition. Alors on dit que les $\alpha_i$ sont les 
conjugués de $\alpha$ sur $K$.
\begin{defn}
On dit qu'une extension $L/K$ est normale ou quasi-galoisienne, si pour tout
$a\in L$, $Irr_k(a)$ a toutes ses racines dans $L$. Ou de façon équivalente, $L$
contient tous les conjugués de $a$.
\end{defn}

\begin{prop}
\label{prop:decompnorm}
Soit $P\in k[X]$ de degré $n\geq 1$ et $K$ un corps de décomposition de $P$ sur
$k$. Alors $K/k$ est normale.
\end{prop}
\begin{proof}
\cite[Chap. VIII, p.~167]{Pol}\\
\end{proof}

Revenons alors au groupe des $k$-automorphismes d'une extension. On a la
proposition suivante :

\begin{prop}
Soit $K/k$ une extension algébrique et soit $\alpha\in K$. Alors pour tout
$g\in\textup{Aut}_k(K)$, $g(\alpha)$ est racine de $\Irr{k}{\alpha}$. Si on
pose $G = \textup{Aut}_k(K)$ alors l'orbite de $\alpha$ sous l'action de $G$,
qu'on notera $G\alpha$ est un ensemble fini de cardinal inférieur ou égal à
$\textup{deg}_k(x)$ et $\Irr{k}{\alpha}$ est divisible par :
\[\prod_{\beta\in G\alpha}{(X - \beta)}\]
\end{prop}
\begin{proof}
Comme $g$ est un morphisme on a $0 = g(P(\alpha)) = P(g(\alpha))$ d'où
$g(\alpha)$ est aussi une racine.\par
Donc pour tout $g\in G$, $X - g(\alpha)$ divise $P$, mais comme $P$ étant de
degré $d := \textup{deg}_k(x)$ a au plus $d$ facteur distincts, $G\alpha$ est 
forcément fini et de cardinal inférieur ou égal à $d$. Le produit des $X -
\beta$ pour $\beta\in G\alpha$ divise alors $P$.\\
\end{proof}

On va pouvoir alors introduire la notion du groupe de Galois d'une extension
galoisienne. On énoncera la définition et quelques propriétés de base sur ce
groupe qui nous servirons implicitement tout le long du rapport. On ne
démontrera pas la plupart des résultats \textit{(sauf s'il me reste du temps
pour peaufiner à fond)}. 

\begin{defn}
\label{def:gal}
Soit $K/k$ une extension de degré fini. On dit que $K/k$ est galoisienne si
$\#\textup{Aut}_k(K) = [K:k]$. Dans ce cas, on note $\textup{Gal}(K/k) :=
\textup{Aut}_k(K)$ et pour tout $\alpha\in K$ et $g\in\textup{Gal}(K/k)$, les
$g(\alpha)$ seront appelés les conjugués de $\alpha$.
On appellera le groupe de Galois d'un polynôme $P$, le groupe de Galois d'un
corps de décomposition de $P$; on pourra le noter $\textup{Gal}(P/k)$.
\end{defn}

\begin{thm}
Soit $K/k$ une extension de degré fini. Les conditions suivantes sont
équivalentes :
\begin{enumerate}
\item L'extension $K/k$ est galoisienne.
\item L'extension $K/k$ est normale et séparable.
\item Le corps $K$ est le corps de décomposition d'un polynôme séparable.
\end{enumerate}
Sous ces conditions, pour tout $\alpha\in K$ on a :
\[\Irr{k}{\alpha} = \prod_{\beta\in G\alpha}{(X - \beta)}\]
\end{thm}
\begin{proof}
\cite[Chap. VIII, p.~169]{Pol}\\
\end{proof}


\begin{thm} Soit $k$ un corps, $P$ un polynôme de degré $n$ dans $k[X]$ et $K$ 
un corps de décomposition de $P$. On a :
\begin{enumerate}
\item Le groupe de Galois de $K/k$ et donc de $P$ est un sous-groupe du groupe
symétrique $\textup{S}_n$; donc son ordre divise $n!$.
\item Si $P$ est irréductible alors le groupe de Galois de $P$ agit
transitivement sur ses racines; son ordre est divisible par $n$.
\end{enumerate}
\end{thm}
\begin{proof}
\cite[Chap. IX, p.~206]{Pol}\\
\end{proof}

Pour les corps finis, déterminer le groupe de Galois est beaucoup plus facile
comme le montre le théorème suivant :

\begin{thm}
Soit $n\geq1$ et $\GF{q}$ le corps à $q=p^r$ éléments, avec $p$ premier.
L'extension $\GF{q^n}/\GF{q}$ est galoisienne et son groupe de Galois
$\textup{Gal}(\GF{q^n}/\GF{q})$ est cyclique d'ordre $n$, engendré par 
l'automorphisme de Frobenius $\textup{Fr}_q$.
\end{thm}
\begin{proof}
On a déjà montré que $\GF{q^n}$ était le corps de décomposition de $Q = X^{q^n} 
- X$, théorème \ref{th:isomGF}, donc que l'extension était normale, proposition 
\ref{prop:decompnorm}. De plus, comme $Q' = -1 \neq 0$, $Q$ est séparable,
proposition \ref{prop:sepderiv}.  D'où $\GF{q^n}/\GF{q}$ est galoisienne et donc
:
\[|\textup{Gal}(\GF{q^n}/\GF{q})| = [\GF{q^n}:\GF{q}] = n\]
Comme le Frobenius est un $\GF{q}$-automorphisme, c'est une morphisme et un 
élément de $\GF{q}$ élevé à la puissance $q^n$ est toujours égal à lui-même, 
c'est un élément de $\textup{Gal}(\GF{q^n}/\GF{q})$. Notons $d$ son ordre, donc
$\textup{Fr}_q^d = Id_{\GF{q^n}}$ ou encore, pour tout $x\in\GF{q^n}$ :
\[\textup{Fr}_q^d(x) = x^{q^d} = x \Leftrightarrow x^{q^d} - x = 0\]
Le polynôme $X^{q^d} - X$ ayant au plus $d$ racines, on a $d\geq n$ et donc $d =
n$ d'où $\textup{Fr}_q$ engendre $\textup{Gal}(\GF{q^n}/\GF{q})$ et donc le
résultat.\\
\end{proof}

%TODO : Groupe de décomposition et groupe d'inertie ?

\subsection{Extensions et polynômes cyclotomiques}
%TODO : Racines de l'unités et quelques mots sur les polynômes 
%cyclotomiques et leurs extensions. Peut-être rajouter un ou deux mots sur le 
%cas des corps finis.
On définit une racine \nroot{n} de l'unité dans un corps $K$ 
comme un élément $u\in K$ tel que pour $n\in\NN$ on ait $u^n = 1$. Ce 
sont exactement les racines du polynômes $X^n - 1$.
On note $U_n$ l'ensemble des racines \nroot{n} de l'unité dans une clôture 
algébrique de $K$ et $U_n(K)$ l'ensemble des racines \nroot{n} de l'unité 
appartenant à $K$.

\begin{prop}
\label{prop:rootcycl}
Soit $K$ un corps, s'il contient $n$ racines \nroot{n} de l'unité alors elles 
forment un groupe cyclique $U_n(K)$ d'ordre $n$ isomorphe à $\zmodn{n}$.
\end{prop}
\begin{proof}
La proposition découle d'un théorème plus général disant que tout sous-groupe 
fini $G$ du groupe multiplicatif $K^*$ est formé de racine de l'unité et est 
cyclique. Lui-même est tiré de deux corollaires du théorème suivant (voir 
\cite[p.~26-27]{Sam} pour une démonstration détaillée) :

\begin{thm}
Soit $A$ un anneau principal, $M$ un $A$-module libre de type fini et $M'$ un 
sous-module de $M$. Alors :
\begin{enumerate}[1)]
\item $M'$ est un module libre de rang $\leq n$.
\item Il existe une base $(e_1,\dots,e_n)$ de $M$, un entier $q\leq n$ et des 
éléments $a_1,\dots,a_q$ non nuls de $A$ tels que $(a_1e_1,\dots,a_qe_q)$ soit 
une base $M'$ avec $a_i|a_{i+1}$ pour tout $1\leq i\leq q-1$.
\end{enumerate}
\end{thm}

\begin{cor}
Soit $E$ un $A$-module de type fini. Alors $E$ est isomorphe à un produit 
$(A/\mathfrak{a}_1)\times\dots\times(A/\mathfrak{a}_n)$, où les $\mathfrak{a}_i$
sont des idéaux de $A$ tels que $\mathfrak{a}_1\supset\mathfrak{a}_2\supset
\dots\supset\mathfrak{a}_n$.
\end{cor}

\begin{cor}
Soit $G$ un groupe commutatif fini. Il existe $x\in G$ dont l'ordre est le 
$ppcm$ des ordres des éléments de $G$.
\end{cor}
\begin{proof}
On sait qu'un groupe commutatif est un $\ZZ$-module (si on le note 
additivement). D'après le corollaire ci-dessus, $G$ est donc isomorphique à 
un produit $\zmodn{a_1}\times\dots\times\zmodn{a_n}$ où $a_1|a_2|\dots|a_n$. 
Aucun des $a_i$ n'est nul, sinon $G$ serait infini; ce serait le produit de 
$\ZZ^r$ avec des groupes finis. On note $y$ la classe de $1$ dans 
$\zmodn{a_n}$ et on pose $x = (0,\dots,0,y)$. Alors $x$ est d'ordre $a_n$ et si 
on prend un élément $z\in G$, avec $z = (z_1,\dots,z_n)$, on a aussi $a_nz = 0$ 
car $a_i|a_n$ pour tout $1\leq i \leq n$. $x$ est donc l'élément recherché.\\
\end{proof}

De là, on en déduit qu'il existe un $z\in G$ d'ordre $n$ tel que $y^n = 1$ pour 
tout $y\in G$. Comme le nombre de racines de $X^n - 1$ sur un corps est au plus 
$n$ alors $G$ a au plus $n$ éléments. Or $z$ est d'ordre $n$ donc $G$ contient 
les éléments $z, z^2,\dots,z^n = 1$ qui sont distincts. Donc $G$ est formé par 
ces éléments et est cyclique d'ordre $n$. La proposition en découle 
naturellement.\\
\end{proof}

\begin{defn}
On appelle racine primitive \nroot{n} de l'unité les racines qui engendrent 
$U_n$.
\end{defn}

\begin{prop}
Les racines primitives \nroot{n} de l'unité forment un sous-groupe $U^{\times}_n
$ 
isomorphe à $(\zmodn{n})^*$. En particulier, il y en a $\varphi(n)$.
\end{prop}
\begin{proof}
D'après \ref{prop:rootcycl} les éléments qui engendrent $U_n$ sont exactement 
les éléments dont l'image engendre $\zmodn{n}$. D'où l'isomorphisme et l'ordre 
de $U^{\times}_n$.\\
\end{proof}

Lorsqu'on les considère sur $\CC$, les racines \nroot{n} de l'unité sont 
de la forme $e^{2ik\pi/n}$ et les racines primitives définissent un polynôme 
qu'on appelle polynôme cyclotomique défini comme suit : 
\[\phi_n(X) = \prod_{\zeta\in U_n^{\times}}{(X - \zeta)}\]
On va montrer qu'il s'agit du polynôme minimal de ces racines sur $\QQ$, 
commençons par la proposition suivante :

\begin{prop}
Le polynôme $\phi_n$ appartient à $\ZZ[X]$ et est de degré égal à 
$\varphi(n)$.
\end{prop}
\begin{proof}
Nous allons utiliser les deux résultats suivants; on ira voir 
\cite[p.~72, 80]{Per} pour des démonstrations détaillées.

\begin{lem}
Soit $n\in\NN^*$ alors on a $n = \sum_{d|n}{\varphi(d)}$.
\end{lem}

\begin{prop}
On a $X^n - 1 = \prod_{d|n}{\phi_d(X)}$.
\end{prop}

On va alors raisonner par récurrence sur $n$. On a $\phi_1(X) = X-1 
\in\ZZ[X]$. Supposons alors que le résultat soit vrai pour $d<n$. Posons 
$F(X) = \prod_{d|n\atop{d\neq n}}{\phi_d(X)}$, alors $F\in\ZZ[X]$ et est 
unitaire. Si on fait la division euclidienne de $X^n - 1$ par $F$ dans 
$\ZZ[X]$, alors on obtient :
\[X^n - 1 = F(X)P(X) + R(X)\]
avec $P, R \in\ZZ[X]$ et $\textup{deg~}R < \textup{deg~}F$. Or, on a déjà
$X^n - 1 = \phi_n(X).F(X)$ dans $\QQ[X]$, \textit{i.e.} $F(X)(\phi_n(X) -
P(X)) = R(X)$, mais le degré de $R$ étant plus petit que celui de $F$, on a 
nécessairement $\phi_n = P \in\ZZ[X]$. Le résultat sur le degré résulte 
de la proposition et du lemme ci-dessus.\\
\end{proof}
\begin{thm}
Le polynôme $\phi_n$ est irréductible sur $\QQ$.
\end{thm}
\begin{proof}
Soit $K$ un corps de décomposition de $\phi_n$ sur $\QQ$, $\zeta$ une 
racine primitive \nroot{n} de l'unité et $p$ un nombre premier ne divisant pas 
$n$. Les racines primitives de l'unité sont toutes de la forme $\zeta^m$ avec 
$(m, n) = 1$ puisqu'elles doivent être d'ordre exactement égal à $n$; cela ce 
soit sur leur expression exponentielle. On veut alors montrer que $\zeta$ et 
$\zeta^p$ ont exactement le même polynôme minimal.\par
Posons alors $f$ et $g$ les polynômes minimaux sur $\QQ$ de $\zeta$ et 
$\zeta^p$ respectivement. Ils sont dans $\ZZ[X]$ puisqu'ils divisent tous
les deux $\phi_n(X)$. En effet, comme $\ZZ[X]$ est factoriel, on a :
\[\phi_n = f_1^{a_1}\dots f_r^{a_r}\]
avec $f_i$ unitaire, irreductible et dans $\ZZ[X]$; puisque $\phi_n$ est
unitaire et dans $\ZZ[X]$. Alors un $f_i$ annule $\zeta$ et un $f_j$ 
annule $\zeta^p$; or ils sont, d'après Gauss, irréductibles et unitaires dans 
$\QQ[X]$, il s'agit donc de $f$ et $g$. En particulier, $f$ et $g$ 
divisent $\phi_n$ dans $\ZZ[X]$.\par
Supposons alors que $f \neq g$. Comme ils sont irréductibles et disctincts, leur
produit $f.g$ divise $\phi_n$. Comme $g(\zeta^p) = 0$, alors $\zeta$ est racine 
de $g(X^p)$, ainsi $f$ divise $g(X^p)$ dans $\QQ[X]$ mais aussi dans 
$\ZZ[X]$ d'après le lemme de Gauss sur les contenus de polynômes. On va 
alors considérer les polynômes dans $\GF{p}[X]$, on note $\bar{f}$ et $\bar{g}$ 
leurs réductions modulo $p$. Par Frobenius, on obtient que $\bar{g}(X^p) = 
\bar{g}(X)^p$, ainsi si on prend $r$ un facteur irréductible de $\bar{f}$ et on 
écrit :
\[\bar{g}(X)^p = \bar{f}(X)\bar{h}(X)\]
alors $r$ divise aussi $\bar{g}$ par le lemme d'Euclide. Puique $f.g$ divise 
$\phi_n$ alors $\bar{f}.\bar{g}$ divise $\bar{\phi_n}$, on a $r^2$ qui divise 
$\bar{\phi_n}$. Ceci implique en particulier que $\bar{\phi_n}$ aurait une 
racine double, donc $\bar{X}^n - 1$ aussi. Or, $\bar{X}^n - 1$ est séparable sur
$\GF{p}$ puisque $(X^n - 1)' = nX^{n-1}$ et $p$ ne divise pas $n$, il ne peut 
donc pas avoir de racines multiples, d'où une contradiction.\par
Reste alors à montrer que tous les conjugués de $\zeta$ sont exactement les 
autres racines primitives \nroot{n} de l'unité. Si on prend une racine primitive
\nroot{n} $\zeta'$ quelconque, alors on a $\zeta' = \zeta^m$ où $m = p_1^{a_1}
\dots p_l^{a_l}$ avec $p_i \nmid n$. Il suffit alors de faire de procéder par 
récurrence en utilisant le raisonnement ci-dessus pour s'appercevoir que $\zeta$
et $\zeta'$ ont effectivement le même polynôme minimal. De sorte que $f$ admet 
toutes les racines primitives comme zéros, \textit{i.e.} $\textup{deg~}f \geq 
\textup{deg~}\phi_n$, et $f\mid\phi_n$ alors on a bien $f = \phi_n$ irréductible
sur $\QQ$ comme on voulait.\\
\end{proof}

\begin{rem}
Comme $\phi_n$ est unitaire alors son contenu est égal à $1$ ce qui implique, 
avec ce qui précède, qu'il est alors aussi irréductible dans $\ZZ[X]$.
\end{rem}
\vspace{0.3cm}
Grâce à ce résultat on peut alors introduire une notion qui nous sera utile dans
la suite, celle de corps cyclotomique.

\begin{defn}
Soit $K$ un corps commutatif, on appelle corps cyclotomique $K^{(n)}$ sur $K$ le
corps de décomposition de $X^n - 1$ sur $K$.
\end{defn}

\begin{rem}
D'après les résultats ci-dessus, on voit que dans le cas $K = \QQ$, le 
corps de décomposition de $X^n - 1$ est exactement $\QQ(\zeta_n) \simeq 
\QQ[X]/\phi_n(X)$, où $\zeta_n$ est une racine primitive. C'est en 
particulier une extension algébrique simple.
\end{rem}

%TODO: Parler des racines de l'unité dans les corps finis avant de foutre le
%résultat.
Le théorème suivant sera cruciale dans la suite du mémoire, il servira notament 
à justifier théoriquement la méthode de Rains.

\begin{thm}
\label{th:polycycldecomp}
Si $K = \GF{q}$ et $(n,q) = 1$, alors $\phi_n$ se factorise en $\varphi(n)/d$ 
polynômes unitaires irréductibles dans $\GF{q}[X]$ de même degré égal à $d$. 
$K^{(n)}$ est le corps de décomposition de n'importe lequel de ces polynôme et 
on a :
\[[K^{(n)}:K] = d\]
avec $d$ l'ordre multiplicatif de $q$ dans $\zmodn{n}$.
\end{thm}
\begin{proof}
Soit $\zeta$ une racine primitive \nroot{n} de l'unité dans $\GF{q}$, alors 
$\zeta$ appartient a un sur-corps $\GF{q^k}$ si et seulement si $\zeta^{q^k} = 
\zeta$; ce qui est équivalent à $q^k \equiv 1 \bmod n$ puisque $\zeta^n = 1$ par
définition. On pose alors $d$ égal au plus petit $K$ satisfaisant cette 
condition, dans ce cas $\zeta\in\GF{q^d}$ et ne peut pas être dans un sous-corps
de celui-ci. Ainsi, le polynôme minimal de $\zeta$ est de degré $d$ et comme on 
a choisi $\zeta$ arbitrairement, on obtient le résultat voulu.\\
\end{proof}

Dans la suite de la sous-section précédente, on peut énoncer un résultat sur les
extensions cyclotomiques sur $\QQ$.

\begin{prop}
Soit $n \geq 2$ et $\zeta$ une racine primitive \nroot{n} de l'unité sur $\QQ$
, alors l'extension $\QQ(\zeta_m)/\QQ$ est galoisienne. En particulier, on a 
les points suivant :

\begin{enumerate}[(i)]
\item L'extension $\QQ(\zeta_m)/\QQ$ est normale.

\item On a $[\QQ(\zeta_m):\QQ] = \varphi(n)$.

\item Et $\textup{Gal}(L/K) = \zmodninv{n} \simeq U^{\times}(n)$.
\end{enumerate}
\end{prop}
\begin{proof}
(i) On a vu que $\phi_n$ était le polynôme minimal de toutes les racines 
primitives \nroot{m} de l'unité et comme ces mêmes racines forment un groupe 
cyclique, elles sont toutes dans $\QQ(\zeta_m)/\QQ$; d'où l'extension est
normale.\par
(ii) $\phi_n$ est irréductible de degré $\varphi(n)$, donc on a bien
$[\QQ(\zeta):\QQ] = \varphi(n)$.\par
(iii) Soit $\sigma\in G := \textup{Gal}(L/K$, comme $\sigma(\zeta)$ est un
conjugué de $\zeta$, nécessairement $\sigma(\zeta) = \zeta^k$ pour un $k$
premier avec $n$. On a donc l'application $\psi : G \to \zmodninv{n}$ telle que
$\psi(\sigma) = k$. De plus, pour $\psi(\sigma') = k'$, on a :
\[\sigma'\circ\sigma(\zeta) = \sigma'(\zeta^k) = \zeta^{kk'}\]
d'où :
\[\psi(\sigma\circ\sigma') = \psi(\sigma)\psi(\sigma')\]
L'application est donc un morphisme de groupes. Elle est aussi injective puisque
$\psi(\sigma) = 1$ est équivalent à $\psi = \textup{Id}$. On conclut alors en
appliquant le (ii), ce qui implique que $\psi$ est un isomorphisme, et en 
rappelant la définition \ref{def:gal}.\\
\end{proof}

\section{Théorie des nombres}
%TODO: Tout ce qui concerne la preuve de Rains avec les idéaux 
%premiers, les normes d'ideaux, la factorisation unique, Galois, les périodes de
%Gauss, les périodes elliptiques (Mihailescu et al.) ou plus précisément ce qui 
%permet d'y arriver.
Dans cette section, on introduira la notion de corps de nombres, de leur
anneaux d'entier et des idéaux premiers sur ses anneaux. Le but va être de 
démontrer l'identité fondamentale dans le cadre des extensions Galoisienne. 
Cela sera utile pour la sous-section \ref{sec:pergauss} et plus précisément le 
théorème \ref{th:gaussnormal}.
\subsection{Entiers sur les corps de nombres}
%TODO :Définitions (normes, traces, éléments algébriques, etc.)
% entiers algébrique, anneaux des entiers, discriminant (?)
Commençons par le théorème suivant :
\begin{thm}
Soit $K$ un corps, $A\subset K$ un sous-anneau et $x\in K$. Alors les trois
points suivants sont équivalents :
\begin{enumerate}[a)]
\item L'élément $x$ est racine d'un polynôme unitaire à coefficients dans $A$.

\item L'anneau $A[x]$ est un $A$-module de type fini.

\item Il existe un $A$-module de type fini $M$ qui contient $A[x]$.

\end{enumerate}
\end{thm}
\begin{proof}
Supposons a) vraie, alors on a $a_0,\dots,a_{n-1}\in A$ tels que :
\[x^n + a_{n-1}x^{n-1} + \dots + a_0 = 0\]
Notons $M$ le $A$-module engendré par $1, x,\dots, x^{n-1}$, on a alors $x^n\in
M$ et on montre par récurrence sur $j$ que $x^{n+j} = -a_{n-1}x^{n-1 +
j}-\dots- a_0x^j$ sont dans $M$. Comme l'anneau $A[x]$ est engendré par les
$x^j$ pour $j\geq0$, donc $A[x] \subseteq M$; l'autre inclusion étant immédiate,
on a donc $A[x] = M$.\par
L'implication de b) vers c) est immédiate. Supposons qu'il existe un $M =
\langle{v_1,\dots,v_n}\rangle$ tel que $A[x]\subset M$ ou encore $xM \subset M$
puisque $x\in M$, avec $M\neq 0$. Alors :
\begin{align*}
xv_1 &= a_{11}v_1 + \dots + a_{1n}v_n\\
&\vdots\\
xv_n &= a_{n1}v_1 + \dots + a_{nn}v_n
\end{align*}
avec les $a_{ij}\in A$. 
\end{proof}



\subsection{Idéaux premiers}
%TODO: Idéaux premiers sur les anneaux d'entiers de corps de nombres,
%décomposition unique, résultat de la théorie de Galois.


\section{Courbes elliptiques}
%TODO: Définitions (j-invariant, nombres de points, tordues, module de
%Tate (?) etc.), résultats généraux, sur les corps finis
On va maintenant introduire les courbes elliptiques et plus en particulier les
courbes elliptiques sur les corps finis. Parmi les notions importantes qui nous
serons utiles, on parlera des points de torsion, des périodes elliptiques et des
tordues d'une courbe elliptique. Dûe à la nature un peu extérieur des courbes
elliptiques par rapport au isomorphisme de corps finis, on ne s'attardera pas 
la plupart du temps, à définir les notions de géométrie algébriques. On invitera
le lecteur à consulter les deux premiers chapitre de \cite{Sil} pour plus de
précision.

\subsection{Définitions}
%TODO: Rapide énoncé des changements de variable, comment on arrive à la forme
%"simple"; liste des différents paramètres/invariants qui nous intéressent.
%j-invariant, groupe d'automorphisme ?> c'est peut-être là que je peux utiliser
%le module de Tate...
Commençons par une définition plutôt générale d'une courbe elliptique.

\begin{defn}
Une courbe elliptique est un couple $(E, \EO)$, où $E$ est une courbe 
projective lisse de genre $1$ et $\EO$ un point de $E$ qu'on appelle origine.
\end{defn}

Par abus de langage, on parlera désormais de la courbe elliptique $E$ en
supposant l'origine $\EO$ fixée. Une courbe projective est une variété de
dimension $1$ dans $\PP{2}$; le genre est un invariant de la courbe, on ne
s'attardera pas dessus; et la notion de courbe lisse sera rapidement définie
plus loin \ref{def:lisse}.\par
On dira qu'une courbe E est définie 
sur $K$ si son équation est à coefficient dans $K$ et on notera $E/K$. Si on ne
précise pas le corps de définition, cela veut dire qu'on considère $E$ sur la
clôture algébrique de $K$.

\begin{defn} 
On appelle équation de Weierstrass, une équation projective de $\PP{2}$ de la
forme :
\begin{equation}
\label{eq:weierfull}
ZY^2 + a_1XYZ + a_3YZ^2 = X^3 + a_2X^2Z + a_4XZ^2 + a_6Z^3
\end{equation}
On appelle courbe de Weierstrass, une courbe projective $W$ définie par 
l'équation \ref{eq:weierfull}, elle possède un point à l'infini $O = 
[0, 1, 0]$.
\end{defn}

On ne s'intéressera principalement au corps de caractéristique différente de $2$
et $3$. Dans ce cas, il existe deux changements de variables, (voir \cite[Chap. 
III,p.~42]{Sil} qui permettent d'obtenir la forme plus connue qui servira pour 
les courbes elliptiques, autrement dit une courbe de Weierstrass $W$ peut se 
mettre sous la forme, en posant $Z = 1$ :
\begin{equation}
\label{eq:weiersimpl}
y^2 = x^3 + Ax + B
\end{equation}
où $A, B\in\bar{K}$. Pour la forme complète, on définit aussi les
quantités suivantes :
\begin{align*}
b_2 &= a_1^2 + 4a_2,\\
b_4 &= 2a_4 + a_1a_3,\\
b_6 &= a_3^2 + 4a_6,\\
b_8 &= a_1^2a_6 + 4a_2a_6 - a_1a_3a_4 + a_2a_3^2 - a_4^2,\\
c_4 &= b_2^2 - 24b_4,\\
c_6 &= b_2^3 + 36b_2b_4 - 216b_6,\\
\Delta &= -b_2^2b_8 - 8b_4^3 - 27b_6^2 + 9b_2b_4b_6,\\
j &= c_4^3/\Delta,\\
\end{align*}

\begin{defn}
On appelle $\Delta$ le discriminant d'une courbe de Weierstrass et $j$ le
$j$-invariant de cette même courbe.
\end{defn}

On va maintenant énoncé le résulat permettant de faire le lien entre les courbes
elliptiques et celles de Weierstrass.

\begin{prop}
Soit $E$ une courbe elliptique définie sur un corps $K$.
\begin{enumerate}[(i)]
\item Il existe des fonctions $x, y\in K(E)$ telles que l'application :
\begin{align*}
\phi :&\;E \longrightarrow \PP{2}\\
&P \longmapsto [x,y,1]
\end{align*}
donne un isomorphisme entre $E/K$ et une courbe de Weierstrass donnée par
l'équation :
\[W : Y^2 + a_1XY + a_3Y = X^3 + a_2X^2 + a_4X + a_6\]
avec $a_1,\dots,a_6\in K$ et satisfaisant $\phi(\EO) = [0, 1, 0]$.

\item Les seuls changements de variables qui conservent la structure d'équation
de Weierstrass sont de la forme :
\[X = u^2X' + r, \quad Y = u^3Y' + su^2X' + t,\]
avec $u\in K^{\times}$ et $r,s,t\in K$.

\item Réciproquement, toutes courbes projectives lisses données par une équation
de Weierstrass \ref{eq:weierfull} est une courbe elliptique de point d'origine
$\EO = [0, 1, O]$.
\end{enumerate}
\end{prop}

On ne va pas démontrer ce résultat. Le point (i) permet entre autre d'utiliser
la formule \ref{eq:weiersimpl} pour les courbes elliptiques définies sur des
corps de caractéristique différente de 2 et 3. Le point (ii) permettra, entre
autre, de calculer les tordues d'une courbe elliptique. Le point (iii) est assez
parlant de lui-même.\par
On va s'attarder un peu sur le caractère lisse d'une courbe de Weierstrass (donc
elliptique) et ses invariants. Commençons par une définition générale :

\begin{defn}
Soit $V$ une variété, $P\in V$ et $f_1,\dots,f_m \in\bar{K}[\bar{X}]$ des 
générateurs de $I(V)$. Alors $V$ est non-singulière ou lisse en $P$ si la 
matrice $m \times n$ suivante :
\[\left(\frac{\partial f_i}{\partial X_j}(P)\right)_{1\leq i \leq m\atop 1\leq j
\leq m}\]
est de rang $n - \textup{dim}(V)$; où $\bar{X} = X_1,\dots,X_n$. La variété $V$
est dite lisse si elle est lisse en tous ses points.
\end{defn}

Pour une courbe $E$ d'équation \ref{eq:weiersimpl}, on définit le discrimant et
le $j$-invariant de $E$ comme suit :
\[\Delta = -16(4A^3 + 27B^2) \etmath j = -1728\frac{(4A)^3}{\Delta}\]
Les seuls changements de variables conservant la forme simplifiée sont les
suivants :
\[x = u^2x' \etmath y = u^3y' \textup{\quad pour un } u\in K^{\times}\]
ce qui nous donne :
\[u^4A' = A, \quad u^6B' = B, \quad u^{12}\Delta' = \Delta\]

\begin{prop}
\label{def:lisse}
Une courbe $E$ donnée par une équation de Weierstrass est lisse si est seulement
si $\Delta \neq 0$.
\end{prop}
\begin{proof}
On peut commencer par montrer que $\EO$ n'est jamais un point singulier. Pour
cela, il suffit de voir que si on pose :
\[F(X,Y,Z) = Y^2Z + a_1XYZ + a_3YZ^2 - X^3 - a_2X^2Z - a_4XZ^2 - a_6Z^3\]
On a bien :
\[F(\EO) = F(0, 1, 0) = 0 \etmath \frac{\partial F}{\partial Z}(\EO) = 1 \neq
0\]
D'où $\EO$ n'est pas singulier.\par
Supposons alors que $E$ possède un point singuler $P = (x_0, y_0)$ et posons
$f(x,y) = F(X/Z,Y/Z,1)$. Comme les changements de variable :
\[x = x' + x_0, \quad y = y' + y_0\]
laissent $\Delta$ inchangé (voir \cite[Chap. III,p.~44-45]{Sil}), 
on peut supposer que $P = (0,0)$. Alors on a :
\begin{align*}
a_6 &= f(0,0),\\
a_4 &= \frac{\partial f}{\partial x}(0,0) = 0,\\
a_3 &= \frac{\partial f}{\partial y}(0,0) = 0,\\
\end{align*}
et en remplaçant ces valeurs dans les formules plus haut,
on obtient $\Delta = 0$.\par
Suppsosons alors que $E$ est lisse. Pour simplifier la preuve on se contentera
de traiter uniquement les cas en caractéristique différente de $2$ et $3$ et
considérons l'équation de Weierstrass de la forme :
\[E : y^2 = x^3 + ax + b\]
La courbe $E$ est singulière si et seulement si il existe un point $P = 
(x_0,y_0)$ tel que :
\[2y_0 = 3x^2 + a = 0\]
Autrement dit, les points singuliers sont de la forme $(x_0, 0)$ où $x_0$ est
racine double du polynôme $x^3 + ax + b$. Or ce polynôme n'a de racine double si
et seulement si sont discriminant, égal à $4a^3 - 27b^2 = \Delta/16$, est nul.
Ainsi, si $E$ n'est pas singulière, $\Delta\neq0$.\\
\end{proof}

\begin{prop}
Soit $j_0\in\bar{K}$, alors il existe une courbe elliptique définie au-dessus de
$K(j_0)$ telle que son $j$-invariant est $j_0$.
\end{prop}
\begin{proof}
Supposons que $j \neq 0, 1728$ et considérons la courbe :
\[E : y^2 + xy = x^3 - \frac{36}{j_0 - 1728}x - \frac{1}{j_0 - 1728}.\]
De longs calculs permettent alors d'obtenir :
\[\Delta = \frac{j_0^3}{(j_0 - 1728)^3} \etmath j = j_0\]
On a donc les courbes qu'il faut pour toutes caractéristiques si $j \neq 0,
1728$.\par
Pour finir, les courbes suivantes complètent la liste :
\begin{align*}
E &: y^2 + y = x^3, \quad \Delta = -27, \quad j = 0,\\
E &: y^2 = x^3 + x, \quad \Delta = -64, \quad j= 1728.
\end{align*}
\end{proof}

Une notion, sinon la notion, importante de courbes elliptiques et le fait que
l'ensemble des points d'une courbe elliptique muni de la loi de composition des
cordes tangentes forme un groupe commutatif avec pour élément neutre le point 
$\EO$.\par
On se contentera d'énoncer les résultats fondamentaux de la loi de groupe sans 
nécessairement tout prouver; voir \cite[Chap.~III, p~51-53]{Sil}. Il faut garder
en mémoire que cette loi est définie pour les équations de Weierstrass et ce 
n'est pas trivial qu'elle soit encore valable pour les courbes elliptiques. 

\begin{prop}
La loi de composition des cordes tangentes a les propriétés suivantes :
\begin{enumerate}[(a)]
\item Si une droite intersecte les points, non nécessairement disctints, $P, Q,
R$, alors :
\[(P\oplus Q) \oplus P = \EO.\]

\item $P\oplus\EO=P$ pour tout $P\in E$.

\item $P\oplus Q = Q \oplus P$ pour tout $P, Q\in E$.

\item Soit $P\in E$, il existe un point de $E$ noté $\ominus P$, tel que :
\[P\oplus(\ominus P) = \EO.\]

\item Soit $P, Q, R \in E$, alors :
\[(P\oplus Q) \oplus R = P\oplus (Q\oplus R).\]

\item Supposons que $E$ est défini sur $K$, alors :
\[E(K) = \lbrace{(x,y)\in K^2 : y^2 = x^3 + ax +
b}\rbrace\cup\lbrace{\EO}\rbrace\]
est un sous-groupe de $E$.
\end{enumerate}
\end{prop}
%TODO : foutre les formulaire, peut-être parler des formes de Montgomery, parler
%des points de torsions rapidement
\subsection{Corps finis}
%TODO: Nombres de points, théorème de Hasse, Frobenius, trace, points de 
%torsions


\subsection{Courbes supersingulières}
%TODO: Définitions Twists quadratiques, courbes $y^2 = x^3 + x$ et 
%$y^2 = x^3 + 1$ et leurs tordues, pour préparer le terrain pour la partie 
%elliptique de Rains

\section{Complexité \& notations}
%TODO: Définir le $O$ et $O^{\sim}$; définir les notations etc. !!p. 720 de
%Modern Computer Algebra!!
Cette courte section va nous permettre de définir quelques notions de complexité
et surtout d'introduire les notations qu'on utilisera pour les futurs analyses
de complexités des algorithmes étudiées.

\part{Isomorphismes de corps finis}
\label{deux}
%TODO: Présentation du problème, énonciation des difficultés etc.
On a vu, théorème \ref{th:isomGF}, qu'un corps fini était unique à isomorphisme
près. De façons équivalente, deux corps finis de même cardinal sont donc 
isomorphes. Malheureusement, la preuve n'est pas constructive et ne permet donc 
pas d'expliciter ou de construire un tel isomorphisme.\par
On a également vu, théorème \ref{th:elemprim}, que tout corps fini, excépté le 
corps premier, s'obtient par adjonction de la racine d'un polynôme irréductible
du degré de l'extension. Il y a donc autant de corps finis "différents" de même
cardinal que de polynômes irréductibles de bon degré sur le corps premier (ou 
$\GF{q}$ si on ne part pas du corps premier). Le problème est donc comment 
passer de l'un à l'autre.\par

\section{Algorithme de Pinch}
%TODO: Énonciation et descriptions de la méthode de Pinch, justification des 
%résultats, explication des inconvénients et autre.
Soit $f$ et $g$ deux polynômes irréductibles de degré $n$ sur $\GF{q}$, 
$k_1 := \GF{q}[X]/(f)$ et $k_2 := \GF{q}[Y]/(g)$. On veut trouver un 
isomorphisme qui relie ces deux corps de cardinal $q^n$. Une première approche 
naïve est de chercher une racine de $f$ dans $k_2$ et d'envoyer $x = \bar{X}$
sur cette même racine. Cependant, cela implique de factoriser $f$ ce qui prend
beaucoup trop de temps. Pour être plus précis, d'après le théorème 14.14 de
\cite[p.~390]{GaGe}, factoriser complètement un polynôme de degré $n$ sur 
$\GF{q}$ prend un $O(nM(n)\textup{log}(qn))$. Et d'après le corollaire 14.16
du même ouvrage \cite[p.~392]{GaGe}, trouver une racine se fait en
$O(M(n)\textup{log\;}n\textup{log}(qn))$. Sachant qu'on cherche une racine sur 
une extension de degré $n$, la complexité devient 
$O(E(n)\textup{log\;}n\textup{log}(nq^n))$.\par
 Une méthode pour faire plus rapide est dûe à Pinch, l'une de ses variantes 
 utililses les racines de l'unité. C'est 
celle qu'on va décrire dans la sous-section suivante. On pourra noter aussi une 
autre  méthode établie par Allombert \cite{All} qui consiste essentiellement en 
de  l'algèbre linéaire, donc une complexité en $O(n^{\omega})$ avec 
$2<\omega<3$.

\subsection{Principe}
%TODO: Explication de la méthode et illustration avec l'exempe de l'article, 
%peut-être ?
Globalement, la méthode de Pinch consiste au départ à choisir un groupe $\Gamma$
défini par des relations algébriques sur $\GF{q}$. Puis, on choisit un élément 
$\gamma\in\Gamma$ tel que :
\vspace{0.3cm}
\begin{itemize}
\item il soit d'ordre $m$ "petit" et défini sur $\GF{q^n}$,
\item il engendre exactement $\GF{q^n}$; et non un de ses sous-corps.
\end{itemize}
\vspace{0.3cm}
Par suite, on exprime alors $\gamma$ en tant que polynôme en $x$ et en
polynôme en $y$ et, pour finir, on utilise ces deux expressions pour exprimer 
$x$ en fonction de $y$, afin d'obtenir l'isomorphisme voulu.\par
Dans la méthode qu'on va décrire, dite cyclotomique, on utilisera $\Gamma =
k_1^{\times}$ et $\gamma$ sera une racine primitive \nroot{m} de l'unité.

\vspace{0.3cm}

Pour qu'une racine \nroot{m} de l'unité soit dans un corps fini, il faut et il 
suffit que $m$ divise l'ordre du groupe multiplicatif. Si c'est le cas, il y a
parmis ces racines, des racines primitives \nroot{m}. Imaginons qu'on ait un 
élément $\zeta_m$ dans $k_1^{\times}$ qui soit une racine primivite \nroot{m} de
l'unité. Un morphisme $\phi$ entre $k_1$ et $k_2$ envoie alors 
$\zeta_m$ sur racine primitive \nroot{m} de l'unité dans $k_2$; le problème est
de trouver laquelle. On choisit $\zeta'_m$ une racine primitive \nroot{m} de 
l'unité dans $k_2$ alors $\zeta_m$ s'enverra sur une puissance $\zeta'_m$
puisque le groupe mulitplicatif d'un corps est cyclique. On définit alors les 
applications $\phi_s : k_1 \to k_2$ pour $0 < s < m$, telles que :
\[\phi(\zeta_m) = (\zeta'_m)^s\]
Le but va être de déterminer quel $\phi_s$ définit un isomorphisme.\par
Pour cela, on va utiliser un peu d'algèbre linéaire. De façon générale, si on a
deux bases $(x_i)_{i\in I}$, $(y_i)_{i\in I}$ de $k_1$, $k_2$ respectivement,
telles qu'il existe un isomorphisme $\phi$ qui envoie $x_i$ sur $y_i$ pour tout
$i\in I = \lbrace{1,\dots,n}\rbrace$, on a alors :
\begin{align*}
x_i &= \sum_{0\leq j < n}{a_{ij}x^j}\\
y_i &= \sum_{0\leq k < n}{b_{ik}y^k}\\
\phi(x^j) &= \sum_{0\leq k < n}{c_{jk}y^k}\\
\end{align*}
avec comme objectif de déterminer la valeur de $c_{jk}$. Or pour tout $i\in I$, 
on a : 
\begin{align*}
\sum_{0\leq j < n}{a_{ij}\phi{(x^j)}}&= \sum_{0\leq j < n}{a_{ij}c_{jk}y^k}\\
&= \sum_{0\leq k < n}{b_{ik}y^k}\\
\end{align*}
Ou de façon équivalente, si on pose $A$, $B$ et $C$ les matrices avec comme 
coefficients respectifs $a_{ij}$, $b_{ik}$ et $c_{kj}$, on a :
\[AC = B\]\par

Reste alors à déterminer les bases $(x_i)$ et $(y_i)$. Par exemple, si on trouve
deux racines $\nu_1$ et $\nu_2$ de $f$ et $g$ respectivement, telles que 
$\phi(\nu_1) = \nu_2$, on pourra prendre comme bases $x_i = \nu_1^i$ et 
$y_i = \nu_2^i$.\par
Mais comme les racines $\zeta_m$ et $\zeta'_m$ sont censées être choisies de 
telle sorte qu'elles n'engendrent aucun des sous-corps de $\GF{q^n}$ cela veut 
dire, en particulier, qu'elles sont racines de $f$ et $g$. On peut donc poser 
$x_i = \zeta_m^i$ et $y_i = (\zeta'_m)^{si}$ selon l'application $\phi_s$ qu'on 
souhaite tester. De $C$ définie comme ci-dessus avec les nouvelles bases,
on déduit l'image de $x$, dont les coefficients dans la base monomial $y$ sont 
sur la deuxième ligne. Alors, si $\phi_s(x)$ est racine de $f$ sur $k_2$,
l'application $\phi_s$ est effectivement un isomorphisme.\par
\vspace{0.3cm}
La méthode peut se résumer ainsi :
\vspace{0.3cm}
\begin{enumerate}[1.]
\item Trouver un entier $m$ "petit" divisant $q^n - 1$ et tel que 
$(q,m) = 1$. 

\item Déterminer une racine \nroot{m} primitive $\zeta_m$ dans $k_1$ et
$\zeta'_m$ dans $k_2$.

\item  Déterminer pour quel $s\in \lbrace{1,\dots,m-1}\rbrace$, l'application
$\phi_s$ telle que $\phi_s(\zeta_m) = (\zeta'_m)^s$, est un isomorphisme.
\end{enumerate}
\vspace{0.3cm}
Dans le premier point, la deuxième condition est là pour assurer que, théorème
\ref{th:polycycldecomp}, une racine \nroot{m} primitive de l'unité 
n'engendre aucun sous-corps de $\GF{q^n}$. Pour le deuxième point, il suffit de 
prendre au hasard un $z\in k_1^{\times}$ et de l'élever à la puissance 
$(q^n - 1)/m$ jusqu'à tomber sur un élément d'ordre $m$, la probabalité 
que cela arrive est de $\varphi(m)/m$; il y a au plus $m$ éléments solutions de
$X^m - 1$ dans $\GF{q^n}$ et seulement $\varphi(m)$ éléments d'ordre exactement
$m$. Si $m$ est premier, ce qui sera bien souvent le cas, la chance de tomber
sur un élément d'ordre exactement $m$ est quasiment assurée par cette méthode.

\begin{ex}
Illustrons la méthode par un exemple directement tiré du papier de Pinch
(\cite{Pin}). On se place dans $\GF{11}$ et on considère les deux polynôme
irréductibles sur $\GF{11}$, $f = X^{23} + 8X^2 + X + 9$ et $g = Y^{23} + 3Y^2 +
4Y + 9$. On pose $m = 829$, on vérifiera qu'il divise bien $11^{23} - 1$ et 
qu'il est premier avec $11$; en fait, il est lui-même premier. On définit $k_1$ 
et $k_2$ comme les corps de ruptures de $f$ et $g$ respectivement.\par
Dans cet exemple, il se trouve que $x^{\tfrac{11^{23} - 1}{829}}$ et 
$y^{\tfrac{11^{23} - 1}{829}}$ sont déjà des racines primitives \nroot{m} de 
l'unité, voici leurs expressions :
\begin{align*}
\alpha(x) =&\hspace{0.15cm} 7 + 8x + 4x^2 + 4x^4 + 4x^5 + 10x^6 + 4x^7 + 3x^8 + 
5x^9\\
& + 2x^{10} + 6x^{11} + 4x^{12} + 8x^{13} + 6x^{14} + 4x^{15} + 4x^{16}\\
& + 5x^{17}  + 7x^{18} + 4x^{19} + x^{20} + 8x^{22}
\end{align*}
et
\begin{align*}
\beta(y) =&\hspace{0.15cm}1 + y + 4y^2 + 4y^3 + 9y^4 + y^5 + 6y^6 + 3y^7 + 3y^8 
+ 3y^9\\
& + 6y^{10} + 5y^{11} + 6y^{12} + 8y^{13} + y^{14} + 9y^{15} + 4y^{16}\\
& + 3y^{17} + 5y^{18} + y^{19} + 10y^{20} + 10y^{21} + 6y^{22}
\end{align*}
Le but va donc être de calculer les matrices de passage entre les bases
$(\alpha(x)^i)_{i\in I}$ et $(\beta(y)^{si})_{i\in I}$, pour $s < 829$.
Dans ce cas précis, il se trouve que la puissance $s = 14$ convient. En 
calculant la matrice de passage, on trouve alors les coefficients de l'image de 
$x$ dans $k_2$ sur la deuxième ligne de ladite matrice et l'image de $x$ est :
\begin{align*}
\phi(x) =&\hspace{0.15cm} 8 + 9y + y^2 + 2y^3 + 7y^4 + 4y^5 + 6y^6 + 10y^7 + 
9y^8\\
& + 10y^9 + 2y^{10} + 2y^{12} + 10y^{13} + 2y^{14} + 7y^{15} + y^{16}\\
& + 3y^{17} + 2y^{18} + 2y^{20} + y^{21} + y^{22}
\end{align*}
On peut alors vérifier que $f(\phi(x)) = 0$.
\end{ex}

\subsection{Limitations}
%TODO: Taille de m, choisir un élément au hasard
Le premier problème de cette méthode est qu'il n'y a pas toujours d'élément
d'ordre "petit" dans $\GF{q^n}^{\times}$. Par exemple, si on prend $p = 2$ et un
premier de Mersenne égal à $2^n - 1$, on ne trouvera pas d'élement d'ordre plus
petit que $2^n - 1$. Il y a donc un risque qu'on soit obligé de chercher des
éléments d'ordre proche de $q^n - 1$, ce qui peut poser problème. Dans l'exemple
ci-dessus, on avait :
\[q^n - 1 = 11^{23} - 1 = 895430243255237372246530\]
cela peut vite devenir laborieux de travailler sur des entiers de cet ordre. Le
scénario peut être encore pire avec des valeurs de $q$ et de $n$ beaucoup plus
grandes.\par
Un autre problème est la recherche au hasard de racine primitive de l'unité. Le
côté aléatoire du choix fait qu'il n'est jamais certain de pouvoir tomber sur
deux racines qui sont dans le même facteur irréductible du polynôme
cyclotomique. Il peut arriver qu'on soit obligé de tester $m-2$ valeurs de $s$
pour pouvoir enfin tomber sur un isomorphisme.\par
Il faudrait alors trouver un moyen de pouvoir travailler avec un $m$
relativement petit même si le cardinal  du groupe des inversibles n'a pas de
facteur petit, et en même temps, pouvoir choisir des éléments uniques sous
l'action du groupe de Galois afin d'être sûr qu'un isomorphisme existe alors
entre les deux. C'est ce que fait, entre autre, l'algorithme de Rains 
cyclotomique qu'on va étudier dans la section suivante.


\section{Algorithme de Rains : méthode cyclotomique}
%TODO  Commencer par une explication concise de la méthode, ce 
%qu'elle change par rapport à Pinch.\par
%Présenter pas à pas l'article de Rains (en gros), en enlevant ce qui nous 
%concernerait moins et en rajoutant ce qu'il n'a pas mis (tous les détails 
%sur l'algorithme convert et sa complexité).
On va maintenant parler de la méthode Rains dites cyclotomique pour calculer des
isomorphismes de corps finis. Elle fera usage d'éléments normaux, pour pouvoir
travailler sur des petites extensions, et de périodes de Gauss, afin d'avoir des
éléments unique sous l'action du groupe de Galois de l'extension.

\subsection{Principe}
%TODO: Passer par une petite extension pour réduire la taille de m, utilisation
%des éléments normaux, utilisation des périodes de Gauss.
La première limitation de la méthode de Pinch est la taille du facteur
$m$, Rains propose alors dans son article (\cite{Rai}) de passer à une petite
extension afin de chercher un $m$ plus petit. En effet, imaginons qu'on ait deux
extensions de petit degré $k'_1$ et $k'_2$ de $k_1$ et $k_2$ respectivement. Si
on applique la méthode de Pinch à $k'_1$ et $k'_2$ et qu'on trouve un
isomorphisme, il suffira alors de le restreindre à $k_1$.\par
\vspace{0.3cm}
L'idée est la suivante, on a les corps $k'_1$ et $k'_2$, extensions de degré $o$
petit de $k_1$ et $k_2$ respectivement, tels que $m$ petit divise $q^{n.o} -
1$ et soit premier avec $q$. On peut alors trouver deux racines primitive
\nroot{m} de l'unité $\zeta_{m}$ et $\zeta'_m$ qui engendrent $k'_1$ et $k'_2$
respectivement et tels que $\phi(\zeta_m) = \phi(\zeta'_m)$. Alors on a : 
\[\phi(\textup{Tr\;}_{k'_1/k_1}(\zeta_{m})) =
\textup{Tr\;}_{k'_2/k_2}(\zeta'_{m})\]
De plus, les traces sont aussi des générateurs de leurs corps respectifs, on 
%TODO: tu dois encore le prouver, on peut sûrement s'inspirer le preuve pour le
%même résultats mais dans le cas d'éléments normaux.
aura besoin de ne manipuler que des matrices de taille $o$ au lieu de taille 
$n.o$.\par
%TODO : Tout ce truc sert vraiment ?! Tu peux éventuellement rajouter l'exemple
%pour illustrer le fait que ce n'est pas si simple de fabriquer des éléments
%uniques. Réponse : Oui ça sert même grave, faudra le prouve- oh wait, c'est
%fait avec le lemme sur les éléments normaux :o.
Le problème de devoir tester plusieurs candidats avant de tomber sur un 
isomorphisme est malheureusement toujours présent. C'est à ce moment qu'on va
utiliser les périodes de Gauss, introduites à la sous-section 
\ref{sec:pergauss}.

\vspace{0.3cm}

Notre but est donc de trouver des générateurs uniques. Il se trouve qu'on peut
encore améliorer le temps asymptotique en exigant que ses générateurs
satisfassent une condition plus forte; cette condition est que les générateurs
soient des éléments normaux. Imaginons qu'on est calculé deux éléments normaux
$v\in k_1$ et $w\in k_2$ tels que $\phi(v) = w$. Pour finir de résoudre
le problème, il faut déterminer les images $\phi(x^i)$ pour $0\leq i < n$. Or
comme $\phi(x^i) = \phi(x)^i$, il suffit même de déterminer uniquement l'image
de $\phi(x)$.\par
Pour cela, il faut pouvoir être capable à partir d'un élément normal $v$ et d'un
autre élément $z$, trouver les coefficients de $z$ dans la base de $v$. Puisque,
si on peut exprimer $x$ sous la forme :
\[x = \sum_{i\in I}{c_iv^{q^i}}\]
alors on aura :
\begin{align*}
\phi(x) &= \sum_{i\in I}{c_i\phi(v)^{q^i}}\\
&= \sum_{i\in I}{c_iw^{q^i}}
\end{align*}
puisque $\phi$ est un morphisme. Il se trouve alors que devoir passer de la base
normale à la base monomial en $x$ peut se faire, non pas en inversant une
matrice, mais en inversant l'élément d'un anneau quotient. C'est ce qui va être
exposé dans la sous-section \ref{sec:elemnorm}.\par
En attendant un algorithme plus détaillé, on peut résumer la recherche d'un
élément normal unique de la façon suivante :
\vspace{0.3cm}
\begin{enumerate}[1)]
\item Trouver un polynôme $h$ de degré $o$ irréductible sur $\GF{q}$ et tel que 
$(o, n) = 1$.

\item Construire l'extension de degré $o$, $k'_1 = k_1[x]/h(x)$.

\item Chercher une racine primitive \nroot{m} $\zeta_m$ dans $k'_1$.

\item Calculer la periode de Gauss de $\zeta_m$.
\end{enumerate}
\vspace{0.3cm}
Les points 1) et 2) ne seront nécessaires que lorsqu'on aura besoin d'utiliser
une extension pour trouver le paramètre $m$ assez petit. Il sera bien possible
que $o = 1$ comme on le montrera dans la sous-section \textit{(la dernière
probablement)}. Le point 3) a déjà été évoqué pour la méthode de Pinch, à ceci
pèrs qu'il faudra rajouter $m$ sans facteur carré. Et le point 4) sera traité 
dans la sous-section \ref{sec:pergauss} comme il a été mentionné plus haut. On 
répète exactement le même processus pour $k_2$, reste alors à calculer l'image 
de $x$ par l'isomorphisme reliant les deux éléments normaux. C'est ce qui va 
être étudier dans la sous-section qui suit.

\subsection{Éléments normaux}
\label{sec:elemnorm}
%TODO: Définitions et résultats utiles pour l'algorithme, justification de
%l'algorithme (peut-être à mettre dans la partie Galois de la section corps
%finis), conversion de la base polynomiale à la base normale.
On a le théorème suivant :
\begin{thm}
Soit $L/K$ une extension galoisienne, alors il existe un $x\in L$ dont l'orbite
par l'action du groupe de Galois forme une base de $L$ en tant que $K$-espace
vectoriel.
\end{thm}
\begin{proof}
%TODO : Je ne sais pas si ça vaut le coup de faire la preuve. J'aimerais bien
%mais bon... Tu as Artin et LiNi11 Th 3.72
\end{proof}
À partir de là, on assure l'existence des éléments normaux :

\begin{defn}
Soit $L/K$ une extension Galoisienne. On dit que $x\in L$ est un élément normal
si son orbite sous l'action de Galois forme une base de $L$ en tant que
$K$-espace vectoriel; ou encore si ses conjugués forment une base de $L$ en tant
que $K$-espace vectoriel.
\end{defn}
Dans le cas des corps finis, on a montré que le Frobenius engendre le groupe de
Galois de $\GF{p^n}$, donc pour un $x\in\GF{p^n}$ ses conjugués sont $x^p,
x^{p^2},\dots,x^{p^{n-1}}$. On va commencer par énoncer quelques résultats sur 
ses éléments sur les corps finis.

\begin{thm}
\label{th:nbelemnorm}
Le nombre d'élements normaux dans $\GF{p^n}$ est égal au nombre d'unité de
l'anneau $\GF{p}[X]/(X^n - 1)$.
\end{thm}
\begin{proof}
%TODO: LiNi2 3.73.. Va y'en avoir des résultats à montrer pour ça.
\end{proof}

On a alors le corollaire suivant :
\begin{cor}
Soit $L/K$ une extension de corps finis de caractéristique $p$, premier. Si
$x\in L$ est un élément normal de $L$ sur $K$, alors $\textup{Tr}_{L/K}(x)$ est 
un élément normal de $K$ sur $\GF{p}$. Si $[L:K]$ est une puissance de $p$ alors
l'inverse est aussi vrai.
\end{cor}
\begin{proof}
Puisque la trace est surjective sur les corps finis, (à prouver malgré tout ou
alors juste pour toi) alors pour tout $y\in K$ il existe un $y'\in L$ tel que $y
= \textup{Tr}_{L/K}(y')$. De là, il suffit alors de prendre l'expression de $y'$
dans la base normale engendrée par $x$ et d'y appliquer la trace; le fait
qu'elle soit linéaire montre que $\textup{Tr}_{L/K}(x)$ est aussi une base 
normale de $K$.\par
Montrons l'autre sens. Supposons que $K = \GF{q}$ et $L = \GF{q^{p^k}}$ où $q
= p^n$. D'après le théorème \ref{th:nbelemnorm}, le nombre d'éléments normaux de
$L$ est égal au nombre d'unité dans $\GF{p}[X]/(X^{p^kn} - 1)$. Comme
$(X^{p^kn} - 1 ) = (X^n - 1)^{p^k}$, alors un élément de $\GF{p}[X]/(X^{p^kn} - 
1)$ est une unité si et seulement si sa réduction modulo $(X^n - 1)$ est une
unité dans $\GF{p}[X]/(X^n - 1)$.
Comme dans ce cas, $(X^n - 1)^{p^k} \subset (X^n - 1)$, on a $\GF{p}[X]/
(X^n - 1) \subset \GF{p}[X]/(X^{p^kn} - 1)$, une unité du premier anneau sera 
donc une unité du second; et comme la réduction de $1$ modulo quoique ce soit 
est toujours $1$, une unité de $\GF{p}[X]/(X^n - 1)$ est nécessairement une 
unité de $\GF{p}[X]/(X^{p^kn} - 1)$.\par
On en déduit que $L$ a $p^k$ fois plus d'éléments normaux que $K$. De façon plus
générale, un élément de $K$ a exactement $p^k$ antécédents pour la trace, le
nombre de conjugués, puisque l'extension est galoisienne; donc $L$ a au plus
$p^k$ fois élément normaux que $L$. La borne est atteinte dans notre cas, ce qui
conclut la démonstration.\\
\end{proof}

On va maintenant développer plus en détails la partie évaluation de
l'isomorphisme. Notamment en illustrant l'isomorphisme entre les matrices
circulantes de taille $n$ sur $\GF{q}$ et l'anneau $\GF{q}[\omega]/(\omega^n -
1)$.\par
Le contexte est le suivant, imaginons qu'on soit parvenu à déterminer deux
éléments normaux $v$ et $w$ tels qu'ils soient reliés par un isomorphisme $\phi
: k_1\to k_2$. Notre but est de calculer cet isomorphisme, c'est-à-dire de
déterminer l'image de $x$ par celui-ci. Comme dit plus haut, on veut trouver les
coefficients de $x$ dans la base normale engendrée par $v$, en déduire son
image dans la base normale engendrée par $w$ et pour finir son image dans la 
base monomial engendrée par $y$. On peut résumer la situation de la manière 
suivante :
\begin{center}
\begin{tikzpicture}
\matrix(m)[matrix of math nodes,
row sep=3em, column sep=5em,
text height=2ex, text depth=2ex]
{\bigoplus\limits_{i\in I}{x^i} & \bigoplus\limits_{i\in I}{y^i}\\
\bigoplus\limits_{i\in I}{v^{p^i}} & \bigoplus\limits_{i\in I}{w^{p^i}}\\};
\path[->,font=\scriptsize,>=angle 90]
(m-1-1) edge node[auto] {$\phi$} (m-1-2)
(m-2-1) edge node[auto] {$\phi$} (m-2-2)
(m-1-1) edge node[left] {$\pi$} (m-2-1)
(m-2-2) edge node[right] {$\pi^{\prime}$} (m-1-2);
\end{tikzpicture}
\end{center}

La flêche $\pi^{\prime}$ est plus ou moins gratuite, il suffit d'exprimer $w$ en
fonction des $y^i$ mais en général les éléments sont déjà exprimés en fonction 
de la base monomiale. La seule flêche qui pose un problème est celle 
correspondant à $\pi$ qui permet d'exprimer $x$ en fonction des $v^{p^i}$. C'est
pour cette flêche là qu'on va effectuer un peu plus de travail.
Soit $z\in\GF{q^n}$ alors il existe des $c_i\in\GF{q}$ tels que :
\[z = \sum_{0 \leq i < n}{c_iv^{q^i}}\]
En utilisant le Frobenius, on obtient pour $0\leq j < n$ :
\[z^{q^{n-j}} = \sum_{0\leq i < n}{c_iv^{q^{i+n-j}}}\]
On va alors appliquer une certaine forme linéaire $\lambda$ (il s'agira de la 
trace) :
\[\lambda\left(z^{q^{n-j}}\right) = \sum_{0\leq i < n}
{c_i\lambda\left(v^{q^{i+n-j}}\right)}\]
Soit maintenant la matrice $B$ définie par :
\[b_{ij} = \lambda\left(v^{q^{i+n-j}}\right)\]
Si la matrice est inversible alors on a terminé, puisque si on note $d_{ij}$ les
coefficients de $B^{-1}$, on a bien :
\[c_i = \sum_{0\leq j < n}{d_{ij}\lambda\left(z^{q^{n-j}}\right)}\]
pour $0\leq i < n$. Reste à savoir si la matrice $B$ sera effectivement 
inversible.
On peut commencer par rappeler un résultat classique d'algèbre linéaire qui
assure que toute forme linéaire peut s'exprimer sous la forme :
\[x \mapsto \Tr{\alpha x}\]
pour un certain $\alpha\in\GF{q}$. On peut donc écrire les coefficients de $B$ 
sous la forme suivante :
\[b_{ij} = \Tr{\alpha v^{q^{i+n-j}}} = \Tr{\alpha^{q^j}v^{q^i}}\]
On va voir que dans notre cas, il nous suffira de choisir $\alpha = v$.

\begin{lem}
\label{lem:mattrinv}
Soit $\alpha,\beta\in\GF{q^n}$ et la matrice $A$ défini par les coefficients 
$a_{ij} = \Tr{\alpha^{q^i}\beta^{q^j}}$, $A$ est inversible si et seulement si 
$\alpha$ et $\beta$ sont des éléments normaux.
\end{lem}
\begin{proof}
Supposons que $\alpha$ ne soit pas un élément normal. Dans ce cas, il existe une
relation de dépendance linéaire entre les conjugués de $\alpha$, alors la 
combinaison linéaires des lignes de $A$ correspondant à cette dépendance est $0$
puisque la trace est linéaire; $A$ ne peut donc pas être inversible.\par
Inversement, supposons que $A$ ne soit pas inversible. Il existe alors une 
combinaison linéaire de ses lignes qui vaut $0$. Ou encore, il existe une 
combinaison linéaire $\alpha^{\prime}$ des conjugués $\alpha$ telle que 
$\Tr{\alpha^{\prime}\beta^{q^j}} = 0$ pour tout $j$ entre $0$ et $n-1$; 
\textit{i.e.} telle que :
\[\Tr{\alpha^{\prime}\beta + \alpha^{\prime}\beta^q + \dots + 
\alpha^{\prime}\beta^{q^{n-1}}}\]
Or la trace est surjective,
%TODO : c'est pas très clair dans ta tête alors faudrait que tu l'expliques un 
%peu
donc si $\beta$ est normal cela implique que $\alpha^{\prime} = 0$ mais alors 
$\alpha$ n'est pas un élément normal; ce qui démontre le lemme.\\
\end{proof}
Reste maintenant à expliquer d'où vient l'anneau 
$\mathbb{F}_q[\omega]/(\omega^n - 1)$. Il se trouve que la matrice $B$ a une 
propriété particulière, elle est ce qu'on appelle une matrice circulante, 
\textit{i.e.} pour obtenir la ligne suivante, il suffit de décaler tous les 
éléments vers la droite en faisant passer le dernier coefficient en première 
position. Ou plus formellement, si :
\[i_1 - j_1 \equiv i_2 - j_2 \bmod n\]
alors on a 
\[b_{i_1j_1} = b_{i_2j_2}\]
et inversement. Donc, si on considère :
\[b_{i+1j+1} = \lambda\left(v^{q^{(i+1) + n - (j+1)}}\right) = 
\lambda\left(v^{q^{i+n-j}}\right)\]
on a bien que $b_{ij} = b_{i+1j+1}$ avec $(i+1) - (j+1) \equiv i - j \bmod n$ 
comme annoncé plus haut. On a alors le théorème suivant :

\begin{thm}
\label{th:matcirciso}
Les matrices circulantes de $M_n(\GF{q})$ forment un anneau qui est isomorphe à 
$\mathbb{F}_q[\omega]/(\omega^n - 1)$ \textup{via} l'isomorphisme d'anneaux
suivant :
\begin{equation*}
\label{eq:isomconvert}
\psi : B \longmapsto \sum_{0\leq j < n}{b_{0j}\omega^j}
\end{equation*}
\end{thm}
\begin{proof}
La matrice identité étant contenue dans l'ensemble des matrices circulantes, 
montrons alors qu'il s'agit d'un sous-anneau. On voit facilement que la sommes 
de deux matrices circulantes est tourjous circulante. Pour le produit, prenons 
$A$ et $B$ deux matrices circulantes de tailles $n$, on notera $C := AB$. On a :
\[c_{ij} = \sum_{k=0}^{n-1}{a_{ik}b_{kj}}\]
Donc si on défini, en tenant compte de la structure des matrices circulantes, 
$a_{i\,-1} := a_{in-1}$ et $b_{-1j} := b_{n-1j}$, on arrive à :
\begin{align*}
c_{i+1j+1} &= \sum_{k=0}^{n-1}{a_{i+1k}b_{kj+1}}\\
&= \sum_{k=1}^{n-1}{a_{ik-1}b_{k-1j}} + a_{in-1}b_{n-1j}\\
&= \sum_{k^{\prime}=0}^{n-1}{a_{ik^{\prime}}b_{k^{\prime}j}}\\
&= c_{ij}
\end{align*}
ce qui montre que le produit est encore une matrice circulante; il s'agit donc 
bien d'un sous-anneau.\par
Montrons ensuite que l'application $\psi$ est un morphisme d'anneaux. On a bien 
évidemment $\psi(I_n) = 1$ et de manière relativement immédiate :
\begin{align*}
\psi(A + B) &= \sum_{0\leq j < n}{(a_{0j} + b_{0j})\omega^j}\\
&= \sum_{0\leq j < n}{a_{0j}\omega^j} + \sum_{0\leq j < n}{b_{0j}\omega^j}\\
&= \psi(A) + \psi(B)\\
\end{align*}
Maintenant, d'après la formule du produit de deux polynômes, on a :
\begin{align*}
\psi(A)\psi(B) &= \sum_{0\leq k < n}
{\bigg(\sum_{i+j=k}{a_{0i}b_{0j}}\bigg)\omega^k}\\
&= \sum_{0\leq k < n}{\bigg(\sum_{i\equiv k-j \bmod n}{a_{0i}b_{0(k-i \bmod n)}}
\bigg)\omega^k}\\
\end{align*}
Mais $b_{0k-i} = b_{ik}$ puisque $0 -(k-i) \equiv i-k \bmod n$; en tenant compte
du fait que $i$ parcourt tout $\zmodn{n}$, on arrive finalement à :
\[\sum_{0\leq k < n}{\bigg(\sum_{0\leq i < n}{a_{0i}b_{ik}}\bigg)\omega^k} = 
\psi(AB)\]\par
Comme les $\omega^k$ sont linéairement indépendants par définition, pour que 
l'image d'une matrice circulante soit égale à $0$, il faut que sa première 
ligne, et donc toutes les autres, soit nulle; d'où $\textup{Ker}\,\psi = 
\{0_n\}$. La surjectivité est immédiate, il suffit de prendre les coefficients 
d'un élément quelconque $u$ de $\GF{q}[\omega]/(\omega^n - 1)$ et de définir la 
matrice $A$ de coefficients $a_{ij} = u_{(i-j \bmod n)}$ pour $i,j\in\zmodn{n}$,
où les $u_i$ sont les coefficients $u$; on aura alors $\psi(A) = u$.\\
\end{proof}




\subsection{Périodes de Gauss}
%TODO: Énoncé et justifier les résultats donnés par Rains (notamment le lemme
%expliquant la forme de $\GF{q}[X]/I$ pour $I$ sous-groupe du groupe de Galois);
%peut-être à mettre directement dans la partie cyclotomique de la section
%algorithme de Rains cyclotomique.
\label{sec:pergauss}
On sait désormais le type d'élément qui nous intéresse, il faut maintenant 
trouver un moyen d'en trouver qui soient "uniques". Globalement, cela revient à
trouver des éléments qui ont la même orbite sous l'action du groupe de Galois de
l'extension, ou encore, à regrouper les différentes orbites du polynôme
cyclotomiques (puisqu'il se décompose dans $\GF{q^n}$). Une bonne façon de faire
est d'utiliser les périodes de Gauss.

\begin{defn}
Soit $n, k, m\in\NN$ tels que $\phi(m) = nk$. On pose $m = m_1m_2$ où $m_1$ est
le facteur sans carrés de $m$ et on définit, pour $l$ premier :
\[g(X) = X^{m_2}\prod_{l\mid m_2}{\sum_{1\leq i\leq
v_l(m_2)}{X^{ml^{-i}}}}\in\ZZ[X].\]
Soit $q$ une puissance d'un nombre premier avec $(q, m) = 1$,
$\zeta\in\GF{q^{nk}}$ une racine primitive \nroot{m} de l'unité et $\mathcal{K}$
un sous-groupe de $\zmodninv{m}$ d'ordre $k$. La période de Gauss de type $(n,
\mathcal{K})$ sur $\GF{q}$ est définie par :
\begin{equation}
\label{eq:gausgen}
\alpha = \sum_{a\in\mathcal{K}}{g(\zeta^a)}.
\end{equation}
\end{defn}

Si $m$ est sans facteur carré, alors $m_2 = 1$ et on a la formule suivante :
\begin{equation}
\label{eq:gaussf}
\alpha = \sum_{a\in\mathcal{K}}{\zeta^a}
\end{equation}
Le but va être de savoir à quelles conditions un élément de la forme
\ref{eq:gaussf} sera un élément normal de $\GF{q^n}$ sur $\GF{q}$. Le reste de
la sous-section sera donc consacré à prouver le théorème suivant :

\begin{thm}
\label{th:gaussnormal}
En conservant les notations précédentes, une période de Gauss de type
$(n,\mathcal{K})$ est un élément normal de $\GF{q^n}$ si et seulement si
$\langle q,\mathcal{K}\rangle = \zmodninv{m}$.
\end{thm}
On verra à la fin que ce n'est possible que si $m$ est sans facteurs carrés,
ce qui permettra alors de revenir à la forme simplifiée. On peut d'ores et déjà
montrer la nécessité de la condition, regardons le lemme suivant :

\begin{lem}
Posons $\alpha$ comme dans la formule \ref{eq:gausgen}, alors
$\alpha\in\GF{q^s}$, où $s$ est l'ordre multiplicatif de $q$ modulo
$\mathcal{K}$. En particulier, si $\langle
q,\mathcal{K}\rangle\neq\zmodninv{m}$, alors $\alpha$ n'est pas un élément
normal.
\end{lem}
\begin{proof}
En particulier, on a $q^s = 1 \bmod m$ dans $\mathcal{K}$, on en déduit  que 
$\alpha^{q^s} = \alpha$ :
\[\alpha^{q^s} = \left(\sum_{a\in\mathcal{K}}{g(\zeta^a)}\right)^{q^s} =
\sum_{a\in\mathcal{K}}{g(\zeta^{aq^s})} = \sum_{a\in\mathcal{K}}{g(\zeta^a)} =
\alpha\]
puisque le Frobenius est un morphisme, d'où la première assertion. L'ordre $s$ 
de $q$ modulo $\mathcal{K}$ est égal à $\#\langle q,\mathcal{K}\rangle/k$ 
puisque $\langle q,\mathcal{K}\rangle$ est l'union disjointe (?!) de 
$q^i\mathcal{K}$ pour $0\leq i\leq < s$. En particulier, si $\langle 
q,\mathcal{K}\rangle\neq\zmodninv{m}$, alors $s$ est plus petit que $n$, puisque
$\varphi(m) = nk$, et $\alpha$ ne peut décemment pas être un élément normal.\\
\end{proof}

Le prochain lemme nous permettra de supprimer le caractère aléatoire de la
méthode. En particulier, il assure que, sous certaines conditions, deux périodes
de Gauss $\alpha$ et $\alpha'$ sont conjugués sur $\GF{q}$. En particulier, cela
implique que $\langle{q}\rangle$ agit transitivement sur les périodes de Gauss,
donc assure qu'elles ont le même polynôme minimal et qu'elles sont toujours
reliées par un isomorphisme.

\begin{lem}
\label{lem:gaussconj}
Soit $\zeta, \zeta'\in\GF{q^{nk}}$ deux racines primitive \nroot{m} de l'unité
et $\alpha, \alpha'\in\GF{q^{n}}$ les périodes de Gauss correspondantes. Si
$\langle q,\mathcal{K}\rangle=\zmodninv{m}$, alors $\alpha$ et $\alpha'$ sont
conjuguées sur $\GF{q}$.
\end{lem}
\begin{proof}
Puisque le groupe des racines de l'unité est cyclique, alors il existe un $s$
compris entre $1$ et $m-1$, tel que $(s, m) = 1$ et $\zeta' = \zeta^s$. Puisque
$\langle q, \mathcal{K}\rangle = \zmodninv{m}$ alors il existe
$j\in\lbrace{0,\dots,n-1}\rbrace$ tel que $s\in q^j\mathcal{K}$. Donc :
\[\alpha' = \sum_{a\in\mathcal{K}}{g(\zeta^{as})} =
\sum_{a\in\mathcal{K}}{g(\zeta^{aq^j})} =
\left(\sum_{a\in\mathcal{K}}{g(\zeta^a)}\right)^{q^j} = \alpha^{q^j}\]
ce qu'il fallait démontrer.\\
\end{proof}

%TODO: Foutre le schéma qui passe de la carac 0 à la carac finie, peut-être ?
%Dépend de comment je veux aborder le problème..

\subsection{Algorithme \& analyse de complexité}
%TODO: Détailler l'algorithme de l'article/code de Luca + analyser la complexité
%de ces algos et les "prouver" (montrer que ça marche bien, cela dit c'est 
%peut-être largement faisable avant).
Maintenant qu'on a tous les résultats théoriques qu'il nous faut, on va pouvoir
énoncer et justifier les aglorithmes pour déterminer l'isomorphisme. Rappelons
le contexte et les paramètres.\par
On a $f$ et $g$ deux polynômes irréductibles de degré $n$ distincts sur
$\GF{q}$, on note :
\[k_1 := \GF{q}[X]/(f)\etmath k_2 := \GF{q}[Y]/(g)\]
avec $q = p^r$ pour $r$ un entier naturel non nul et $p$ un nombre premier
impair. On va construire, en utilisant des racines primitive \nroot{m} de 
l'unité, me sans facteurs carrés, des éléments uniques de $k_1$ et $k_2$ sous 
l'action du groupe de Galois de $\GF{q^n}$ de telle sorte qu'ils soient reliés 
par un isomorphisme. On en déduira l'image de $x$ et on aura déterminé
l'isomorphisme.

\subsubsection*{Recherche d'un élément normal unique}
On commence par chercher un entier $m$ divisant $q^n - 1$ tel que :
\vspace{0.3cm}
\begin{enumerate}[(1)]
\item $\langle{q}\rangle \times \mathcal{K} = \zmodninv{m}$ pour $\mathcal{K}$ 
un sous-groupe de $\zmodninv{m}$ d'ordre $\varphi(m)/n$,
\item $(m, q) = 1$,
\item $m$ soit sans facteurs carrés.
\end{enumerate}
\vspace{0.3cm}
La condition (1) est équivalente à demander que $\langle{q}\rangle$ soit d'ordre
$n$ dans $\zmodninv{m}$. Sous ces conditions, on peut appliquer le théorème 
\ref{th:gaussnormal} pour calculer une période de Gauss $\eta(\zeta_m) = 
\sum_{a\in\mathcal{K}}{\zeta_m^a}$ afin d'obtenir un élément unique dans chaque
extension de corps.

\begin{rem}
Il se trouve que l'algorithme ne peut pas fonctionner si $q$ est une puissance
$r$ de $p$, qui ne soit pas premier avec $n$. En effet, on a 
$\langle{q,\mathcal{K}}\rangle = \zmodninv{m}$, donc par définition $p = 
sq^a\bmod m$ pour un $s\in\mathcal{K}$ et $0 \leq a < n$, alors
$q=s^rq^{ar} \bmod m$. Or, $q$ ne peut être écrit que d'une seule façon comme 
produit d'un élément de $\mathcal{K}$ et d'une puissance
de $q$, d'où $s^r = 1$ et $q^{ar - 1} = 1 \bmod m$. Ce qui implique que $n$
divise $ar - 1$ et donc que $(n, r) = 1$.\par
Une solution possible serait de calculer un générateur
$\alpha$ de $k/\GF{p^{d/c}}$, son polynôme minimal serait alors uniquement
défini sur $\GF{p^{d/c}}$ mais se scinderait en $c$ facteurs sur $\GF{q}$, ce
qui nous obligerait à réutiliser la méthode trial-and-error de Pinch.
\end{rem}
Un autre problème est la taille du paramètre $m$, il se peut qu'il soit de
taille proche de $q^n - 1$ ce qui est beaucoup trop important. Une solution
possible proposée par Rains (\cite{Rai}) est d'utiliser une petite extension de
$\GF{q^n}$ de degré $o$. Le choix d'utiliser une extension ou non sera détaillé 
dans la partie implémentation \ref{trois}; globablement, il s'agira d'une borne
sur $m$. Commençons par décrire plus précisément l'algorithme dans le cas où 
aucune extension supplémentaire n'est utilisée.

\vspace{0.3cm}
$\bullet\; o = 1$\par
\vspace{0.2cm}
Imaginons qu'on ait trouvé un $m$ relativement petit, satisfaisant les points
(1), (2) et (3) et divisant $q^n - 1$. Dans ce cas, on peut travailler
directement dans les corps $k_1$ et $k_2$. L'algorithme est donc lui suivant :

\begin{algorithm}
\caption{Détermination d'un élément normal unique dans un corps fini}
\label{alg:gausspersansext}
\begin{algorithmic}[1]
\REQUIRE $k$, $m$ \COMMENT{$k$ un corps fini, $m$ un entier divisant $q^n - 1$
satisfaisant les conditions (1), (2) et (3)}
\ENSURE $\eta$ \COMMENT{Un élément normal unique de $k$
sous l'action de son groupe de Galois}
\bigskip
\REPEAT
    \STATE $\zeta_m = \textup{rand}(k)^{(q^n - 1)/m}$
\UNTIL{$\zeta_m^m = 1$\quad\&\quad$\forall d\mid m,\;d\neq m,\;\zeta_m^d\neq1$}
\STATE $\eta = \sum_{a\in\mathcal{K}}{\zeta_m^a}$
\RETURN $\eta$
\end{algorithmic}
\end{algorithm}
Pour trouver une racine primitive \nroot{m} de l'unité, on prend un élément au
hasard dans $\GF{q^n}$, on l'élève à la puissance $(q^n - 1)/m$ et on teste si
le résultat élevé à la puissance $m/l$ vaut $1$ ou non pour $l$ un diviseur 
premier strict de $m$. On s'attend à ce que cela réussisse après $O(1)$ essaies;
chaque essaie prenant $O(E(n)n\,\textup{log\;}q)$ opérations.\par
Maintenant, on doit calculer la période elliptiques $\eta := \eta(\zeta_m)$,
d'après ce qui précéde cela revient à faire $\varphi(m)/n$ exponentiation dans
$\GF{q^n}$ d'exposant au plus $m$, donc un $O((m\textup{log\;}m)E(n)/n)$.\par
Ainsi, le coût total de l'algorithme est un $O(E(n)(n\,\textup{log\;}q +
m\,\textup{log\;}m/n))$. Tant que $m\in o(n^2)$, on peut négliger les termes en
$m$ et se contenter d'une complexite en $O(E(n)n\,\textup{log\;}q)$.

\begin{prop}
L'algorithme \ref{alg:gausspersansext} est correct dans le sens où l'élément
produit :
\[\eta(\zeta_m) = \sum_{a\in\mathcal{K}}{\zeta_m^a}\]
est un élément normal de $\GF{q^n}$ unique sous l'action du groupe de
$\textup{Gal}(\GF{q^n}/\GF{q}) = \langle{q}\rangle$.
\end{prop}
\begin{proof}
Les conditions (1), (2) et (3) permettent d'appliquer le théorème
\ref{th:gaussnormal} qui assure que $\eta(\zeta_m)$ est un élément normal. Le
fait qu'il soit unique sous l'action du groupe de Galois est justifié par le
lemme \ref{lem:gaussconj}.
\end{proof}

\vspace{0.3cm}
$\bullet\; o > 1$
Dans le cas où on ait besoin d'utiliser une extension, il faut modifier
légèrement les conditions sur $m$ :

\vspace{0.3cm}
\begin{enumerate}[(1')]
\item $\langle{q}\rangle\times\mathcal{K} = \zmodninv{m}$ pour $\mathcal{K}$ 
un sous-groupe d'ordre $\varphi(m)/(n.o)$,

\item 
\end{enumerate}
\vspace{0.3cm}




\begin{algorithm}
\caption{Détermination d'un élément normal unique dans un corps finis
\textit{via} une extension}
\label{alg:gaussperavecext}
\begin{algorithmic}[1]
\REQUIRE $k$, $m$, $h$ \COMMENT{$k$ un corps fini, $m$ un entier divisant $q^n -
1$satisfaisant les conditions \textit{TODO}, $h$ \textit{TODO}}
\ENSURE $\eta$ \COMMENT{Un élément normal unique de $k$ sous l'action de son
groupe de Galois}
\STATE $k' := k/(h)$
\REPEAT
    \STATE $\zeta_m = \textup{rand}(k')^{(q^{n.o} - 1)/m}$
\UNTIL{$\zeta_m^m = 1$\quad\&\quad$\forall d\mid m,\;d\neq m,\;\zeta_m^d\neq1$}
\STATE $\alpha = \sum_{a\in\mathcal{K}}{\zeta_m^a}$
\STATE $\eta = \textup{Tr}_{k'/k}(\alpha)$
\RETURN $\eta$
\end{algorithmic}
\end{algorithm}

\begin{prop} 
Les algorithmes \ref{alg:gaussperavecext} sont
corrects.
\end{prop}

\subsubsection*{Détermination de l'isomorphisme}
Reste alors à déterminer exactement l'isomorphisme, c'est-à-dire calculer son
image en $x$. Comme on l'a expliqué au début de la sous-section, il suffit juste
de récupérer les coefficients de $x$ par rapport à la base normale engendrée par
$\eta(\zeta_m)$; alors il nous restera plus qu'à reporter les coefficients 
trouvés dans la base normale engendrée par $\eta(\zeta'_m)$.

\begin{algorithm}
\caption{Conversion de la base polynomiale vers la base normale}
\label{alg:convert}
\begin{algorithmic}[1]
\REQUIRE $z$, $v$, $n$, $p$, $\omega$ \COMMENT{$z\in\GF{q^n}$, $v$ élément 
normal, $n$ degré de l'extension, $p$ caractéristique, $\omega$ défini 
ci-dessus}
\ENSURE $c$ \COMMENT{Tuple contenant les coefficients de $z$ dans la base 
normale engendrée par $v$}
\bigskip
\FOR{$i = 0$ \TO $n-1$}
    \STATE $B_i = \Tr{v\times v^{q^{n-(i+1)}}}$
\ENDFOR
\medskip
\STATE $I = (\sum_{i = 0}^{n-1}{B_i\cdot \omega^i})^{-1}$
\medskip
\FOR{$i = 0$ \TO $n-1$}
    \STATE $T_i = \Tr{v\times z^{q^{n-i}}}$
\ENDFOR
\medskip
\FOR{$i = 0$ \TO $n-1$}
    \STATE $c_i = \sum_{j=0}^{n-1}{I_{(j-i)\bmod n}T_j}$
\ENDFOR
\RETURN $c$

\end{algorithmic}
\end{algorithm}
On commence par élever à la puissance $q$ plusieurs polynômes sur $\GF{q}$, 
on les multiplie par un autre polynôme et enfin on calcule la trace de chacun 
de ces produits; on effectue alors $n$ fois cette opération. D'où un 
$O(nE(n)\textup{log}\;q + nE(n)) + n\textup{T}(n) = O(nE(n)\textup{log}\;q + 
nE(n))$ opérations, 
où $\textup{T}(n)$ est la complexité du calcul de la trace d'un élément
de $\GF{q^n}$ et $\textup{T}(n) \in O(E(n))$.
Par la suite, il s'agit dans un premier temps de la muliplication par un 
scalaire d'éléments de $\mathbb{F}_p[\omega]/(\omega^n - 1)$ puis de $n$ sommes 
de polynômes. Ensuite, on inverse le polynôme et on prend la trace de l'élément 
qu'il représente; ce qui nous donne $O(n + E(n).\textup{log }n)$.\par
Des lignes $5$ à $7$, il s'agit de la même opération que des lignes 1 à 3, d'où 
encore une complexité de $O(nE(n)\textup{log}\;q + nE(n))$.
Les lignes 8 à 10 consistent en $n$ sommes de $n$ multiplications dans le corps 
de base $\GF{q}$, \textit{i.e.} un grand $O(n)$.
Au final, la complexité de l'algorithme est égale à:
\[O(nE(n) + nE(n)\textup{log }q +  E(n)\textup{log }n + n\textup{T}(n) + n)\] 
ou de manière plus simple, elle est en $O(nE(n)\textup{log}\;q)$ qui domine le 
reste.

\begin{prop}
L'algorithme \ref{alg:convert} est correct.
\end{prop}
\begin{proof}
D'après le point précédent, on a :
\[\textup{Tr}\left(z^{p^{n-j}}\right) = \sum_{0\leq i < n}{c_i\textup{Tr}
\left(v^{p^{i+n-j}}\right)}\]
et la matrice $B$ définie par :
\[b_{ij} = \Tr{\alpha v^{p^{i+n-j}}} = \Tr{\alpha^{p^j}v^{p^i}}\]
On sait qu'elle est inversible grâce au lemme \ref{lem:mattrinv}, donc si on
note $d_{ij}$ les coefficients de la matrice $B^{-1}$, on a bien :
\[c_i = \sum_{0\leq j < n}{d_{ij}\textup{Tr}\left(z^{p^{n-j}}\right)}\]
Le théorème \ref{th:matcirciso} permet de conclure.\\
\end{proof}


\section{Algorithme de Rains : méthode elliptique}
%TODO: Résumer/Détailler le papier de Luca, rajouter les résultats sur
%les twists, les périodes, détaillés l'algorithme et sa complexité, le prouver 
%(probablement fait avec le papier de Luca et Mihailescu pour les périodes ?)
%, détailler les différentes étapes, les paramètres, la façon de trouver m 
%etc.
\subsection{Principe}
%TODO: Choix d'un point d'ordre m sur une bonne courbe elliptique engendre 
%l'extension, utilisation des périodes elliptiques afin d'avoir un élément 
%stable par l'action du groupe de Galois (Énoncer les lemmes etc.)

\subsection{Périodes elliptiques}
%TODO: Prouver et définir les périodes elliptiques (peut-être que ce serait à
%faire directement dans la partie algorithme elliptique de Rains ?)


\subsection{Choix des paramètres}
%TODO: Justifier le choix de m par rapport à n et q, tout ce qui concerne la 
%valeur propre d'ordre n et donc le choix de traces qui s'en suit, expliquer 
%comment on pick les courbes (à ce moment là on pourra aussi justifier avec le 
%résultat sur les tordues).

\subsection{Algorithme \& Analyse de complexité}
%TODO: Décrire l'algorithme et donner sa complexité.

\part{Implémentation}
\label{trois}
%TODO: Comparer les deux méthodes ou même avec l'implémentation de base de SAGE 
%(retrouver une simple racine). Détailler les limitations dû à SAGE, peut-être
%prendre une partie de la conclusion en expliquant ce qui pourrait être changer.

\section{Trouver $m$}
%TODO: Des détails sur m et comment on le trouve (forme an + 1, premier ou
%puissance d'un nombre premier; pour la méthode cyclotomique et elliptique

\section{Résultats numériques}
%TODO: Je ne sais pas vraiment quoi mettre, je pourrais juste montrer
%l'évolution de certains paramètres, comparer la progression de certaines
%valeurs comme le nombre de trace et le nombre de courbe ou je sais plus
%vraiment quoi... Ah et faire une comparaison avec la méthode elliptique aussi,
%éventuellement montrer que lorsqu'on a besoin d'une extension la méthode
%elliptique est bien chouette; à l'inverse, la méthode cyclotomique gère bien
%sans extension

\section{Comparaison avec d'autres méthodes}
%TODO: Comparaison avec la méthode naïve, Allombert, autre ?

\part{Conclusion}
\label{quatre}
%TODO: Expliquer ce qu'il peut rester à faire (cas général, car(K) = 2 et n 
%composé, couper et appliquer l'une des deux méthodes selon la nature du m ou 
%s'il y a besoin de calculer une extension, etc.), où est-ce qu'on peut 
%récupérer des performances (produit scalaire).

\addcontentsline{toc}{part}{Références}
\begin{thebibliography}{LC}

\bibitem{Rai} \emph{Efficient Computation of Isomorphism Between Finite Fields},
\bsc{Eric M. Rains}, 1996.

\bibitem{Pin} \emph{Recognising Elements of Finite Fields}, \bsc{Richard G.E. 
Pinch}, Cryptography and coding II, p. 193-197, Oxford University Press, 1992.

\bibitem{LiNi1} \emph{Finite Fields}, \bsc{Rudolf Lidl} \& 
\bsc{Harald Niederreiter}, Encyclopedia of mathematics and its applications vol.
20, Cambridge, 1983.

\bibitem{LiNi2} \emph{Introduction to Finite Fields and their Applications},
\bsc{Rudolf Lidl} \& \bsc{Harald Niederreiter}, Cambridge University Press,
1986.

\bibitem{Per} \emph{Cours d'algèbre}, \bsc{Daniel Perrin}, ellipses, 1996.

\bibitem{Sil} \emph{The Arithmetic of Elliptic Curves}, 
\bsc{Joseph H. Silverman}, Graduate texts in mathematics, Springer, 2nd ed. 
2009.

\bibitem{Was2} \emph{Elliptic Curves, Number Theory and Cryptography},
\bsc{Lawrence C. Washington}

\bibitem{Sam} \emph{Théorie algébrique des nombres}, \bsc{Pierre Samuel}, 
Hermann, 1971.

\bibitem{Kim} \emph{Number Theory II}, \bsc{Kimbal Martin}, University of
Oklahoma, 2011, \bsc{url :} \url{http://www2.math.ou.edu/~kmartin/ntii/ntii.pdf}

\bibitem{Lan} \emph{Algebraic Number Theory}, \bsc{Serge Lang}, Graduate Texts
in Mathematics, 2nd ed., Springer, 2000.

\bibitem{Nek} \emph{Théorie de Galois}, \bsc{Jan Nekov\'a\v{r}}, Université 
Pierre et Marie Curie, 2003, \bsc{url :} 
\url{http://www.math.jussieu.fr/~nekovar/co/ln/gal/g.pdf}.

\bibitem{Esc} \emph{Théorie de Galois}, \bsc{Jean-Pierre Escofier}, Dunod, 2nde
éd., 2000.

\bibitem{Pol} \emph{Algèbre et théorie de Galois}, \bsc{Patrick Polo}, 
Université Pierre et Marie Curie, 2007, \bsc{url :} 
\url{http://www.math.jussieu.fr/~polo/M1/}.

\bibitem{Was1} \emph{Introduction to Cyclotomic Fields}, \bsc{Lawrence C. 
Washington}, Graduate texts in mathematics, Springer-Verlag, 1982.

\bibitem{MuPa} \emph{Handbook of Finite Fields}, \bsc{Gary L. Mullen} \& 
\bsc{Daniel Panario}, Discrete mathematics and its applications, Series Editor 
Kenneth H. Rosen, CRC Press.

\bibitem{GaGe} \emph{Modern Computer Algebra}, \bsc{Joachim von zur Gathen} \&
\bsc{Jürgen Gerhard}, 3rd edition, Cambridge

\bibitem{FeGaSho} \emph{Normal Basis \textup{via} General Gauss Periods},
\bsc{Sandra Feisel}, \bsc{Joachim von zur Gathen} \& \bsc{M. Amin Shokrollahi}, 
Mathematics of computation, Volume 68, Number 225, January 1999, p. 271-290

\bibitem{MiMoScho} \emph{Computing the Eigenvalue in the Schoof-Elkies-Atkin
Algorithm using Abelian Lifts}, \bsc{P. Mih\u{a}ilescu}, \bsc{F. Morain} \&
\bsc{É. Schost}, 2007, \bsc{url :}
\url{http://hal.inria.fr/LIX/inria-00130142/en/}

\bibitem{All} \emph{Explicit Computation of Isomorphisms between Finite
Fields}, \bsc{Bill Allombert}, Elsevier Science, \bsc{url :}
\url{http://www.sciencedirect.com/science/article/pii/S1071579701903442}

\bibitem{Dat} \emph{Théorie algébrique des nombres}, \bsc{Jean-François Dat},
Université Pierre et Marie Curie, 2013, \bsc{url}
\url{https://www.imj-prg.fr/~jean-francois.dat/enseignement/TNM2/TNM2.pdf}

\end{thebibliography}
\end{document}

\documentclass[a4paper]{article} % papier A4
\usepackage[utf8]{inputenc}      % accents dans le source
\usepackage[T1]{fontenc}         % accents dans le pdf
\usepackage{textcomp}            % symboles complémentaires (euro)
\usepackage[frenchb]{babel}      % titres en français
\usepackage{amsmath}
\usepackage{amsthm}
\usepackage{amssymb}
\usepackage[colorlinks=false]{hyperref}
\usepackage{enumerate}
\usepackage{algorithmic}
\usepackage{pgf}
\usepackage{tikz}
\usepackage{tikz-cd}
\usetikzlibrary{matrix,arrows,decorations.pathmorphing}
\numberwithin{equation}{section}
\newcommand\nroot[1]{\textit{#1}\up{\textit{ième}}}
\newcommand\zmodn[1]{\mathbb{Z}/#1\mathbb{Z}}
\newcommand\zmodninv[1]{(\mathbb{Z}/#1\mathbb{Z})^{\times}}
\newcommand\GF[1]{\mathbb{F}_{#1}}
\newcommand\Irr[2]{\textup{Irr}_{#1}(#2)}
\renewcommand{\algorithmicrequire}{\textbf{Input:}}
\renewcommand{\algorithmicensure}{\textbf{Ouput:}}
\newcommand\Tr[1]{\textup{Tr}\left(#1\right)}
\begin{document}
\newtheorem{thm}{Thèorème}[section]
\newtheorem{lem}[thm]{Lemme}
\newtheorem{cor}{Corollaire}[thm]
\newtheorem{prop}[thm]{Proposition}
\theoremstyle{definition}
\newtheorem{defn}[thm]{Définition}
\newtheorem*{ex}{Exemple}
\theoremstyle{remark}
\newtheorem{rem}{Remarque}[thm]
\section{Introduction}
%Soit $\mathbb{F}_q$ un corps finis de caractéristique $p$, on considére alors sa clôture algébrique $\overline{\mathbb{F}}_q$. Si théoriquement la structure de cet objet est relativement bien connu, lorsqu'il s'agit de faire des calculs de manière effective dans celui-ci les performances ne sont pas toujours aussi optimales qu'on le voudrait.\par
%Cependant, une solution théorique dû à DeFeo, Schost \& Doliskani (rajouter la référence du papier) donne de bons espoirs quant à la possibilité d'une implémentation avec une complexité relativement raisonnable. Le but du présent travail, en plus d'exposer de manière plus ou moins consise la théorie, est d'implémenter cette méthode sous SAGE (et/ou en C).\par
%On commencera par exposer et rappeler plusieurs résultats théoriques qui serviront de bases aux différentes méthodes exposées. Puis on exposera le problème de trouver un isomorphisme explicite entre deux corps finis de même cardinal pour enfin finir sur une description et une implémentation de la méthode De Feo, Doliskani \& Schost, plus éventuellement des graphiques de performances pour comparer les différentes méthodes.
\section{Rappels et résultats préliminaires}
Dans cette section on rappellera différents résultats et notions utiles à la compréhensions du sujet. À chaque sous-section on commencera par donner les définitions et quelques résultats fondamentaux pour finir sur des résultats beaucoup plus spécifiques au contexte de ce mémoire.
\subsection{Extensions de corps}
On ne considérera uniquement des corps commutatifs.
\subsubsection{Définitions}
%TODO : Donner les définitions de bases, extension algébrique, fini, etc.

Soit $K$ et $L$ deux corps de caratéristiques nulle ou égale à $p$ premier. S'il existe un morphisme $\varphi : K \to L$ celui-ci est forcément injectif. En effet, $\textup{Ker}\,\varphi$ est un idéal propre de $K$ (puisque $\varphi(1) = 1$), ainsi comme un corps n'a que deux idéaux, $(0)$ et $(1)$, on a bien $\textup{Ker}\,\varphi = \lbrace0\rbrace$.

\begin{defn}
On dit que $K$ est une extension de corps de $k$ s'il existe un morphisme de corps $\varphi : k \to K$. Ou de façon équivalente, si $k \subseteq K$ alors $K$ est une extension (de corps) de $k$. On note aussi $K/k$ une extension de corps.\par
Si on a $k\subseteq L \subseteq K$, alors $L/k$ est une sous-extension de $K/k$.
\end{defn}

\begin{defn}
Soit $K/k$ une extension de corps et $S$ une partie de $K$. Le sous-corps $L := k(S)$ de $K$ engendré par $S$ sur $k$ est le plus petite sous-corps de $K$ contenant $S$ et $l$. Si $S = \lbrace x_1,\dots,x_n \rbrace$ est fini, alors on note $L = K(x_1,\dots,x_n)$, on dit alors que l'extension est de type fini. L'extension $L/k$ est dite monogène ou simple si elle est engendré par un seul élément.\par
Si $K/k$ est une extension on peut voir $K$ comme un $k$-espace vectoriel ou une $k$-algèbre. On appelle $[K:k] := dim_k(K)$ le degré de l'extension. On dit qu'une extension $K/k$ est de degré fini si $[K:k] < \infty$.\par
On appelle sous-corps premier d'un corps $K$ le corps engendré par $1_{K}$. En pratique, il s'agira soit de $\mathbb{Q}$ pour les corps de caractèristiques $0$ et $\mathbb{F}_p$ pour les corps de caractèristiques $p$, avec $p$ premier.
\end{defn}

\begin{thm}
Soit $k \subseteq L \subseteq L$ des extensions de corps de degré fini. Alors on a :
\[[K:k] = [K:L][L:k]\]
\end{thm}
\begin{proof}
Soit $[K:L] = m$ et $[L:k] = n$. On a donc que $K$ est un $L$-espace vectoriel de dimension $m$ et $L$ est un $k$-espace vectoriel de dimension $n$, le théorème revient à montrer que $K$ est un $k$-espace vectoriel de dimension $mn$. Or, d'après ce qui précède, on a $L \simeq k^n$ et $K \simeq L^m$, d'où
\[K \simeq \underbrace{L \oplus\dots\oplus L}_{m fois}\simeq\underbrace{k^n\oplus\dots\oplus k^n}_{m fois} \simeq k^{nm}\]
ce qui prouve le théorème.\\
\end{proof}

On dit qu'un élément $x\in K$ est algèbrique sur $k$ s'il existe un polynôme unitaire à coefficient dans $k$ qui annule $x$. L'ensemble des éléments algébrique d'un corps (sur un sous-corps) forme un corps \textit{(CF. Cours d'algèbre de Perrin ou autre)}. On dit qu'une extension $K/k$ est algébrique si tous les éléments de $K$ sont algébriques sur $k$.

\begin{prop}
Toute extension de degré fini est algébrique et de type fini.
\end{prop}
\begin{proof}
Si l'extension $K/k$ est de degré fini alors elle admet une base finie $(\alpha_1,\dots,\alpha_n)$ en tant que $k$-espace vectoriel, on a alors $K = k(\alpha_1,\dots,\alpha_n)$. Comme $[K:k] = n < \infty$, si $\alpha\in K$ alors $1, \alpha, \dots, \alpha^n$ satisfont une relation de dépendance linéaire, \textit{i.e.} il existe $a_0, \dots, a_n$ dans $k$ tels que :
\[a_0 + a_1.\alpha + \dots + a_n.\alpha^n = 0\]
puisque $K$ est un $k$-espace vectoriel. D'où tout $\alpha$ est algébrique sur $k$.\\
\end{proof}

\begin{prop}
Soit $\alpha\in L$ un élément algébrique sur $k$ et $f = X^d + a_1X^{d-1} + \dots + a_d$ son polynôme minimal sur $k$. Si on pose $d := \textup{deg }f$ alors on a :
\begin{enumerate}[(i)]
\item Les éléments $1, \alpha,\dots,\alpha^{d-1}$ forment une base de $k[\alpha]$ en tant qu'espace vectoriel.
\item $k[\alpha]$ est un corps, on a alors $k[\alpha] = k(\alpha)$.
\item On a $[k(\alpha):k] = d$.
\end{enumerate}
\end{prop}
\begin{proof}
(i) Il suffit de multiplier l'identité $\alpha^d = -a_1\alpha^{d-1} - \dots - a_d$ par $\alpha^i$ pour $i\geq0$ et on montre par récurrence que :
\[\alpha^{d+i} \in k\cdot1 + \dots k\cdot\alpha^{d-1}\]
d'où $k[\alpha] = k\cdot1 + \dots k\cdot\alpha^{d-1}$. De plus, les éléments $1,\alpha,\dots,\alpha^{d-1}$ sont linéairement indépendant puisque si :
\[u_0\cdots1 + u_1\cdots\alpha + \dots u_{d-1}\cdots\alpha^{d-1} = 0\]
avec les $u_i\neq0$ cela contredirait la minimalité du degré de $f$.\par
(ii) $k[\alpha] \subset L$ est le sous-anneau d'un corps, il est donc intègre. De plus, pour tout $\beta\in k[\alpha]$ l'application $m_{\beta}$ de la multiplication par $\beta$ dans $k[\alpha]$ est $k$-linéaire et injective, puisque l'anneau est intègre. Pour finir, elle est aussi surjective puisque nous sommes en dimension finie, donc il existe $x\in k[\alpha]$ tel que $m_{\beta}(x) = \beta x = 1$.\par
(iii) Il résulte directement des deux points précédents.\\
\end{proof}

\subsubsection{Corps de rupture et corps de décomposition}

On peut aussi construire des extensions de corps \textit{via} des polynômes irréductibles. Concrètement, soit $k$ un corps et $P$ un polynôme irréductible unitaire dans $k[X]$. Comme $k[X]$ est principal (il est même euclidien) alors $(P)$ est un idéal maximal, on peut alors faire le quotient $k[X]$ par $(P)$ est on obtient un corps $K \simeq k[X]/(P)$. Dans ce cas, la classe de $X$ qu'on notera $x := \overline{X}$, est une racine du polynôme $P$ et engendre $K$ sur $k$.\par
 Pour encore aller plus loin, considérons une extension $L/k$ telle que $P$ admette une racine $\alpha\in L$. Si on note $\Irr{k}{\alpha}$ le polynôme minimal de $\alpha$, alors $\Irr{k}{\alpha}$ divise $P$, donc est égal à $\lambda P$ pour $\lambda\in k^{\times}$, puisque $P$ est irréductible. Alors le morphisme $k$-algèbre $\phi : k[X] \to L$ défini par $\phi(X) = \alpha$ induit un morphisme de $k$-algèbre $\varphi : K \to L$ tel que $\varphi(x) = \alpha$. Ce morphisme est unique puisque $x$ engendre $K = k(x)$.\\\par
 On vient de montrer en particulier le théorème suivant :

\begin{thm}
Soit $k$ un corps et $P$ un polynôme irréductible dans $k[X]$. Alors $K := k[X]/(P)$ est un sur-corps de $k$ dans lequel $P$ a au moins une racine, la classe de $\overline{X} = x$. On l'appelle le corps de rupture de $P$ sur $k$.\par
\end{thm}

Une autre notion importante est celle de corps de décomposition d'un polynôme non constant sur un corps $k$. Il s'agit du sur-corps $K \supset k$ contenant toutes les racines de $P$, qui est engendré par les racines susnommées.

\label{cordéc}
\begin{thm}
Tout $P\in k[X]$ non constant admet un corps de décomposition unique à isomorphisme près.
\end{thm}
\begin{proof}
%TODO: À toi de voir si tu fais toute la démonstration ou non. C'est un peu long, mais il y a des résultats sympas.
Pour prouver l'existence du corps de décomposition, il suffit de se placer dans la clôture algèbrique de $k$, alors si on note $K_0$ le corps engendré par $\alpha_1,\dots,\alpha_n$ les racines de $P$, il est clair que $K_0$ est le corps de décomposition de $P$.\par

\end{proof}

\subsubsection{Sur les corps finis}
%TODO : Existence des extensions, unicité des corps de cardinal q = p^n, Théorème de l'élément primitif ?
On désignera par $\GF{q}$ le corps fini à $q = p^n$ éléments, de caractéristique $p$, avec $p$ premier.




\subsection{Théorie de Galois}

Dans cette section, on introduira les résultats de base de la théorie de Galois sans forcément tout démontrer. Le but étant d'exposer les résultats qui nous serviront tout au long de ce mémoire.

\subsubsection{Rapide introduction}
%TODO : Prendre une partie de l'exposé par Samuel et une partie du polycopié de P. Polo pour le groupe de Galois d'un polynôme (comment il agit sur les racines et autres; ça pourrait être intéressant)

\subsubsection{Sur les corps finis}

\subsection{Corps de nombres}

On appelle un corps de nombres une extension finie de $\mathbb{Q}$. On va alors exposer différents résultats s'appliquant à ses extensions qui nous serviront indirectement mais qui restent très puissants et vastes.

\subsubsection{Anneaux des entiers}

\subsubsection{Décomposition en idéaux premiers}



\subsection{Racines de l'unité et corps cyclotomiques}
Nous allons introduire les racines de l'unités sur un corps et les extensions/anneaux associés. Cela servira à expliquer la méthode de Rains permettant de calculer des isomorphimes entre deux corps finis.

\subsubsection{Racines de l'unités dans un corps}
\begin{defn}
On définit une racine \nroot{n} de l'unité dans un corps $k$ comme un élément $u\in k$ tel que pour $n\in\mathbb{N}$ on ait $u^n = 1$. Ce sont exactement les racines du polynômes $X^n - 1$.\par
 On note $U_n$ l'ensemble des racines \nroot{n} de l'unité dans une clôture algébrique de $k$ et $U_n(k)$ l'ensemble des racines \nroot{n} de l'unité appartenant à $k$.
\end{defn}

\begin{prop}
\label{proprootcycl}
Soit $k$ un corps, s'il contient $n$ racines \nroot{n} de l'unité alors elles forment un groupe cyclique $U_n(k)$ d'ordre $n$ isomorphe à $\zmodn{n}$.
\end{prop}
\begin{proof}
La proposition découle d'un théorème plus général disant que tout sous-groupe fini $G$ du groupe multiplicatif $k^*$ est formé de racine de l'unité et est cyclique. Lui-même est tiré de deux corollaires du théorème suivant (voir \cite[p.~26-27]{Sam} pour une démonstration détaillée) :

\begin{thm}
Soit $A$ un anneau principal, $M$ un $A$-module libre de type fini et $M'$ un sous-module de $M$. Alors :
\begin{enumerate}[1)]
\item $M'$ est un module libre de rang $\leq n$.
\item Il existe une base $(e_1,\dots,e_n)$ de $M$, un entier $q\leq n$ et des éléments $a_1,\dots,a_q$ non nuls de $A$ tels que $(a_1e_1,\dots,a_qe_q)$ soit une base $M'$ avec $a_i|a_{i+1}$ pour tout $1\leq i\leq q-1$.
\end{enumerate}
\end{thm}

\begin{cor}
Soit $E$ un $A$-module de type fini. Alors $E$ est isomorphe à un produit $(A/\mathfrak{a}_1)\times\dots\times(A/\mathfrak{a}_n)$, où les $\mathfrak{a}_i$ sont des idéaux de $A$ tels que $\mathfrak{a}_1\supset\mathfrak{a}_2\supset\dots\supset\mathfrak{a}_n$.
\end{cor}

\begin{cor}
Soit $G$ un groupe commutatif fini. Il existe $x\in G$ dont l'ordre est le $ppcm$ des ordres des éléments de $G$.
\end{cor}
\begin{proof}
On sait qu'un groupe commutatif est un $\mathbb{Z}$-module (si on le note additivement). D'après le corollaire ci-dessus, $G$ est donc isomorphique à un produit $\zmodn{a_1}\times\dots\times\zmodn{a_n}$ où $a_1|a_2|\dots|a_n$. Aucun des $a_i$ n'est nul, sinon $G$ serait infini; ce serait le produit de $\mathbb{Z}^r$ avec des groupes finis. On note $y$ la classe de $1$ dans $\zmodn{a_n}$ et on pose $x = (0,\dots,0,y)$. Alors $x$ est d'ordre $a_n$ et si on prend un élément $z\in G$, avec $z = (z_1,\dots,z_n)$, on a aussi $a_nz = 0$ car $a_i|a_n$ pour tout $1\leq i \leq n$. $x$ est donc l'élément recherché.\\
\end{proof}

De là, on en déduit qu'il existe un $z\in G$ d'ordre $n$ tel que $y^n = 1$ pour tout $y\in G$. Comme le nombre de racines de $X^n - 1$ sur un corps est au plus $n$ alors $G$ a au plus $n$ éléments. Or $z$ est d'ordre $n$ donc $G$ contient les éléments $z, z^2,\dots,z^n = 1$ qui sont distincts. Donc $G$ est formé par ces éléments et est cyclique d'ordre $n$. La proposition en découle naturellement.\\
\end{proof}

\subsubsection{Polynômes cyclotomiques}

\begin{defn}
On appelle racine primitive \nroot{n} de l'unité les racines qui engendrent $U_n$.
\end{defn}

\begin{prop}
Les racines primitives \nroot{n} de l'unité forment un sous-groupe $U^{\times}_n$ isomorphe à $(\zmodn{n})^*$. En particulier, il y en a $\varphi(n)$.
\end{prop}
\begin{proof}
D'après \ref{proprootcycl} les éléments qui engendrent $U_n$ sont exactement les éléments dont l'image engendre $\zmodn{n}$. D'où l'isomorphisme et l'ordre de $U^{\times}_n$.\\
\end{proof}

Lorsqu'on les considère sur $\mathbb{C}$, les racines \nroot{n} de l'unité sont de la forme $e^{2ik\pi/n}$ et les racines primitives définissent un polynôme qu'on appelle polynôme cyclotomique défini comme suit : 
\[\phi_n(X) = \prod_{\zeta\in U_n^{\times}}{(X - \zeta)}\]
On va montrer qu'il s'agit du polynôme minimal de ces racines sur $\mathbb{Q}$, commençons par la proposition suivante :

\begin{prop}
Le polynôme $\phi_n$ appartient à $\mathbb{Z}[X]$ et est de degré égal à $\varphi(n)$.
\end{prop}
\begin{proof}
Nous allons utiliser les deux résultats suivants; on ira voir \cite[p.~72, 80]{Per} pour des démonstrations détaillées.

\begin{lem}
Soit $n\in\mathbb{N}^*$ alors on a $n = \sum_{d|n}{\varphi(d)}$.
\end{lem}

\begin{prop}
On a $X^n - 1 = \prod_{d|n}{\phi_d(X)}$.
\end{prop}

On va alors raisonner par récurrence sur $n$. On a $\phi_1(X) = X-1 \in\mathbb{Z}[X]$. Supposons alors que le résultat soit vrai pour $d<n$. Posons $F(X) = \prod_{d|n\atop{d\neq n}}{\phi_d(X)}$, alors $F\in\mathbb{Z}[X]$ et est unitaire. Si on fait la division euclidienne de $X^n - 1$ par $F$ dans $\mathbb{2}[X]$, alors on obtient :
\[X^n - 1 = F(X)P(X) + R(X)\]
avec $P, R \in\mathbb{Z}[X]$ et $\textup{deg~}R < \textup{deg~}F$. Or, on a déjà $X^n - 1 = \phi_n(X).F(X)$ dans $\mathbb{Q}[X]$, \textit{i.e.} $F(X)(\phi_n(X) - P(X)) = R(X)$ mézalor pour une raison de degré, le degré de $R$ étant plus petit que celui de $F$, on a nécessairement $\phi_n = P \in\mathbb{Z}[X]$. Le résultat sur le degré résulte de la proposition et du lemme ci-dessus.\\
\end{proof}
\begin{thm}
Le polynôme $\phi_n$ est irréductible sur $\mathbb{Z}$ et donc sur $\mathbb{Q}$.
\end{thm}
\begin{proof}
Soit $K$ un corps de décomposition de $\phi_n$ sur $\mathbb{Q}$, $\zeta$ une racine primitive \nroot{n} de l'unité et $p$ un nombre premier ne divisant pas $n$. Les racines primitives de l'unité sont toutes de la forme $\zeta^m$ avec $(m, n) = 1$ puisqu'elles doivent être d'ordre exactement égal à $n$; cela ce soit sur leur expression exponentielle. On veut alors montrer que $\zeta$ et $\zeta^p$ ont exactement le même polynôme minimal.\par
Posons alors $f$ et $g$ les polynômes minimaux sur $\mathbb{Q}$ de $\zeta$ et $\zeta^p$ respectivement. Ils sont dans $\mathbb{Z}[X]$ puisqu'ils divisent tous les deux $\phi_n(X)$. En effet, comme $\mathbb{Z}[X]$ est factoriel, on a :
\[\phi_n = f_1^{a_1}\dots f_r^{a_r}\]
 avec $f_i$ unitaire, irreductible et dans $\mathbb{Z}[X]$; puisque $\phi_n$ est unitaire et dans $\mathbb{Z}[X]$. Alors un $f_i$ annule $\zeta$ et un $f_j$ annule $\zeta^p$; or ils sont, d'après Gauss, irréductibles et unitaires dans $\mathbb{Q}[X]$, il s'agit donc de $f$ et $g$. En particulier, $f$ et $g$ divisent $\phi_n$ dans $\mathbb{Z}[X]$.\par
Supposons alors que $f \neq g$. Comme ils sont irréductibles et disctincts, leur produit $f.g$ divise $\phi_n$. Comme $g(\zeta^p) = 0$, alors $\zeta$ est racine de $g(X^p)$, ainsi $f$ divise $g(X^p)$ dans $\mathbb{Q}[X]$ mais aussi dans $\mathbb{Z}[X]$ d'après le lemme de Gauss sur les contenus de polynômes. On va alors considérer les polynômes dans $\GF{p}[X]$, on note $\bar{f}$ et $\bar{g}$ leurs réductions modulo $p$. Par Frobenius, on obtient que $\bar{g}(X^p) = \bar{g}(X)^p$, ainsi si on prend $r$ un facteur irréductible de $\bar{f}$ et on écrit :
\[\bar{g}(X)^p = \bar{f}(X)\bar{h}(X)\]
alors $r$ divise aussi $\bar{g}$ par le lemme d'Euclide. Puique $f.g$ divise $\phi_n$ alors $\bar{f}.\bar{g}$ divise $\bar{\phi_n}$, on a $r^2$ qui divise $\bar{\phi_n}$. Ceci implique en particulier que $\bar{\phi_n}$ aurait une racine double, donc $\bar{X}^n - 1$ aussi. Or, $\bar{X}^n - 1$ est séparable sur $\GF{p}$ puisque $(X^n - 1)' = nX^{n-1}$ et $p$ ne divise pas $n$, il ne peut donc pas avoir de racines multiples, d'où une contradiction.\par
Reste alors à montrer que tous les conjugués de $\zeta$ sont exactement les autres racines primitives \nroot{n} de l'unité. Si on prend une racine primitive \nroot{n} $\zeta'$ quelconque, alors on a $\zeta' = \zeta^m$ où $m = p_1^{a_1}\dots p_l^{a_l}$ avec $p_i \nmid n$. Il suffit alors de faire de procéder par récurrence en utilisant le raisonnement ci-dessus pour s'appercevoir que $\zeta$ et $\zeta'$ ont effectivement le même polynôme minimal. De sorte que $f$ admet toutes les racines primitives comme zéros, \textit{i.e.} $\textup{deg~}f \geq \textup{deg~}\phi_n$, et $f\mid\phi_n$ alors on a bien $f = \phi_n$ irréductible sur $\mathbb{Q}$ comme on voulait. Finalement, $\phi_n$ est unitaire alors son contenu est égal à $1$ ce qui implique, avec ce qui précède, qu'il est alors aussi irréductible dans $\mathbb{Z}[X]$.\\
\end{proof}


\subsubsection{Corps cyclotomiques}

Grâce à ce résultat on peut alors introduire une notion qui nous sera utile dans la suite, celle de corps cyclotomique.

\begin{defn}
Soit $K$ un corps commutatif, on appelle corps cyclotomique $K^{(n)}$ sur $K$ le corps de décomposition de $X^n - 1$ sur $K$.
\end{defn}

\begin{rem}
D'après la section ci-dessus, on voit que dans le cas $K = \mathbb{Q}$, le corps de décomposition de $X^n - 1$ est exactement $\mathbb{Q}(\zeta_n) \simeq \mathbb{Q}[X]/\phi_n(X)$, où $\zeta_n$ est une racine primitive. C'est en particulier une extension algébrique simple.
\end{rem}

Le théorème suivant sera cruciale dans la suite du mémoire, il servira notament à justifier théoriquement la méthode de Rains.

\begin{thm}
\label{polycycldecomp}
Si $K = \GF{q}$ et $(n,q) = 1$, alors $\phi_n$ se factorise en $\varphi(n)/d$ polynômes unitaires irréductibles dans $\GF{q}[X]$ de même degré égal à $d$. $K^{(n)}$ est le corps de décomposition de n'importe lequel de ces polynôme et on a :
\[[K^{(n)}:K] = d\]
avec $d$ l'ordre multiplicatif de $q$ dans $\zmodn{n}$.
\end{thm}
\begin{proof}
 Soit $\zeta$ une racine primitive \nroot{n} de l'unité dans $\GF{q}$, alors $\zeta$ appartient a un sur-corps $\GF{q^k}$ si et seulement si $\zeta^{q^k} = \zeta$; ce qui est équivalent à $q^k \equiv 1 \bmod n$ puisque $\zeta^n = 1$ par définition. On pose alors $d$ égal au plus petit $k$ satisfaisant cette condition, dans ce cas $\zeta\in\GF{q^d}$ et ne peut pas être dans un sous-corps de celui-ci. Ainsi, le polynôme minimal de $\zeta$ est de degré $d$ et comme on a choisi $\zeta$ arbitrairement, on obtient le résultat voulu.
\end{proof}




\subsection{Courbes elliptiques}

\section{Isomorphisme de corps finis}

Si l'unicité des corps finis à isormophisme près et donc l'existence d'isomorphisme entre deux corps finis de même cardinal est connue et démontrée, trouver explicitement cette isomorphisme demande un peu plus d'effort. \par
De façon plus formelle, notons $\mathbb{F}_p$ le corps fini à $p$ éléments, $p$ premier. Soit $f$ et $g$ deux polynômes irréductibles de degré $n$ sur $\mathbb{F}_p$. On obtient alors deux corps, les quotients $k_1 := \mathbb{F}_p[X]/(f)$ et $k_2 := \mathbb{F}_p[X]/(g)$; l'objectif est d'expliciter un isomorphisme de corps entre $k_1$ et $k_2$ qui sera entièrement déterminer par l'image de $x = \overline{X}$ en fonction de $y = \overline{Y}$.

\subsection{Idée générale}
\label{ideegen}

L'idée principale, dûe à Pinch \cite{Pin}, est qu'une fois qu'on a déterminé l'image d'une base du corps $k_1$, en tant qu'espace vectoriel, par un certain isomorphisme, on peut déterminer l'image de n'importe quelle autre base en utilisant simplement de l'algèbre linéaire.\par
Plus particulièrement, soit $(x_i)_{i\in I}$ et $(y_i)_{i\in I}$, avec $I = \lbrace1,\dots,n\rbrace$, des bases de $k_1$ et $k_2$ respectivement. On suppose qu'il existe un isomorphisme $\phi : k_1 \to k_2$ avec $\phi(x_i) = y_i$ pour $i\in I$. On a alors :

\begin{align*}
x_i &= \sum_{0\leq j < n}{a_{ij}x^j}\\
y_i &= \sum_{0\leq k < n}{b_{ik}y^k}\\
\phi(x^j) &= \sum_{0\leq k < n}{c_{jk}y^k}\\
\end{align*}

avec comme objectif de déterminer la valeur de $c_{jk}$. Pour tout $i\in I$, on a : 

\begin{align*}
\sum_{0\leq j < n}{a_{ij}\phi{(x^j)}}&= \sum_{0\leq j < n}{a_{ij}c_{jk}y^k}\\
&= \sum_{0\leq k < n}{b_{ik}y^k}\\
\end{align*}

Ou de façon équivalente, si on pose $A$, $B$ et $C$ les matrices avec comme coefficients respectifs $a_{ij}$, $b_{ik}$ et $c_{kj}$, le problème revient à résoudre le système suivant :
\[AC = B\]\par

Reste alors à déterminer les bases $(x_i)$ et $(y_i)$. Par exemple, si on trouve deux racines $x$ et $y$ de $f$ et $g$ respectivement, telles que $\phi(x) = y$ alors on pourra prendre comme base $x_i = x^i$ et $y_i = y^i$.



\subsection{Méthode de Pinch}

La méthode de Pinch permet de déterminer des générateurs qui correspondent \textit{via} un isomorphisme. Elle repose essentiellement sur l'idée suivante : exprimer $x$ en un polynôme de degré au plus $n$ en $y$. On exposera deux façons d'y arriver, une première dite cyclotomique qui utilisera les racines de l'unité, et une seconde dite elliptique qui utilisera les points d'une courbe elliptique défini sur $\mathbb{F}_p$.\\\par

Plus précisément, on se donne un groupe $\Gamma$ défini par des opérations algébriques sur $\mathbb{F}_p$. On cherche alors à trouver un élément $\gamma\in \Gamma$ d'ordre $m$ petit "défini" sur $\mathbb{F}_{p^n}$. Puis on exprime $\gamma$, ou ses composantes pour la méthode elliptique, en tant que polynômes en $x$ et en $y$, on les écrira $\alpha(x)$ et $\beta(y)$. Le choix de $m$ doit être fait de tel sorte que $\gamma$, ou ses composantes, engendre $\mathbb{F}_{p^n}$ et non un de ses sous-corps. On utilisera alors les deux représentations de $\gamma$ afin d'exprimer $x$ en tant que polynôme de $y$.

\subsubsection{Méthode cyclotomique}

Dans cette méthode, on prendra pour $\Gamma$ le groupe des inversibles du corps, i.e. $\Gamma = \mathbb{F}_{p^n}^*$. Le but est de trouver une racine primitive \nroot{m} de l'unité qui engendre $\mathbb{F}_{p^n}$. Pour cela, on peut choisir un élément $\alpha\in\mathbb{F}_{p^n}$ au hasard et l'élever à la puissance $(p^n - 1)/m$, on aura alors une probabilité de $\varphi(m)/m$ \textit{(ça pourra éventuellement être prouvé)} de tomber sur une racine primitive; on posera $\gamma = \zeta_m = \alpha^{(p-1)/m}$.\par

\begin{ex}
Commençons par illustrer un exemple directement tiré de l'article \cite{Pin}. On se place dans le corps $\mathbb{F}_{11}$ et on considère les deux polynômes irréductibles suivant : $f = X^{23} + 8X^2 + X + 9$ et $g = Y^{23} + 3Y^2 + 4Y + 9$. Comme décrit ci-dessus, on prend $\Gamma = (\mathbb{F}_{11^{23}})^*$, on remarque aussi, après divisions successives, que $829$ divise l'ordre $\Gamma$.\par
Il faut alors vérifier que $829$ ne divise aucun ordre de sous-corps multiplicatifs de $\mathbb{F}_{11^{23}}$, mais le seul sous-corps est $\mathbb{F}_{11}$ lui-même, puisque 23 est premier, alors c'est vérifié dans ce cas; et donc un élément d'ordre $829$ engendrera $\mathbb{F}_{11^{23}}$ sur $\mathbb{F}_{11}$. \textit{(À prouver, probablement dans la section du dessus)}\par
L'étape suivante consiste à prendre un élément au hasard et de l'élever à la puissance $l = |\Gamma|/829$ et vérifier qu'il s'agit bien d'une racine primitive de l'unité. Il se trouve que dans ce cas $x = \overline{X}$ et $y = \overline{Y}$ sont déjà de bons candidats. On obtient alors les deux égalités suivantes :
\begin{align*}
\alpha(x) =&\hspace{0.15cm} 7 + 8x + 4x^2 + 4x^4 + 4x^5 + 10x^6 + 4x^7 + 3x^8 + 5x^9\\
& + 2x^{10} + 6x^{11} + 4x^{12} + 8x^{13} + 6x^{14} + 4x^{15} + 4x^{16}\\
& + 5x^{17}  + 7x^{18} + 4x^{19} + x^{20} + 8x^{22}
\end{align*}
et
\begin{align*}
\beta(y) =&\hspace{0.15cm}1 + y + 4y^2 + 4y^3 + 9y^4 + y^5 + 6y^6 + 3y^7 + 3y^8 + 3y^9\\
& + 6y^{10} + 5y^{11} + 6y^{12} + 8y^{13} + y^{14} + 9y^{15} + 4y^{16}\\
& + 3y^{17} + 5y^{18} + y^{19} + 10y^{20} + 10y^{21} + 6y^{22}
\end{align*}
Comme $\alpha(x)$ et $\beta(y)$ engendrent les corps $k_1$ et $k_2$ respectivement, si on a $\phi : k_1\to k_2$ un isomorphisme, alors nécessairement $\phi(\alpha(x)) = \beta(y)^s$ pour un $s < 829$ puisque les groupes multiplicatifs sont cycliques; et accessoirement, $829$ est premier donc toutes les racines de l'unité différentes de $1$ sont primitives.\par
On applique alors la méthode décrite en \ref{ideegen} ou dans \cite{Rai} avec $x_i = \alpha(x)$ et $y_i = \beta(y)^s$ avec $0 < s < 829$. On obtiendra alors à chaque $s$ une application linéaire $\phi_s$ correspondant à la matrice $C$ calculée. Pour déterminer s'il s'agit bien d'un isomorphisme ou non, il faudra alors calculer $\phi_s(x)$ et vérifier si $f(\phi_s(x)) = 0$. Puisque si c'est un isomorphisme, une racine de $f$ dans $k_1$ doit toujours l'être dans $k_2$.\par
Dans ce cas précis, il se trouve que la puissance $s = 14$ convient. En calculant la matrice de passage, on trouve alors les coefficients de l'image de $x$ dans $k_2$ sur la deuxième ligne de ladite matrice et l'image de $x$ est donc :

\begin{align*}
\phi(x) =&\hspace{0.15cm} 8 + 9y + y^2 + 2y^3 + 7y^4 + 4y^5 + 6y^6 + 10y^7 + 9y^8\\
& + 10y^9 + 2y^{10} + 2y^{12} + 10y^{13} + 2y^{14} + 7y^{15} + y^{16}\\
& + 3y^{17} + 2y^{18} + 2y^{20} + y^{21} + y^{22}
\end{align*}


\end{ex}

\begin{rem}
\label{probunicite}
Un problème apparait avec cette méthode: le $\zeta_m$ calculé n'est pas nécessairement unique dans le sens où si on a $\zeta_m\in k_1$ et $\zeta'_m\in k_2$, il peut arriver qu'il n'existe pas d'isomorphisme avec $\phi(\zeta_m) = \zeta'_m$. Cela s'illustre très bien dans l'exemple ci-dessus lorsqu'on est obligé de calculer différentes puissances de $\beta(y)$ afin de pouvoir trouver une racine de $f$ dans $k_2$; par conséquent on voit bien qu'il n'y a pas d'isomorphisme $\phi$ tels que $\phi(\alpha(x)) = \beta(y)$.\par
La solution donnée dans le même exemple pose problème. Elle oblige à calculer et à inverser autant de matrices que nécessaire afin de trouver un élément de la base du corps d'arrivé qui permet d'obtenir effectivement un isomorphisme. Pour pallier à ce problème, il faudrait alors pouvoir trouver des éléments ayant des propriétés uniques leur assurant d'avoir nécessairement un isomorphisme qui les relie. C'est ce qui sera décrit, en partie, dans la méthode de Rains un peu plus loin.
\end{rem}

%TODO : Reste à expliquer le problème du m petit (à moins que ça pourra être fait dans la description de Rains ?)
\subsubsection{Méthode elliptique}

\subsection{Méthode de Rains (Peut-être à mettre directement avec la méthode de Pinch en expliquant les améliorations apportées etc.)}
\label{sectrains}

Comme expliqué dans la remarque \ref{probunicite}, la méthode de Pinch pose un problème pour trouver des éléments uniques. Si on se base sur l'exemple donné par Rains dans \cite{Rai}, il y a plusieurs piste pour trouver de tels éléments.

\subsubsection{Élément normal}

\subsubsection{Algorithme}

\subsubsection{Structure des anneaux cyclotomiques}

Pour construire les éléments normaux nécessaires à l'application de la méthode, nous allons devoir utiliser des résultats sur les anneaux cyclotomiques. Soit $p$ un nombre premier et $m$ un entier naturel non divisible par $p$. On considère l'anneau :
\[\mathcal{R}_m :=\mathbb{F}_p[X]/\phi_m(X)\]
où $\phi_m(X)$ est le \nroot{m} polynôme cyclotomique. Il est alors intéressant de considérer l'anneau :
\[R_m := \mathbb{Z}[X]/\phi_m(X)\]
en remarquant que $\mathcal{R}_m$ n'est que la réduction modulo $p$ de $R_m$. En particulier on a le diagramme commutatif suivant :

\begin{center}
\begin{tikzpicture}
\matrix(m)[matrix of math nodes,
row sep=2.6em, column sep=3em,
text height=1.5ex, text depth=0.25ex]
{R_m & \mathcal{R}_m\\
\mathbb{Z} & \mathbb{F}_p\\};
\path[->,font=\scriptsize,>=angle 90]
(m-1-1) edge node[auto] {$\bmod p$} (m-1-2)
(m-2-1) edge node[auto] {$\bmod p$} (m-2-2);
\path[-,font=\scriptsize,>=angle 90]
(m-2-1) edge node[auto] {$\pi$} (m-1-1)
(m-2-2) edge node[right] {$\bar{\pi}$} (m-1-2);
\end{tikzpicture}
\end{center}

Commençons par énoncer le théorème suivant, il nous permettra en particulier d'utiliser des résultats de la théorie algébrique des nombres :

\begin{thm} 
\label{ramrm}
L'anneau $R_m$ défini comme ci-dessus est l'anneau des entiers du corps cyclotomique $\mathbb{Q}(\zeta_m) = \mathbb{Q}[X]/\phi_m(X)$. De plus, le premier $p$ est ramifié dans $R_m$ si et seulement si $p$ divise $m$.
\end{thm}

\begin{proof}
Pour le moment je laisse la preuve de côté. On pourrait mettre ça dans un annexe puisque ça a pas troooop de rapport avec le sujet pour l'instant. Mais je sais pas si j'aurais le temps de rédiger un tas d'annexe de partout...
\end{proof}

Puisque $\mathbb{Q}(\zeta_m)/\mathbb{Q}$ est une extension galoisienne de groupe de Galois $G := \textup{Gal}(Q(\zeta_m)/\mathbb{Q}) = (\zmodn{m})^{\times}$ \textit{(faire une référence à théorème démontré plus haut)}, alors le groupe agit par restriction sur $R_m$ et $\mathcal{R}_m$ \textit{via} les automorphismes de la forme :
\[\sigma_k(x) = x^k\]
pour $k\in G$. Par suite, si on a $I$ un sous-groupe de $G$, alors on défini $R_I$ et $\mathcal{R}_I$ les invariants sous l'action de $I$ des anneaux $R_m$ et $\mathcal{R}_m$ respectivement. On veut alors pouvoir déterminer la structure de $\mathcal{R}_I$.\par
D'après le schéma précédent, on a la situation suivante :

\begin{center}
\begin{tikzpicture}
\matrix(m)[matrix of math nodes,
row sep=2.6em, column sep=3em,
text height=1.5ex, text depth=0.25ex]
{(p) \subset \mathbb{Z}&pR_I \subset R_I & R_I/pR_I \simeq \mathcal{R}_I\\};
\path[->,font=\scriptsize,>=angle 90]
(m-1-2) edge node[auto] {$\bmod p$} (m-1-3);
\path[->,font=\scriptsize,>=angle 90]
(m-1-1) edge node[auto] {$\pi$} (m-1-2);
\end{tikzpicture}
\end{center}

D'après le théorème \textit{(référence au théorème de la décomposition en idéaux premier quand l'extension est galoisienne)}, comme $\mathbb{Q}(\zeta_m)/\mathbb{Q}$ est galoisienne, on a:
\[pR_I = \prod_{i = 1}^g{\mathfrak{B}_i^e}\]
où les $\mathfrak{B}_i$ sont premiers distincts, $g$ est le nombre d'idéaux de $R_I$ au-dessus de $(p)$. De plus, $e = 1$, puisque $pR_m$ n'est pas ramifié dans $R_m$, et à fortiori dans $R_I$ non plus, d'après le théorème \ref{ramrm} ($p$ ne divise pas $m$). On a donc d'après le théorème des restes chinois:
\[\mathcal{R}_I \simeq R_I/\prod_{i = 1}^g{\mathfrak{B}_i} \simeq \bigoplus_{i=1}^g{R_I/\mathfrak{B}_i}\]
Les $R_I/\mathfrak{B}_i$ sont des corps, comme l'anneau des entiers d'un corps de nombres est un anneau de Dedekind, et de même cardinal puisqu'ils ont tous le même degré d'inertie $f = [R_I/\mathfrak{B}_i:\mathbb{F}_p]$, d'après le théorème \textit{(remettre la même référence qu'au-dessus)}. Reste alors à déterminer le nombre $g$, pour cela on utilisera le lemme suivant :

\begin{lem}[Rains\cite{Rai}]
Soit $p$ relativement premier à $m$. Alors le nombre d'idéaux premiers de $R_I$ au-dessus de $(p)$ est égal à l'index du sous-groupe engendré par $p$ dans $G/I$.
\end{lem}
\begin{proof}
On commence par le cas $I = (1)$. Dans ce cas $R_I = R_m$ et soit $\mathfrak{B}$ un premier au-dessus de $(p)$, alors $R_m/\mathfrak{B}$ est un sur-corps de $\mathbb{F}_p$. Le degré d'inertie de $\mathfrak{B}$ par rapport à $(p)$ est donc $f = [R_m/\mathfrak{B}:\mathbb{F}_p]$. Aussi, d'après le théorème \ref{polycycldecomp}, il y a $g = \varphi(m)/f$ idéaux $\mathfrak{B}$ de $R_m$ qui sont au-dessus de $(p)$; ils correspondent aux facteurs distincts de $\phi_m$ dans $\mathbb{F}_p$ \textit{via} l'isomorphisme d'anneaux :
\[R_m/pR_m \simeq \mathbb{F}_p[X]/\phi_m(X)\]\par
 Maintenant, comme le groupe de Galois du corps résiduel $R_m/\mathfrak{B}$ n'est composé que d'éléments de la forme $\sigma_{p^k}$ %TODO : t'as toujours ça à prouver, mec
, on a bien que son degré par rapport à $\mathbb{F}_p$ est égal à l'ordre du sous-groupe engendré par $p$ dans $G$, d'où le résultat pour le cas $I = (1)$ puisque $\varphi(m) = |G| = |(\zmodn{m})^{\times}|$, $f = |\langle p\rangle|$ et donc :
\[g = |G|/|\langle p\rangle| = [G:\langle p\rangle]\]
d'après le résultat Lagrange sur l'ordre des groupes finis.\par
Considérons maintenant le cas général où $I$ est un sous-groupe de $G$. Alors pour n'importe quel idéal premier $\mathfrak{B}$ de $R_m$, l'idéal premier en-dessous de $\mathfrak{B}$ en tant qu'idéal de $R_I$ est donné par :
\[\prod_{i\in(I_{\mathfrak{B}}I)/I_{\mathfrak{B}}}{\sigma_i(\mathfrak{B})}\]
%TODO : Mieux détailler la preuve, histoire de montrer que t'as vraiment compris. Ou encore mieux, comprendre la preuve !
où $I_{\mathfrak{B}}$ est le sous-groupe d'isotropie de $\mathfrak{B}$; ou encore le sous-groupe de $G$ tel que pour $\sigma\in I_{\mathfrak{B}}$, $\sigma(\mathfrak{B}) = \mathfrak{B}$. Il s'agit là en fait du produit des conjugués de $\mathfrak{B}$ \textit{via} l'action du groupe de Galois. On peut aussi voir ce produit comme la norme de l'idéal $\mathfrak{B}$ de $R_m$ sur $\mathbb{Z}$, celle définie comme suit : 
\[N_{L/K}(\mathfrak{B}) = \prod_{\sigma\in\textup{Gal}(L/K)}{\sigma(\mathfrak{B})\cap\mathcal{O}_K}\]
avec ici $L = \mathbb{Q}(\zeta_m)$ et $K = \mathbb{Q}$.
\end{proof}

\begin{cor}
\label{coreng}
L'anneau $\mathcal{R}_I$ se factorise en la somme directe de $[G/I:\langle p \rangle]$ corps, tous isomorphes à $\GF{p^n}$, où $n$ est l'ordre de $p$ dans $G/I$.
\end{cor}
\begin{proof}
Il suffit d'appliquer directement le lemme au raisonnement fait plus haut. %TODO: En vrai, je crois pas que ça soit aussi simple, il faudra vraiment creuser encore...
\end{proof}

Soit maintenant l'élement suivant :
\begin{equation}
z_I^{(k)} = \sum_{i\in{I}}{\zeta_m^{ik}}
\end{equation}

Alors cet élément est dans $\mathcal{R}_I$, puisque lui appliquer un $\sigma_i\in{I}$ revient juste à permuter les termes de la somme. Alors on a le théorème suivant :
 
\begin{thm}[Rains\cite{Rai}]
Si on suppose que $m$ est sans facteur carré, alors les $z_I^{(k)}$ engendrent $\mathcal{R}_I$, pour $k\in G/I$.
\end{thm}

\begin{proof}
Commençons par considérer le cas $I = \langle1\rangle$ et $m$ premier. Alors on a $z_I = \zeta_m^k$ et, par définition, les $\zeta_m^k$ engendrent $\mathcal{R}_m$ pour $0 \leq k\leq m$. Il suffit alors de montrer que $1$ peut s'exprimer en fonction des $\zeta_m^k$ pour $k > 0$, puisque $k = 0$ n'est pas dans $(\zmodn{m})^{\times}$. Mais $m$ est premier, alors le polynôme $\phi_m$ donne exactement :
\[\textup{Tr}(\zeta_m) = \sum_{0 < k < m}{\zeta_m^k} = 1\]
les $\zeta_m^k$ étant les conjugués de $\zeta_m$ pour $0 < k < m$; à nouveau car $m$ est premier.\\\par
Changeons alors seulement l'hypothèse $m$ premier en $m$ sans facteur carré, on va raisonner par récurrence. En particulier, on peut écrire $m = m_1m_2$ avec $(m_1,m_2) = 1$, alors le résutlat ci-dessus est encore valable pour $m_1$ et $m_2$ par hypothèse de récurrence. Maintenant, d'après le théorème des restes chinois, on a :
\[\zmodninv{m} \simeq \zmodninv{m_1} \times \zmodninv{m_2}\]
Ainsi, \textit{via} cette isomorphisme on peut exprimer un élément $z_I^{(k)}$ dans $\mathcal{R}_m$ comme un produit de deux éléments $z_I^{(k_1)}$ et $z_I^{(k_2)}$ dans $\mathcal{R}_{m_1}$ et $\mathcal{R}_{m_2}$ respectivement; et inversement.\par
Prenons un $k\in\zmodninv{m}$ alors on peut l'écrire sous la forme :
\[k = uk_2m_1 + vk_1m_2\]
avec $k_1\in\zmodninv{m_1}$, $k_2\in\zmodninv{m_2}$ et $u,v\in\mathbb{Z}$ tels qu'ils soient premiers avec $m_2$ et $m_1$ respectivement. On a donc :

\begin{align*}
z_I^{(k)} &= \zeta_m^k\\
&= \zeta_m^{uk_2m_1 + vk_1m_2}\\
&= (\zeta_m^{um_1})^{k_1}(\zeta_m^{vm_2})^{k_2}\\
\end{align*}

On a bien que $\zeta_m^{um_1}$ est une racine \nroot{$m_2$} puisque $m_1m_2 = m$, mais c'est aussi une racine primitive car $(u,m_2) = 1$ et $(m_1, m_2) = 1$; donc la seul façon d'avoir $kum_1 = m$ est de poser $k = m_2$. Le raisonnement est exactement le même pour l'autre terme. D'où le résultat :
\[z_I^{(k)} = z_I^{(k_1)}z_I^{(k_2)}\]\par
Pour l'autre sens, on prend le produit suivant :
\[z_I^{(k_1)}z_I^{(k_2)} = \zeta_{m_1}^{k_1}\zeta_{m_2}^{k_2}\]
Si on l'élève à la puissance $m_1$ ou $m_2$, le produit ne pourra pas être égal à $1$. En revanche, si on l'élève à la puissance $m$, on obtiendra bien $1$, d'où le fait que c'est une racine primitive \nroot{m} de l'unité. Alors on pourra toujours le réécrire sous la forme : 
\[\zeta_m^k = z_I^{(k)}\]
avec $k\in\zmodninv{m}$, ce qu'on cherchait.\par
Ainsi, comme les $z_I^{(k_1)}$ engendrent $\mathcal{R}_{m_1}$ et les $z_I^{(k_2)}$ engendrent $\mathcal{R}_{m_2}$ alors les $z_I^{(k)}$ engendrent $\mathcal{R}_m$.\\\par
Pour finir, on considère $I$ sous-groupe quelconque du groupe de Galois. Prenons un $x$ dans $\mathcal{R}_I$, alors on peut écrire :
\[x = \sum_{i\in I}{\sigma_i(y)}\]
pour un $y$ dans $\mathcal{R}_m$; l'action d'un $\sigma_i$ sur $x$ ne sera alors qu'une permutation des termes de la somme. De plus, on peut aussi écrire $y$ sous la forme :
\[y = \sum_{(k,m)=1}{c_k\sigma_k(\zeta_m)}\]
c'est ce qu'on a prouvé plus haut. En combinant les deux expression on obtient alors :
\begin{align*}
x &= \sum_{i\in I}{\sum_{(k,m)=1}{c_k\sigma_i\sigma_k(\zeta_k)}}\\
&= \sum_{(k,m)=1}{c_kz_I^{(k)}}\\
\end{align*}
ce qu'il fallait montrer et on achève la preuve.\\
\end{proof}

\begin{rem}
On remarque que si $p$ engendre $\zmodninv{m}/I$ alors, d'après le corollaire \ref{coreng}, $\mathcal{R}_I$ est un corps puisque dans ce cas $[\zmodninv{m}/I:\langle p\rangle] = 1$; et finalement, $z_I^{(k)}$ est alors un élément normal de ce corps. On vient donc de finir de justifier l'algorithme de Rains.
\end{rem}

\subsection{Algorithme de Rains : méthode elliptique}


\begin{thebibliography}{LC}
\bibitem{Sam} \emph{Théorie algébrique des nombres}, \bsc{Pierre Samuel}, Hermann, 1971.
\bibitem{Nek} \emph{Théorie de Galois}, \bsc{Jan Nekov\'a\v{r}}, Université Pierre et Marie Curie, 2003, \bsc{url :} \url{http://www.math.jussieu.fr/~nekovar/co/ln/gal/g.pdf}.
\bibitem{Per} \emph{Cours d'algèbre}, \bsc{Daniel Perrin}, ellipses, 1996.
\bibitem{Rai} \emph{Efficient computation of isomorphism between finite fields}, \bsc{Eric M. Rains}, 2008.
\bibitem{LiNi} \emph{Finite fields}, \bsc{Rudolf Lidl} \& \bsc{Harald Niederreiter}, Encyclopedia of mathematics and its applications vol. 20, Cambridge, 1983.
\bibitem{Pin} \emph{Recognising elements of finite fields}, \bsc{Richard G.E. Pinch}, Cryptography and coding II, p. 193-197, Oxford University Press, 1992.
\bibitem{Pol} \emph{Algèbre et théorie de Galois}, \bsc{Patrick Polo}, Université Pierre et Marie Curie, 2007, \bsc{url :} \url{http://www.math.jussieu.fr/~polo/M1/ATG07chIV.pdf}.
\bibitem{Law} \emph{Introduction to cyclotomic fields}, \bsc{Lawrence C. Washington}, Graduate texts in mathematics, Springer-Verlag, 1982.
\bibitem{Sil} \emph{The arithmetic of elliptic curves}, \bsc{Joseph H. Silverman}, Graduate texts in mathematics, Springer, 2nd ed. 2009.
\bibitem{GarD} \emph{Handbook of finite fields}, \bsc{Gary L. Mullen} \& \bsc{Daniel Panario}, Discrete mathematics and its applications, Series Editor Kenneth H. Rosen, CRC Press.
\end{thebibliography}


\end{document}

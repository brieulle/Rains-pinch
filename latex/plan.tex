\documentclass[a4paper]{article} % papier A4
\usepackage[utf8]{inputenc}      % accents dans le source
\usepackage[T1]{fontenc}         % accents dans le pdf
\usepackage{textcomp}            % symboles complémentaires (euro)
\usepackage[frenchb]{babel}      % titres en français
\usepackage{amsmath}
\usepackage{amsthm}
\usepackage{amssymb}
\usepackage[colorlinks=false]{hyperref}
\usepackage{enumerate}
\usepackage{algorithmic}
\usepackage{pgf}
\usepackage{tikz}
\usepackage{tikz-cd}
\usetikzlibrary{matrix,arrows,decorations.pathmorphing}
\numberwithin{equation}{section}
\newcommand\nroot[1]{\textit{#1}\up{\textit{ième}}}
\newcommand\zmodn[1]{\mathbb{Z}/#1\mathbb{Z}}
\newcommand\zmodninv[1]{(\mathbb{Z}/#1\mathbb{Z})^{\times}}
\newcommand\GF[1]{\mathbb{F}_{#1}}
\newcommand\Irr[2]{\textup{Irr}_{#1}(#2)}
\renewcommand{\algorithmicrequire}{\textbf{Input:}}
\renewcommand{\algorithmicensure}{\textbf{Ouput:}}
\newcommand\Tr[1]{\textup{Tr}\left(#1\right)}
\begin{document}
\newtheorem{thm}{Thèorème}[section]
\newtheorem{lem}[thm]{Lemme}
\newtheorem{cor}{Corollaire}[thm]
\newtheorem{prop}[thm]{Proposition}
\theoremstyle{definition}
\newtheorem{defn}[thm]{Définition}
\newtheorem*{ex}{Exemple}
\theoremstyle{remark}
\newtheorem{rem}{Remarque}[thm]
\section*{Remerciements}
\section*{Introduction}
Exposé rapide de la situation, mise en contexte cryptographique/avec le projet de recherche (pour avoir un point de vue plus large).
\section{Préliminaire/contexte théorique}
\subsection{Corps finis}
Définitions, extensions, galois (Frobenius etc.), racines de l'unités, polynôme/corps cyclotomique.
\subsection{Courbes elliptiques}
Définitions (j-invariant, nombres de points, tordues, module de Tate (?) etc.), résultats généraux, sur les corps finis
\subsection{Théorie des nombres/Idéaux}
Tout ce qui concerne la preuve de Rains avec les idéaux premiers, les normes d'ideaux, la factorisation unique, Galois, les périodes de Gauss, les périodes elliptiques (Mihailescu et al.) ou plus précisément ce qui permet d'y arriver.
\section{Isomorphismes de corps finis}
Présentation du problème, énonciation des difficultés etc.
\subsection{Algorithme de Pinch (peut-être à mettre dans l'introduction de la section ?)}
Énonciation et descriptions de la méthode de Pinch, justification des résultats, explication des inconvénients et autre.
\subsection{Algorithme de Rains}
Commencer par une explication concise de la méthode, ce qu'elle change par rapport à Pinch.
\subsubsection{Méthode cyclotomique}
Présenter pas à pas l'article de Rains (en gros), en enlevant ce qui nous concernerait moins et en rajoutant ce qu'il n'a pas mis (tous les détails sur l'algorithme convert et sa complexité).
\subsubsection{Méthode elliptique}
Résumer/Détailler le papier de Luca, rajouter les résultats sur les twists, les périodes, détaillés l'algorithme et sa complexité, le prouver (probablement fait avec le papier de Luca et Mihailescu pour les périodes ?), détailler les différentes étapes, les paramètres, la façon de trouver m etc.
\subsubsection{Comparatif}
Comparer les deux méthodes ou même avec l'implémentation de base de SAGE (retrouver une simple racine).
\section{"Conclusion"}
Expliquer ce qu'il peut rester à faire (cas général, coupe et appliquer l'une des deux méthodes selon la nature du m ou s'il y a besoin de calculer une extension, etc.), où est-ce qu'on peut récupérer des performances (produit scalaire)

















\end{document}

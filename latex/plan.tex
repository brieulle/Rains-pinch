\documentclass[a4paper]{article} % papier A4
\usepackage[utf8]{inputenc}      % accents dans le source
\usepackage[T1]{fontenc}         % accents dans le pdf
\usepackage{textcomp}            % symboles complémentaires (euro)
\usepackage[frenchb]{babel}      % titres en français
\usepackage{amsmath}
\usepackage{amsthm}
\usepackage{amssymb}
\usepackage[colorlinks=false]{hyperref}
\usepackage{enumerate}
\usepackage{algorithmic}
\usepackage{pgf}
\usepackage{tikz}
\usepackage{tikz-cd}
\usetikzlibrary{matrix,arrows,decorations.pathmorphing}
\numberwithin{equation}{section}
\newcommand\nroot[1]{\textit{#1}\up{\textit{ième}}}
\newcommand\zmodn[1]{\mathbb{Z}/#1\mathbb{Z}}
\newcommand\zmodninv[1]{(\mathbb{Z}/#1\mathbb{Z})^{\times}}
\newcommand\GF[1]{\mathbb{F}_{#1}}
\newcommand\Irr[2]{\textup{Irr}_{#1}(#2)}
\renewcommand{\algorithmicrequire}{\textbf{Input:}}
\renewcommand{\algorithmicensure}{\textbf{Ouput:}}
\newcommand\Tr[1]{\textup{Tr}\left(#1\right)}
\begin{document}
\newtheorem{thm}{Thèorème}[section]
\newtheorem{lem}[thm]{Lemme}
\newtheorem{cor}{Corollaire}[thm]
\newtheorem{prop}[thm]{Proposition}
\theoremstyle{definition}
\newtheorem{defn}[thm]{Définition}
\newtheorem*{ex}{Exemple}
\theoremstyle{remark}
\newtheorem{rem}{Remarque}[thm]
\section*{Remerciements}
\section*{Introduction}
Soit $\GF{q}$ le corps fini à $q = p^r$ éléments, avec $p$ un nombre premier. On
définit alors $\overline{\GF{q}}$ sa clôture algébrique. Lorsqu'on a besoin de
faire des calculs dans cette clôture, il est peut être plus aisé de calculer
directement sur un corps finis en particulier. Il est donc important de pouvoir
se déplacer rapidement entre les différents sous-corps de cette clôture.\par
Il est connu et prouvé que de deux corps finis de même cardinal sont reliés par
un isomorphisme. Cependant la preuve de ce résultat n'est pas constructive,
trouver un tel isomorphisme demande alors un travail supplémentaire.\par
La situation est la suivante, nous avons deux extensions de $\GF{q}$ de même
degré et définies par deux polynôme irréductible distincts $f$ et $g$. On notera
:
\[k_1=\GF{q}[X]/(f)\textup{ et }k_2=\GF{q}[Y]/(g)\]
le but est donc de trouver un isormophisme reliant ces deux corps. Une méthode
simple et immédiate est d'envoyer $x = \bar{X}$ sur une racine de $f$ dans
$k_2$. Le problème est que cela revient à factoriser le polynôme $f$ ce qui est
beaucoup trop lent.\par
On va donc dans ce rapport étudier deux méthodes permettant, \textit{via}
l'utilisation de racines de l'unité et de courbes elliptique, de calculer
rapidement de tels isomorphismes.


\section{Préliminaire/contexte théorique}
\subsection{Corps finis}
Définitions, extensions, galois (Frobenius etc.), racines de l'unités, 
polynôme/corps cyclotomique.
\subsection{Théorie des nombres/Idéaux}
Tout ce qui concerne la preuve de Rains avec les idéaux premiers, les normes
 d'ideaux, la factorisation unique, Galois, les périodes de Gauss, les 
 périodes elliptiques (Mihailescu et al.) ou plus précisément ce qui permet 
 d'y arriver.
\subsection{Courbes elliptiques}
Définitions (j-invariant, nombres de points, tordues, module de Tate (?) 
etc.), résultats généraux, sur les corps finis


\section{Isomorphismes de corps finis}
Présentation du problème, énonciation des difficultés etc.
\subsection{Algorithme de Pinch (peut-être à mettre dans l'introduction de 
la section ?)}
Énonciation et descriptions de la méthode de Pinch, justification des 
résultats, explication des inconvénients et autre.
\subsection{Algorithme de Rains : méthode cyclotomique}
\textit{Résumé :}Commencer par une explication concise de la méthode, ce 
qu'elle change par rapport à Pinch.\par
Présenter pas à pas l'article de Rains (en gros), en enlevant ce qui nous 
concernerait moins et en rajoutant ce qu'il n'a pas mis (tous les détails 
sur l'algorithme convert et sa complexité).
\subsubsection{Principe}
\subsubsection{Éléments normaux}
\subsubsection{Algorithme}
\subsubsection{Analyse de complexité}
\subsection{Algorithme de Rains : méthode elliptique}
\textit{Résumé :}Résumer/Détailler le papier de Luca, rajouter les résultats sur
les twists, les périodes, détaillés l'algorithme et sa complexité, le prouver 
(probablement fait avec le papier de Luca et Mihailescu pour les périodes ?)
, détailler les différentes étapes, les paramètres, la façon de trouver m 
etc.
\subsubsection{Principe}
\subsubsection{Choix des paramètres}
\subsubsection{Analyse de complexité}
\subsection{Comparatif}
Comparer les deux méthodes ou même avec l'implémentation de base de SAGE 
(retrouver une simple racine).
\section{"Conclusion"}
Expliquer ce qu'il peut rester à faire (cas général, couper et appliquer 
l'une des deux méthodes selon la nature du m ou s'il y a besoin de calculer 
une extension, etc.), où est-ce qu'on peut récupérer des performances 
(produit scalaire)

\end{document}

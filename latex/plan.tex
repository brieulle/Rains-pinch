\documentclass[a4paper]{article} % papier A4
\usepackage[utf8]{inputenc}      % accents dans le source
\usepackage[T1]{fontenc}         % accents dans le pdf
\usepackage{textcomp}            % symboles complémentaires (euro)
\usepackage[frenchb]{babel}      % titres en français
\usepackage{amsmath}
\usepackage{amsthm}
\usepackage{amssymb}
\usepackage[colorlinks=false]{hyperref}
\usepackage{enumerate}
\usepackage{tocloft}
\usepackage{algorithmic}
\usepackage{pgf}
\usepackage{tikz}
\usepackage{tikz-cd}
\usetikzlibrary{matrix,arrows,decorations.pathmorphing}
\numberwithin{section}{part}
\numberwithin{equation}{section}
\newcommand\nroot[1]{\textit{#1}\up{\textit{ième}}}
\newcommand\zmodn[1]{\mathbb{Z}/#1\mathbb{Z}}
\newcommand\zmodninv[1]{(\mathbb{Z}/#1\mathbb{Z})^{\times}}
\newcommand\GF[1]{\mathbb{F}_{#1}}
\newcommand\Irr[2]{\textup{Irr}_{#1}(#2)}
\renewcommand{\algorithmicrequire}{\textbf{Input:}}
\renewcommand{\algorithmicensure}{\textbf{Ouput:}}
\newcommand\Tr[1]{\textup{Tr}\left(#1\right)}
\setlength{\cftsecnumwidth}{3em}
\setlength{\cftsubsecnumwidth}{3em}


\begin{document}
\newtheorem{thm}{Thèorème}[section]
\newtheorem{lem}[thm]{Lemme}
\newtheorem{cor}{Corollaire}[thm]
\newtheorem{prop}[thm]{Proposition}
\theoremstyle{definition}
\newtheorem{defn}[thm]{Définition}
\newtheorem*{ex}{Exemple}
\theoremstyle{remark}
\newtheorem{rem}{Remarque}[thm]

\addcontentsline{toc}{part}{Remerciements}
\part*{Remerciements}

\tableofcontents

\addcontentsline{toc}{part}{Introduction}
Soit $\GF{q}$ le corps fini à $q = p^r$ éléments, avec $p$ un nombre premier. On
définit alors $\overline{\GF{q}}$ sa clôture algébrique. Lorsqu'on a besoin de
faire des calculs dans cette clôture, il est peut être plus aisé de calculer
directement sur un corps finis en particulier. Il est donc important de pouvoir
se déplacer rapidement entre les différents sous-corps de cette clôture.\par
Il est connu et prouvé que de deux corps finis de même cardinal sont reliés par
un isomorphisme. Cependant la preuve de ce résultat n'est pas constructive,
trouver un tel isomorphisme demande alors un travail supplémentaire.\par
La situation est la suivante, nous avons deux extensions de $\GF{q}$ de même
degré et définies par deux polynôme irréductible distincts $f$ et $g$. On notera
:
\[k_1=\GF{q}[X]/(f)\textup{ et }k_2=\GF{q}[Y]/(g)\]
le but est donc de trouver un isormophisme reliant ces deux corps. Une méthode
simple et immédiate est d'envoyer $x = \bar{X}$ sur une racine de $f$ dans
$k_2$. Le problème est que cela revient à factoriser le polynôme $f$ ce qui est
beaucoup trop lent.\par
On va donc dans ce rapport étudier deux méthodes permettant, \textit{via}
l'utilisation de racines de l'unité et de courbes elliptique, de calculer
rapidement de tels isomorphismes.


\part{Contexte théorique}

\section{Corps finis}
Dans cette section on va rappeler les définitions et démontrer certains 
résultats liés aux corps finis. Principalement, on étudiera les extensions 
de corps et la théorie de base de Galois sur les corps finis.

\subsection{Définitions}

Soit $K$ 

\subsection{Extensions de corps}
Esquisse de la théorie général + résultats particuliers sur les corps finis
\subsection{Théorie de Galois}
Définitions et résutlats de base (sans forcément tout prouver), Frobenius, 
Groupe de Galois, élément primitif, extensions galoisiennes, etc.

\subsection{Extensions et polynômes cyclotomiques}
Racines de l'unités et quelques mots sur les polynômes cyclotomiques et leurs 
extensions. Peut-être rajouter un ou deux mots sur le cas des corps finis (ça 
pourrait se mettre dans la section du dessus).

\section{Idéaux et théorie des nombres}
\textit{Résumé :} Tout ce qui concerne la preuve de Rains avec les idéaux 
premiers, les normes d'ideaux, la factorisation unique, Galois, les périodes de 
Gauss, les périodes elliptiques (Mihailescu et al.) ou plus précisément ce qui 
permet d'y arriver.
\subsection{Corps de nombres}
Définitions (normes, traces, éléments algébriques, etc.)
\subsection{Idéaux premiers}
 Idéaux premiers sur les anneaux d'entiers de corps de nombres,
 décomposition unique, résultat de la théorie de Galois.
\subsection{Périodes de Gauss}
Énoncé et justifier les résultats donnés par Rains (notamment le lemme
expliquant la forme de $\GF{q}[X]/I$ pour $I$ sous-groupe du groupe de Galois);
peut-être à mettre directement dans la partie cyclotomique de la section
algorithme de Rains cyclotomique.

\section{Courbes elliptiques}
\textit{Résumé :}Définitions (j-invariant, nombres de points, tordues, module de
Tate (?) etc.), résultats généraux, sur les corps finis
\subsection{Définitions}
Rapide énoncé des changements de variable, comment on arrive à la forme
"simple"; liste des différents paramètres/invariants qui nous intéressent.
\subsection{Sur les corps finis}
Nombres de points, théorème de Hasse, Frobenius, trace
\subsection{Périodes elliptiques}
Prouver et définir les périodes elliptiques (peut-être que ce serait à
faire directement dans la partie algorithme elliptique de Rains ?)
\subsection{Tordues}
Twists quadratiques, courbes $y^2 = x^3 + x$ et $y^2 = x^3 + 1$ et leurs
tordues, pour préparer le terrain pour la partie elliptique de Rains

\section{Complexité \& Notations}
Définir le $O$ et $O^{\sim}$; définir les notations etc.


\part{Isomorphismes de corps finis}
Présentation du problème, énonciation des difficultés etc.

\section{Algorithme de Pinch}
Énonciation et descriptions de la méthode de Pinch, justification des 
résultats, explication des inconvénients et autre.
\subsection{Principe}
Explication de la méthode et illustration avec l'exempe de l'article, peut-être
?
\subsection{Limitations}
Taille de m, choisir un élément au hasard

\section{Algorithme de Rains : méthode cyclotomique}
\textit{Résumé :}Commencer par une explication concise de la méthode, ce 
qu'elle change par rapport à Pinch.\par
Présenter pas à pas l'article de Rains (en gros), en enlevant ce qui nous 
concernerait moins et en rajoutant ce qu'il n'a pas mis (tous les détails 
sur l'algorithme convert et sa complexité).
\subsection{Principe}
Passer par une petite extension pour réduire la taille de m, utilisation des
éléments normaux, utilisation des périodes de Gauss.
\subsection{Éléments normaux}
Définitions et résultats utiles pour l'algorithme, justification de
l'algorithme (peut-être à mettre dans la partie Galois de la section corps
finis), conversion de la base polynomiale à la base normale.
\subsection{Algorithme \& analyse de complexité}
Détailler l'algorithme de l'article/code de Luca + analyser la complexité de ces
algos et les "prouver" (montrer que ça marche bien, cela dit c'est peut-être
largement faisable avant).

\section{Algorithme de Rains : méthode elliptique}
\textit{Résumé :}Résumer/Détailler le papier de Luca, rajouter les résultats sur
les twists, les périodes, détaillés l'algorithme et sa complexité, le prouver 
(probablement fait avec le papier de Luca et Mihailescu pour les périodes ?)
, détailler les différentes étapes, les paramètres, la façon de trouver m 
etc.
\subsection{Principe}
Choix d'un point d'ordre m sur une bonne courbe elliptique engendre 
l'extension, utilisation des périodes elliptiques afin d'avoir un élément 
stable par l'action du groupe de Galois (Énoncer les lemmes etc.)
\subsection{Choix des paramètres}
Justifier le choix de m par rapport à n et q, tout ce qui concerne la valeur 
propre d'ordre n et donc le choix de traces qui s'en suit, expliquer comment on
pick les courbes (à ce moment là on pourra aussi justifier avec le résultat 
sur les tordues).
\subsection{Algorithme \& Analyse de complexité}
Décrire l'algorithme et donner sa complexité.

\part{Implémentation}
Comparer les deux méthodes ou même avec l'implémentation de base de SAGE 
(retrouver une simple racine). Détailler les limitations dû à SAGE, peut-être
prendre une partie de la conclusion en expliquant ce qui pourrait être changer.

\section{Résultats numériques}

\section{Comparaison avec d'autres méthodes}

\part{Conclusion}
Expliquer ce qu'il peut rester à faire (cas général, couper et appliquer 
l'une des deux méthodes selon la nature du m ou s'il y a besoin de calculer 
une extension, etc.), où est-ce qu'on peut récupérer des performances 
(produit scalaire)

\end{document}

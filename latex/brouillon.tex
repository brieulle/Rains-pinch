\documentclass[a4paper]{article} % papier A4
\usepackage[utf8]{inputenc}      % accents dans le source
\usepackage[T1]{fontenc}         % accents dans le pdf
\usepackage{textcomp}            % symboles complémentaires (euro)
\usepackage[frenchb]{babel}      % titres en français
\usepackage{amsmath}
\usepackage{amsthm}
\usepackage{amssymb}
\usepackage[colorlinks=false]{hyperref} %Apparemment ça sert à rien...
\usepackage{enumerate}
\usepackage{tocloft}             % Pour la table des matières
\usepackage{algorithm}
\usepackage{algorithmic}
\usepackage{pgf}
\usepackage{tikz}
\usepackage{tikz-cd}
\usepackage{rotating}
\usetikzlibrary{matrix,arrows,decorations.pathmorphing}
\usepackage{array}

% Numérotation des sections, sous-sections, équations, etc.
\numberwithin{section}{part}
\numberwithin{equation}{section}

% Maccros pour les commandes de math
\newcommand\nroot[1]{\textit{#1}-ième}
\newcommand\zmodn[1]{\mathbb{Z}/#1\mathbb{Z}}
\newcommand\zmodninv[1]{(\mathbb{Z}/#1\mathbb{Z})^{\times}}
\newcommand\GF[1]{\mathbb{F}_{#1}}
\newcommand\Irr[2]{\textup{Irr}_{#1}(#2)}
\renewcommand{\algorithmicrequire}{\textbf{Input:}}
\renewcommand{\algorithmicensure}{\textbf{Ouput:}}
\newcommand\Tr[1]{\textup{Tr}\left(#1\right)}
\newcommand\QQ{\mathbb{Q}}
\newcommand\ZZ{\mathbb{Z}}
\newcommand\NN{\mathbb{N}}
\newcommand\CC{\mathbb{C}}
\newcommand\RR{\mathbb{R}}
\newcommand\EO{\mathcal{O}}
\newcommand\PP[1]{\mathbb{P}^{#1}}
\newcommand\etmath{\textup{\quad et \quad}}
\newcommand\M[1]{\textup{M}(#1)}
\newcommand\E[1]{\textup{E}(#1)}
\newcommand\I[1]{\textup{I}(#1)}
\newcommand\tO[1]{\widetilde{O}(#1)}
\newcommand\groupgen[1]{\langle{#1}\rangle}
\newcommand\ord[2]{\textup{ord}_{#1}(#2)}

% Éviter l'overlap dans la table des matières pour les sections, sous-sections
% etc.
\setlength{\cftsecnumwidth}{3em}    
\setlength{\cftsubsecnumwidth}{3em} 


\sloppy
\begin{document}
\title{Calcul d'isomorphisme de corps finis}
\author{Ludovic Brieulle}
\newtheorem{thm}{Théorème}[section]
\newtheorem{lem}[thm]{Lemme}
\newtheorem{cor}[thm]{Corollaire}
\newtheorem{prop}[thm]{Proposition}
\theoremstyle{definition}
\newtheorem{defn}[thm]{Définition}
\newtheorem*{ex}{Exemple}
\theoremstyle{remark}
\newtheorem*{rem}{Remarque}

\maketitle
\part*{Remerciements}

\addcontentsline{toc}{part}{Introduction}
\part*{Introduction}
Soit $\GF{q}$ le corps fini à $q = p^r$ éléments, avec $p$ un nombre premier. On
définit alors $\overline{\mathbb{F}}_q$ sa clôture algébrique. Lorsqu'on a 
besoin de faire des calculs dans cette clôture, il faut faire les calculs
directement sur un corps fini en particulier. Il n'y a pas d'unique façon de
partir d'un corps premier pour \og atteindre\fg\, la clôture algébrique. Les
différentes façons d'y parvenir peuvent être représenter par ce qu'on appelle 
un réseau de corps finis. Par exemple, un réseau de corps fini pour 
$\overline{\mathbb{F}}_5$ se présente sous la forme suivante :

\begin{center}
\begin{tikzpicture}
\matrix(m)[matrix of math nodes,
row sep=1em, column sep=3em,
text height=2ex, text depth=0.5ex]
{ & &\overline{\mathbb{F}}_q &  &\\
\GF{5^4} & & \GF{5^6} & &\\
& \GF{5^2} & & \GF{5^3} & \GF{5^5}\\
& & \GF{5} & & \\};
\path[dashed,-,font=\scriptsize,>=angle 90]
(m-1-3) edge (m-2-1)
(m-1-3) edge (m-2-3)
(m-1-3) edge (m-3-5);
\path[-,font=\scriptsize,>=angle 90]
(m-2-1) edge (m-3-2)
(m-2-3) edge (m-3-2)
(m-2-3) edge (m-3-4)
(m-2-1) edge (m-3-2)
(m-4-3) edge (m-3-2)
(m-4-3) edge (m-3-4)
(m-4-3) edge (m-3-5);
\end{tikzpicture}
\end{center}

Pour passer d'une branche à l'autre, il faut plonger un corps de l'ancienne
branche dans un corps de la nouvelle. Seulement, on n'utilise pas nécessairement
la même représentation\footnote{Par exemple, si on quotiente par deux polynômes
irréductibles différents ou si on utilise des bases différentes.} pour les deux 
branches. Une façon de procéder est de construire un isomorphisme entre le corps
d'une branche et le sous-corps isomorphe de l'autre branche.\par
Il est connu et prouvé que deux corps finis de même cardinal sont reliés par
un isomorphisme. Cependant la preuve de ce résultat ne permet pas d'expliciter
directement un tel isomorphisme, le trouver demande alors un travail 
supplémentaire. La situation est la suivante, on a deux extensions de 
$\GF{q}$ de même degré et définies par deux polynômes irréductibles distincts 
$f$ et $g$. On notera :
\[k_1=\GF{q}[X]/(f)\etmath k_2=\GF{q}[Y]/(g)\]
le but est donc de trouver un isormophisme reliant ces deux corps. 
Une méthode simple et immédiate est d'envoyer $x = \bar{X}$ la classe de $X$
dans $k_1$ sur une racine de $f$ dans $k_2$. Le problème est que cela revient à 
factoriser le polynôme $f$ ce qui est beaucoup trop lent. On étudiera dans ce
rapport deux variantes d'un algorithme dû à Rains\cite{Rai}, l'une dite 
cyclotomique et l'autre elliptique, lui-même étant une amélioration d'un 
algorithme donné par Pinch\cite{Pin}, pour déterminer des isomorphismes entre 
corps finis de même cardinal.

\newpage
\tableofcontents
\newpage

\part{Rappels et contexte théorique}
\label{un}

Cette partie consiste principalement à rappeler et énoncer les résultats dont
nous aurons besoin afin de justifier et mettre en place les deux algorithmes 
pour calculer les isomorphismes de corps finis. Cette partie n'a pas pour 
vocation de faire un exposé complet des notions abordées, pour un traité plus 
détaillé, le lecteur pourra consulter au choix \cite{LiNi1}, \cite{MuPa} ou 
\cite[chap.~III]{Per} pour les corps finis;
\cite[chap.~VI]{Sam} ou \cite{Esc} pour la théorie de Galois; \cite{Sam}, 
\cite{Was1} ou \cite{Lan} pour la théorie des nombres; \cite{Sil} ou 
\cite{Was2} pour la théorie sur les courbes elliptiques; et pour finir 
\cite{GaGe} pour tout ce qui concerne le calcul formel en général. On 
s'efforcera néamoins de démontrer les résultats les plus importants ou 
fondamentaux. Le c\oe ur du sujet se trouve lui dans la partie \ref{deux}.

\section{Corps finis}
Dans cette section on rappelle les définitions et démontrer certains 
résultats liés aux corps finis. Principalement, on étudiera les extensions 
de corps et la théorie de Galois sur les corps finis. Tous les corps seront
considérés commutatifs. On notera $(a,b)$ le pgcd entre deux entiers $a$ et $b$.

\subsection{Définitions}
Un corps fini $K$ est un corps à $q = p^r$ éléments, où $p$ un nombre premier
s'appelle la \emph{caractéristique} de $K$ telle que pour tout $x\in K$ :
\begin{equation}
p\cdot x = \underbrace{x+\dots+x}_{p\textup{ fois}} = 0.
\end{equation}
Un corps fini est muni de l'automorphisme de Frobenius définie par :
\begin{equation}
\phi_p : x \mapsto x^p.
\end{equation}
On a aussi :
\begin{equation}
\phi_q = \underbrace{\phi\circ\cdots\circ\phi}_{r\textup{
fois}},
\end{equation}
qui à tout $x\in \bar{K}$ associe $x^q$.\par
On caractérise les éléments de $q$ par l'identité $x^q - x = 0$. En effet, le
groupe des inversibles de $K$ est d'ordre $q-1$, donc pour tout $x\in K$ on a
$x^{q-1} = 1$ et il suffit de multiplier par $x$ pour obtenir l'identité.
On peut complètement caractériser les corps finis par le théorème suivant. Il
assure entre autre que deux corps finis de même cardinal sont isomorphes, ce qui
sera le point de départ de notre travail.

\begin{thm}
\label{th:isomGF}
Pour tout nombre premier $p$ et tout entier strictement positif $n$, il existe 
un corps fini à $p^n$ éléments. Tout corps fini à $q = p^n$ éléments est 
isomorphe au corps de décomposition de $X^q - X$ sur $\GF{p}$. On parlera 
alors du corps fini à $q$ éléments et on le notera $\GF{q}$.
\end{thm}
\begin{proof}
\textit{(Existence)} Pour $q = p^n$, on considère le polynôme $X^q - X$ dans 
$\GF{p}[X]$ et on note $K$ son \emph{corps de décomposition}, définition 
\ref{def:dec}, sur $\GF{p}$. Le polynôme est \emph{séparable}, définition 
\ref{def:sep}, ou n'a aucune racine multiple puisque sa dérivée est égale à 
$qX^{q-1} - 1 = -1$, proposition \ref{prop:sepderiv}. Posons $S=\lbrace x\in K 
: x^q - x = 0\rbrace$, alors $S$ est un sous-corps de $K$; $0$ et $1$ sont 
dans $K$ et d'après les propriétés sur le Frobenius et les résultats obtenus 
plus haut on a :
\begin{equation}
(a - b)^q = a^q - b^q = a - b \etmath(ab^{-1})^q = a^qb^{-q} = ab^{-1}.
\end{equation}
Ainsi, $S$ contient toutes les racines de $X^q - X$ mais comme $K$ a déjà $q$
éléments, alors $K = S$ est un corps à $q$ éléments.\par
\textit{(Unicité)} Soit $K$ le corps à $q = p^n$ éléments, il est de 
caractéristique $p$ et contient $\GF{p}$. On en déduit que $K$ est un corps de 
décomposition de $X^q - X$ sur $\GF{p}$, puisqu'il est scindé sur $K$, et 
l'unicité se déduit de l'unicité des corps de décomposition, corollaire 
\ref{cor:dec}.\\
\end{proof}

\subsection{Extension de corps}
\label{def:degext}
Soient $k$ et $K$ deux corps, s'il existe un homomorphisme de $k$ vers $K$ alors
celui-ci est nécessairement injectif. Supposons qu'on ait $f(x) = 0$, cela
s'écrit aussi $x.f(1) = 0$ donc $x = 0$ puisque un corps n'a pas de diviseur de
$0$ et $f(1) = 1$. 
On dit que $K$ est une extension de corps de $k$ s'il existe un morphisme de 
corps $\varphi : k \to K$ ou de façon équivalente, si $k \subseteq K$ alors $K$
est une extension (de corps) de $k$. On notera $K/k$ une extension de corps.
Si on a $k\subseteq L \subseteq K$, alors $L/k$ est une sous-extension de
$K/k$. On appelle corps premier un corps qui n'a pas de sous-corps ou de
sous-extensions.\par
Soit $K/k$ une extension de corps et soit $S$ une partie de $K$. Le sous-corps 
$L := k(S)$ de $K$ engendré par $S$ sur $k$ est le plus petit sous-corps de $K$ 
contenant $S$ et $k$. Si $S = \lbrace x_1,\dots,x_n \rbrace$ est fini, alors on 
note $L = K(x_1,\dots,x_n)$, on dit alors que l'extension est de \emph{type
fini}. L'extension $L/k$ est dite \emph{monogène} ou \emph{simple} si elle est
engendrée par un seul élément. Si $K/k$ est une extension de corps, alors on 
peut voir $K$ comme un $k$-espace vectoriel ou une $k$-algèbre. On appelle 
$[K:k] := dim_k(K)$ le \emph{degré} de l'extension. On dit qu'une extension 
$K/k$ est de degré fini si $[K:k] < \infty$.\par

\begin{thm}
Soient $k \subseteq L \subseteq K$ des extensions de corps de degré fini. Alors 
on a :
\begin{equation}
[K:k] = [K:L][L:k].
\end{equation}
\end{thm}
\begin{proof}
On pose $[K:L] = m$ et $[L:k] = n$. On a donc que $K$ est un $L$-espace 
vectoriel de dimension $m$ et $L$ est un $k$-espace vectoriel de dimension $n$, 
le théorème revient à montrer que $K$ est un $k$-espace vectoriel de dimension 
$mn$. Or, d'après ce qui précède, on a $L \simeq k^n$ et $K \simeq L^m$, d'où
\begin{equation}
K \simeq \underbrace{L \oplus\dots\oplus L}_{m fois}\simeq\underbrace
{k^n\oplus\dots\oplus k^n}_{m fois} \simeq k^{nm},
\end{equation}
ce qui prouve le théorème.\\
\end{proof}

On dit qu'un élément $x\in K$ est \emph{algèbrique} sur $k$ s'il existe un 
polynôme unitaire à coefficients dans $k$ qui annule $x$. L'ensemble des 
éléments algébriques d'un corps (sur un sous-corps) forme un corps \cite[p.~64, 
théorème 1.14]{Per}. On dit qu'une extension $K/k$ est algébrique si tous les 
éléments de $K$ sont algébriques sur $k$. En particulier, toute extension de
degré fini est algébrique et de type finie.


Soit $L/K$ une extension de corps et soit $\alpha\in L$ un élément algébrique 
sur $K$. Si on note $\Irr{K}{\alpha}$ le \emph{polynôme minimal} de $\alpha$ sur
$K$ et on pose $n := \textup{deg\,}\Irr{K}{\alpha}$ alors l'anneau des polynômes
sur $K$ en $\alpha$ noté $K[\alpha]$ est égal au corps des fractions rationnels
en $\alpha$ $K(\alpha)$ et c'est un $K$-espace vectoriel de dimension de $n$ qui
admet pour base la base $1,\alpha,\dots,\alpha^{n-1}$ qu'on appellera la 
\emph{base monomiale}. En résumé, $K(\alpha)/K$ est une extension de corps de
degré $n$, on l'appelle le \emph{corps de rupture} de $\Irr{K}{\alpha}$ et
s'obtient en quotientant l'anneau des polynômes sur $K$ par l'idéal engendré par
le polynôme minimal de $\alpha$. De façon plus générale, le corps de rupture 
d'un polynôme $P\in K[X]$ est l'extension de $K$ obtenue en adjoignant une 
racine $\alpha$ de $P$ à $K$ :
\begin{equation}
K(\alpha)\simeq K[X]/(P).
\end{equation}

Considérons une extension $L/K$ telle que $P\in K[X]$ irréductible admette une 
racine $\alpha\in L$. Alors $\Irr{K}{\alpha}$ divise $P$, donc est égal à 
$\lambda P$ pour $\lambda\in K^{\times}$, puisque $P$ est irréductible. Alors 
le morphisme $K$-algèbre $\phi : K[X] \to L$ défini par $\phi(X) = \alpha$ 
induit un morphisme de $K$-algèbre $\varphi : K[X]/(P) \to L$ tel que 
$\varphi(x) = \alpha$. Ce morphisme est unique puisque $x$ engendre $K[X]/(P)$. 
Ce qui montre en particulier l'existence et l'unicité à isomorphisme près des 
corps de rupture.\par
\vspace{0.3cm}
Une autre notion importante est celle de corps de décomposition d'un polynôme 
non constant sur un corps $k$.

\begin{defn}
\label{def:dec}
Soit $K$ un corps et soit $\overline{K}$ une clôture algébrique de $K$. On 
appelle corps de décomposition de $P\in K[X]$, le corps $L\subset\overline{K}$ 
tel que $P$ soit scindé dans $L$ et que ses racines engendrent $L$ sur $K$.
\end{defn}

\begin{thm}
\label{cor:dec}
Tout $P\in k[X]$ non constant admet un corps de décomposition unique à 
isomorphisme près.
\end{thm}

Penchons-nous alors plus en détails sur les corps finis. On peut se demander
étant donné un corps fini $\GF{p^n}$ quels peuvent être les corps qui le
contiennent. Dans un premier temps, il faut que les deux corps aient la même
caractéristique. Supposons qu'il existe un morphisme $f : \GF{p^n} \to 
\GF{\ell^m}$ pour $p$ et $\ell$ premiers. En particulier, on a pour 
$x\in\GF{p^n}$ :
\begin{equation}
f(p.x) = 0 = p.f(x),
\end{equation}
pour $f(x)\in\GF{\ell^m}$, donc nécessairement $\ell = p$.\par
Ainsi, si le morphisme existe alors $\GF{p^n}^{\times}$ est un sous-groupe
d'ordre $p^n - 1$ de $\GF{p^m}^{\times}$ d'ordre $p^m - 1$. Donc, on a $p^n - 1$
qui divise $p^m - 1$. Si on écrit $m = q.n + r$ avec $r < n$ alors on obtient :
\[p^m - 1 = p^r((p^n)^q - 1) + p^r - 1\]
Or $p^n - 1$ divise $(p^n)^q - 1$ donc comme $p^n - 1$ divise aussi $p^m - 1$ la
seule possibilité pour $r$ est $0$ et donc $n|m$. On vient de prouver la
nécessité du critère suivant :

\begin{prop}
Soit $\GF{q}$ un corps fini à $q = p^m$ éléments, alors tout ses sous-corps sont
de la forme $\GF{p^n}$ avec $n|m$. Réciproquement, soit $n|m$ alors il existe un
(et un seul) sous-corps de $\GF{q}$ de cardinal $p^n$.
\end{prop}
\begin{proof}
Pour montrer l'autre sens, il suffit de remarquer que $\GF{p^n}$ correspond aux
éléments satisfaisant l'équation $X^{p^n} - X$ puisque $p^n - 1$ divise $p^m -
1$.\\
\end{proof}

\begin{thm}
\label{th:elemprim}
Toute extension séparable (\ref{def:sep}) et de degré fini est simple.
\end{thm}
\begin{proof}
On va se contenter de le prouver dans le cas d'un corps fini; le cas général
demande un peu plus de travail. On invitera le lecteur à consulter
\cite[p.~87]{Esc}.\par
Si $K$ est un corps fini alors son groupe mulitplicatif $K^{\times}$ est
cyclique (voir par exemple la preuve de la proposition \ref{prop:rootcycl} ou
\cite[p.~50]{LiNi1}). Dans ce cas, il suffit de prendre pour générateur d'une
extension $L/k$, le générateur du groupe mulitplicatif de $L$.\\
\end{proof}

%TODO : Trace, norme, multiplication par un élément ? Non dégénérescence de Tr ?

Deux notions importantes pour les extensions de corps sont la trace et la norme
d'un élément d'une extension par rapport à un sous-corps. On les définit comme
suit:

\begin{defn}
Soit $L/K$ une extension de dimension finie et soit $\alpha\in L$. On appelle
endormorphisme de multiplication par $\alpha$ l'application :
\begin{align*}
m_{\alpha} :&L \longrightarrow L\\
&x\longmapsto\alpha.x.
\end{align*}
On définit alors la trace et la norme de $\alpha$ comme la trace et le 
déterminant de l'endormorphisme $m_{\alpha}$. Les formules suivantes, pour 
$\alpha, \alpha'\in L$ :
\begin{equation}
m_{\alpha + \alpha'} = m_{\alpha} + m_{\alpha'} \etmath
m_{\alpha\alpha'} = m_{\alpha}\circ m_{\alpha'}. 
\end{equation}
induisent les résultats classiques de la norme et la trace, \textit{i.e.} :
\begin{align}
\textup{Tr}_{L/K}(\alpha + \alpha') &= \textup{Tr}_{L/K}(\alpha) +
\textup{Tr}_{L/K}(\alpha'),\\
N_{L/K}(\alpha\alpha') &= N_{L/K}(\alpha)N_{L/K}(\alpha').
\end{align}
\end{defn}

\begin{prop}
Soit un $K$ un corps de caractéristique 0 ou fini et soit $L/K$ une extension
algébrique de degré $n$ sur $K$. Soient $x_1,\dots, x_n$ les racines du polynôme
minimal de $x$, dans le corps de décomposition de celui-ci, chacune répétée 
$[L:K(x)]$ fois; alors on a :
\begin{equation}
Tr_{L/K}(x) = x_1 + \dots + x_n \etmath N_{L/K}(x) = x_1\dots x_n.
\end{equation}
De plus, le polynôme caractéristique de $x$ relativement à $L$ et $K$ est $(X -
x_1)\dots(X - x_n)$.
\end{prop}
%TODO: La preuve, éventuellement.
%TODO: Je voulais en déduire que la trace est surjective, mais je sais plus trop
%comment, si tu t'en souviens un jours...

\subsection{Groupe de Galois d'une extension de corps finis}
%TODO :Définitions et résutlats de base (sans forcément tout prouver),
%Frobenius, action de groupes ? Ça me parait raisonnable...
%Groupe de Galois, élément primitif, extensions galoisiennes, etc.
Soit un corps $L$ et soit l'ensemble $G$ des automorphismes de $L$. L'ensemble 
des éléments de $x\in L$ tels que $\sigma(x) = x$ pour tout $\sigma\in G$ est un
sous-corps $K$ de $L$, c'est le corps des invariants de $G$. Soit $L/K$ et
$L'/K$ deux extensions de corps, on appelle $K$-homomorphisme de $L$ dans $L'$ 
les morphismes de $K$-algèbre de $L$ dans $L'$. Si $L = L'$ alors on parle de
$K$-automorphisme et ceux-ci forment un groupe pour la loi de composition.

\begin{thm}
Soit $L/K$ une extension de corps de type fini de degré $n$, avec $K$ un corps
fini ou de caractéristique nulle. Les conditions suivantes sont équivalentes :
\begin{enumerate}[(i)]
	\item $K$ est le corps des invariants du groupe $G$ des $K$-automorphismes
de $L$.
	\item Pour tout $x\in L$, le polynôme minimal de $x$ sur $K$ a toutes ses
racines dans $L$.
	\item $L$ est engendré par les racines d'un polynôme sur $K$.
\end{enumerate}
Dans ces conditions, le groupe $G$ est d'ordre égal à $n$.
\end{thm}

\begin{defn}
\label{def:gal}
Une extension de corps est galoisienne si elle satisfait les
conditions du théorème. Dans ce cas, $G$ s'appelle le groupe de Galois de $L/K$,
on le note $G := Gal(L/K)$. Si $G$ est abélien (resp. cyclique) alors 
l'extension est dite abélienne (resp. cyclique).
\end{defn}

Un polynôme irréductible $f$ est \emph{séparable} sur un corps $K$ si toutes
ses racines dans un corps de décompositions sont distinctes; la notion est
indépendante du corps de décomposition choisi, puisqu'il existe un $K$-morphisme
reliant les racines et dans ce cas leurs images sont aussi distinctes. 
Un élément $\alpha$ est \emph{séparable} sur $K$ si son polynôme minimal est 
séparable sur $K$.

\begin{defn}
\label{def:sep}
Une extension algébrique $L/K$ est \emph{séparable} si tout les $\alpha\in L$
sont séparables.
\end{defn}

\begin{rem}
On peut définir la séparabilité d'un polynôme réductible en se basant sur ses
facteurs irréductibles, un polynôme sera séparable si tous ses facteurs
irréductibles le sont.
\end{rem}

\begin{prop}
\label{prop:sepderiv}
Soit $P$ irréductible sur un corps $K$, on a l'équivalence suivante :
\begin{equation}
P\textup{ séparable } \Leftrightarrow P'\neq 0.
\end{equation}
\end{prop}

Soit $f$ un polynôme irréductible de degré $n$ sur un corps $K$. Soit $\alpha_1 
= a$ une racine de $f$ et $\alpha_2,\dots,\alpha_n$ les autres racines de $f$ 
dans un corps de décomposition. Alors on dit que les $\alpha_i$ sont les 
conjugués de $\alpha$ sur $K$.
\begin{defn}
Une extension $L/K$ est \emph{normale} ou \emph{quasi-galoisienne}, si pour tout
$a\in L$, $\Irr{k}{a}$ a toutes ses racines dans $L$ ou, de façon équivalente,
si $L$ contient tous les conjugués de $a$.
\end{defn}

\begin{prop}
\label{prop:decompnorm}
Soit $P\in k[X]$ de degré $n\geq 1$ et soit $K$ un corps de décomposition de $P$
sur $k$. Alors $K/k$ est normale.
\end{prop}

On peut désormais énoncer un résultat qui permet de déterminer de façon plus
pratique si une extension est galoisienne ou non.

\begin{thm}
Soit $K/k$ une extension de degré fini. Les conditions suivantes sont
équivalentes :
\begin{enumerate}
\item L'extension $K/k$ est galoisienne.
\item L'extension $K/k$ est normale et séparable.
\item Le corps $K$ est le corps de décomposition d'un polynôme séparable.
\end{enumerate}
Sous ces conditions, pour tout $\alpha\in K$ on a :
\begin{equation}
\Irr{k}{\alpha} = \prod_{\beta\in G\alpha}{(X - \beta)}.
\end{equation}
\end{thm}

\begin{thm} Soit $k$ un corps, soit $P$ un polynôme de degré $n$ dans $k[X]$ et
soit $K$ un corps de décomposition de $P$. On a :
\begin{enumerate}
\item Le groupe de Galois de $K/k$ et donc de $P$ est un sous-groupe du groupe
symétrique $\textup{S}_n$; donc son ordre divise $n!$.
\item Si $P$ est irréductible alors le groupe de Galois de $P$ agit
transitivement sur ses racines; son ordre est divisible par $n$.
\end{enumerate}
\end{thm}

Pour les corps finis, déterminer le groupe de Galois est beaucoup plus facile
comme le montre le théorème suivant :

\begin{thm}
Soit $n\geq1$ et soit $\GF{q}$ le corps à $q=p^r$ éléments, avec $p$ premier.
L'extension $\GF{q^n}/\GF{q}$ est galoisienne et son groupe de Galois
$\textup{Gal}(\GF{q^n}/\GF{q})$ est cyclique d'ordre $n$, engendré par 
l'automorphisme de Frobenius $\phi_q$.
\end{thm}
\begin{proof}
On a déjà montré que $\GF{q^n}$ était le corps de décomposition de $Q = X^{q^n} 
- X$, théorème \ref{th:isomGF}, donc que l'extension était normale, proposition 
\ref{prop:decompnorm}. De plus, comme $Q' = -1 \neq 0$, $Q$ est séparable,
proposition \ref{prop:sepderiv}, d'où $\GF{q^n}/\GF{q}$ est galoisienne
et donc :
\[|\textup{Gal}(\GF{q^n}/\GF{q})| = [\GF{q^n}:\GF{q}] = n\]
Comme le Frobenius est un $\GF{q}$-automorphisme, c'est un morphisme et un 
élément de $\GF{q}$ élevé à la puissance $q^n$ est toujours égal à lui-même, 
c'est un élément de $\textup{Gal}(\GF{q^n}/\GF{q})$. Notons $d$ son ordre, alors
$\phi_q^d = Id_{\GF{q^n}}$ ou encore, pour tout $x\in\GF{q^n}$ :
\[\phi_q^d(x) = x^{q^d} = x \Leftrightarrow x^{q^d} - x = 0\]
Le polynôme $X^{q^d} - X$ ayant au plus $d$ racines, on a $d\geq n$ et donc $d =
n$ d'où $\phi_q$ engendre $\textup{Gal}(\GF{q^n}/\GF{q})$ et donc le
résultat.\\
\end{proof}

\begin{defn}
\label{def:elemnorm}
Soit $L/K$ une extension galoisienne. On dit que $x\in L$ est un élément
\emph{normal} de $L$ si son orbite sous l'action du groupe de Galois forme une 
base de $L$ en tant que $K$-espace vectoriel; ou encore si ses conjugués forment
une base de $L$ en tant que $K$-espace vectoriel. On appelera cette base la 
\emph{base normale} de $L$.
\end{defn}

\begin{thm}
Soit $L/K$ une extension galoisienne, alors il existe un $x\in L$ dont l'orbite
par l'action du groupe de Galois forme une base de $L$ en tant que $K$-espace
vectoriel.
\end{thm}
\begin{proof}
TODO
%TODO : Je ne sais pas si ça vaut le coup de faire la preuve. J'aimerais bien
%mais bon... Tu as Artin et LiNi11 Th 3.72
\end{proof}

%TODO : Groupe de décomposition et groupe d'inertie ?
\subsection{Extensions et polynômes cyclotomiques}
%TODO : Racines de l'unités et quelques mots sur les polynômes 
%cyclotomiques et leurs extensions. Peut-être rajouter un ou deux mots sur le 
%cas des corps finis.
Soit $A$ un anneau et soit $n\in\NN$, une racine \nroot{n} de l'unité dans $A$ 
est un élément $u$ de $A$ tel que $u^n = 1$. On note $\mu_{n,A}$ l'ensemble des 
racines \nroot{n} de l'unité dans $A$ et pour tout corps $K$, on note 
$\mu_{n,\bar{K}}$ l'ensemble des racines de l'unité dans une clôture algébrique 
de $K$. Comme il ne dépend pas du choix de la clôture algébrique, on le note 
$\mu_{n,K}$. Dans le cas particulier des corps finis, une condition d'existence 
est le divisibilité de l'ordre du groupe multiplicatif par l'entier $n$.\par

\begin{prop}
\label{prop:rootcycl}
L'ensemble $\mu_{n,K}$ forme un groupe pour la multiplication dans $K$ et si 
$|\mu_{n,K}| = n$ alors il est non canoniquement isomorphe à $\zmodninv{n}$.
\end{prop}
\begin{proof}
La proposition découle d'un résultat plus général affirmant que tout sous-groupe
fini $G$ du groupe multiplicatif $K^*$ est formé de racine de l'unité et est 
cyclique. 
\begin{lem}
Soit $G$ un groupe commutatif fini. Il existe $x\in G$ dont l'ordre est le 
$ppcm$ des ordres des éléments de $G$.
\end{lem}
En effet, on sait qu'un groupe commutatif est un $\ZZ$-module (si on le note 
additivement). D'après la classification des modules de type fini, $G$ est 
isomorphe à un produit $\zmodn{a_1}\times\dots\times\zmodn{a_n}$ où 
$a_1|a_2|\dots|a_n$. Aucun des $a_i$ n'est nul, sinon $G$ serait infini; 
ce serait le produit de $\ZZ^r$ avec des groupes finis. On note $y$ la classe 
de $1$ dans $\zmodn{a_n}$ et on pose $x = (0,\dots,0,y)$. Alors $x$ est d'ordre 
$a_n$ et si on prend un élément $z\in G$, avec $z = (z_1,\dots,z_n)$, on a aussi
$a_nz = 0$ car $a_i|a_n$ pour tout $1\leq i \leq n$. L'élément $x$ est donc
l'élément recherché.\par
On en déduit qu'il existe un $z\in G$ d'ordre $n$ tel que $y^n = 1$ pour 
tout $y\in G$. Comme le nombre de racines de $X^n - 1$ sur un corps est au plus 
$n$ alors $G$ a au plus $n$ éléments. Or $z$ est d'ordre $n$ donc $G$ contient 
les éléments $z, z^2,\dots,z^n = 1$ qui sont distincts. Donc $G$ est formé par 
ces éléments et est cyclique d'ordre $n$. La proposition en découle 
naturellement.\\
\end{proof}

\begin{defn}
On appelle racine \emph{primitive} \nroot{n} un générateur de $\mu_{n,K}$. On 
note $\mu_{n,K}^{\times}$ l'ensemble de ces racines.
\end{defn}

Lorsqu'on les considère sur $\CC$, on définit à partir des racines primitives de
l'unité le polynôme suivant :
\begin{equation}
\Phi_n(X) = \prod_{\zeta\in \mu_n^{\times}}{(X - \zeta)}.
\end{equation}

C'est un polynôme irréductible à coefficients dans $\ZZ$ et de degré
$\varphi(n)$ où est $\varphi$ est l'indicatrice d'Euler. On l'appelle le
\emph{polynôme cyclotomique}, il s'agit du polynôme minimal des racines
\nroot{n} de l'unité. On appelle \emph{corps cyclotomique} le corps
de décomposition de $\Phi_n$.

\begin{rem}
Dans le cas $K = \QQ$, le corps de décomposition de $X^n - 1$ est exactement 
$\QQ(\zeta_n) \simeq \QQ[X]/\phi_n(X)$, où $\zeta_n$ est une racine primitive. 
C'est en particulier une extension algébrique simple.
\end{rem}

Le théorème crucial pour la suite du mémoire, il sert notamment à justifier une
partie de l'algorithme de Rains cyclotomique.

\begin{thm}
\label{th:polycycldecomp}
Si $K = \GF{q}$ et $(n,q) = 1$, alors $\phi_n$ se factorise en $\varphi(n)/d$ 
polynômes unitaires irréductibles dans $\GF{q}[X]$ de même degré égal à $d$. Si
on note $K^{(n)}$ le corps de décomposition de n'importe lequel de ces facteurs,
on a :
\[[K^{(n)}:K] = d\]
avec $d$ l'ordre multiplicatif de $q$ dans $\zmodn{n}$.
\end{thm}
\begin{proof}
Soit $\zeta$ une racine primitive \nroot{n} de l'unité dans $\GF{q}$, alors 
$\zeta$ appartient a un sur-corps $\GF{q^k}$ si et seulement si $\zeta^{q^k} = 
\zeta$; ce qui est équivalent à $q^k \equiv 1 \bmod n$ puisque $\zeta^n = 1$ par
définition. On pose alors $d$ égal au plus petit $K$ satisfaisant cette 
condition, dans ce cas $\zeta\in\GF{q^d}$ et ne peut pas être dans un sous-corps
de celui-ci. Ainsi, le polynôme minimal de $\zeta$ est de degré $d$ et comme on 
a choisi $\zeta$ arbitrairement, on obtient le résultat voulu.\\
\end{proof}

\begin{prop}
\label{prop:cyclgal}
Soit $n \geq 2$ et soit $\zeta$ une racine primitive \nroot{n} de l'unité sur 
$\QQ$, alors l'extension $\QQ(\zeta_m)/\QQ$ est galoisienne. En particulier, on 
a les points suivant :

\begin{enumerate}[(i)]
\item L'extension $\QQ(\zeta_m)/\QQ$ est normale.

\item On a $[\QQ(\zeta_m):\QQ] = \varphi(n)$.

\item Et $\textup{Gal}(L/K) = \zmodn{n} \simeq \mu_{n,\QQ}$.
\end{enumerate}
\end{prop}
\begin{proof}
(i) On a vu que $\phi_n$ était le polynôme minimal de toutes les racines 
\nroot{m} de l'unité et comme ces mêmes racines forment un groupe 
cyclique, elles sont toutes dans $\QQ(\zeta_m)/\QQ$; d'où l'extension est
normale.\par
(ii) $\phi_n$ est irréductible de degré $\varphi(n)$, donc on a bien
$[\QQ(\zeta):\QQ] = \varphi(n)$.\par
(iii) Soit $\sigma\in G := \textup{Gal}(L/K$, comme $\sigma(\zeta)$ est un
conjugué de $\zeta$, nécessairement $\sigma(\zeta) = \zeta^k$ pour un $k$
premier avec $n$. On a donc l'application $\psi : G \to \zmodninv{n}$ telle que
$\psi(\sigma) = k$. De plus, pour $\psi(\sigma') = k'$, on a :
\[\sigma'\circ\sigma(\zeta) = \sigma'(\zeta^k) = \zeta^{kk'}\]
d'où :
\[\psi(\sigma\circ\sigma') = \psi(\sigma)\psi(\sigma')\]
L'application est donc un morphisme de groupes. Elle est aussi injective puisque
$\psi(\sigma) = 1$ est équivalent à $\psi = \textup{Id}$. On conclut alors en
appliquant le (ii), ce qui implique que $\psi$ est un isomorphisme, et en 
rappelant la définition \ref{def:gal}.\\
\end{proof}

\section{Décomposition d'idéaux d'anneaux d'entiers}
%TODO: Tout ce qui concerne la preuve de Rains avec les idéaux 
%premiers, les normes d'ideaux, la factorisation unique, Galois, les périodes de
%Gauss, les périodes elliptiques (Mihailescu et al.) ou plus précisément ce qui 
%permet d'y arriver.
Dans cette section, on introduira la notion de corps de nombres, de leur
anneaux d'entier et des idéaux premiers sur ses anneaux. Le but est d'énoncer 
l'identité fondamentale dans le cadre des extensions Galoisiennes. Cela sera 
utile pour la sous-section \ref{sec:gaussper} et plus précisément le 
théorème. %\ref{th:gaussnormal}.
\subsection{Définitions}
%TODO :Définitions (normes, traces, éléments algébriques, etc.)
% entiers algébrique, anneaux des entiers, discriminant (?)

\subsection{Identité fondamentale}
%TODO: Idéaux premiers sur les anneaux d'entiers de corps de nombres,
%décomposition unique, résultat de la théorie de Galois.
\begin{defn}
Soit $L/K$ une extension de corps de nombres. Soit maintenant $\mathfrak{p}$ un 
idéal de $K$ et soit $\mathfrak{p}\EO_L = \prod_{i=1}^{g}{\mathfrak{P}_i^{e_i}}$
sa décomposition en idéaux premiers dans $L$ avec chaque $\mathfrak{P}_i$
distincts. On a appelle \emph{indice de ramification} de $\mathfrak{P}_i$ sur 
$\EO_K$ la valeur 
\begin{equation}
e(\mathfrak{P}_i\vert \mathfrak{p}) := e_i
\end{equation}
et le \emph{degré d'inertie} ou \emph{degré résiduel} de $\mathfrak{P}_i$ sur
$\EO_K$ la valeur
\begin{equation}
f(\mathfrak{P}_i\vert \mathfrak{p}) := f_i =
[\EO_K/\mathfrak{P}_i:\EO_K/\mathfrak{p}].
\end{equation}
\end{defn}

\begin{thm}[L'identité fondamentale]
\label{th:fundid}
Soit $L/K$ une extension de corps de nombres de degré $n$, en reprenant les 
notations ci-dessus, on a :
\begin{equation}
\sum_{i=0}^g{e_if_i} = n = [L:K]
\end{equation}
\end{thm}

\begin{thm}
\label{th:fundidgal}
Supposons que $L/K$ soit une extension de corps galoisienne de degré $n$. Soit
$\mathfrak{p}$ un idéal de $K$ et soit $\mathfrak{p}\EO_L = 
\prod_{i=1}^{g}{\mathfrak{P}_i^{e_i}}$ sa décomposition en idéaux premiers dans
$\EO_L$. Alors le groupe $\textup{Gal}(L/K)$ agit transitivement sur
$\mathfrak{P}_1,\dots,\mathfrak{P}_g$, en particulier $e_1 = \dots = e_g$ et
$f_1 = \dots = f_g$. Si de plus on pose $e = e_1$ et $f = f_1$, alors on a :
\begin{equation}
\mathfrak{p}\EO_L = \mathfrak{P}_1^e\dots\mathfrak{P}_g^e
\end{equation}
et l'identité fondamentale devient :
\begin{equation}
n = efg
\end{equation}
\end{thm}

\begin{thm}
\label{th:entiercycl}
Soit $m$ un entier positif et soit $p$ un nombre premier, l'anneau $R_m := 
\ZZ[X]/\Phi_m(X)$ est l'anneau des entiers de $\QQ[X]/\Phi_m(X)$. De plus, 
l'idéal premier $pR_m$ se ramifie dans $R_m$ si et seulement si $p$ ne divise 
pas $m$.
\end{thm}
\section{Courbes elliptiques}
%TODO: Définitions (j-invariant, nombres de points, tordues, module de
%Tate (?) etc.), résultats généraux, sur les corps finis
On introduit maintenant les courbes elliptiques et plus en particulier les
courbes elliptiques sur les corps finis. Parmi les notions importantes qui nous
serons utiles, on parlera des points de torsion, des périodes elliptiques et des
tordues d'une courbe elliptique. Dûe à la nature un peu extérieur des courbes
elliptiques par rapport au isomorphisme de corps finis, on ne s'attardera pas 
la plupart du temps, à définir les notions de géométrie algébriques. On invitera
le lecteur à consulter les trois premiers chapitre du livre de 
Silvermann\cite{Sil} pour plus de précision.

\subsection{Définitions}
%TODO: Rapide énoncé des changements de variable, comment on arrive à la forme
%"simple"; liste des différents paramètres/invariants qui nous intéressent.
%j-invariant, groupe d'automorphisme ?> c'est peut-être là que je peux utiliser
%le module de Tate...
Soit $K$ un corps et soit $f$ un polynôme homogène dans $K[X_0,\dots,X_n]$, on
appelle \emph{courbe projective} tout ensemble de la forme :
\begin{equation}
C := \lbrace{[x,y,z]\in \PP{2}(K) : f(x,y,z) = 0}\rbrace
\end{equation}
\begin{defn}
Une \emph{courbe elliptique} est un couple $(E,\EO)$ où $E$ est une courbe 
projective lisse de genre $1$ et $\EO$ est un point de $E$ qu'on appelle 
l'origine.
\end{defn}
Par abus de langage, on désignera une courbe elliptique par sa courbe projective
$E$. Il se trouve qu'il existe des fonctions $x$ et $y$ dans l'anneau de 
fonctions $K(E)$ de $E$ qui permettent d'identifier $E$ à une courbe de 
Weierstrass en envoyant $\EO$ sur $[0,1,0]$. Pour les corps $K$ de 
caractéristique différente de $2$ et $3$, cela revient à définir $E$, après 
plusieurs changements de variables et en posant $z = 1$, de la façon suivante :
\begin{equation}
\label{eq:weiersimpl}
E : y^2 = x^3 + ax + b\quad\textup{avec}\quad \Delta := -16(4a^3 + 27b^2)\neq 0,
\end{equation}
c'est cette définition qu'on utilisera pour désigner une courbe elliptique. On
appelle $\Delta$ le \emph{discriminant} de $E$ et la condition $\Delta\neq0$ est
équivalente au fait que $E$ soit lisse. Un invariant important des courbes 
elliptiques s'appelle le \emph{$j$-invariant} de $E$ :
\begin{equation}
j(E) = -1728\dfrac{(4a)^3}{\Delta}.
\end{equation}
On écrira simplement $j$ quand il n'y aura pas de confusion possible. Les deux
résultats suivant sont particulièrement intéressants :
\begin{prop}
\label{prop:j-invariant}
\begin{enumerate}[(i)]
    \item Deux courbes elliptiques sont isomorphes si et seulement si elles ont
    le même $j$-invariant.
    \item Soit $j_0\in\overline{K}$ alors il existe une courbe elliptique
    $E/K(j_0)$ telle que $j(E) = j_0$.
\end{enumerate}
\end{prop}
On note $E/K$ pour signifier que les coefficients $a, b$ sont dans $K$, c'est
équivalent à dire que $E$ est définie sur $K$. On parle de points rationnels 
d'un corps $L\subset\overline{K}$ pour les points
$P := [x,y,1]$ dans $E$ tel que $x, y$ sont dans $L$ et on note leur ensemble
$E(L)$.\par
\vspace{0.3cm}
Si on munit $E$ de la loi de composition de \emph{corde tangente}, désignée par
$\oplus$, alors $E$ possède une structure de groupe abélien avec pour élément 
neutre $\EO$. On dit qu'un point $P$ est un \emph{point de torsion} s'il existe 
un $m\in\ZZ$ tel que 
\begin{equation}
[m]P = \underbrace{P\oplus\dots\oplus P}_{\textup{m fois}} = \EO.
\end{equation}
On note alors l'ensemble des points de \emph{$m$-torsion} :
\begin{equation}
E[m] = \lbrace{P\in E(\overline{K}) : [m]P = \EO}\rbrace.
\end{equation}
On a les résultats suivants :
\vspace{0.3cm}
\begin{itemize}
\item Si $\textup{car}(K) = 0$ ou $\textup{car}(K) = p > 0$ et $p\not|\,m$ 
alors :
\begin{equation}
E[m] = \zmodn{m}\times\zmodn{m}
\end{equation}
\item Si $\textup{car}(K) = p > 0$ alors on a l'une des deux situations :
    \begin{align} 
    E[p^e] &= \lbrace{\EO}\rbrace, \quad \forall e\in\NN^{*}\\
    E[p^e] &= \zmodn{p^e}\quad \forall e\in\NN^{*}
    \end{align}
\end{itemize}
\vspace{0.3cm}

Soient $E_1$ et $E_2$ deux courbes elliptiques, une \emph{isogénie} est un
morphisme de $\varphi : E_1 \rightarrow E_2$ tel que $\varphi(\EO) = \EO$. C'est
en fait un homomorphisme de groupe et on note $\textup{Hom}(E_1, E_2)$
l'ensemble des isogénies de $E_1$ vers $E_2$. L'ensemble $\textup{End}(E)$ des
endormorphisme de $E$ est un anneau de caractéristique $0$, intègre et de rang
au plus $4$ en tant que $\ZZ$-module. L'objet qui nous intéressera plus
particulièrement est l'ensemble des endormorphisme inversible, qu'on note
$\textup{Aut}(E)$. Le théorème suivant détaille sa structure.
\begin{thm}
\label{th:autell}
Soit $E/K$ une courbe elliptique, alors $\textup{Aut}(E)$ est un groupe fini
d'ordre divisant $24$. Plus précisément, on a le tableau suivant :

\begin{equation}
\begin{tabular}{|c|c|c|}
    \hline
    \#Aut$(E)$ & $j(E)$ & car$(K)$\\
    \hline\hline
    $2$ & $j(E)\neq0,1728$ & $-$\\
    \hline
    $4$ & $j(E) = 1728$ & car$(K)\neq 2, 3$\\
    \hline
    $6$ & $j(E) = 0$ & car$(K)\neq 2, 3$\\
    \hline
    $12$ & $j(E) = 0 = 1728$ & car$(K) = 3$\\
    \hline
    $24$ & $j(E) = 0 = 1728$ & car$(K) = 2$\\
    \hline
\end{tabular}
\end{equation}
\end{thm}
Les automorphismes pour les courbes elliptiques de la forme
(\ref{eq:weiersimpl}) sont tous définis par : 
\begin{equation}
x = u^2x'\etmath y = u^3y',
\end{equation}
pour $u\in\overline{K}^{\times}$; ce sont les seuls changements de variables qui
conservent cette forme.
\subsection{Corps finis}
%TODO: Nombres de points, théorème de Hasse, Frobenius, trace, points de 
%torsions
On peut aussi considérer les courbes elliptiques sur les corps finis $\GF{q}$.
Dans ce cas, le nombre de point d'une courbe elliptique $E/\GF{q}$ n'a qu'un
nombre fini de point. Plus précisément, on a le théorème suivant :
\begin{thm}[Hasse]
\label{th:hasse}
Soit $E/\GF{q}$ une courbe elliptique défini sur $\GF{q}$ alors on a l'inégalité
suivante :
\begin{equation}
\vert{q + 1 - \#E(\GF{q})}\vert\leq 2\sqrt{q}.
\end{equation}
\end{thm}
Le morphisme de Frobenius $\phi_q$ agit sur les coordonnées des points de
$E(\overline{\mathbb{F}}_q)$ :
\begin{equation}
\phi_q(x,y) = (x^q, y^q)\etmath \phi_q(\EO) = (\EO),
\end{equation}
il s'agit d'un endomorphisme de $E$. En particulier, on a complètement 
caractérisé les points rationnels d'une courbe sur les corps finis :
\begin{equation}
P\in E(\GF{q}) \Leftrightarrow \phi_q(P) = P.
\end{equation}
\begin{prop}
Soit $E/\GF{q}$ une courbe elliptique, si on pose $t = q + 1 - \#E(\GF{q})$
alors on a :
\begin{equation}
\label{eq:frpolmin}
\phi_q^2 - t\phi_q + q = 0,
\end{equation}
en tant qu'endomorphisme de $E$ et $t$ est l'unique entier tel que l'égalité
soit vérifiée. De plus, pour $m$ premier avec $q$, $t$ est aussi l'unique entier
tel que :
\begin{equation}
t \equiv \Tr{\phi_q} \bmod m.
\end{equation}
\end{prop}
En particulier, le polynôme (\ref{eq:frpolmin}) est le polynôme minimal du
Frobenius de $E$. Par abus de langage, on parlera parfois de $t$ comme de la 
trace de $E$.

\subsection{Courbes supersingulières}
\label{sec:singcurve}
%TODO: Définitions Twists quadratiques, courbes $y^2 = x^3 + x$ et 
%$y^2 = x^3 + 1$ et leurs tordues, pour préparer le terrain pour la partie 
%elliptique de Rains
\begin{prop}
\label{prop:trtwist}
Soient $E/\GF{q}$ et $E'/\GF{q}$ deux courbes définies sur le corps à $q = p^r$
éléments, avec $p\neq2,3$. Soit $u\in\overline{K}$ tel que l'application :
\begin{equation}
\overline{u} : E\longrightarrow E,\quad(x,y)\longmapsto(u^2x,u^3y)
\end{equation}
soit $\overline{K}$-isomorphisme de courbes elliptiques. Alors on a l'une des
deux situations suivantes :
\begin{enumerate}[(i)]
    \item $u^{q-1}\notin\GF{p}$ et $E, E'$ sont supersingulières;
    \item $u^{q-1}\in\GF{p}$ et $t_{E'} = \alpha t_E$ où, si on note $n$ l'ordre
    de $u^{q-1}$ dans $\GF{p}$, alors $\alpha$ est le réprésentant de l'unique
    racine de $X^n - 1 \bmod q$ de module $\vert{\alpha}\vert < q/2$ tel que
    $\alpha = u^{q-1}\bmod p$.
\end{enumerate}
\end{prop}
\begin{proof}
TODO
\end{proof}

Concrètement, cela veut dire que si deux courbes sont isomorphes alors soit 
elles sont supersingulières auquel cas leurs traces sont toutes deux égales à 
$0$, soit leurs traces varies d'une racine de l'unité. On peut résumer la
situation de la manière suivante :
\vspace{0.3cm}
\begin{itemize}
    \item Si $j\neq0, 1728$ alors la courbe $E$ de $j$-invariant $j$ n'a qu'une 
    tordue, la tordue quadratique $E(u)$ pour $u$ un non-résidu quadratique. 
    Alors on a :
    \begin{equation}
    t_E = t_{E(u)} = 0\quad\textup{ou}\quad t_E = -t_{E(u)}
    \end{equation}

    \item Si $j = 1728$, alors la courbe $E$ de $j$-invariant $1728$ a quatre
    tordues, ce sont les $E(u^i)$ pour $i\in\lbrace{0,\dots,3}\rbrace$ et $u$
    non-résidu quartique. Alors on a pour tout $i\in\lbrace{0,\dots,3}\rbrace$ 
    et pour $\zeta_4$ une racine \nroot{$4$} primitive de l'unité:
        \begin{align}
        t_{E(u^i)} &= t_E = 0,\quad\textup{si\;}q\neq1\bmod4,\\
        t_{E(u^i)} &= \zeta_4^it_E\bmod q,\quad\textup{sinon.}
        \end{align}

    \item Si $j = 0$, alors la courbe $E$ de $j$-invariant $0$ a six tordues, ce
    sont les $E(u^i)$ pour $i\in\lbrace{0,\dots,5}\rbrace$ et $u$ un non-résidu
    sextique. Alors on a pour tout $i\in\lbrace{0,\dots,5}\rbrace$ et pour
    $\zeta_6$ une racine \nroot{$6$} primitive de l'unité :
        \begin{align}
        t_{E(u^i)} &= t_E = 0,\quad\textup{si\;}q\neq1\bmod3,\\
        t_{E(u^i)} &= \zeta_6^it_E\bmod q,\quad\textup{sinon.}
        \end{align}

\end{itemize}
\vspace{0.3cm}

\section{Complexité et notations}
Cette courte section nous permettra de définir quelques notions de complexité
et surtout d'introduire les notations qu'on utilisera pour les futures analyses
de complexités des algorithmes étudiés.
On exprimera le nombre d'opérations en nombre d'opérations sur $\GF{q}$. On note
$\M{n}$ le nombre d'opérations qu'il faut pour multiplier deux polynômes
de degré au plus $n$ sur $\GF{q}$. Elle a les propriétés suivantes :
\begin{equation}
\label{eq:M1}
\M{n}/n\geq\M{m}/M\textup{ si }n\geq m,\quad \M{mn} \leq m^2\M{n},
\end{equation}
pour tout $n, m\in\NN$. En particulier, elle est \emph{superlinéaire} :
\begin{equation}
\M{mn}\geq m.\M{n},\quad \M{n + m}\geq\M{n} + \M{m}, \quad \M{n}\geq n.
\end{equation}
La seconde propriété de (\ref{eq:M1}) implique que M est au plus quadratique et 
donc pour toute constante positive $c$ on a $\M{cn}\in O(\M{n})$.\par
Le nombre d'opérations nécessaires pour effectuer une mulitplication dans
$\GF{q^n}$ sera noté $\E{n}\in O(\M{n})$, elle est aussi superlinéaire. De même,
on note $\I{n}$ le nombre d'opérations pour inverser un élément dans $\GF{q^n}$,
c'est habituellement supposé être en $O(\M{n}\,\textup{log\,}n)$.
On utilisera aussi la notation $\tO$. Grossièrement, on a que $f(n)\in\tO{g(n)}$
est équivalent à $f(n)\in O(g(n)\,\textup{log}^k\,g(n)$.
\begin{defn}
On désignera $\textup{rand}(F)$ comme la fonction retournant un élément choisi
au hasard dans un ensemble $F$.
\end{defn}
\part{Isomorphisme de corps finis}
\label{deux}
D'après le théorème \ref{th:isomGF}, deux corps finis de même cardinal sont
isomorphes. Le théorème \ref{th:elemprim} assure que toute extension de corps 
fini s'obtient par adjonction d'une racine d'un polynôme irréductible. Il y
a donc autant de façons de définir une extension que de polynômes 
irréductibles de bon degré. Le problème est comment passer de l'une à 
l'autre.\par
%TODO Retravailler peut-être cette partie pour la rendre plus compréhensible
Soient $f$ et $g$ deux polynômes irréductibles de degré $n$ sur $\GF{q}$, on a :
\begin{equation}
k_1 := \GF{q}[X]/(f) \etmath k_2 := \GF{q}[Y]/(g). 
\end{equation}
On veut trouver un isomorphisme qui relie ces deux corps. Une première approche 
naïve est de chercher une racine de $f$ dans $k_2$ et d'envoyer $x = \bar{X}$ 
sur cette même racine. Cependant, cela implique de factoriser $f$, d'après 
\cite[th. 14.14]{GaGe}, factoriser complètement un polynôme de degré $n$ sur 
$\GF{q}$ prend $O(n\,\M{n}\,\textup{log}(qn))$ ou $\tO{n^2\,\textup{log\,}q}$
opérations sur $\GF{q}$. Et d'après \cite[cor. 14.16]{GaGe}, trouver une racine 
se fait en $O(M(n)\,\textup{log\;}n\textup{log}(qn))$ ou 
$\tO{n\,\textup{log\,}q}$ opérations sur $\GF{q}$. Sachant qu'on cherche une 
racine sur une extension de degré $n$, la complexité devient 
$O(\E{n}\,\textup{log\;}n\,\textup{log}(nq^n))$ ou $\tO{n^2\,\textup{log\,}q}$.
% = O~(n^3) ?
On note aussi l'existence d'une méthode établie par Allombert\cite{All} qui 
consiste essentiellement à utiliser l'algèbre linéaire pour déterminer un
isomorphisme, donc une complexité en $O(n^{\omega})$ avec $2<\omega<3$.\par
Un premier pas vers une méthode plus performante est la méthode de Pinch, 
nous allons exposé une de ses variantes dans la section suivante.

\section{Algorithme de Pinch}
La méthode de Pinch consiste à choisir un groupe $\Gamma$ défini par des 
relations algébriques sur $\GF{q}$ et un élément $\gamma\in\Gamma$ tel que:
\vspace{0.3cm}
\begin{itemize}
\item il soit d'ordre $m$ petit et défini sur $\GF{q^n}$,
\item il engendre exactement $\GF{q^n}$; et non un de ses sous-corps.
\end{itemize}
\vspace{0.3cm}
On exprime alors $\gamma$ en tant que polynôme en $x := \overline{X}$ et en 
polynôme en $y := \overline{Y}$ et on utilise ces deux expressions pour exprimer
$x$ en fonction de $y$ afin d'obtenir l'isomorphisme voulu.\par

\subsection{Méthode cyclotomique}

Pour la méthode cyclotomique on utilisera $\Gamma = k_1^{\times}$ et $\gamma$ 
sera une racine primitive \nroot{m} de l'unité.
Un corps fini contient toutes les racines \nroot{m} de l'unité si et seulement 
si l'ordre de son groupe multiplicatif est divisible par $m$. En particulier, il
contient alors des racines \nroot{m} primitives de l'unité. Soit $\zeta_m$ une
racine \nroot{m} primitive de l'unité dans $k_1$, alors l'image de $\zeta_m$ par
n'importe quel morphisme $\phi : k_1 \to k_2$ est une racine \nroot{m} primitive
de l'unité.\par
On choisit $\zeta'_m$ une racine primitive \nroot{m} de l'unité dans $k_2$
alors $\zeta_m$ s'enverra sur une puissance $\zeta'_m$ puisque le groupe 
mulitplicatif d'un corps est cyclique. On définit pour $0 < s < m$ les
applications $\phi^s : k_1 \to k_2$ , telles que :
\begin{equation}
\phi^s(\zeta_m) = (\zeta'_m)^s.
\end{equation}
Le but est de déterminer quel $\phi^s$ définit un isomorphisme.\par
Pour cela, on utilise de l'algèbre linéaire. Si on a deux bases 
$(x_i)_{i\in I}$, $(y_i)_{i\in I}$ de $k_1$, $k_2$ respectivement, telles 
qu'il existe un isomorphisme $\phi$ qui envoie $x_i$ sur $y_i$, alors il nous
suffit de calculer la matrice de passage de la base $(x_i)_{i\in I}$ à
$(y_i)_{i\in I}$ pour pouvoir déterminer ensuite $\phi(x)$. En posant $A$ la
matrice avec pour lignes les $x_i$, $B$ la matrice avec pour ligne les $y_i$ et
$C$ la matrice de passage, il faut alors résoudre le système 
\begin{equation}
AC = B.
\end{equation}
\par
Il faut maintenant déterminer les bases $(x_i)_{i\in I}$ et $(y_i)_{i\in I}$. 
Par exemple, si on trouve deux racines $\nu_1$ et $\nu_2$ de $f$ et $g$ 
respectivement, telles que $\phi(\nu_1) = \nu_2$, on pourra prendre comme bases 
$x_i = \nu_1^i$ et $y_i = \nu_2^i$.\par
Mais comme les racines $\zeta_m$ et $\zeta'_m$ sont choisies de telle sorte 
qu'elles n'engendrent aucuns des sous-corps de $\GF{q^n}$ cela veut 
dire, en particulier, qu'elles sont racines de $f$ et $g$. On pose donc 
$x_i = \zeta_m^i$ et $y_i = (\zeta'_m)^{si}$ selon l'application $\phi^s$ qu'on 
souhaite tester. On déduit l'image de $x$ de la matrice $C$, dont les 
coefficients dans la base monomiale engendrée par $y$ sont sur la deuxième 
ligne. Alors, si $\phi^s(x)$ est racine de $f$ sur $k_2$, l'application $\phi^s$
est effectivement un isomorphisme. La méthode peut se résumer ainsi :
\vspace{0.3cm}
\begin{enumerate}[1.]
\item Trouver un entier $m$ "petit" divisant $q^n - 1$ tel que 
$q$ soit premier avec $m$. 

\item Déterminer une racine \nroot{m} primitive $\zeta_m$ dans $k_1$ et
$\zeta'_m$ dans $k_2$.

\item  Déterminer pour quel $s\in \lbrace{1,\dots,m-1}\rbrace$, l'application
$\phi^s$ telle que $\phi^s(\zeta_m) = (\zeta'_m)^s$, est un isomorphisme.
\end{enumerate}
\vspace{0.3cm}
La condition $q$ premier avec $m$ est là pour assurer qu'une racine primitive
\nroot{m} de l'unité engendre exactement $\GF{q^n}$, théorème
\ref{th:polycycldecomp}. Pour le deuxième point, il suffit de prendre au hasard 
un $z\in k_1^{\times}$ et de l'élever à la puissance $(q^n - 1)/m$ jusqu'à 
tomber sur un élément d'ordre $m$, la probabalité que cela arrive est de 
$\varphi(m)/m$; il y a au plus $m$ éléments solutions de $X^m - 1$ dans 
$\GF{q^n}$ et seulement $\varphi(m)$ éléments d'ordre exactement $m$. Si $m$ est
premier et plus grand que $5$, ce qui sera bien souvent le cas, alors la chance 
de tomber sur une racine primitive de l'unité est de plus de $80\%$ et elle 
atteint $95\%$ dès que $m \geq 23$; ce n'est donc pas ce qui demandera le plus
de ressources. 
\begin{ex}
Illustrons la méthode par un exemple directement tiré de l'article de 
Pinch\cite{Pin}. On considère les deux polynômes irréductibles sur $\GF{11}$, 
$f = X^{23} + 8X^2 + X + 9$ et $g = Y^{23} + 3Y^2 + 4Y + 9$. On pose $m = 829$, 
on vérifiera qu'il divise bien $11^{23} - 1$ et qu'il est premier avec $11$; 
en fait, il est lui-même premier. On définit $k_1$ et $k_2$ comme les corps de 
rupture de $f$ et $g$ respectivement.\par
Dans cet exemple, il se trouve que $x^{\tfrac{11^{23} - 1}{829}}$ et 
$y^{\tfrac{11^{23} - 1}{829}}$ sont déjà des racines primitives 829-ième de 
l'unité, voici leurs expressions dans les bases monomiales engendrées par $x$ et
$y$ :
\begin{align*}
\alpha(x) =&\hspace{0.15cm} 7 + 8x + 4x^2 + 4x^4 + 4x^5 + 10x^6 + 4x^7 + 3x^8 + 
5x^9\\
& + 2x^{10} + 6x^{11} + 4x^{12} + 8x^{13} + 6x^{14} + 4x^{15} + 4x^{16}\\
& + 5x^{17}  + 7x^{18} + 4x^{19} + x^{20} + 8x^{22},
\end{align*}
\begin{align*}
\beta(y) =&\hspace{0.15cm}1 + y + 4y^2 + 4y^3 + 9y^4 + y^5 + 6y^6 + 3y^7 + 3y^8 
+ 3y^9\\
& + 6y^{10} + 5y^{11} + 6y^{12} + 8y^{13} + y^{14} + 9y^{15} + 4y^{16}\\
& + 3y^{17} + 5y^{18} + y^{19} + 10y^{20} + 10y^{21} + 6y^{22}
\end{align*}
Le but est de calculer les matrices de passage entre les bases
$(\alpha(x)^i)_{i\in I}$ et $(\beta(y)^{si})_{i\in I}$, pour $s < 829$.
Dans ce cas précis, il se trouve que la puissance $s = 14$ convient. En 
calculant la matrice de passage, on trouve alors les coefficients de l'image de 
$x$ dans $k_2$ sur la deuxième ligne de ladite matrice et l'image de $x$ est :
\begin{align*}
\phi(x) =&\hspace{0.15cm} 8 + 9y + y^2 + 2y^3 + 7y^4 + 4y^5 + 6y^6 + 10y^7 + 
9y^8\\
& + 10y^9 + 2y^{10} + 2y^{12} + 10y^{13} + 2y^{14} + 7y^{15} + y^{16}\\
& + 3y^{17} + 2y^{18} + 2y^{20} + y^{21} + y^{22}
\end{align*}
On peut alors vérifier que $f(\phi(x)) = 0$.
\end{ex}

\subsection{Limitations}
Le premier problème de cette méthode est qu'il n'y a pas toujours d'élément
d'ordre "petit" dans $\GF{q^n}^{\times}$. Par exemple, si on prend $p = 2$ et un
premier de Mersenne égal à $2^n - 1$, on ne trouvera pas d'élement d'ordre plus
petit que $2^n - 1$. Il y a donc un risque qu'on soit obligé de chercher des
éléments d'ordre proche de $q^n - 1$, ce qui peut poser problème. Dans l'exemple
ci-dessus, on avait :
\[q^n - 1 = 11^{23} - 1 = 895430243255237372246530\]
On obtiendrait alors une complexité exponentielle. Le scénario peut être encore 
pire avec des valeurs de $q$ et de $n$ beaucoup plus grandes.\par
Un autre problème est la recherche au hasard des racines primitives de l'unité. 
Le côté aléatoire du choix fait qu'il n'est jamais certain de pouvoir tomber sur
deux racines qui sont dans le même facteur irréductible du polynôme
cyclotomique. Il peut arriver qu'on soit obligé de tester $m-2$ valeurs de $s$
pour pouvoir enfin tomber sur un isomorphisme.\par
Il faudrait alors trouver un moyen de pouvoir travailler avec un $m$
relativement petit même si le cardinal du groupe des inversibles n'a pas de
facteur petit, et en même temps, pouvoir choisir des éléments uniques sous
l'action du groupe de Galois afin d'être sûr qu'un isomorphisme existe alors
entre les deux éléments. C'est ce que fait, entre autre, l'algorithme de Rains 
cyclotomique qu'on va étudier dans la section suivante.


\section{Algorithme de Rains : méthode cyclotomique}
La méthode de Rains cyclotomique est une extension de la méthode de Pinch
cyclotomique. Les principaux changements sont l'utilisation de petites
extensions des corps considérés afin de pouvoir trouver un $m$ plus petit et
l'usage des périodes de Gauss afin de déterminer des éléments uniques sous
l'action du groupe de Galois reliés par un isomorphisme.

\subsection{Principe}

L'algorithme se résume de la façon suivante. On commence par chercher un $m$ qui
satisfait certaines conditions permettant de trouver des racines \nroot{m}
primitive de l'unité qui engendrent les corps qui nous intéressent.\par
Si on ne trouve pas de $m$ assez petit, il se peut qu'on ait besoin de
travailler dans de petites extensions de $k_1$ et $k_2$, voir
\ref{sec:extension}. Il faut alors construire des élements uniques 
reliés par un isomorphisme entre $k_1$ et $k_2$ ou leurs extensions, ce qui est
effectué \textit{via} les \emph{périodes de Gauss} dans la section 
\ref{sec:gaussper}.\par
Si l'utilisation d'une extension a été nécessaire, alors on calcule la trace des
deux éléments obtenus afin d'avoir des générateurs de l'extension qui nous
intéresse. Une fois ceux-ci obtenus, il ne reste plus qu'à déterminer
l'isomorphisme. Ici le choix fait fera usage d'éléments dit normaux, ce qui est
traité à la section \ref{sec:elemnorm}.

\subsection{Utilisation d'une extension}
\label{sec:extension}
La première limitation de la méthode de Pinch est la taille du facteur
$m$, Rains propose alors dans son article \cite{Rai} de passer par une petite
extension de $\GF{q^n}$ afin de trouver un $m$ plus petit. En effet, imaginons 
qu'on ait deux extensions de petit degré $k'_1$ et $k'_2$ de $k_1$ et $k_2$ 
respectivement. Si on applique la méthode de Pinch à $k'_1$ et $k'_2$ et 
qu'on trouve un isomorphisme, il suffira alors de le restreindre à $k_1$.\par
L'idée est la suivante, on a les corps $k'_1$ et $k'_2$, extensions de degré $o$
petit de $k_1$ et $k_2$ respectivement, tels que $m$ petit divise $q^{n.o} -
1$ et soit premier avec $q$. On peut alors trouver deux racines primitive
\nroot{m} de l'unité $\zeta_{m}$ et $\zeta'_m$ qui engendrent $k'_1$ et $k'_2$
respectivement et tels que $\phi(\zeta_m) = \zeta'_m$. Alors on a : 
\begin{equation}
\phi(\textup{Tr\;}_{k'_1/k_1}(\zeta_{m})) =
\textup{Tr\;}_{k'_2/k_2}(\zeta'_{m}).
\end{equation}
La surjectivité de la trace permet d'assurer en particulier que ces
éléments engendrent toujours $k_1$ et $k_2$. On résume la situation par le 
schéma suivant :

\begin{equation}
\begin{tikzpicture}
\matrix(m)[matrix of math nodes,
row sep=0.5em, column sep=1.5em,
text height=2ex, text depth=1ex]
{\GF{q}[\zeta_m]& & \\
&\GF{q}[\Tr{\zeta_m}]&\\
& & \GF{p}[\Tr{\zeta_m}]\\
\GF{q^o}& & \\
&\GF{q}& \\
& & \GF{p}\\};
\path[-,font=\scriptsize,>=angle 90]
(m-1-1) edge node[below] {$o$} (m-2-2)
(m-4-1) edge node[below] {$o$} (m-5-2)
(m-2-2) edge node[below] {$r$} (m-3-3)
(m-5-2) edge node[below] {$r$} (m-6-3)
(m-1-1) edge node[left] {$n$} (m-4-1)
(m-2-2) edge node[left] {$n$} (m-5-2)
(m-3-3) edge node[left] {$n$} (m-6-3);
\end{tikzpicture}
\end{equation}

Le problème de devoir tester plusieurs candidats avant de tomber sur un 
isomorphisme est malheureusement toujours présent. C'est à ce moment qu'on 
utiliser les périodes de Gauss, introduites à la sous-section 
\ref{sec:gaussper}.

\subsection{Éléments normaux}
\label{sec:elemnorm}
Il se trouve qu'on peut améliorer le temps nécessaire pour déterminer
l'isomorphisme en exigant que les générateurs satisfassent une condition plus 
forte qu'être unique; cette condition est que les générateurs soient des 
éléments normaux. Imaginons qu'on ait calculé deux éléments normaux $v\in k_1$ 
et $w\in k_2$ tels que $\phi(v) = w$. Pour finir de résoudre le problème, il 
faut déterminer les images $\phi(x^i)$ pour $0\leq i < n$. Or comme $\phi(x^i) =
\phi(x)^i$, il suffit même de déterminer uniquement l'image de $\phi(x)$.\par
Pour cela, il faut pouvoir être capable à partir d'un élément normal $v$ et d'un
autre élément $z$, trouver les coefficients de $z$ dans la base de $v$. Puisque,
si on peut exprimer $x$ sous la forme :
\[x = \sum_{i\in I}{c_iv^{q^i}}\]
alors on aura :
\begin{align*}
\phi(x) &= \sum_{i\in I}{c_i\phi(v)^{q^i}}\\
&= \sum_{i\in I}{c_iw^{q^i}}
\end{align*}
puisque $\phi$ est un morphisme. Devoir passer de la base normale à la base 
monomial en $x$ peut se faire, non pas en inversant une matrice, mais en 
inversant l'élément d'un anneau quotient. C'est ce qui va être exposé dans cette
sous-section. Dans le cas des corps finis, on a montré que le Frobenius engendre
le groupe de Galois de $\GF{p^n}$, donc pour un $x\in\GF{p^n}$ ses conjugués 
sont $x^p, x^{p^2},\dots,x^{p^{n-1}}$. 
\begin{thm}
\label{th:nbelemnorm}
Le nombre d'élements normaux dans $\GF{p^n}$ est égal au nombre d'unités de
l'anneau $\GF{p}[X]/(X^n - 1)$.
\end{thm}
\begin{proof}
%TODO: LiNi2 3.73.. Va y'en avoir des résultats à montrer pour ça.
\end{proof}
\begin{cor}
\label{cor:tracenorm}
Soit $L/K$ une extension de corps finis de caractéristique $p$, premier. Si
$x\in L$ est un élément normal de $L$ sur $K$, alors $\textup{Tr}_{L/K}(x)$ est 
un élément normal de $K$ sur $\GF{p}$. Si de plus$[L:K]$ est une puissance de 
$p$ alors la réciproque est aussi vraie.
\end{cor}
\begin{proof}
Puisque la trace est surjective sur les corps finis,alors pour tout $y\in K$ il 
existe un $y'\in L$ tel que $y = \textup{Tr}_{L/K}(y')$. De là, il suffit alors 
de prendre l'expression de $y'$ dans la base normale engendrée par $x$ et d'y 
appliquer la trace; le fait qu'elle soit linéaire montre que 
$\textup{Tr}_{L/K}(x)$ est aussi une base normale de $K$.\par
Montrons l'autre sens. Supposons que $K = \GF{q}$ et $L = \GF{q^{p^k}}$ où $q
= p^n$. D'après le théorème \ref{th:nbelemnorm}, le nombre d'éléments normaux de
$L$ est égal au nombre d'unité dans $\GF{p}[X]/(X^{p^kn} - 1)$. Comme
$(X^{p^kn} - 1 ) = (X^n - 1)^{p^k}$, alors un élément de $\GF{p}[X]/(X^{p^kn} - 
1)$ est une unité si et seulement si sa réduction modulo $(X^n - 1)$ est une
unité dans $\GF{p}[X]/(X^n - 1)$.
Comme dans ce cas, $(X^n - 1)^{p^k} \subset (X^n - 1)$, on a $\GF{p}[X]/
(X^n - 1) \subset \GF{p}[X]/(X^{p^kn} - 1)$, une unité du premier anneau sera 
donc une unité du second; et comme la réduction de $1$ modulo quoique ce soit 
est toujours $1$, une unité de $\GF{p}[X]/(X^n - 1)$ est nécessairement une 
unité de $\GF{p}[X]/(X^{p^kn} - 1)$.\par
On en déduit que $L$ a $p^k$ fois plus d'éléments normaux que $K$. De façon plus
générale, un élément de $K$ a exactement $p^k$ antécédents pour la trace, le
nombre de conjugués, puisque l'extension est galoisienne; donc $L$ a au plus
$p^k$ fois élément normaux que $L$. La borne est atteinte dans notre cas, ce qui
conclut la démonstration.\\
\end{proof}

On développe maintenant plus en détail la partie évaluation de
l'isomorphisme. Notamment en illustrant l'isomorphisme entre les matrices
circulantes de taille $n$ sur $\GF{q}$ et l'anneau $\GF{q}[\omega]/(\omega^n -
1)$.\par
Le contexte est le suivant, imaginons qu'on soit parvenu à déterminer deux
éléments normaux $v$ et $w$ tels qu'ils soient reliés par un isomorphisme $\phi
: k_1\to k_2$. Notre but est de calculer cet isomorphisme, c'est-à-dire de
déterminer l'image de $x$ par celui-ci. Comme dit plus haut, on veut trouver les
coefficients de $x$ dans la base normale engendrée par $v$, en déduire son
image dans la base normale engendrée par $w$ et pour finir son image dans la 
base monomial engendrée par $y$. La situation se résume de la façon suivante :
\begin{equation}
\begin{tikzpicture}
\matrix(m)[matrix of math nodes,
row sep=3em, column sep=5em,
text height=2ex, text depth=2ex]
{\bigoplus\limits_{i\in I}{\GF{q}\cdot x^i} & \bigoplus\limits_{i\in
I}{\GF{q}\cdot y^i}\\
\bigoplus\limits_{i\in I}{\GF{q}\cdot v^{p^i}} & \bigoplus\limits_{i\in
I}{\GF{q}\cdot w^{p^i}}\\};
\path[->,font=\scriptsize,>=angle 90]
(m-1-1) edge node[auto] {$\phi$} (m-1-2)
(m-2-1) edge node[auto] {$\phi$} (m-2-2)
(m-1-1) edge node[left] {$\pi$} (m-2-1)
(m-2-2) edge node[right] {$\pi^{\prime}$} (m-1-2);
\end{tikzpicture}
\end{equation}

L'application $\GF{q}$-linéaire $\pi^{\prime}$ est plus ou moins gratuite, il 
suffit d'exprimer $w$ en fonction des $y^i$ ce qui prend $O(...)$ opérations, 
mais en général les éléments sont déjà exprimés en fonction de la base 
monomiale. La seule flêche qui pose un problème est celle correspondant à $\pi$ 
qui permet d'exprimer $x$ en fonction des $v^{p^i}$. C'est pour cette 
application qu'on va devoir effectuer tout le travail. Soit $z\in\GF{q^n}$ alors
il existe des $c_i\in\GF{q}$ tels que :
\begin{equation}
z = \sum_{0 \leq i < n}{c_iv^{q^i}}.
\end{equation}
En appliquant $\phi_{q^{n-j}}$ puis une forme $\GF{q}$-linéaire $\lambda
: \GF{q^n} \to \GF{q}$, on obtient :
\begin{equation}
\lambda\left(z^{q^{n-j}}\right) = \sum_{0\leq i < n}
{c_i\lambda\left(v^{q^{i+n-j}}\right)}.
\end{equation}
Soit maintenant la matrice $B = (b_{ij})_{(i,j)\in I^2}$ définie par :
\begin{equation}
b_{ij} = \lambda\left(v^{q^{i+n-j}}\right).
\end{equation}

Si la matrice est inversible alors on a terminé, puisque si on note $d_{ij}$ les
coefficients de $B^{-1}$, on a bien :
\begin{equation}
c_i = \sum_{0\leq j < n}{d_{ij}\lambda\left(z^{q^{n-j}}\right)},
\end{equation}
pour $0\leq i < n$. Reste à savoir si la matrice $B$ sera effectivement 
inversible.
D'après un résultat classique d'algèbre linéaire, toute forme linéaire peut
s'exprimer sous la forme :
\begin{equation}
x \mapsto \Tr{\alpha x},
\end{equation}
pour un certain $\alpha\in\GF{q}$. On peut donc réécrire les coefficients de $B$
sous la forme suivante :
\begin{equation}
b_{ij} = \Tr{\alpha v^{q^{i+n-j}}} = \Tr{\alpha^{q^j}v^{q^i}}.
\end{equation}
On va voir que dans notre cas, il nous suffira de choisir $\alpha = v$.

\begin{lem}
\label{lem:mattrinv}
Soit $\alpha,\beta\in\GF{q^n}$ et soit la matrice $A$ défini par les 
coefficients $a_{ij} = \Tr{\alpha^{q^i}\beta^{q^j}}$, $A$ est inversible si et 
seulement si $\alpha$ et $\beta$ sont des éléments normaux.
\end{lem}
\begin{proof}
Supposons que $\alpha$ ne soit pas un élément normal. Dans ce cas, il existe une
relation de dépendance linéaire entre les conjugués de $\alpha$, alors la 
combinaison linéaires des lignes de $A$ correspondant à cette dépendance est $0$
puisque la trace est linéaire; $A$ ne peut donc pas être inversible.\par
Inversement, supposons que $A$ ne soit pas inversible. Il existe alors une 
combinaison linéaire de ses lignes qui vaut $0$. Ou encore, il existe une 
combinaison linéaire $\alpha^{\prime}$ des conjugués $\alpha$ telle que 
$\Tr{\alpha^{\prime}\beta^{q^j}} = 0$ pour tout $j$ entre $0$ et $n-1$; 
\textit{i.e.} telle que :
\[\Tr{\alpha^{\prime}\beta + \alpha^{\prime}\beta^q + \dots + 
\alpha^{\prime}\beta^{q^{n-1}}}\]
Or la trace est surjective, donc si $\beta$ est normal cela implique que 
$\alpha^{\prime} = 0$ mais alors $\alpha$ n'est pas un élément normal; ce qui 
démontre le lemme.\\
\end{proof}
Reste maintenant à expliquer d'où vient l'anneau 
$\mathbb{F}_q[\omega]/(\omega^n - 1)$. Il se trouve que la matrice $B$ a une 
propriété particulière, elle est ce qu'on appelle une matrice \emph{circulante},
\textit{i.e.} pour obtenir la ligne suivante, il suffit de décaler tous les 
éléments vers la droite en faisant passer le dernier coefficient en première 
position. Ou plus formellement, si :
\[i_1 - j_1 \equiv i_2 - j_2 \bmod n\]
alors on a 
\[b_{i_1j_1} = b_{i_2j_2}\]
et inversement. Donc, si on considère :
\[b_{i+1j+1} = \lambda\left(v^{q^{(i+1) + n - (j+1)}}\right) = 
\lambda\left(v^{q^{i+n-j}}\right)\]
on a bien que $b_{ij} = b_{i+1j+1}$ avec $(i+1) - (j+1) \equiv i - j \bmod n$ 
comme annoncé plus haut. On a alors le théorème suivant :

\begin{thm}
\label{th:matcirciso}
Les matrices circulantes de $M_n(\GF{q})$ forment un anneau qui est isomorphe à 
$\mathbb{F}_q[\omega]/(\omega^n - 1)$ \textup{via} l'isomorphisme d'anneaux
suivant :
\begin{equation*}
\label{eq:isomconvert}
\psi : B \longmapsto \sum_{0\leq j < n}{b_{0j}\omega^j}.
\end{equation*}
\end{thm}
\begin{proof}
Montrons que l'application $\psi$ est un morphisme d'anneaux. On a bien 
évidemment $\psi(I_n) = 1$ et de manière relativement immédiate :
\begin{align*}
\psi(A + B) &= \sum_{0\leq j < n}{(a_{0j} + b_{0j})\omega^j}\\
&= \sum_{0\leq j < n}{a_{0j}\omega^j} + \sum_{0\leq j < n}{b_{0j}\omega^j}\\
&= \psi(A) + \psi(B).\\
\end{align*}
Maintenant, d'après la formule du produit de deux polynômes, on a :
\begin{align*}
\psi(A)\psi(B) &= \sum_{0\leq k < n}
{\bigg(\sum_{i+j=k}{a_{0i}b_{0j}}\bigg)\omega^k}\\
&= \sum_{0\leq k < n}{\bigg(\sum_{i\equiv k-j \bmod n}{a_{0i}b_{0(k-i \bmod n)}}
\bigg)\omega^k}.\\
\end{align*}
Mais $b_{0k-i} = b_{ik}$ puisque $0 -(k-i) \equiv i-k \bmod n$; en tenant compte
du fait que $i$ parcourt tout $\zmodn{n}$, on arrive finalement à :
\[\sum_{0\leq k < n}{\bigg(\sum_{0\leq i < n}{a_{0i}b_{ik}}\bigg)\omega^k} = 
\psi(AB).\]\par
Comme les $\omega^k$ sont linéairement indépendants par définition, pour que 
l'image d'une matrice circulante soit égale à $0$, il faut que sa première 
ligne, et donc toutes les autres, soit nulle; d'où $\textup{Ker}\,\psi = 
\{0_n\}$. La surjectivité est immédiate, il suffit de prendre les coefficients 
d'un élément quelconque $u$ de $\GF{q}[\omega]/(\omega^n - 1)$ et de définir la 
matrice $A$ de coefficients $a_{ij} = u_{(i-j \bmod n)}$ pour $i,j\in\zmodn{n}$,
où les $u_i$ sont les coefficients $u$; on aura alors $\psi(A) = u$.\\
\end{proof}

\subsection{Périodes de Gauss}
\label{sec:gaussper}
On connait désormais le type d'élément qui nous intéresse, il faut maintenant 
trouver un moyen d'en trouver qui soient uniques. Cela revient à trouver des 
éléments qui ont la même orbite sous l'action du groupe de Galois de
l'extension, ou encore, à regrouper les différentes orbites du polynôme
cyclotomiques (puisqu'il se décompose dans $\GF{q^n}$). Une bonne façon de faire
est d'utiliser les périodes de Gauss qui en plus d'être uniques sous l'action du
groupe de Galois, fournissent aussi des éléments normaux.\par
Soit $K$ un corps de nombre et $\EO_K$ son anneau des entiers. Posons 
$\mathfrak{p}$ l'idéal de $\EO_K$ au-dessus de $p$ tel que $\EO_K/\mathfrak{p} =
\GF{q}$, pour $q = p^r$. D'après le théorème \ref{th:entiercycl}, si $m$ et $p$
sont premiers entre eux, alors $p$ est non ramifié dans $R_m$ ou par abus de 
langage dans $K[\mu_m]/K := K[X]/\Phi(X)$ où $\mu_m := \mu_{m,\overline{\CC}}$. 
Alors on a le schéma suivant :
\begin{equation}
\label{schema:gaussnorm}
\begin{tikzpicture}
\matrix(m)[matrix of math nodes,
row sep=1em, column sep=3em,
text height=1.5ex, text depth=0.25ex]
{ &K[\mu_m] & \GF{q}[\mu_m] & \\
\QQ[\mu_m]& & &\GF{p}[\mu_m]\\
& K & \GF{q}& \\
\QQ& & &\GF{p}\\};
\path[-,font=\scriptsize,>=angle 90]
(m-2-1) edge node[left] {$\zmodninv{m}$} (m-4-1)
(m-3-2) edge node[left] {$G$} (m-1-2)
(m-1-3) edge node[right] {$\groupgen{q}$} (m-3-3)
(m-4-4) edge node[right] {$\groupgen{p}$} (m-2-4)
(m-2-1) edge node[auto] {} (m-1-2)
(m-4-1) edge node[auto] {} (m-3-2)
(m-1-3) edge node[auto] {} (m-2-4)
(m-3-3) edge node[auto] {} (m-4-4);
\path[->>,font=\scriptsize,>=angle 90]
(m-4-1) edge node[auto] {$\bmod p$} (m-4-4)
(m-3-2) edge node[auto] {$\bmod\mathfrak{p}$} (m-3-3)
(m-1-2) edge node[auto] {$\bmod\mathfrak{P}$} (m-1-3);
\end{tikzpicture}
\end{equation}
Où $\zmodninv{m}$ est le groupe de Galois de $\QQ[\mu_m]/\QQ$, $G$ est le groupe
de Galois de $K[\mu_m]/K$, sous-groupe de $\zmodninv{m}$ et $\groupgen{q}$
le groupe de décomposition de 
$\mathfrak{P}/\mathfrak{p}$, %($= \GF{q}[\mu_m]/\GF{q}$ ?)
on a en particulier l'inclusion $\groupgen{q}\subset G
\subset\zmodninv{m}$.

%%%%%%%%%%%%%%%%%%%%%%%%%%%%%%%%%%%%%%%%%%%%%%%%%%%%%%%%%%%%%%%%%%%%%%%%%%%%%%%%
\iffalse
\begin{equation}
\#G/\groupgen{q}=g\etmath\#\groupgen{q} = f = [K[\mu_m]:K]/g.
\end{equation}
\fi
%%%%%%%%%%%%%%%%%%%%%%%%%%%%%%%%%%%%%%%%%%%%%%%%%%%%%%%%%%%%%%%%%%%%%%%%%%%%%%%%
%TODO: Rajouter ce qui reste ? À voir ce dont on a besoin pour la
%démonstration...

\begin{defn}
Supposons qu'il existe un sous-groupe $S\subset\zmodninv{m}$ tel que
$\zmodninv{m} = \groupgen{q}\times S$. Pour toute racine \nroot{m} de
l'unité $\zeta_m$, on définit la \emph{période de Gauss} comme suit :
\begin{equation}
\eta(\zeta_m) = \sum_{\sigma\in S}{\zeta_m^{\sigma}}.
\end{equation}
\end{defn}
\begin{prop}
\label{prop:gaussperconj}
Le groupe de Galois $\textup{Gal}(\GF{q}[\mu_m]/\GF{q})\simeq\groupgen{q}$
agit transitivement sur les périodes de Gauss $\eta(\zeta_m)$ pour tout
$\zeta_m\in\eta_m^{\times}$.
\end{prop}
\begin{proof}
Soit $\tau\in\groupgen{q}$, alors :
\begin{equation}
\eta(\zeta_m)^{\tau} = \sum_{\sigma\in S}{\zeta_m^{\sigma\tau}}
\end{equation}
comme $\zmodninv{m} = \groupgen{q}\times S$ alors pour chaque
$\tau\in\groupgen{q}$ on a $\zeta_m^{\tau} = \zeta_m'$ un autre générateur
de $\mu_m$ ou encore $\eta(\zeta_m)^{\tau} = \eta(\zeta'_m)$, d'où
$\groupgen{q}$ agit transitivement sur les $\eta(\zeta_m)$.\\
\end{proof}
Le but de la section va être de démontrer le théorème suivant :

\begin{thm}
\label{th:gausspernorm}
Soit $\GF{q^n}/\GF{q}$ une extension de degré $n$ et soit un entier $m$ premier 
avec $q$, tel que :
\vspace{0.3cm}
\begin{enumerate}[(i)]
    \item $\zmodninv{m} = \groupgen{q}\times S$ pour $S$ un sous-groupe,
    \item $(\varphi(m), r) = 1$,
    \item $m$ est sans facteur carré.
\end{enumerate}
\vspace{0.3cm}
Si $\#\groupgen{q} = n$ alors les périodes $\eta(\zeta_m)$ forment une base
normale de $\GF{q^n}$ sur $\GF{q}$.
\end{thm}

La condition (ii) implique que $\groupgen{p} = \groupgen{q}$ et donc que
$\eta(\zeta_m)$ engendre $\GF{q}[\mu_m]/\GF{q}$ si et seulement s'il engendre
$\GF{p}[\mu_m]/\GF{p}$. Cela permet d'éviter des discussions supplémentaires 
lors de la preuve, on se contentera donc de traiter le cas $q = p$.\par
Pour démontrer le théorème, on va remonter d'un niveau dans la généralité pour
finalement retomber sur le résultat qu'on souhaite obtenir. On a l'anneau 
\begin{equation}
\EO_{m,p} := \GF{p}[X]/\Phi_m(X),
\end{equation}
réduction modulo $\mathfrak{p}$ de 
\begin{equation}
\EO_m := \ZZ[X]/\Phi_m(X),
\end{equation}
où $\EO_m$ est l'anneau des entiers de $\QQ[X]/\Phi_m(X)$ d'après le théorème 
\ref{th:entiercycl} et $\mathfrak{p} = p\EO_m$ est l'idéal de $\EO_m$ au-dessus 
de $p$. Comme $\QQ[X]/\Phi_m(X)$ est galoisienne, proposition 
\ref{prop:cyclgal}, le groupe de Galois $\zmodninv{m}$ agit sur $\EO_m$ et donc 
sur $\EO_{m,p}$. On notera :
\begin{equation}
\sigma_k(x) = x^k
\end{equation}
l'action du groupe de Galois. Soit maintenant $S$ un sous-groupe de 
$\zmodninv{m}$, on va étudier la forme de l'invariant $\EO_{m,p}^S$  de 
$\EO_{m,p}$ sous l'action de $S$, réduction modulo $\mathfrak{p}$ de $\EO_m^S$ 
l'invariant de $\EO_m$ sous l'action de $S$.\par
On a $\mathfrak{p} = \prod_{i=0}^g{\mathfrak{P}_i}$, où $g$ est le nombre
d'idéaux de $\EO_m$ au-dessus de $p$ et les $\mathfrak{P}_i$ sont des premiers
de $\EO_m$. En effet, l'extension est galoisienne et $p$ est ramifié dans 
$\EO_m$ puisque $m$ et $p$ sont premiers entre eux, théorème 
\ref{th:entiercycl}. Ainsi on a :
\begin{equation}
\EO_{m,p} = \EO_m/p\EO_m = \EO_m/\prod_{i=0}^g{\mathfrak{P}_i}
\end{equation}
D'après le théorème des restes chinois généralisé, on obtient finalement :
\begin{equation}
\EO_{m,p} = \bigoplus_{i=0}^g{\EO_m/\mathfrak{P}_i}
\end{equation}
Or, les $\EO_m/\mathfrak{P}_i$ sont des extensions de corps de $\ZZ/p\ZZ\simeq
\GF{p}$ de même degré, le degré résiduel $f = [\EO_m/\mathfrak{P}_i:\GF{p}]$. 
Dans ce cas, on a :
\begin{equation}
\label{eq:structinvSmodp}
\EO_{m,p}^S = \bigoplus_{i=0}^{g'}{\GF{p^{f'}}}.
\end{equation}
où $g'$ est le nombre d'idéaux $\mathfrak{P}'_i$ de $\EO_m^S$ tels que $p\EO_m^S
= \prod_{i = 0}^{g'}{\mathfrak{P}'_i}$ et $f' = 
[\EO_m^S/\mathfrak{P}'_i:\GF{p}]$.

\begin{lem}
\label{lem:nbprimeabove}
Soit $p$ un nombre premier et tel que $(m, p) = 1$. Alors le nombre d'idéaux
premiers de $\EO_m^S$ au-dessus de $p$ est l'index du sous-groupe généré par $p$
dans $\zmodninv{m}/S$.
\end{lem}
\begin{proof}
Commençons par le cas $S = \groupgen{1}$, alors d'après l'identité fondamentale,
on a $\varphi(m) = gf$, $g$ et $f$ définis comme ci-dessus. En prenant en compte
le fait que le corps résiduel $\EO_m/p\EO_m$ est un corps fini, l'action du 
groupe de Galois est uniquement constitué d'éléments de la forme $\sigma_{p^k}$.
Mais cela implique que le degré du corps résiduel est donc l'ordre de $p$ dans 
$\zmodninv{m}$ ce qui donne le résultat pour $S = \groupgen{1}$ car dans ce cas 
:
\begin{equation}
g = \varphi(m)/f = [\zmodninv{m}:\groupgen{p}].
\end{equation}

Considérons maintenant $S$ comme un sous-groupe quelconque. Pour n'importe quel
$\mathfrak{p}$ de $\EO_m$, l'idéal premier en-dessous de celui-ci dans $\EO_m^S$
est l'idéal :
\begin{equation}
\prod_{i\in\left(S_{\mathfrak{p}}S\right)/S_{\mathfrak{p}}}
{\sigma_i(\mathfrak{p})\cap\EO_m^S},
\end{equation}
où $S_{\mathfrak{p}}$ est le stabilisateur de $\mathfrak{p}$ sous l'action de
$S$. Le groupe quotient $(S_{\mathfrak{p}}S)$ permet de trouver les conjugués de
$\mathfrak{p}$ multiplier par une puissance de celui-ci. En quotientant par
$S_{\mathfrak{p}}$, on obtient une puissance de $\mathfrak{p}$ égale à $1$ et en
intersectant avec $\EO_m^S$, on obtient alors la décomposition de $p\EO_m^S$.
Réciproquement, si $i\notin S_{\mathfrak{p}}S$ alors $\sigma_i(\mathfrak{p})$ 
n'est pas un conjugué de $\mathfrak{p}$ et donc n'est pas au-dessus de $p$.\par
Le cardinal de $(S_{\mathfrak{p}}S)/S_{\mathfrak{p}}$ est le même pour tous les
$\mathfrak{p}$ au-dessus de $p$ puisqu'ils sont conjugués. Alors le nombre de
premiers de $\EO_m^S$ au-dessus de $p$ est égal à :
%TODO: Transition un poil trop rapide...
\begin{equation}
\dfrac{[\zmodninv{m}:\groupgen{p}]}{[\groupgen{p}S:\groupgen{p}]} =
[\zmodninv{m}:\groupgen{p}S],
\end{equation}
d'après le troisième théorème d'isomorphisme. Ce qu'on peut encore écrire
:
\begin{equation}
[\zmodninv{m}/S:\groupgen{p}].
\end{equation}
% Je comprends pas tout le bordel que Rains fait pour obtenir le nombre d'idéaux
% au-dessus de p. Comme L^S/Q est aussi de Galois, il suffirait d'utiliser les
% résultats sur le cardinal du groupe de décomposition ? Il y aurait juste une
% petite discussion par rapport au fait qu'on considère un sous-groupe de G =
% Gal(L/Q).
\vspace{0.3cm}
\emph{Démonstration alternative du cas général}\par
Considérons mainenant $S$ sous-groupe quelconque, comme $\QQ[\mu_m]/\QQ$ est 
commutative alors $\QQ[\mu_m]^S/\QQ$ est aussi galoisienne de groupe de Galois 
$\zmodninv{m}/S$. Alors on peut reporter l'étude à l'extension 
$\QQ[\mu_m]^S/\QQ$, on regarde l'action de $\zmodninv{m}/S$ sur les
$\mathfrak{P}'_i$ tels que $p\EO_m^S = \prod_{i=0}^{g'}{\mathfrak{P}'_i}$. Comme
il n'y a qu'une seule orbite de cardinal $g'$ alors la formule des classes
donne :
\begin{equation}
g' = \frac{\#\zmodninv{m}/S}{\#\groupgen{p}} = 
[\zmodninv{m}/S:\groupgen{p}].
\end{equation}
\end{proof}

\begin{cor}
L'anneau $\EO_{m,p}^S$ se factorise en une somme directe de
$[\zmodninv{m}/S:\groupgen{p}]$ corps tous isomorphes à $\GF{p^n}$, où $n$ est 
l'ordre de $p$ dans $\zmodninv{m}/S$.
\end{cor}
\begin{proof}
Il suffit d'appliquer le lemme \ref{lem:nbprimeabove}, l'équation
(\ref{eq:structinvSmodp}) et le fait que $\groupgen{p}$ étant le groupe
décomposition de $\mathfrak{p}$ son cardinal $\#\groupgen{p}$ est égal à $e'f'$,
où $e'$ est l'indice de ramification de $\mathfrak{p}$ dans $\EO_m^S$, qui est 
égal à $1$, et $f'$ le degré du corps résiduel, qui est égal à $n$.\\
\end{proof}
En particulier si $[\zmodninv{m}/S:\groupgen{p}] = 1$ alors $\EO_{m,p}^S$ est
un corps; c'est en fait exactement $\GF{p^n}$, avec $\#\groupgen{p} = n$. Il 
reste alors à montrer que les périodes de Gauss engendrent effectivement 
$\EO_{m,p}^S$.
\begin{lem}
Soit $m$ un entier sans facteur carré, alors les périodes $\eta(\zeta_m^k)$
engendrent $\EO_{m,p}^S$, où $k\in\zmodninv{m}/S$.
\end{lem}
\begin{proof}

Commençons par considérer le cas $S = \langle1\rangle$ et $m$ premier. Alors on 
a $\eta(\zeta_m^k) = \zeta_m^k$ et, par définition, les $\zeta_m^k$ engendrent 
$\EO_{m,p}$ pour $0 \leq k\leq m$. Il suffit alors de montrer que $1$ peut 
s'exprimer en fonction des $\zeta_m^k$ pour $k > 0$, puisque $k = 0$ n'est pas 
dans $\zmodninv{m}$. Mais $m$ est premier, alors le polynôme $\phi_m$ 
donne exactement :
\begin{equation}
\textup{Tr}(\zeta_m) = \sum_{0 < k < m}{\zeta_m^k} = 1.
\end{equation}

Changeons alors seulement l'hypothèse $m$ premier en $m$ sans facteur carré, on 
va raisonner par récurrence. En particulier, on peut écrire $m = m_1m_2$ avec 
$(m_1,m_2) = 1$ et le résutlat ci-dessus est encore valable pour $m_1$ et 
$m_2$. D'après le théorème des restes chinois, on a :
\[\zmodninv{m} \simeq \zmodninv{m_1} \times \zmodninv{m_2},\]
ainsi, \textit{via} cette isomorphisme on peut exprimer un élément
$\eta(\zeta_m^k)$ de $\EO_{m,p}$ comme un produit de deux éléments
$\eta(\zeta_m^{(k_1)})$ et $\eta(\zeta_m^{k_1})$ dans $\EO_{m_1,p}$ et 
$\EO_{m_2,p}$, avec $k_1\in\zmodninv{m_1}$ et $k_2\in\zmodninv{m_2}$ 
respectivement, et inversement.
%TODO: C'est vraiment utile tout ce qui suit ?
%%%%%%%%%%%%%%%%%%%%%%%%%%%%%%%%%%%%%%%%%%%%%%%%%%%%%%%%%%%%%%%%%%%%%%%%%%%%%%%%
\iffalse
En effet, prenons un $k\in\zmodninv{m}$ alors on peut l'écrire sous la forme :
\[k = uk_2m_1 + vk_1m_2\]
avec $k_1\in\zmodninv{m_1}$, $k_2\in\zmodninv{m_2}$ et $u,v\in\mathbb{Z}$ tels 
qu'ils soient premiers avec $m_2$ et $m_1$ respectivement. On a donc :
\begin{align*}
z_S^{(k)} &= \zeta_m^k\\
&= \zeta_m^{uk_2m_1 + vk_1m_2}\\
&= (\zeta_m^{um_1})^{k_1}(\zeta_m^{vm_2})^{k_2}\\
\end{align*}
On a bien que $\zeta_m^{um_1}$ est une racine \nroot{$m_2$} puisque 
$m_1m_2 = m$, mais c'est aussi une racine primitive car $(u,m_2) = 1$ et 
$(m_1, m_2) = 1$; donc la seul façon d'avoir $kum_1 = m$ est de poser $k = m_2$.
Le raisonnement est exactement le même pour l'autre terme. D'où le résultat :
\[z_S^{(k)} = z_S^{(k_1)}z_S^{(k_2)}\].\par
Pour l'autre sens, on prend le produit suivant :
\[z_S^{(k_1)}z_S^{(k_2)} = \zeta_{m_1}^{k_1}\zeta_{m_2}^{k_2}\]
Si on l'élève à la puissance $m_1$ ou $m_2$, le produit ne pourra pas être égal 
à $1$. En revanche, si on l'élève à la puissance $m$, on obtiendra bien $1$, 
d'où le fait que c'est une racine primitive \nroot{m} de l'unité. Alors on 
pourra toujours le réécrire sous la forme : 
\[\zeta_m^k = z_S^{(k)}\]
avec $k\in\zmodninv{m}$, ce qu'on cherchait.\par
\fi
%%%%%%%%%%%%%%%%%%%%%%%%%%%%%%%%%%%%%%%%%%%%%%%%%%%%%%%%%%%%%%%%%%%%%%%%%%%%%%%%
Ainsi, comme les $\eta(\zeta_m^{k_1})$ engendrent $\EO_{m_1,p}$ et les 
$\eta(\zeta^{k_2})$ engendrent $\EO_{m_2,p}$ par réccurence, alors les 
$\eta(\zeta^k)$ engendrent $\EO_{m,p}$, à nouveau par le théorème des restes 
chinois.\par
Pour finir, on considère $S$ un sous-groupe quelconque du groupe de Galois. 
Prenons un $x$ dans $\EO_{m,p}^S$, alors on peut écrire :
\begin{equation}
x = \sum_{i\in S}{\sigma_i(y)}
\end{equation}
pour un $y$ dans $\EO_{m,p}$; l'action d'un $\sigma_i$ sur $x$ ne sera alors
qu'une permutation des termes de la somme. De plus, on peut aussi écrire $y$ 
sous la forme :
\begin{equation}
y = \sum_{(k,m)=1}{c_k\sigma_k(\zeta_m)}
\end{equation}
c'est ce qu'on a prouvé plus haut. En combinant les deux expression on obtient 
alors :
\begin{align*}
x &= \sum_{i\in S}{\sum_{(k,m)=1}{c_k\sigma_i\sigma_k(\zeta_m)}}\\
&= \sum_{(k,m)=1}{c_k\eta(\zeta_m^k)}\\
\end{align*}
ce qu'il fallait démontrer.\\
\end{proof}

Ainsi, si $\zmodninv{m}/S = \groupgen{p}$ et $\#\groupgen{p} = n$, alors les 
$\eta(\zeta_m^k)$ avec $k\in\groupgen{p}$ sont conjugués dans $\EO_{m,p}^S = 
\GF{p^n}$, d'après la proposition \ref{prop:gaussperconj}, et si $m$ est sans 
facteurs carrés alors ils l'engendrent d'après le lemme précédent, ce sont donc 
des élements normaux de $\GF{p^n}$; la condition (ii) implique que ce sont aussi
des éléments normaux de $\GF{q^n}$. Ce qui achève la preuve du théorème 
\ref{th:gausspernorm}.

\subsection{Algorithme et analyse de complexité}
\label{sec:algcompcycl}
Maintenant qu'on a tous les résultats théoriques qu'il nous faut, on va pouvoir
énoncer et justifier les aglorithmes pour déterminer l'isomorphisme. 
On rappelle brièvement la situation, on a $f$ et $g$ deux polynômes 
irréductibles de degré $n$ distincts sur $\GF{q}$, on note :
\[k_1 := \GF{q}[X]/(f)\etmath k_2 := \GF{q}[Y]/(g)\]
avec $q = p^r$ pour $r$ un entier naturel non nul et $p$ un nombre premier
impair. Il nous faut alors déterminer deux éléments normaux uniques sous
l'action du groupe de Galois de $\GF{q^n}/\GF{q}$ et en déduire l'image de $x :=
\overline{X}$ pour déterminer complètement l'isomorphisme. 

\subsubsection*{Recherche d'un élément normal unique}
Comme cela a été exposé au point \ref{sec:extension}, si on ne trouve pas de $m$
assez petit alors il faut passer par une extension de petit degré $o$ de
$\GF{q^n}$ afin de pouvoir appliquer les résultats à l'extension $\GF{q^{n.o}}$
et redescendre sur $\GF{q^n}$.\par
La façon dont on va trouver $m$ est détaillé dans la section
\ref{sec:recherchemcycl}. En résumé, on cherche le plus petit entier
$m$ sans facteurs carrés tel qu'il existe $o\in\NN^*$ tel que :
\vspace{0.3cm}
\begin{enumerate}[(1)]
    \item $(q, m) = 1$ et $q$ soit d'ordre $\ord{m}{q} :=n.o$ dans 
    $\zmodninv{m}$,
    \item $\varphi(m)/\ord{m}{q}$ et $\ord{m}{q}$ soient premiers entre eux,
    \item $n$ et $o$ soient premiers entre eux,
    \item $(\varphi(m), r) = 1$ pour $q = p^r$,
    \item $\zmodninv{m} = \groupgen{q^o}\times S$ pour $S$ un sous groupe
    de $\zmodninv{m}$ d'ordre $\varphi{m}/(n.o)$.
\end{enumerate}
\vspace{0.3cm}
Les conditions $(1)$, $(2)$ et $(5)$ sont là pour assurer que les racines 
\nroot{m} primitives engendrent exactement $\GF{q^{n.o}}$ (\emph{cf.} 
\ref{schema:gaussnorm}). La condition $(3)$ est d'ordre pratique et permet de
construire des extensions à partir de polynôme à coefficients directement dans 
$\GF{q}$ si besoin est. La quatrième condition permet de s'assurer que
$\groupgen{p} = \groupgen{q}$ et donc que les résultats sur
$\GF{p}$ soient encore valables sur $\GF{q}$. Enfin, la dernière condition est
aussi la plus importante puisqu'elle permet d'appliquer le théorème 
\ref{th:gausspernorm}, elle est aussi en particulier la conséquence des points
$(1)$ et $(2)$.
\begin{rem}
Il se trouve que l'algorithme ne peut pas fonctionner si $q$ est une puissance 
$r$ de $p$ non première avec $n$, le degré de l'extension. En effet, on a 
$\groupgen{q}\times S = \zmodninv{m}$, donc par définition $p = sq^a\bmod 
m$ pour un $s\in S$ et $0 \leq a < n$, alors $q=s^rq^{ar} \bmod m$. 
Or, $q$ ne peut être écrit que d'une seule façon comme produit d'un élément de 
$S$ et d'une puissance de $q$, d'où $s^r = 1$ et $q^{ar - 1} = 1 \bmod m$. 
Ce qui implique que $n$ divise $ar - 1$ et donc que $(n, r) = 1$.\par
Une solution possible serait de calculer un générateur $\alpha$ de 
$k/\GF{p^{d/c}}$, son polynôme minimal serait alors uniquement défini sur 
$\GF{p^{d/c}}$ mais se scinderait en $c$ facteurs sur $\GF{q}$, ce qui nous 
obligerait à réutiliser la méthode trial-and-error de Pinch.
\end{rem}
\vspace{0.3cm}
Alors selon la valeur de $o$ lorsqu'on cherche $m$, \emph{i.e.} si $o = 
\ord{m}{q} /n = 1$ ou $> 1$, il y aura deux façons différentes de procéder, 
c'est ce qu'on va détailler plus bas.
\paragraph{$\bullet$ Cas rationnel $o = 1$}
Imaginons qu'on ait trouvé un $m$ satisfaisant les conditions $(1)$ à $(5)$ avec
$o = 1$ ou $q$ directement d'ordre $n$ . Dans ce cas, on peut travailler 
directement dans les corps $k_1$ et $k_2$ pour trouver calculer les périodes de
Gauss. Le tout est résumé dans l'algorithme \ref{alg:gausspersansext}.

\begin{algorithm}
\caption{Détermination d'un élément normal unique dans un corps fini}
\label{alg:gausspersansext}
\begin{algorithmic}[1]
\REQUIRE $k$ un corps fini de cardinal $q^n$; $m$ un entier divisible par $n$
satisfaisant les conditions (1) à (5).
\ENSURE $\eta$, un élément normal unique de $k$ sous l'action de Galois.
\bigskip
\REPEAT
    \STATE $\zeta_m \leftarrow \textup{rand}(k)^{(q^n - 1)/m}$
\UNTIL{$\zeta_m^m = 1$\quad\&\quad$\forall d\mid m,\;d\neq m,\;\zeta_m^d\neq1$}
\STATE $\eta \leftarrow \sum_{a\in S}{\zeta_m^a}$
\RETURN $\eta$
\end{algorithmic}
\end{algorithm}
On veut déterminer sa complexité, pour trouver une racine primitive 
\nroot{m} de l'unité, on prend un élément au hasard dans $\GF{q^n}$, on l'élève 
à la puissance $(q^n - 1)/m$ et on teste si le résultat élevé à la puissance 
$m/\ell$ vaut $1$ ou non pour $\ell$ un diviseur premier strict de $m$. On 
s'attend à ce que cela réussisse après $O(1)$ essaies; chaque essaie prenant 
$O(\E{n}n\,\textup{log\;}q)$ opérations.\par
On doit ensuite calculer la période de Gauss $\eta := \eta(\zeta_m)$,
d'après ce qui précéde cela revient à faire $\varphi(m)/n$ exponentiation dans
$\GF{q^n}$ d'exposant au plus $m$, cela se fait $O((m\textup{log\;}m)\E{n}/n)$
opérations.\par
Ainsi, le coût total de l'algorithme est de $O(\E{n}(n\,\textup{log\;}q +
m\,\textup{log\;}m/n))$ opérations dans $\GF{q}$. Tant que $m\in o(n^2)$, on 
peut négliger les termes en $m$ et se contenter d'une complexite en 
$O(\E{n}n\,\textup{log\;}q)$.

\begin{prop}
\label{prop:algsansext}
L'algorithme \ref{alg:gausspersansext} est correct et de complexité :
\begin{equation}
O(\E{n}n\,\textup{log\,}q).
\end{equation}
\end{prop}
\begin{proof}
Les conditions (1), (2) et (5) permettent d'appliquer le théorème 
\ref{th:gausspernorm} qui assure que $\eta$ est un élément normal de 
$\GF{q^n}$ sur $\GF{q}$. Le fait qu'il soit unique sous l'action du groupe de 
Galois est justifié par la proposition \ref{prop:gaussperconj}.\\
\end{proof}

\paragraph{$\bullet$ Cas général $o > 1$}
Si jamais $q$ n'est pas d'ordre $n$ alors on doit passer par une extension de
degré $o = \ord{m}{q}/n$ premier avec $n$; si $(o,n)\neq1$ on passe au
$m$ suivant (\emph{cf.} \ref{sec:recherchemcycl}). Pour cela on définit $h$ dans
$\GF{q}[X]$, irréductible de degré $o$ et on définit les extensions :
\begin{equation}
k_1' := k_1[U]/(h) \etmath k_2' := k_2[V]/(h)
\end{equation}
dans lesquelles on travaillera. Si on pose $\eta_1 = \sum_{\sigma\in
S}{\zeta_m^{\sigma}}$ et $\eta_2 = \sum_{\sigma\in S}{\zeta_m'^{\sigma}}$ alors
on peut résumer la situation au schéma suivant :
\begin{equation}
\begin{tikzpicture}
\matrix(m)[matrix of math nodes,
row sep=3em, column sep=3em,
text height=1.5ex, text depth=0.25ex]
{\GF{q}[\eta_1] & \GF{q}[\eta_2]\\
\GF{q}[\Tr{\eta_1}] & \GF{q}[\Tr{\eta_2}]\\};
\path[->,font=\scriptsize,>=angle 90]
(m-1-1) edge node[auto] {$\phi$} (m-1-2)
(m-2-1) edge node[auto] {$\phi$} (m-2-2);
\path[-,font=\scriptsize,>=angle 90]
(m-1-1) edge node[left] {$o$} (m-2-1)
(m-1-2) edge node[right] {$o$} (m-2-2);
\end{tikzpicture}
\end{equation}

\begin{rem}
Si on rajoute en plus la condition $(n,r) = 1$, on peut alors directement 
prendre des polynômes à coefficients dans $\GF{p}$, cela est aussi valable pour 
l'algorithme \ref{alg:gausspersansext}.
\end{rem}
\vspace{0.3cm}

\begin{algorithm}
\caption{Détermination d'un élément normal unique dans un corps finis
\textit{via} une extension}
\label{alg:gaussperavecext}
\begin{algorithmic}[1]
\REQUIRE $k$ un corps fini de cardinal $q^n$, $m$ un entier satisfaisant les 
conditions (1) à (5), $h$ un polynôme de degré $o$ à coefficients dans $\GF{q}$.
\ENSURE $\eta$, un élément normal unique de $k$ sous l'action de Galois.
\bigskip
\STATE $k' \leftarrow k/(h)$
\REPEAT
    \STATE $\zeta_m \leftarrow \textup{rand}(k')^{(q^{n.o} - 1)/m}$
\UNTIL{$\zeta_m^m = 1$\quad\&\quad$\forall 
    d\mid m,\;d\neq m,\;\zeta_m^d\neq1$}
\STATE $\alpha \leftarrow \sum_{a\in S}{\zeta_m^a}$
\STATE $\eta \leftarrow \textup{Tr}_{k'/k}(\alpha)$
\RETURN $\eta$
\end{algorithmic}
\end{algorithm}
%TODO: Analyse de complexité
\begin{prop} 
\label{prop:algavecext}
L'algorithme \ref{alg:gaussperavecext} est correct et de complexité :
\begin{equation}
O(\E{no}(no\textup{log\,}q + m\,\textup{log\,}m/(no))).
\end{equation}
\end{prop}
\begin{proof}
Le théorème \ref{th:gausspernorm} et la proposition \ref{prop:gaussperconj}
assure que $\alpha$ est un élément normal unique de $\GF{q^{n.o}}$. Il suffit 
alors d'appliquer le corollaire \ref{cor:tracenorm} pour assurer que $\eta :=
\Tr{\alpha}$ soit un élément normal unique de $\GF{q^n}$.\\
\end{proof}

\subsubsection*{Détermination de l'isomorphisme}
Reste alors à déterminer exactement l'isomorphisme, c'est-à-dire calculer son
image en $x$. Comme on l'a expliqué au début de la sous-section, il suffit juste
de récupérer les coefficients de $x$ par rapport à la base normale engendrée par
$\eta(\zeta_m)$; alors il nous restera plus qu'à reporter les coefficients 
trouvés dans la base normale engendrée par $\eta(\zeta'_m)$. On rappelle
l'existence de l'isomorphisme reliant les matrices circulantes de tailles $n$ 
sur $\GF{q}$ et l'anneau $\GF{q}[\omega]/(\omega^n - 1)$.

\begin{algorithm}
\caption{Conversion de la base polynomiale vers la base normale}
\label{alg:convert}
\begin{algorithmic}[1]
\REQUIRE $z\in\GF{q^n}$, $v$ élément normal, $n$ degré de l'extension, $p$ 
caractéristique, $\omega$ défini ci-dessus.
\ENSURE $c$, tuple contenant les coefficients de $z$ dans la base 
normale engendrée par $v$.
\bigskip
\FOR{$i = 0$ \TO $n-1$}
    \STATE $B_i \leftarrow \Tr{v\times v^{q^{n-(i+1)}}}$
\ENDFOR
\STATE $I \leftarrow (\sum_{i = 0}^{n-1}{B_i\cdot \omega^i})^{-1}$
\FOR{$i = 0$ \TO $n-1$}
    \STATE $T_i \leftarrow \Tr{v\times z^{q^{n-i}}}$
\ENDFOR
\FOR{$i = 0$ \TO $n-1$}
    \STATE $c_i \leftarrow \sum_{j=0}^{n-1}{I_{(j-i)\bmod n}T_j}$
\ENDFOR
\RETURN $c$

\end{algorithmic}
\end{algorithm}
On commence par élever à la puissance $q$ plusieurs polynômes sur $\GF{q}$, 
on les multiplie par un autre polynôme et enfin on calcule la trace de chacun 
de ces produits; on effectue alors $n$ fois cette opération. Donc un 
$O(n\E{n}\textup{log}\;q + n\E{n}) + n\textup{T}(n) = O(n\E{n}\textup{log}\;q + 
n\E{n})$ opérations, 
où $\textup{T}(n)$ est la complexité du calcul de la trace d'un élément
de $\GF{q^n}$ et $\textup{T}(n) \in O(\E{n})$.
Par la suite, il s'agit dans un premier temps de la muliplication par un 
scalaire d'éléments de $\mathbb{F}_p[\omega]/(\omega^n - 1)$ puis de $n$ sommes 
de polynômes. Ensuite, on inverse le polynôme et on prend la trace de l'élément 
qu'il représente; ce qui nous donne $O(n + \E{n}.\textup{log }n)$.\par
Des lignes $5$ à $7$, il s'agit de la même opération que des lignes 1 à 3, d'où 
encore une complexité de $O(n\E{n}\textup{log}\;q + n\E{n})$.
Les lignes 8 à 10 consistent en $n$ sommes de $n$ multiplications dans le corps 
de base $\GF{q}$, \textit{i.e.} un grand $O(n)$.
Au final, la complexité de l'algorithme est égale à:
\[O(n\E{n} + n\E{n}\textup{log }q +  \E{n}\textup{log }n + n\textup{T}(n) + n)\]
ou de manière plus simple, elle est en $O(n\E{n}\textup{log}\;q)$ qui domine le 
reste.

\begin{prop}
\label{prop:algconvert}
L'algorithme \ref{alg:convert} est correct et de complexité :
\begin{equation}
O(n\E{n}\textup{log}\;q).
\end{equation}
\end{prop}
\begin{proof}
D'après le point précédent, on a :
\[\textup{Tr}\left(z^{p^{n-j}}\right) = \sum_{0\leq i < n}{c_i\textup{Tr}
\left(v^{p^{i+n-j}}\right)}\]
et la matrice $B$ définie par :
\[b_{ij} = \Tr{\alpha v^{p^{i+n-j}}} = \Tr{\alpha^{p^j}v^{p^i}}\]
On sait qu'elle est inversible grâce au lemme \ref{lem:mattrinv}, donc si on
note $d_{ij}$ les coefficients de la matrice $B^{-1}$, on a bien :
\[c_i = \sum_{0\leq j < n}{d_{ij}\textup{Tr}\left(z^{p^{n-j}}\right)}\]
Le théorème \ref{th:matcirciso} permet de conclure.\\
\end{proof}

\subsubsection*{Algorithme principal}

On peut désormais énoncer complètement l'algorithme de Rains cyclotomique : on
commence par chercher le plus petit entier $m$ qui satisfait les conditions
$(1)$ à $(5)$, on calcule alors les périodes de Gauss dans $k_1$ et $k_2$ ou
dans leurs extensions et on finit par calculer les coefficients de $x$ dans la
base normale engendrée par $\eta_2$; en réécrivant $\eta_2$ en fonction de $y$,
on retrouve alors l'image de $x$ dans la base monomiale engendrée par $y$ et on
peut calculer l'image de n'importe quel élément de $k_1$.

\begin{algorithm}
\caption{Détermination d'un isomorphisme entre deux corps finis de même
cardinal}
\label{alg:rainscycl}
\begin{algorithmic}[1]
\REQUIRE $k_1$, $k_2$ deux corps finis à $q^n$ éléments
\ENSURE $\phi(x)$ l'image de la classe de $X$ par un isomorphisme $\phi : k_1
\to k_2$. 
\bigskip
\STATE Chercher le plus petit entier $m$ satisfaisant $(1)$ à $(5)$
\STATE $o \leftarrow \ord{m}{q}/n$
\IF{$o = 1$}
    \STATE Calculer $\eta_1$ en utilisant l'algorithme \ref{alg:gausspersansext}
    \STATE Calculer $\eta_2$ en utilisant l'algorithme \ref{alg:gausspersansext}
\ELSE
    \STATE Trouver un polynôme irréductible $h$ de degré $o$ sur $\GF{q}$
    \STATE Calculer $\theta_1$ en utilisant l'algorithme 
    \ref{alg:gaussperavecext}
    \STATE Calculer $\theta_2$ en utilisant l'algorithme 
    \ref{alg:gaussperavecext}
    \STATE $\eta_1 \leftarrow \Tr{\theta_1}$
    \STATE $\eta_2 \leftarrow \Tr{\theta_2}$
\ENDIF
\STATE Construire $\GF{q}[\omega]/(\omega^n - 1)$
\STATE Calculer les $c_i$ en utilisant l'algorithme \ref{alg:convert}

\RETURN $\sum_{i\in\lbrace{0,\dots,n-1}\rbrace}{c_i\eta_2^{q^i}}$

\end{algorithmic}
\end{algorithm}

\begin{prop}
L'algorithme \ref{alg:rainscycl} est correct.
\end{prop}
\begin{proof}
Il suffit d'appliquer successivement les propositions $\ref{prop:algsansext}$ et
$\ref{prop:algconvert}$ ou $\ref{prop:algavecext}$ et $\ref{prop:algconvert}$,
selon la valeur de $o$.\\
\end{proof}

\section{Algorithme de Rains : méthode elliptique}
%TODO: Résumer/Détailler le papier de Luca, rajouter les résultats sur
%les twists, les périodes, détaillés l'algorithme et sa complexité, le prouver 
%(probablement fait avec le papier de Luca et Mihailescu pour les périodes ?)
%, détailler les différentes étapes, les paramètres, la façon de trouver m 
%etc.
Utiliser pour groupe algébrique $\Gamma$ le groupe des points d'une courbe
elliptique avait déjà été proposé par Pinch\cite{Pin}. Comme pour la méthode
cyclotomique, on cherche des points d'ordre petit $m$ divisant le nombre de
point d'une courbe elliptique et, en utilisant les mêmes techniques que pour la
méthode faisant usage des racines de l'unité, on essaie de trouver un
isomorphisme reliant leurs abscisses. Cependant, on peut éviter d'utiliser la
méthode trial-and-error de Pinch en utilisant judicieusement la structure des
groupe de Galois en jeu. 
\subsection{Principe}
%TODO: Choix d'un point d'ordre m sur une bonne courbe elliptique engendre 
%l'extension, utilisation des périodes elliptiques afin d'avoir un élément 
%stable par l'action du groupe de Galois (Énoncer les lemmes etc.)
On commence par chercher un entier $m$ et une courbe $E/\GF{q}$ satisfaisant
plusieurs conditions, notamment des conditions reliant la trace du Frobenius de
$E/\GF{q}$, $m$ et le nombre de points rationnels de $E$ sur $\GF{q^n}$; ce sera
étudié plus en détail dans la section \ref{sec:choixparam}.\par
Il faut alors calculer des points d'ordre exactement $m$ sur $E/k_1$ et $E/k_2$.
Selon le $j$-invariant et la trace (du Frobenius) de $E$, on calcule alors  
ce qu'on appelle les \emph{périodes elliptiques} qui consistent principalement à
sommer les abscisses de l'orbites d'un point de $m$-torsion sous l'action d'un
sous-groupe de $\zmodninv{m}$, ce sera détailler dans la section
\ref{sec:perell}. On aura alors obtenu des éléments normaux et il nous suffira 
de procéder comme dans la méthode cyclotomique \ref{sec:algcompcycl} afin
d'obtenir l'isomorphisme désiré.

\subsection{Choix des paramètres}
\label{sec:choixparam}
%TODO: Justifier le choix de m par rapport à n et q, tout ce qui concerne la 
%valeur propre d'ordre n et donc le choix de traces qui s'en suit, expliquer 
%comment on pick les courbes (à ce moment là on pourra aussi justifier avec le 
%résultat sur les tordues).
On rappelle qu'une courbe elliptique $E/\GF{q}$ est munie de l'endomorphisme de
Frobenius $\phi_q : [X, Y, Z] \mapsto [X^q, Y^q, Z^q]$ qui a pour polynôme
minimal $X^2 -tX + q$ avec $\#E(\GF{q}) = q + 1 - t$. Le polynôme
minimal se scinde dans une extension quadratique et devient $(X -
\alpha)(X - \beta)$. On rappelle aussi que $\#E(\GF{q^n}) = q^n + 1 - (\alpha^n
+ \beta^n)$.\par
Si $m$ est premier avec $p$ alors le groupe de $m$-torsion $E[m]$ est de rang
$2$ et pour toute base fixée $\phi_q$ agit comme une matrice $2\times2$ à
coefficients dans $\zmodn{m}$ sur $E[m]$.
\begin{defn}
Soit $E/\GF{q}$ une courbe elliptique, un nombre premier $\ell$ est appelé
nombre premier de Elkies pour $E$ si le polynôme minimal de $\phi_q$ se scinde
en deux facteurs distincts dans $\zmodn{\ell}$.
\end{defn}
Si $\ell$ est un premier de Elkies, alors pour une base de $E[\ell]$, $\phi_q$
agit comme une matrice diagonale :
\begin{equation}
\begin{pmatrix}
\alpha & 0\\
0 & \beta
\end{pmatrix}
\end{equation}
et pour un entier $n$, le Frobenius de $E/\GF{q^n}$ agit comme une puissance
\nroot{n} de cette matrice sur $E[\ell]$. Ainsi, l'extension de degré
$k = \textup{min}(\ord{\ell}{\alpha},\ord{\ell}{\beta})$ de 
$\GF{q}$ est la plus petite extension contenant un point d'ordre $\ell$, si on
prend $P\in E[m]$ dans l'espace propre de $\alpha$  et que $\ord{\ell}{\alpha} =
k$ alors :
\begin{equation}
\phi_{q^k}(P) = [\alpha^k]P = [1]P = P
\end{equation}
ce qui est le critère d'appartenance à $E/\GF{q^k}$.

Ce résultat serait généralisé au
puissance d'un nombre premier de Elkies par relèvement $\ell$-adique.\par 
%TODO: rajouter le cas d'une puissance d'un elkie en parlant du lift en l-adic
Autrement dit, si $\textup{min}(\ord{m}{\alpha},\ord{m}{\beta}) = n$, on obtient
alors des générateurs de $\GF{q^n}$. Pour cela, il faut prendre un bon $m$ et
surtout une bonne courbe elliptique, \emph{i.e.} prendre pour $m$ un premier
Elkies ou une puissance d'un tel nombre premier semble être la meilleure
solution. Pour $m$ premier ou puissance d'un premier de Elkies, on a :
\begin{equation}
X - \overline{t_E}X + \overline{q} = (X - \overline{\alpha})(X -
\overline{\beta})\bmod m
\end{equation}
Si $n$ divise $m$ alors il nous suffit de chercher les éléments
$z\in\zmodninv{m}$ tels que $z^n = 1$ et de vérifier si :
\begin{equation}
\ord{m}{q/z} > n \etmath \vert{\overline{z + q/z}}\vert\leq\sqrt{2q}
\end{equation}
où dans la seconde équation, on prend le relevé de $z + q/z$ dans $\ZZ$. Les $z$
vérifiant ses conditions seront les candidats pour les classes de traces
admissibles. Elles serviront à déterminer quelle courbe elliptique on utilisera
pour appliquer l'algorithme. Pour sélectionner une courbe, il suffit de calculer
la trace de son Frobenius et la trace du Frobenius de ses tordues en accord avec
la proposition \ref{prop:trtwist}.

\subsection{Périodes elliptiques}
\label{sec:perell}
%TODO: Prouver et définir les périodes elliptiques 
La caractéristique principale de l'algorithme de Rains, comme on a pu le voir
dans la section précédente, est l'utilisation d'éléments normaux uniques sous
l'action d'un certains groupes. Ici, il se trouve que l'unicité sera déduite de
l'action du Frobenius de la courbe elliptique choisie $E/\GF{q}$ sur des points
de torsion d'ordre $m$ de la courbe $E(\GF{q^n})$.
%TODO: Parler des automorphismes non triviaux
\begin{defn}
Soit $E/\GF{q}$ une courbe elliptique avec $j_E\neq0$ et $p\neq2, 3$ et soit $m$
une puissance d'un nombre premier de Elkies. Supposons que
$\#\textup{Aut}_{\GF{q}}(E)$ divise $\varphi(m)$ et considérons
$\textup{Aut}_{\GF{q}}(E)$ comme un sous-groupe de $\zmodninv{m}$. Posons
$\alpha\in\zmodninv{m}$ une valeur propre du Frobenius modulo $m$ et supposons
que $\zmodninv{m}/\textup{Aut}_{\GF{q}}(E) = \groupgen{\alpha}\times S$. Alors
pour tout point $P$ d'ordre exactement $m$ dans l'espace propre engendré par
$\alpha$, on définit la période elliptique :
\begin{equation}
\eta_{\alpha}(P) = \sum_{\sigma\in S}
{\left([\sigma]P\right)_X^{\#\textup{Aut}_{\GF{q}}(E)/2}}
\end{equation}
où $P_X$ désigne l'abscisse du point $P$.
\end{defn}
\begin{prop}
L'endomorphisme de Frobenius $\phi_q$ agit transitivement sur les périodes 
elliptiques $\eta_{\alpha}(P)$.
\end{prop}
\begin{proof}
Comme $P$ est dans l'espace propre de la valeure propre $\alpha$, lui appliquer 
$\phi_q$ revient à le multiplier par $\alpha$. Si on prend 
$\tau\in\groupgen{\alpha}$ alors on a pour $d\geq1$ tel que $\alpha^d = \tau 
\bmod m$ :
\begin{equation}
\phi_q^d(\eta_{\alpha}(P)) = \sum_{\sigma\in
S}{\left([\sigma][\tau]P\right)_X^{\#\textup{Aut}(E)/2}}
\end{equation}
Or, on a $\zmodninv{m}/\textup{Aut}(E) = \groupgen{\alpha} \times S$ donc
$[\tau]P$ est encore un point primitif d'ordre $m$ dans l'espace propre de 
$\alpha$.\\
\end{proof}
\subsection{Algorithme et analyse de complexité}
\label{sec:algcompell}
%TODO: Décrire l'algorithme et donner sa complexité.
Comme la partie détermination de l'isomorphisme est la même pour la partie
cyclotomique que pour la partie elliptique, on ne va traiter que la recherche
d'élément normal unique (sous l'action du Frobenius).\par

\subsubsection*{Recherche d'un élément normal unique}
On cherche donc un entier $m$ puissance d'un nombre premier de Elkies ou bien
puissance de $p$, tel que $n\mid\varphi(m)$ et tel qu'il existe au moins un 
$t\in\zmodninv{m}$ tel que :

\vspace{0.3cm}
\begin{enumerate}[(1)]
    \item si $m$ est égal à $p$, alors $\ord{m}{t} = n$;

    \item si $m$ est un premier de Elkies, alors
    $\textup{min}(\ord{m}{\alpha},\ord{m}{\beta}) = n$, où $t = 
    \overline{\alpha} + \overline{\beta}$.

    \item Selon le cas, on doit aussi avoir
    $\zmodninv{p^k}/\textup{Aut}_{\GF{q}}(E) = \groupgen{t}\times S$ ou
    $\zmodninv{m}/\textup{Aut}_{\GF{q}}(E) = \groupgen{\alpha}\times S$ si en
    plus $\alpha$ est la valeur propre d'ordre minimal égal à $n$.
\end{enumerate}
\vspace{0.3cm}
Les trois points sont nécessaires, selon le cas dans lequel on se trouve, pour
que les périodes elliptiques définies plus haut forment effectivement une base
normale de $\GF{q^n}$.\par
La méthode qu'on utilise pour sélectionner la courbe elliptique sur la quelle on
va travailler est assez simple pour être expliquée succintement. On commence par
regarder les courbes de $j$ invariant $0$ ou $1728$ sur $\GF{q}$; selon la
valeur de $q$ modulo $3$ pour le premier et modulo $4$ pour le second, on teste
si les classes des différentes traces décrites dans la section
\ref{sec:singcurve} sont parmi les bons candidats associés à l'entier $m$
(\emph{cf} \ref{sec:recherchermell}). Si aucune des dix courbes testées ne
convient, alors on teste chaque courbe de $j$-invariant différent de $0$ et 
$1728$ et leurs tordues quadratiques en parcourant tout $\GF{q}$ jusqu'à tomber 
sur une qui marche (Ça serait bien d'avoir une preuve que ça arrive forcément.. 
Peut-être \cite{CasHen} ?).\par
Une fois la courbe et $m$ obtenus, on peut exécuter l'algorithme de Rains
proprement dit, \emph{cf} algorithme \ref{alg:gaussell}.
\begin{algorithm}
\caption{Détermination d'un élément normal unique}
\label{alg:gaussell}
\begin{algorithmic}[1]
\REQUIRE $m$ un entier satisfaisant les conditions (1) et (2), $E/\GF{q^n}$ une 
courbe elliptique avec les bonnes caractéristiques
\ENSURE $\eta$ un élément normal unique de $\GF{q^n}$
\bigskip
\REPEAT
    \STATE $P \leftarrow [\frac{\#E/\GF{q^n}}{m}]\textup{rand}(E/\GF{q^n})$
\UNTIL{$[m]P = \EO\quad\&\quad\forall d\mid m, d\neq m, [d]P\neq\EO$}
\STATE $\eta \leftarrow \sum_{\sigma\in
S}{\left([\sigma]P\right)_X^{\#\textup{Aut}(E)/2}}$
\RETURN $\eta$
\end{algorithmic}
\end{algorithm}

\begin{prop}
L'algorithme \ref{alg:gaussell} est correct et de complexité :
\begin{equation}
O((m\,\textup{log\,}m)\I{n}/n).
\end{equation}
\end{prop}


\part{Implémentation}
\label{trois}
%TODO: Comparer les deux méthodes ou même avec l'implémentation de base de SAGE 
%(retrouver une simple racine). Détailler les limitations dû à SAGE, peut-être
%prendre une partie de la conclusion en expliquant ce qui pourrait être changer.
Dans cette partie, nous discuterons de l'implémentation des algorithmes, des
résultats numériques obtenues et nous comparrons ces résultats avec d'autres
méthodes déjà mises en place.\par
\emph{(Les caracs ne sont plus à jour)}\,
Les tests sont effectués sur une machine virtuel sous le système Linux 
3.12.9-1-ARCH, dotée d'un processeur Intel Xeon E312xx 4 c\oe urs à 
$3092.968$MHz et de $1.23$go de RAM. Les algorithmes ont été implémentés et 
exécutés sous le logiciel \href{http://www.sagemath.org/}{SAGE} en version 
$6.2$ et \emph{(peut-être)} une partie du code fait usage de la bibliothèque
\href{http://www.flintlib.org/}{FLINT}. Des limitations du logiciel SAGE nous
empêche de traiter le cas des extensions de $\GF{q}$ avec $q = p^r$ et $r>1$; on
n'implémentera alors que le cas $q = p$.

\section{Recherche du paramètre \emph{m}}
%TODO: Des détails sur m et comment on le trouve (forme an + 1, premier ou
%puissance d'un nombre premier; pour la méthode cyclotomique et elliptique
La détermination d'une borne optimale pour $m$ nécessiterait un peu plus
d'expérimentations que ce qui a été effectué jusqu'à maintenant. La solution 
adoptée dans le programme est de fixer une borne arbitraire et de chercher un
nombre premier :
\begin{equation}
\label{eq:formedem}
m := an + 1 
\end{equation}
qui satisfasse les conditions nécessaires pour le bon fonctionnement
des différents algorithmes. En effet, il y a de bonne chance qu'un entier de ce
type nous convienne puisque nous avons besoin d'un élément d'ordre $n$ dans
$\zmodninv{m}$ et que cette forme implique forcément que $\varphi(m) = 
an$ soit divisible par $n$. Bien entendu, le théorème de la progression
arithmétique nous assure que tout cela est bien possible.\par
On teste alors tous les entiers de la forme (\ref{eq:formedem}) en augmentant la
valeur de $a$ jusqu'à tomber sur le premier bon candidat ou atteindre la borne.
Si la borne est atteinte sans qu'un bon candidat ait été trouvé, alors 
l'algorithme a échoué et il faut recommencer avec une borne plus grande. On 
rappelle que les critères de sélection pour l'entier $m$ sont donnés au 
\ref{sec:algcompcycl}, pour la méthode cyclotomique, et au \ref{sec:algcompell},
pour la méthode elliptique.

\subsection{Pour la méthode cyclotomique}
\label{sec:recherchemcycl}

La recherche de $m$ dans la méthode cyclotomique est plutôt directe. On regarde
les entiers de la forme (\ref{eq:formedem}) et on test si $m$ est effectivement
premier, si $\ord{m}{p}$ est divisible par $n$ et si le cofacteur
$o := \ord{m}{p}/n$ est premier avec $n$ et dans ce cas si 
$(\varphi(m)/\ord{m}{p}, \ord{m}{p}) = 1$,. Les autres conditions ne sont plus 
nécessaires :
\vspace{0.3cm}
\begin{itemize}
    \item $(\varphi(m), r) = 1$ disparait automatiquement comme on
    ne considère uniquement le cas $q = p$;
    
    \item $m$ est nécessairement sans facteur carré puisqu'on le prend premier;

    \item $\zmodninv{m} = \groupgen{p^o}\times S$ est une conséquence des
    conditions déjà énoncées.
\end{itemize}
\vspace{0.3cm}

On sélectionne le plus petit $m$ qui vérifie ces conditions et on continue alors
l'exécution de l'algorithme.

\begin{rem}
Sélectionner le plus petit $m$ n'est pas nécessairement la solution la plus
judicieuse, il se pourrait que les suivants ne nécessitent pas l'utilisation
d'extensions supplémentaires par exemple. Cependant, devoir tester tous les $m$
possibles, ici cela peut aller jusqu'à $m = q^n - 1$, peut prendre beaucoup de
temps et traiter avec des $m$ de cette taille implique des complexité
exponentielle. Une solution pourrait être d'utiliser une deuxième borne plus
restrictive, de garder en mémoire les $m$ testés et de vérifier si les $m$ entre
l'ancienne borne et la nouvelle fonctionnent et ainsi de suite jusqu'à atteindre
la borne de départ.
\end{rem}

\subsection{Pour la méthode elliptique}
\label{sec:recherchermell}
Pour la méthode elliptique, la situation est à peine plus délicate, il y a la
condition du nombre de trace admissible pour un $m$ en plus. Comme pour la 
méthode cyclotomique, au départ on cherche un $m$ premier de la forme $an + 1$. 
Pour la suite, le traitement est différent selon si $m$ est égal à $p$ ou non.
Dans les deux cas, il faut trouver les éléments d'ordre $n$ de $\zmodninv{m}$.
Une fois ceux-ci obtenus, on effectue les tests suivants :
\vspace{0.3cm}
\begin{itemize}
    \item Si $m = p$, alors on teste simplement si le relevé dans $\ZZ$ des
    classes $\overline{t}$ d'ordre $n$ sont dans le bon intervalle, 
    \emph{i.e.} si $t\in[-\sqrt{2q},\sqrt{2q}]$.\vspace{0.2cm}

    \item Si $m\neq p$, alors il faut effectuer un peu plus de travail. On
    veut que la valeur propre du Frobenius de plus petit ordre soit d'ordre $n$.
    Comme on a vu plus haut, le polynôme se scinde comme suit 
    \[(X - \overline{\alpha})(X- \overline{\beta})= X -\overline{t}X + 
    \overline{q}\bmod m.\] 
    Il faut alors tester si pour $\overline{\alpha}\in\zmodninv{m}$ 
    d'ordre $n$, on a $\overline{q/\alpha}$ d'ordre plus grand que $n$ et 
    $\vert{\alpha + q/\alpha}\vert\leq\sqrt{2q}$.
\end{itemize}
\vspace{0.3cm}
Si $m$ n'admet aucun candidat pour des traces admissibles, alors on passe
au suivant, sinon on continue l'algorithme.

\begin{rem}
Contrairement au cas précédent, l'expérience montre que prendre le premier $m$ 
semble être la solution la plus judicieuse. En effet, tant que $m$ est plus 
petit que $\sqrt{2q}$, le nombre de candidat semble être relativement 
raisonnable, une fois cette borne dépassée, le nombre de candidats par valeur 
diminue drastiquement. Pour plus de détails, on peut consulter l'article 
\cite{CasHen} (perso j'y vois rien pour le moment...).\par
Un autre changement possible est le nombre de trace admissibles minimal qu'on
souhaite obtenir pour valider un candidat. Pour le moment, on ne demande d'avoir
qu'un seul bon candidat, mais on peut imaginer en demander $d>1$. 
L'idée serait que cela puisse améliorer la probabilité avec laquelle on pourrait
directement tomber sur une bonne trace. Cela pourrait être utile dans le cas où
le degré de l'extension devien plus grand que la borne de Hasse comme on le
verra plus tard.\par
Comme les courbes sont choisies sur le corps de base $\GF{q}$, si $q$ est petit
alors le nombre de trace disponible est moindre, ainsi les chances de tomber sur
une bonne courbe diminue. (Je suis plus convaincu par cette deuxième partie) 
%TODO: On pourrait faire un tableau pour tout ça ?
De plus, le degré de l'extension $n$ doit diviser $\varphi(m)$, alors si $n > 
\sqrt{2q}$ \emph{a fortiori} $m >\sqrt{2q}$ et le nombre de bons candidats 
baisse à nouveau. Il serait intéressant de pousser l'étude de ces critères 
(avec plus de temps.. ?) afin de déterminer quelle méthode utiliser dans un 
algorithme général qui regrouperait les deux méthodes.
%TODO: On pourrait même comparé les temps entre la méthode elliptique avec un n
%plus grand que la borne de Hasse et la méthode cyclotomique qui aurait besoin
%d'une extension; histoire de savoir si on garde quand même la méthode
%elliptique avec peu de traces (de classe de trace).
\end{rem}

\section{Résultats numériques}
% TODO:
% - Temps d'exécution complet pour les deux méthodes sur différents paramètres
% (déterminer le (ou les) éléments + calcule de l'isomorphisme);
%
% - Temps d'exécution pour la détermination des éléments uniques pour les deux
% méthodes;
%
% - Comparer éventuellement le nombre de traces trouvées et le nombre de courbes
% testées avant de trouver une bonne courbe
%
% - Temps pour la détermination de l'isomorphisme (montrer plus tard que
% finalement, cette méthode est pas si géniale :<);
%
% - Autre chose ?

\section{Comparaison avec d'autres méthodes}
%TODO: 
% - Comparer cyclotomique vs elliptic, avec et sans extensions;
%
% - Comparer cyclo vs allombert
%
% - Comparer allombert vs ellpitic
%
% - Comparer détermination de l'isomorphisme entre l'algo de javad et le mien
% (ou bien je le fais avant ?)


\part{Conclusion}
\label{quatre}
%TODO: Expliquer ce qu'il peut rester à faire :  car(K) = 2; n  composé; couper 
% et appliquer l'une des deux méthodes selon la nature du m ou s'il y a besoin 
% de calculer une extension, etc.); toute la partie sur la détermination de
% l'isomorphisme (plus profond etc.)
La première chose à noter est l'absence des caractéristiques $2$ et $3$. Si on
peut appliquer théoriquement sans trop de problème les algorithmes à la
caractéristique $3$, la caractéristique $2$ quant à elle est plus
pathologique.\par
Plus précisément, pour les deux algorithmes de Rains cyclotomique un des 
critères nécessaires au fonctionnement de la méthode est $(\varphi(m)/n, n) = 1$
ou de manière équivalente $(\varphi(m)/\ord{m}{p},\ord{m}{p}) = 1$
lorsque $\ord{m}{p} = n$. Or, il se trouve que $2$ est un résidu
quadratique modulo n'importe quel premier $m$ de la forme $8k + 1$, alors on
aura toujours $(\varphi(m)/\ord{m}{2},\ord{m}{2}) > 1$ pour un tel
$m$. Donc si $p = 2$ et $8\mid n$, il nous est impossible de trouver un $m$
satisfaisant les conditions.\par
Rains propose déjà deux solutions dans son article pour la méthode cyclotomique,
il serait intéressant de pouvoir les implémenter et vérifier si elles
fonctionnent pour la méthode elliptique ou, à défaut, d'en déterminer de
nouvelles.

\vspace{0.3cm}

En ce qui concerne l'implémentation elle-même, les paramètres $m$ et $n$ ont 
toujours été sélectionnés de telle sorte qu'ils soient premiers. Pour pouvoir 
traiter le cas général, une solution est de factorisé $n$, de faire l'étude sur 
chacun de ses facteurs et enfin recoller les morceaux.\par
En particulier, pour chaque facteur on cherche alors des $m_i$ premiers de la 
forme $ap_i^{v_{p_i}(n)} + 1$, pour $p_i$ divisant $n$, et selon si 
l'utilisation d'une extension ou non est nécessaire, on utilise la méthode 
elliptique ou cyclotomique; le $m$ de la méthode serait donc le produit des 
$m_i$. L'idée serait alors de combiner le tout dans un programme linéaire.\par
Quant au cas $q = p^r$ avec $r>1$, cela ne dépend malheureusement que du
logiciel utilisé et de la façon dont il gère les corps finis et leurs
extensions. Tant qu'une façon de considérer les extensions d'extensions 
n'est pas correctement implémentée, il faudra se contenter du cas $r = 1$,
\emph{i.e.} de partir du corps premier $\GF{p}$.

\vspace{0.3cm}

Enfin, un autre problème conséquent du calcul d'isomorphismes de corps fini est 
la détermination de l'image du générateur de la base monomial. Comme on a pu le
voir, si les algorithmes ont des performances acceptables pour déterminer deux
éléments définissant un isomorphisme, la solution proposée pour calculer l'image
du générateur de la base monomiale est quant à elle plutôt médiocre comparée à
d'autres algorithmes déjà implémenté.\par
Il se trouve que c'est un problème riche et qui mérite une attention toute
particulière. On peut imaginer qu'en combinant un algorithme performant
résolvant ce problème et ceux présentés dans ce mémoire, on puisse finalement
résoudre le problème d'isomorphisme dans des temps acceptables; il restera alors
à traiter l'évaluation de cet isomorphisme.

\addcontentsline{toc}{part}{Références}
\begin{thebibliography}{LC}

\bibitem{All} \emph{Explicit Computation of Isomorphisms between Finite
Fields}, \bsc{Bill Allombert}, Elsevier Science, 2002, \bsc{url :}
\url{http://www.sciencedirect.com/science/article/pii/S1071579701903442}

\bibitem{CasHen} \emph{The distribution of the number of points modulo an
integer on elliptic curves over finite fields}, \bsc{Wouter Castryck} \&
\bsc{Hendrik Hubrechts}, 2009, \url{http://arxiv.org/abs/0902.44332} 

\bibitem{Esc} \emph{Théorie de Galois}, \bsc{Jean-Pierre Escofier}, Dunod, 2nde
éd., 2000.

\bibitem{FeGaSho} \emph{Normal Basis \textup{via} General Gauss Periods},
\bsc{Sandra Feisel}, \bsc{Joachim von zur Gathen} \& \bsc{M. Amin Shokrollahi}, 
Mathematics of computation, Volume 68, Number 225, January 1999, p. 271-290

\bibitem{GaGe} \emph{Modern Computer Algebra}, \bsc{Joachim von zur Gathen} \&
\bsc{Jürgen Gerhard}, 3rd edition, Cambridge

\bibitem{Lan} \emph{Algebraic Number Theory}, \bsc{Serge Lang}, Graduate Texts
in Mathematics, 2nd ed., Springer, 2000.

\bibitem{LiNi1} \emph{Finite Fields}, \bsc{Rudolf Lidl} \& 
\bsc{Harald Niederreiter}, Encyclopedia of mathematics and its applications vol.
20, Cambridge, 1983.

\bibitem{LiNi2} \emph{Introduction to Finite Fields and their Applications},
\bsc{Rudolf Lidl} \& \bsc{Harald Niederreiter}, Cambridge University Press,
1986.

\bibitem{MuPa} \emph{Handbook of Finite Fields}, \bsc{Gary L. Mullen} \& 
\bsc{Daniel Panario}, Discrete mathematics and its applications, Series Editor 
Kenneth H. Rosen, CRC Press.

\bibitem{MiMoScho} \emph{Computing the Eigenvalue in the Schoof-Elkies-Atkin
Algorithm using Abelian Lifts}, \bsc{P. Mih\u{a}ilescu}, \bsc{F. Morain} \&
\bsc{É. Schost}, 2007, \bsc{url :}
\url{http://hal.inria.fr/LIX/inria-00130142/en/}

\bibitem{Per} \emph{Cours d'algèbre}, \bsc{Daniel Perrin}, ellipses, 1996.

\bibitem{Pin} \emph{Recognising Elements of Finite Fields}, \bsc{Richard G.E. 
Pinch}, Cryptography and coding II, p. 193-197, Oxford University Press, 1992.

\bibitem{Rai} \emph{Efficient Computation of Isomorphism Between Finite Fields},
\bsc{Eric M. Rains}, 1996.

\bibitem{Sam} \emph{Théorie algébrique des nombres}, \bsc{Pierre Samuel}, 
Hermann, 1971.

\bibitem{Sil} \emph{The Arithmetic of Elliptic Curves}, 
\bsc{Joseph H. Silverman}, Graduate texts in mathematics, Springer, 2nd ed. 
2009.

\bibitem{Was1} \emph{Introduction to Cyclotomic Fields}, \bsc{Lawrence C. 
Washington}, Graduate texts in mathematics, Springer-Verlag, 1982.

\bibitem{Was2} \emph{Elliptic Curves, Number Theory and Cryptography},
\bsc{Lawrence C. Washington}, Chapman \& Hall/CRC, 2nd éd., 2008

\end{thebibliography}
\end{document}

\documentclass[a4paper]{article} % papier A4
\usepackage[utf8]{inputenc}      % accents dans le source
\usepackage[T1]{fontenc}         % accents dans le pdf
\usepackage{textcomp}            % symboles complémentaires (euro)
\usepackage[frenchb]{babel}      % titres en français
\usepackage{amsmath}
\usepackage{amsthm}
\usepackage{amssymb}
\usepackage[colorlinks=false]{hyperref}
\usepackage{enumerate}
\usepackage{algorithmic}
\usepackage{pgf}
\usepackage{tikz}
\usepackage{tikz-cd}
\usetikzlibrary{matrix,arrows,decorations.pathmorphing}
\numberwithin{equation}{section}
\newcommand\nroot[1]{\textit{#1}\up{\textit{ième}}}
\newcommand\zmodn[1]{\mathbb{Z}/#1\mathbb{Z}}
\newcommand\zmodninv[1]{(\mathbb{Z}/#1\mathbb{Z})^{\times}}
\newcommand\GF[1]{\mathbb{F}_{#1}}
\newcommand\Irr[2]{\textup{Irr}_{#1}(#2)}
\renewcommand{\algorithmicrequire}{\textbf{Input:}}
\renewcommand{\algorithmicensure}{\textbf{Ouput:}}
\newcommand\Tr[1]{\textup{Tr}\left(#1\right)}
\begin{document}
\newtheorem{thm}{Thèorème}[section]
\newtheorem{lem}[thm]{Lemme}
\newtheorem{cor}{Corollaire}[thm]
\newtheorem{prop}[thm]{Proposition}
\theoremstyle{definition}
\newtheorem{defn}[thm]{Définition}
\newtheorem*{ex}{Exemple}
\theoremstyle{remark}
\newtheorem{rem}{Remarque}[thm]

\begin{thm}Soit $K = \GF{q}$ un corps fini à $q = p^r$ éléments, avec $p\neq2,3$. Soit $E/K$ et $E'/K$ deux courbes elliptiques et $u\in\overline{K}$ tel que l'application :

\begin{center}
\begin{tikzpicture}
    \matrix(m)[matrix of math nodes,
    row sep=0em, column sep=3.5em,
    text height=1.5ex, text depth=0.25ex]
    {E & E^{\prime}\\
    (x,y) & (u^2x, u^3y)\\};
    \path[->,font=\scriptsize,>=angle 90]
    (m-1-1) edge node[auto] {\normalsize{$\overline{u}$}} (m-1-2);
    \path[|->,font=\scriptsize,>=angle 90]
    (m-2-1) edge node[auto] {} (m-2-2);
\end{tikzpicture}
\end{center}
soit un $\overline{K}$-isomorphisme de courbes elliptiques. On a l'une des deux situations suivantes :

\begin{enumerate}[i)]
    \item $u^{q-1}\not\in \GF{p}$ et $E$, $E^{\prime}$ sont supersingulières,
    \item $u^{q-1}\in \GF{p}$ et $t^{\prime} = \alpha.t$ avec $\alpha = u^{q-1} \bmod p$ et $|\alpha| < \tfrac{q}{2}$.
\end{enumerate}
où $t$ et $t^{\prime}$ sont les traces du Frobenius sur les courbes $E$ et $E^{\prime}$ respectivement.
\end{thm}
\begin{proof} 
i) Pour commencer, si $u\in K$ alors $u^{q-1} = 1\in \GF{p}$ et il s'agit du cas suivant. Prenons alors $u\in\overline{K}\setminus K$ et supposons que $u^{q-1}\not\in \GF{p}$. \par
On note :
\[E : y^2 = x^3 + ax + b\] 
avec $a$, $b\in K$. On suppose aussi que $j(E) = j(E^{\prime})\neq 0, 1728$ alors en appliquant l'isomorphisme, on a :
\[E^{\prime} : y^2 = x^3 + u^4ax + u^6b\]
avec $u^4, u^6\in K$; donc $u^2\in K$. Comme $u\not\in K$ cela veut dire que $u = \sqrt{c}$ pour $c\in K$ non-résidu quadratique. Mézalor, on a la situation suivante :
\[c^{q-1} = 1 = c^{\tfrac{q-1}{2}} = u^{q-1}\in \GF{p}\]
ce qui est en contradiction avec les hypothèses de départ. Les seuls cas possibles sont donc $j = 0$ ou $1728$.\par
Commençons par le cas $j = 0$, on a alors :
\[E : y^2 = x^3 + 1 \textup{ et } E^{\prime} : y^2 = x^3 = u^6.\]
Autrement dit $u^6 = c \in K$, donc $c^{\tfrac{q-1}{6}} \notin \GF{p}$ par hypothèse. Si $6$ ne divise pas $q-1$ alors il n'y a pas de racine $6^{\textit{ième}}$ de l'unité dans $\GF{q}$ donc dans $\GF{p}$.\\
Si $6$ divise $q-1$ alors $\omega = c^{\tfrac{q-1}{6}}\in K$ racine $6^{\textit{ième}}$ de l'unité et en plus $\omega\notin\GF{p}$; on a donc encore $\mu_6\not\subset\GF{p}$. Or, on a vu plus haut \textit{(pas encore écrit)} que les seuls automorphismes des courbes de $j$-invariant nul étaient définis par des racines $6^{\textit{ième}}$ de l'unité; l'élément $\omega$ définit donc un automorphisme non-$\GF{p}$-rationnel, \textit{i.e.} $\overline{\omega}^p\neq \overline{\omega}$.\par
Considérons alors le diagramme suivant :

\begin{center}
\begin{tikzpicture}
    \matrix(m)[matrix of math nodes,
    row sep=4em, column sep=3.5em,
    text height=1.5ex, text depth=0.25ex]
    {E/\GF{p} & E^{\prime}/\GF{p}\\
    E/\GF{p} & E^{\prime}/\GF{p}\\
    E/\GF{p}\\};
    \path[->,font=\scriptsize,>=angle 90]
    (m-1-1) edge node[auto] {$\overline{\omega}$} (m-1-2)
    (m-1-1) edge node[auto] {$\pi$} (m-2-1)
    (m-1-2) edge node[auto] {$\pi^{\prime}$} (m-2-2)
    (m-2-1) edge node[auto] {$\overline{\omega}$} (m-3-1)
    (m-2-2) edge node[auto] {$\overline{\omega}^{-p+1}$} (m-3-1);
\end{tikzpicture}
\end{center}
En partant sur la droite, on a :
\begin{center}
\begin{tikzpicture}
    \matrix(m)[matrix of math nodes,
    row sep=4em, column sep=3.5em,
    text height=1.5ex, text depth=0.25ex]
    {(x,y) & (\omega^2x,\omega^3y) & (\omega^{2p}x^p,\omega^{3p}y^p) & (\omega^2x^p,\omega^3y^p)\\};
    \path[|->,font=\scriptsize,>=angle 90]
    (m-1-1) edge node[auto] {$\overline{\omega}$} (m-1-2)
    (m-1-2) edge node[auto] {$\pi^{\prime}$} (m-1-3)
    (m-1-3) edge node[auto] {$\overline{\omega}^{-p+1}$} (m-1-4);
\end{tikzpicture}
\end{center}
et en partant en bas :
\begin{center}
\begin{tikzpicture}
    \matrix(m)[matrix of math nodes,
    row sep=4em, column sep=3.5em,
    text height=1.5ex, text depth=0.25ex]
    {(x,y) & (x^p,y^p) & (\omega^2x^p,\omega^3y^p)\\};
    \path[|->,font=\scriptsize,>=angle 90]
    (m-1-1) edge node[auto] {$\pi$} (m-1-2)
    (m-1-2) edge node[auto] {$\overline{\omega}$} (m-1-3);
\end{tikzpicture}
\end{center}
De manière plus compact, on a montré que :
\[\overline{\omega}\circ\pi = \overline{\omega}^{-p+1}\circ\pi\Leftrightarrow\pi\circ\overline{\omega} = \overline{\omega}^p\circ\pi\]
Or, $\overline{\omega}^p\neq \overline{\omega}$, on a donc au moins un automorphisme qui ne commute pas avec le Frobenius, ce qui équivaut \textit{(mettre une référence ici)} à ce que les courbes soient supersingulières.\par
Le cas $j = 1728$ se traite exactement de la même façon, à ceci près que les courbes s'écrivent :
\[ E : y^2 = x^3 + x \textup{ et } E^{\prime} : y^2 = x^3 + u^4x.\]
Ainsi, $u^4\in K$, ce qui nous permet alors de montrer comme ci-dessus qu'il n'y a pas d'automorphismes $\GF{p}$-rationnels et donc, au moyen du schéma précédent, on montre que dans ce cas aussi on trouve deux automorphismes qui ne commutent pas avec $\pi$, donc que les deux courbes sont supersingulières.




\end{proof}

\end{document}
